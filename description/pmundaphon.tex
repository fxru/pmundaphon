\documentclass[a4paper,]{article}
\usepackage{lmodern}
\usepackage{amssymb,amsmath}
\usepackage{ifxetex,ifluatex}
\usepackage{fixltx2e} % provides \textsubscript
\ifnum 0\ifxetex 1\fi\ifluatex 1\fi=0 % if pdftex
  \usepackage[T1]{fontenc}
  \usepackage[utf8]{inputenc}
\else % if luatex or xelatex
  \ifxetex
    \usepackage{mathspec}
  \else
    \usepackage{fontspec}
  \fi
  \defaultfontfeatures{Ligatures=TeX,Scale=MatchLowercase}
\fi
% use upquote if available, for straight quotes in verbatim environments
\IfFileExists{upquote.sty}{\usepackage{upquote}}{}
% use microtype if available
\IfFileExists{microtype.sty}{%
\usepackage[]{microtype}
\UseMicrotypeSet[protrusion]{basicmath} % disable protrusion for tt fonts
}{}
\PassOptionsToPackage{hyphens}{url} % url is loaded by hyperref
\usepackage[unicode=true]{hyperref}
\hypersetup{
            pdfborder={0 0 0},
            breaklinks=true}
\urlstyle{same}  % don't use monospace font for urls
\usepackage{longtable,booktabs}
% Fix footnotes in tables (requires footnote package)
\IfFileExists{footnote.sty}{\usepackage{footnote}\makesavenoteenv{long table}}{}
\IfFileExists{parskip.sty}{%
\usepackage{parskip}
}{% else
\setlength{\parindent}{0pt}
\setlength{\parskip}{6pt plus 2pt minus 1pt}
}
\setlength{\emergencystretch}{3em}  % prevent overfull lines
\providecommand{\tightlist}{%
  \setlength{\itemsep}{0pt}\setlength{\parskip}{0pt}}
\setcounter{secnumdepth}{0}
% Redefines (sub)paragraphs to behave more like sections
\ifx\paragraph\undefined\else
\let\oldparagraph\paragraph
\renewcommand{\paragraph}[1]{\oldparagraph{#1}\mbox{}}
\fi
\ifx\subparagraph\undefined\else
\let\oldsubparagraph\subparagraph
\renewcommand{\subparagraph}[1]{\oldsubparagraph{#1}\mbox{}}
\fi
\usepackage{graphicx}
% set default figure placement to htbp
\makeatletter
\def\fps@figure{htbp}
\makeatother

\setmainfont{Charis SIL}

\date{}

\begin{document}

\section{Towards a Historical Phonology of the Munda
languages}\label{towards-a-historical-phonology-of-the-munda-languages}

\textbf{Version 0.1}

Felix Rau University of Cologne (f.rau@uni-koeln.de)

This paper aims at presenting a historical phonology of the Munda
languages. In its current state it aims for detailed information and
less for readability. It still remains very much work in progress and
should be taken as such.

Pinnow (1959) still state of the art for the reconstruction of
proto-Munda. The book covers the historical phonology of the Munda
branch and connects it with other branches of the Austroasiatic family,
despite its modest title ``Versuch einer Historischen Lautlehre der
Kharia‐Sprache'' (``Attempt of a historical phonology of the Kharia
language''). However, it has not received the reception it deserves.
This may be partially due to the fact that this immense work was
published in German and partially owed to the fact that it is a very
dense text. This aims to be a less comprehensive, but more accessible
update on our understanding of the historical phonology of the Munda
languages.

The Munda branch consists of approximately twenty languages, some very
closely related, some lexically and structurally quite distinct. Views
on what variety is granted the status of a separate languages differ,
but Glottolog lists 22 Munda languages: Gtaʔ, Gutob, Remo, Juang,
Kharia, Sora, Juray, Gorum, Korku, Asuri, Bijori, Birhor, Ho, Mundari,
Koda, Kodaku, Korwa, Majhwa, Turi, Kol, Mahali, Santali. This can be
taken a a reasonable reflection of the literature. Some subgrouping can
be regarded as well established. These well established subgroups are
Kherwarian, North Munda, Sora-Gorum, Remo-Gutob. Kherwarian consisting
of Asuri, Bijori, Birhor, Ho, Mundari, Koda, Kodaku, Korwa, Majhwa,
Turi, Kol, Mahali, Santali. The languages are normally grouped into the
Santali branch -- consisting of Santali, Kol, and Mahali -- and a
Mundari branch -- consisting of Asuri, Bijori, Birhor, Ho, Mundari,
Koda, Kodaku, Korwa, Majhwa, Turi. Kherwarian and Korku are generally
considered to for the North Munda branch.

The remaining languages are sometimes grouped as South Munda, but its
status as a monophyletic group is unproven and has been disputed
(Anderson \ldots{}). However, at least two groups have been identified
among the southern languages: Sora-Gorum (also called Sora-Juray-Gorum)
and Remo-Gutob. The other languages -- Gtaʔ, Juang, Kharia -- have to be
regarded as isolated branches.

The groupings Kharia-Juang and Gutob-Remo-Gtaʔ have been proposed
(CITE). The evidence for both groupings is too inconsistent to be
regarded as definite. The variety Juray has been regarded as a third
languages in the Sora-Gorum branch. Juray very closely related to Sora
and not well documented. It remains unclear that it merits to be treated
separated from other varieties of Sora.

While the differences between the branches of the Munda languages can be
substantial, the similarities inside one branch can be very high.

In this reconstruction, the following branches of Munda are assumed:
Gutob-Remo, Sora-Gorum, North Munda (consisting of the Kherwarian group
and Korku), and the individual languages Kharia, Juang, Gtaʔ. The term
\emph{southern languages} is used to designate all languages except the
North Munda branch. In our current understanding, it constitues a
paraphyletic goup.

For the current systematic reconstruction a subset of the 22 languages
has been select. The main criterium was the coverage of all known
subbranches of the Munda languages. However, the availability of lexical
material was also important. As a result twelve languages were selected
for systematic comparison: Gorum, Sora, Remo, Gutob, Kharia, Juang,
Gtaʔ, Santali, Mundari, Ho, Korwa, Korku.

The current reconstruction is based on 100 cognate sets from these
languages. These sets of cognates were collected based on their
reliability and th wide attestation among the twelve languages. From
these sets 354 correspondence sets for phonemes were identified. Where
it was possible, a proto-phoneme was posited for the correspondence set
and the phonological form of the proto-Munda word was reconstructed. In
some cases, regularities with the reconstructed forms in the MKCD were
discussed.

\subsubsection{Abreviations of Lexical Resources (to be
done)}\label{abreviations-of-lexical-resources-to-be-done}

\begin{itemize}
\tightlist
\item
  \textbf{AG08} Griffith, Arlo 2008. ``Gutob'' in Anderson, Gregory D.S.
  (ed.) \emph{The Munda Languages} London: Routledge. p.631-688.
\item
  \textbf{BAHL} Bahl, Kali Charan 1967. ``Korwa Lexicon.'' The
  University of Chicago, South Asian Languages Program. Typescript.
\item
  \textbf{BDBH} Bhattacharya, Sudhibushan 1968. \emph{A Bonda
  dictionary}. Poona: Deccan College.
\item
  BMED:
\item
  BSDV1, \ldots{}
\item
  CDES
\item
  CDSE
\item
  DHED
\item
  DSBO
\item
  DSGU
\item
  DSJU
\item
  DSKH
\item
  DSKO
\item
  DSKW
\item
  EMV5, EMV12, EMV13
\item
  \textbf{FR} Lexical material collected/compiled by myself
\item
  GGEG
\item
  \textbf{HLKS} Pinnow, Heinz-Jürgen 1959. \emph{Versuch einer
  historischen Lautlehre der Kharia-Sprache}. Wiesbaden: Harrassowitz.
\item
  HOGV
\item
  \textbf{JLIC} Anonymous n.d. ``Revised Munda lexical items list
  (Juang).'' Typescript.
\item
  MJTL
\item
  MKCD
\item
  NKEV
\item
  PGEG
\item
  \textbf{PJDW} Pinnow, Heinz-Jürgen n.d. ``Wörterverzeichnis
  Juang-Deutsch-Englisch mit etymologischen Angaben.'' Typescript.
\item
  \textbf{PKED} Peterson, John 2009. \emph{A Kharia-English Lexicon}.
  Himalayan Linguistics Archive 5.
\item
  \textbf{RSED} Ramamurti, Rao Sahib G.V. 1938. \emph{Sora-English
  Dictionary}. Madras: Government Press. Reprinted 1986, Delhi: Mittal
  Publications.
\item
  \textbf{Z1963} Zide, Norman H. 1963. ``Gutob Monosyllables.''
  Typescript.
\item
  \textbf{Z1965} Zide, Norman H. 1965. ``Gutob verb morphemes for South
  Munda work.'' (Papers on South Munda phonology III). The University of
  Chicago. Typescript.
\end{itemize}

\subsection{Comparative-Phonological Studies of the Munda
languages}\label{comparative-phonological-studies-of-the-munda-languages}

Gtaʔ /æ/ might be a phonemization of an allophone of /a/, raising the
possibility that Proto-Gtaʔ had a five vowel system, as well.

Proto-Khewarian has been reconstructed with 7 (Zide and Munda 1966,
Pinnow 1959) and 5 (Osada 1996) vowel phonemes. If Osada 1996 is to be
believed, Proto-Kherwarian had a five vowel system.

Zide (1982) reconstructs proto-Sora-Gorum (or proto-Sora-Juray-Gorum in
her nomenclature) as having a ten vowel system.

\begin{longtable}[]{@{}llll@{}}
\toprule
& Front & Central & Back\tabularnewline
\midrule
\endhead
Closed & *i & *ɨ & *u\tabularnewline
Mid & *e(e²) & *ə & *o\tabularnewline
Open-Mid & *ɛ & *ʌ & *ɔ\tabularnewline
Open & & *a &\tabularnewline
\bottomrule
\end{longtable}

(Zide 1982, p.~656)

All but Sora and Proto-Sora-(Juray)-Gorum have five vowel systems or may
have to be reconstructed as having had five vowel systems.

There is no consensus on the situation in proto-Munda. However,
correspondences are so varied and numerous that we have to assume that
the five vowel systems of most modern Munda languages are the result of
a variety of mergers in the development of the Munda languages.

There are approximately 40 distinct correspondence sets for vowels, even
if we allow for some minor variation inside one set. As a result it is
extremely difficult and often impossible to identify a proto-vowel based
on poorly attested etyma. A presumably Austroasiatic etymon for elephant
is attested in Gorum as \emph{raʔ}, Sora as \emph{raʔ}, and Gutob as
\emph{roʔ}. While the initial consonant is a clear reflex of \emph{*r},
/ʔ/ in these three languages could be either a reflex of \emph{*ˀk} or a
prosodic device to produce a heavy syllable, corresponding to
proto-Munda \emph{*Vˀ}. Thus without further investigation of the vowel,
we would already end up with two possible forms: \emph{*rVˀk} and
\emph{*rVVˀ}.

If we compare the vowel reflexes with other correspondence sets we can
identify two candidates \emph{*a} (as reflected in set \emph{*a₅}) and
\emph{*ə} as attested in \emph{*ə₁}. Additionally, the set is compatible
with an isolated reflex of unknown phological value: the second vowel of
the etymon *tv₍₁₉₎ŋv₍₂₅₎n/tv₍₁₉₎nv₍₂₅₎ŋ\_ `stand (v)' (\#0050-4 labelled
VS-025 below).

\begin{longtable}[]{@{}lllllllllllll@{}}
\toprule
Gorum & Sora & Remo & Gutob & Kharia & Juang & Gtaʔ & Santali & Mundari
& Ho & Korwa & Korku & set\tabularnewline
\midrule
\endhead
a & a & ? & o & ? & ? & ? & ? & ? & ? & ? & ? & elephant\tabularnewline
a & a & ɔ & o & o & o & o & ∅ & ∅ & ∅ & ∅ & ∅ &
\emph{*a₅}\tabularnewline
a & a(ə/o:0 & o & o & o(a) & o & wa & e̠ & e & e & e & -- &
\emph{*ə₁}\tabularnewline
a & a & (o) & o & o & o & (wa) & o & u & u & --- & e &
VS-025\tabularnewline
\bottomrule
\end{longtable}

This leaves us with the folowing possible reconstructions: \emph{*raˀk},
\emph{*raaˀ}, \emph{*rəˀk}, \emph{*rəəˀ}, \emph{*rv₍₂₅₎ˀk}, and
\emph{*rv₍₂₅₎v₍₂₅₎ˀ}.

MKCD 1930 \emph{*r{[} {]}uəs} favours \emph{*rəəˀ}, but without top-down
information all six forms remain possible.

\subsubsection{Previous Accounts}\label{previous-accounts}

\paragraph{Pinnow (1959)}\label{pinnow-1959}

Pinnow makes a distinction between Urmunda (proto-Munda?) and a later
state that he calls \emph{jüngeres Stadium} (younger state).

\subparagraph{Proto-Munda (Urmunda)}\label{proto-munda-urmunda}

\begin{longtable}[]{@{}llllllll@{}}
\toprule
& Front & Central & Back & & Front & Central & Back\tabularnewline
\midrule
\endhead
\textbf{Closed} & *i & *ɨ & *u & & *i: & & *u:\tabularnewline
\textbf{Mid} & *ɛ & *ə & *ɔ & & *ɛ: & *ə: & *ɔ:\tabularnewline
\textbf{Open} & & *a & & & & *a: &\tabularnewline
\bottomrule
\end{longtable}

\subparagraph{Proto-Munda (younger
stage)}\label{proto-munda-younger-stage}

\begin{longtable}[]{@{}llllllll@{}}
\toprule
& Front & Central & Back & & Front & Central & Back\tabularnewline
\midrule
\endhead
\textbf{Closed} & *i & *ɨ & *u & & *i: & & *u:\tabularnewline
\textbf{Mid} & *e & *ə & *o & & *e: & *ə: & *o:\tabularnewline
\textbf{Open} & *ɛ & *a & *ɔ & & *ɛ: & *a: & *ɔ:\tabularnewline
\bottomrule
\end{longtable}

Pinnow's younger proto-Munda has nine vowel phonemes. All vowels --
except \emph{*ɨ} -- have long variants that Pinnow considers phonemic
(CITE).

\paragraph{Stampe (1963), Zide (1965, 1966), Munda
etc}\label{stampe-1963-zide-1965-1966-munda-etc}

In the decades after the publication of Pinnow's Historische Lautlehre,
most authors favour a seven vowel system for proto-Munda -- a system
more or less identical with his pre-Munda, but lacking the vowel length
distinction.

\begin{longtable}[]{@{}llll@{}}
\toprule
& Front & Central & Back\tabularnewline
\midrule
\endhead
Closed & i & ɨ & u\tabularnewline
Mid & e & ə & o\tabularnewline
Open & & a &\tabularnewline
\bottomrule
\end{longtable}

While the empirical base in Pinnow (1959) is abundant to the point of
being confusing, the evidence for the seven vowel system is scarce or ar
least unpublished.

\subsection{Phonological Profile of the Munda
Languages}\label{phonological-profile-of-the-munda-languages}

\subsubsection{Overview}\label{overview}

\paragraph{Vowels}\label{vowels}

\begin{longtable}[]{@{}ll@{}}
\toprule
Inventory & Languages\tabularnewline
\midrule
\endhead
/i, e, a, o, u/ & Kharia, Mundari, Ho, Korku, Gorum, Remo, Juang,
Gutob\tabularnewline
/i, e, æ, a, o, u/ & Gtaʔ\tabularnewline
/i, e, ɛ, a, ɔ, o, u/ & Santali\tabularnewline
/i, e, ɛ, ɨ, ə, a, ɔ, o, u/ & Sora\tabularnewline
\bottomrule
\end{longtable}

\subsubsection{Gorum}\label{gorum}

\subsubsection{Sora}\label{sora}

Five vowel systems are very dominant.

Sora Ramamurti:

\begin{longtable}[]{@{}llllllll@{}}
\toprule
& Front & Central & Back & & Front & Central & Back\tabularnewline
\midrule
\endhead
\textbf{Closed} & i & & u & & i: & & u:\tabularnewline
\textbf{Closed} & ɪ & & ʊ & & ɪ: & & ʊ:\tabularnewline
\textbf{Mid} & e & ə & o & & e: & & o:\tabularnewline
\textbf{Open} & & a & & & & a: &\tabularnewline
\bottomrule
\end{longtable}

Stampe 1965:

\begin{longtable}[]{@{}llll@{}}
\toprule
& Front & Central & Back\tabularnewline
\midrule
\endhead
Closed & i & ɨ & u\tabularnewline
Mid & e & ə & o\tabularnewline
Open-Mid & ɛ & & ɔ\tabularnewline
Open & & a &\tabularnewline
\bottomrule
\end{longtable}

Sora (Anderson Harrison):

\begin{longtable}[]{@{}llll@{}}
\toprule
& Front & Central & Back\tabularnewline
\midrule
\endhead
Closed & i & ɨ & u\tabularnewline
Near-close & & & ʊ\tabularnewline
Mid & e & (ə) & o\tabularnewline
Open & & a &\tabularnewline
\bottomrule
\end{longtable}

\subsubsection{Remo}\label{remo}

\subsubsection{Gutob}\label{gutob}

\subsubsection{Kharia}\label{kharia}

\subsubsection{Juang}\label{juang}

\subsubsection{Gtaʔ}\label{gtaux294}

\subsubsection{Kherwarian}\label{kherwarian}

\subsubsection{Korku}\label{korku}

\subsection{Phonological
Reconstruction}\label{phonological-reconstruction}

\subsubsection{Onset}\label{onset}

\begin{longtable}[]{@{}lllll@{}}
\toprule
& bilabial & dental/alveolar & palatal & velar\tabularnewline
\midrule
\endhead
voiceless stop & *p & *t & (*c???) & *k\tabularnewline
voiced stop & *b & *d & *ɟ & *g\tabularnewline
nasal & *m & *n & *ɲ & *ŋ\tabularnewline
sibilant & & *s & &\tabularnewline
lateral & & *l & &\tabularnewline
rhotic & & *r & &\tabularnewline
approximants & & & *j &\tabularnewline
\bottomrule
\end{longtable}

\subsubsection{Coda}\label{coda}

\begin{longtable}[]{@{}llllll@{}}
\toprule
& bilabial & dental/alveolar & palatal & velar & glottal\tabularnewline
\midrule
\endhead
glottalized stop & *ˀp & *ˀt & *ˀc & *ˀk & (*Vˀ?)\tabularnewline
nasal & *m & *n & *ɲ & *ŋ &\tabularnewline
lateral & & *l & & &\tabularnewline
rhotic & & *r & & &\tabularnewline
approximants & & & *j & &\tabularnewline
\bottomrule
\end{longtable}

\subsubsection{Nuclei}\label{nuclei}

\begin{longtable}[]{@{}llll@{}}
\toprule
& Front & Central & Back\tabularnewline
\midrule
\endhead
Closed & *i & & *u\tabularnewline
Mid & *e & *ə & *o\tabularnewline
Open & *ɛ & *a &\tabularnewline
\bottomrule
\end{longtable}

\subsubsection{Proto-Munda Vocalism, a
hypothesis}\label{proto-munda-vocalism-a-hypothesis}

Maximally a nine vowel system /i, e, ɛ, ɨ, ə, a, ɔ, o, u/ (pM?) and
maybe an epenthetic schwa /ᵊ/.

Vowel length is not consistent and seems to reflect prosodic principles
in individual languages, if they do not constitute a bias of the
researcher that expects phonological vowel length disctinctions in South
Asian languages.

Syllable weight needs to be reconstructed.

Assigned so far /i, e, ə, a, o, u/ and the epenthetic /ᵊ/.

\begin{longtable}[]{@{}llll@{}}
\toprule
& Front & Central & Back\tabularnewline
\midrule
\endhead
Closed & *i & & *u\tabularnewline
Mid & *e & *ə & *o\tabularnewline
Open & *ɛ & *a &\tabularnewline
\bottomrule
\end{longtable}

\paragraph{Monophthongization}\label{monophthongization}

There is no evidence for etymological diphthongs in Munda. Diphthongs
are absent or rare, except for Gtaʔ, but the diphthogs in Gtaʔ are
demonstrably (relatively late) innovations in this one particular
language.

Diphthongs in MKCD do not seem to correspond any specific correspondence
sets.

(check MKCD \emph{*iə})

\subsubsection{Proto-Munda Consonantism}\label{proto-munda-consonantism}

\begin{longtable}[]{@{}llllll@{}}
\toprule
& bilabial & dental/alveolar & palatal & velar & glottal\tabularnewline
\midrule
\endhead
voiceless stop & *p & *t & (*c???) & *k & (*Vˀ?)\tabularnewline
voiced stop & *b & *d & *ɟ & *g &\tabularnewline
glottalized stop & *ˀp & *ˀt & *ˀc & *ˀk &\tabularnewline
nasal & *m & *n & *ɲ & *ŋ &\tabularnewline
sibilant & & *s & & &\tabularnewline
lateral & & *l & & &\tabularnewline
rhotic & & *r & & &\tabularnewline
approximants & & & *j & &\tabularnewline
\bottomrule
\end{longtable}

\begin{itemize}
\tightlist
\item
  \emph{*Kʰ} Pinnow (1959 p.232-234) \emph{*q} etc.
\item
  \emph{*dʲ}
\end{itemize}

The consonantism is very close to proto-AA. Major changes:

\begin{itemize}
\tightlist
\item
  pAA \emph{*h} \textgreater{} pM ∅
\item
  pAA \emph{*ʔ} \textgreater{} pM ∅
\item
  final pAA \emph{*s} \textgreater{} pM ∅
\item
  pAA *CVC \textgreater{} pM *CVˀC
\end{itemize}

Status of \emph{*c} unclear, seems to to have merged with \emph{*s},
probaby already at the stage of proto-Munda. possibly later.

\subsubsection{Consonant
cluster-splitting}\label{consonant-cluster-splitting}

Consonant cluster-splitting epenthetic vowel

\begin{itemize}
\tightlist
\item
  short Vᵢ: CCVᵢ(C) → CᵊCVᵢC
\item
  long Vᵢ: CCVVˀᵢ(C) → CVᵢCVVˀᵢ(C)
\end{itemize}

Four attested types of cluster-splitting epenthesis.

\subparagraph{Type 1 CᵢCᵢᵢ → CᵢVCᵢᵢ (cluster-splitting epenthetic
vowel)}\label{type-1-cux1d62cux1d62ux1d62-cux1d62vcux1d62ux1d62-cluster-splitting-epenthetic-vowel}

\begin{itemize}
\tightlist
\item
  \emph{*bl} → \emph{*bVl}: MKCD \emph{*bluuʔ} → pM \emph{*buluuˀ}
  `thigh'
\item
  \emph{*kl} → \emph{*KVl}: MKCD \emph{*klaʔ} → pM \emph{*kᵊla} `tiger'
\item
  \emph{*cl} → \emph{*ɟVl}: MKCD \emph{*{[}c{]}limʔ};
  \emph{*{[}c{]}liəmʔ}; \emph{*{[}c{]}laim{[} {]}} → pM \emph{*ɟal} `to
  lick'
\item
  \emph{*dr} → \emph{*dVr}: MKCD \emph{*d₂raŋ} → \emph{*dᵊraŋ} `horn'
\item
  \emph{*j{[}n{]}ŋ} → \emph{*sVŋ}: MKCD \emph{*j{[}n{]}ŋəl} → pM
  \emph{*sᵊŋəl} `fuel'
\end{itemize}

Variation (non-regular cluster-splitting epenthesis ):

\begin{itemize}
\tightlist
\item
  \emph{*ʔt} → \emph{*bVt} MKCD \emph{*ʔt₁uuŋ}? → pM \emph{*bVtoŋ}
  `fear'
\item
  \emph{*kt} → \emph{*lVt}: MKCD \emph{*kt₂uur}; \emph{*kt₂uər} → pM
  \emph{*lutur} `ear'
\end{itemize}

These correspondences might represent intial consonantloss with later
prefixation of \emph{*bV-} and \emph{*lV-}

\subparagraph{Type 2a CᵢCᵢᵢ → Cᵢ (second consonant
loss)}\label{type-2a-cux1d62cux1d62ux1d62-cux1d62-second-consonant-loss}

\begin{itemize}
\tightlist
\item
  \emph{*cʔ} → \emph{*ɟ}: MKCD \emph{*cʔaaŋ} ; \emph{*cʔaiŋ};
  \emph{*cʔi{[} {]}ŋ} → pM \emph{*ɟaŋ} `bone'
\item
  \emph{*bh} → \emph{*b}: MKCD \emph{*bhiiʔ} → pM \emph{*bv₍₃₁₎} `sated
  (v)'
\end{itemize}

\subparagraph{Type 2b CᵢCᵢᵢ → Cᵢᵢ (initial consonant
loss)}\label{type-2b-cux1d62cux1d62ux1d62-cux1d62ux1d62-initial-consonant-loss}

\begin{itemize}
\tightlist
\item
  \emph{*kd} → \emph{*d}: MKCD \emph{*kdiil}; \emph{*kdiəl};
  \emph{*kdəl} → pM \emph{*dal} `to cover'
\end{itemize}

\subparagraph{Type 2c CᵢCᵢᵢ → VCᵢᵢ (epenthesis and initial consonant
loss)}\label{type-2c-cux1d62cux1d62ux1d62-vcux1d62ux1d62-epenthesis-and-initial-consonant-loss}

\begin{itemize}
\tightlist
\item
  \emph{*pl} → \emph{*∅Vl}: MKCD \emph{*{[}p{]}laŋ}; \emph{*{[}p{]}laiŋ}
  → pM \emph{*Vlaŋ} `thatch'
\item
  \emph{*pr} → \emph{*∅Vr}: MKCD \emph{*pril}/\emph{*priəl} → pM
  \emph{*arel} `hail/pebble'
\item
  \emph{*sl} → \emph{*∅Vl}: MKCD \emph{*slaʔ} → pM
  \emph{*Olaaˀ}/\emph{*Olaˀk} `leaf' V₁
\end{itemize}

While consonant cluster-splitting epenthesis is a well documented
phonological process, it is rare as a regular sound change in the
historical development of languages (Blevins 2017). It is however very
frequent with loan words and Blevins states it is a indication for a
process where speakers learn a non-native language and misperceive a
cluster CCV as CVCV because their native languages demands a CVC
stransition.

If Blevins is correct with her assessment, it would suggest that the
speakers of proto-Munda (or a significant number of speakers) were
unaccustomed to initial consonant clusters and misheard or misanalysed
these typical Austroasiatic clusters, giving rise to the consonant
cluster-splitting epenthesis characteristic for languages in the Munda
family.

\paragraph{Prosodic Restructuring}\label{prosodic-restructuring}

\begin{longtable}[]{@{}llllll@{}}
\toprule
Language & pMunda state & step 1 & step 2 & step 3 & current
state\tabularnewline
\midrule
\endhead
Santali: & *kᵊˈla → & *ˈkəla → & *ˈkula → & & kul\tabularnewline
Gutob: & *kᵊˈla → & *ˈkəla → & *ˈkila → & *ˈkil → & gikil\tabularnewline
Gorum: & *kᵊˈla → & *kəˈlaʔ → & & & kuˈlaʔ\tabularnewline
\bottomrule
\end{longtable}

There was a later resurrection of consonant clusters (or sesquisyllabic
structures) in Gtaʔ (and possibly Remo).

\paragraph{Vowel harmonic processes}\label{vowel-harmonic-processes}

There seem to be some vowel harmonic processes ar work, that

\subsubsection{Issues}\label{issues}

There are several remaining issues. For some we can state why these
reflexes might be problematic, for some the reasons remains completely
unexplaiined.

\paragraph{\texorpdfstring{reflexes of epenthetic vowels:
\emph{*ᵊ}}{reflexes of epenthetic vowels: *ᵊ}}\label{reflexes-of-epenthetic-vowels-ux1d4a}

irregular correspondences

\begin{longtable}[]{@{}lllllllllllll@{}}
\toprule
Gorum & Sora & Remo & Gutob & Kharia & Juang & Gtaʔ & Santali & Mundari
& Ho & Korwa & Korku &\tabularnewline
\midrule
\endhead
u & i & u & i & i & i & u & u & u & u & u: & u &
\emph{*ə(tiger)}\tabularnewline
\bottomrule
\end{longtable}

\subparagraph{\texorpdfstring{\emph{*kᵊla}
`tiger'}{*kᵊla tiger}}\label{kux1d4ala-tiger}

kulaʔ, FR, kina:, RSED.p140, ŋku, AG08.p733, gikil, AG08.p651, kiɽoʔ,
PKED.p102, kiɭog, PJDW.p224, nku, PGEG.p36, kul, CDES.p201, kula:,
BMED.p98, kula, HOGV.p183, ku:l, BAHL.p33, kula, NKEV.p319, *kᵊla,
tiger, \#0004, V281, 197,

\begin{itemize}
\tightlist
\item
  Pinnow 1959: V281 / MKCD: 197 \emph{*klaʔ}
\end{itemize}

We would reconstruct \emph{*ə} (or because it is an epenthetic vowel
\emph{*ᵊ}) for V₁ and accordingly\_*kᵊla\_ based on the reflexes and
Shorto's \emph{*klaʔ} (MKCD 197). However, the set for V₁ is so far
uniquely attested in the reflexes of this one etymon.

The set is very close to\_*u₁\_ as attested in \emph{*bul} `drunk (v)'
or in both vowels of \emph{*lutur} `ear' among others:

\begin{longtable}[]{@{}llllllllllll@{}}
\toprule
Gorum & Sora & Remo & Gutob & Kharia & Juang & Gtaʔ & Santali & Mundari
& Ho & Korwa & Korku\tabularnewline
\midrule
\endhead
u & u & u & i & u & u & u & u & u & u & u & u\tabularnewline
\bottomrule
\end{longtable}

but also to the reflexes of epenthetic \emph{*u} in \emph{*buluuˀ}
(\emph{*u₂} ):

\begin{longtable}[]{@{}llllllllllll@{}}
\toprule
Gorum & Sora & Remo & Gutob & Kharia & Juang & Gtaʔ & Santali & Mundari
& Ho & Korwa & Korku\tabularnewline
\midrule
\endhead
u & u & u/i & i & u & u & u & u & u & u & u & u\tabularnewline
\bottomrule
\end{longtable}

\subsubsection{\texorpdfstring{Proto-Munda
\emph{*a}}{Proto-Munda *a}}\label{proto-munda-a}

Proto-Munda *a is reasonably well understood.

\begin{itemize}
\tightlist
\item
  \emph{*a₁} general
\item
  \emph{*a₂} velar coda (\emph{*aŋ} and \emph{*aˀk})
\item
  \emph{*a₃} palatal coda (\emph{*aɲ} and \emph{*aˀc})
\item
  \emph{*a₄} \emph{*aˀp} coda (so far not \emph{*am})
\item
  \emph{*a₅} unclear
\end{itemize}


\resizebox{\textwidth}{!}{%
\begin{longtable}[]{@{}lllllllllllll@{}}
\toprule
Gorum & Sora & Remo & Gutob & Kharia & Juang & Gtaʔ & Santali & Mundari
& Ho & Korwa & Korku &\tabularnewline
\midrule
\endhead
a & a: & a & a(:) & a & a & a & a & a(:) & a & a(:) & a(:) &
\emph{*a₁}\tabularnewline
a & a & a & a & a & a & ia & a & a(:) & a(:) & a(:) & a &
\emph{*a₂}\tabularnewline
a & a & a & a & a & a & æ & e/e̠ & e & e & e & ? &
\emph{*a₃}\tabularnewline
? & ? & o & o & a & ɔ & a & a & a: & a & a: & a: &
\emph{*a₄}\tabularnewline
a & a & o & o & o & o & o & a & a: & a & a & a &
\emph{*a₅}\tabularnewline
\bottomrule
\end{longtable}}

The correspondence set \emph{*ə₁} is problematic. The etyma, where they
do have a proto-Austroasiatic etymology, suggests that it continues
Austroasiatic \emph{*a}. MKCD suggests a reconstruction as \emph{*a},
but this is not easily reconciled with the the sets \emph{*a₁},
\emph{*a₂}, \emph{*a₃}, and \emph{*a₄}. Especially the motivation for
change of \emph{*a} to North Munda /e/ is unclear, if we assume
\emph{*a}.

\paragraph{\texorpdfstring{\emph{*a₁}}{*a₁}}\label{a}

\begin{longtable}[]{@{}llllllllllll@{}}
\toprule
Gorum & Sora & Remo & Gutob & Kharia & Juang & Gtaʔ & Santali & Mundari
& Ho & Korwa & Korku\tabularnewline
\midrule
\endhead
a & a: & a & a(:) & a & a & a & a & a(:) & a & a(:) &
a(:)\tabularnewline
\bottomrule
\end{longtable}

Proto-Munda \emph{*a} has very consistent reflexes across the whole
Munda languages. The reported vowel length is irregular, but arguably
not phonemic in any Munda language and either a non-phonemic lengthening
in certain contexts or lexemes or an artefact of the phonological
description.

\subparagraph{\texorpdfstring{\emph{*bal} `to burn'
(\#0023-2)}{*bal to burn (\#0023-2)}}\label{bal-to-burn-0023-2}

Go. \emph{bal}; So. \emph{ba:l} (RSED.p49); Gu. \emph{bal} (GZ65.43);
Gt. \emph{ba} (PGEG.p9); Sa. \emph{bal} (BSDV1.p1840); Mu. \emph{bal}
(BMED.p18); Ho \emph{bal} (HOGV.p151); Kw. \emph{ba:l} (BAHL.p1050; Ko.
\emph{ba:l} (NKEV.p292)

\begin{itemize}
\tightlist
\item
  Pinnow 1959 --- / MKCD ---
\end{itemize}

A connection to any of the etyma attested in MKCD remains unclear. The
best candidate is MKCD 1671 \emph{*waar}, \emph{*wər}. However, neither
pAA *w nor pAA *r match the reconstructed pM *b and *l.

\subparagraph{\texorpdfstring{\emph{*arel} `hail/pebble' V₁
(\#0032-1)}{*arel hail/pebble V₁ (\#0032-1)}}\label{arel-hailpebble-v-0032-1}

aril, FR, are:l, RSED.p39, are, BDBH.43, arel, HLKS.V225, arel, PKED.p7,
aɭɛn, PJDW.p158, hare, PGEG.p24, arel, CDES.p88, a:ɽil, BMED.p10, aril,
HOGV.p161, a:ril, BAHL.p10, ---, ---, *arel, hail/pebble, \#0032, V225,
1791,

\begin{itemize}
\tightlist
\item
  Pinnow 1959: V225 / MKCD: 1791 \emph{*pril}; \emph{*priəl}
\end{itemize}

The reflexes of \#0032-1 strongly indicate proto-Munda \emph{*a}. From
our current understanding, this is incompatible with the MKCD forms
\emph{*pril} and *priəl\_ which suggest \emph{*ᵊ}.

\subparagraph{\texorpdfstring{\emph{*usal} `skin' V₂
(\#0036-3)}{*usal skin V₂ (\#0036-3)}}\label{usal-skin-v-0036-3}

usal, FR, usal, RSED.p308, usa, BDBH.173, isa:l, HLKS.V149, usal,
PKED.p300, chalo, PJDW.p180, ugsa, PGEG.p6, chal, CDES.p176, , , , , , ,
sa:li, NKEV.p337, , skin, \#0036, V149, ,

\begin{itemize}
\tightlist
\item
  Pinnow 1959: V149 / MKCD: ---
\end{itemize}

The reflexes in Juang, Santali and Korku may belong to a separate
etymon.

\subparagraph{\texorpdfstring{\emph{*ɟal} `to lick'
(\#0043-2)}{*ɟal to lick (\#0043-2)}}\label{ux25fal-to-lick-0043-2}

zaleˀb, FR, ɟa:l, RSED.p119, salep', BDBH.2523, sal, GZ63.228, jal,
PKED.p82, janɔ, JLIC.v372, cca, PGEG.p14, jal, CDES.p112, jal, EM.p1965,
jal, HOGV.p164, (jaɽa:ʔ), BAHL.p60, jal, NKEV.p312, *ɟal, lick (v),
\#0043, V13, 1409,

\begin{itemize}
\tightlist
\item
  Pinnow 1959: V13 / MKCD: ? 1409 \emph{*{[}c{]}limʔ};
  \emph{*{[}c{]}liəmʔ}; \emph{*{[}c{]}laim{[} {]}}
\end{itemize}

Proto-Austro-Asiatic \emph{*{[}c{]}limʔ}; \emph{*{[}c{]}liəmʔ};
\emph{*{[}c{]}laim{[} {]}} seems to correspond closely to Gorum
\emph{zaleˀb} and Remo \emph{salep'} (palatal - vowel - /l/ vowel -
bilabial). The loss of final coda in all but Gorum and Remo is
unexpected. Where it is preserved, it occurs as reflexes of pM
\emph{*ˀp} and not \emph{*m}. Reflexes of the sequence \emph{*mʔ} are
not well understood so far, though. If taken one by one \emph{*m+ʔ}
should become \emph{*m+∅}. So from Shorto's form, we would expect pMunda
\emph{*ɟəlam} or \emph{*ɟəlem} and not \emph{*ɟal} or if Gorum and Remo
are taken as base \emph{*ɟalEˀp}. However, maybe the sequence \emph{*mʔ}
behaves different than the individual proto-phonemes.

\subparagraph{\texorpdfstring{\emph{*dal} `to cover'
(\#0047-2)}{*dal to cover (\#0047-2)}}\label{dal-to-cover-0047-2}

ɖal, FR, dal, RSED.p73, ɖalu, BDBH.1210, ɖal, GZ65.80, ɖal, PKED.p42,
ɖan, MJTL.p96, ɖa, PGEG.p16, dapal/dalo̠p', CDES.p40, dapal/dālob,
BMED.p35/36, dapal/dalop, HOGV.p153, ---, ---, da:l, NKEV.p299, *dal,
cover (v), \#0047, V3, 1745,

\begin{itemize}
\tightlist
\item
  Pinnow 1959: V3 / MKCD: 1745 \emph{*kdiil}; \emph{*kdiəl};
  \emph{*kdəl}
\end{itemize}

Santali \emph{dapal}/\emph{dalo̠p'} (CDES.p40), Mundari
\emph{dapal}/\emph{dālob} (BMED.p35/36), and Ho
\emph{dapal}/\emph{dalop} (HOGV.p153) are not straightforward reflexes
of proto-Munda \emph{*dal}. Santali \emph{dapal} and the parallel forms
in Mundari and Ho suggest \emph{*dal} with an infix \emph{*-p-} or
possibly the etymon \emph{*daˀp} (correspondence set \#0076) with a
suffix \emph{*-Vl}. Santali \emph{dalo̠p'} in turn suggests either
\emph{*dal} with a suffix \emph{*-Vˀp} or \emph{*daˀp} (correspondence
set \#0076) with an infix \emph{*-l-} .

The consistent differences in the vowels -- a-a in the case of Santali
\emph{dapal} and related forms and a-o̠/o in the case of Santali
\emph{dalo̠p'} related forms -- is interesting.

\subparagraph{\texorpdfstring{\emph{*mv₍₄₎raŋ} `big' V₂
(\#0064-4)}{*mv₍₄₎raŋ big V₂ (\#0064-4)}}\label{mvraux14b-big-v-0064-4}

---, ---, maraŋ/məraŋ, RSED.p173/167, munaʔ, BDBH.2121, (moɖo),
AG08.p663, ---, ---, ---, ---, mnaʔ, PGEG.35, maraŋ, CDES.p17, maraŋ,
BMED.p220, maraŋ, DHED.p225, ---, ---, ---, ---, *mxrxŋ, big, \#0064,
K107, ,

\begin{itemize}
\tightlist
\item
  Pinnow 1959: K107 / MKCD: ---
\end{itemize}

Gtaʔ \emph{mnaʔ} and Remo \emph{munaʔ} are irregular reflexes of ,
especially the Gtaʔ form \emph{mnaʔ} should be different, given our
current uderstanding of the phonological developments, since a velar
coda \emph{*aŋ} results in Gtaʔ /ia/. Remo and Gtaʔ /n/ are also
inconsistent as reflexes or either \emph{*r} or \emph{*ŋ}. Gtaʔ
\emph{mnaʔ} and Remo \emph{munaʔ} are consistently parallel to one
another.

\subparagraph{\texorpdfstring{\emph{*bar} `two'
(\#0078-2)}{*bar two (\#0078-2)}}\label{bar-two-0078-2}

bagu, FR, bar, RSED.p48, mbaʔr, BDBH.2214, umbar, AG08.p646, ubar,
PKED.p205, umba, PJDW.p291, mbar, PGEG.p34, bar, CSED.p42, baria,
BMED.p20, bar, DHED.p27, ---, ---, ba:r, NKEV.293, *bar, two, \#0078,
V49, 1562,

\begin{itemize}
\tightlist
\item
  Pinnow 1959: V49 / MKCD: 1562 \emph{*biʔaar} \textgreater{}
  \emph{*ɓaar}, Pre-Khmer \emph{*{[}ɓ{]}ir}, Pre-Palaungic \&c.
  \emph{*ʔaar}
\end{itemize}

\subparagraph{\texorpdfstring{\emph{*ɟa(ˀt)} `additive.particle'
(\#0079-2)}{*ɟa(ˀt) additive.particle (\#0079-2)}}\label{ux25faux2c0t-additive.particle-0079-2}

zaˀd, FR, ɟa:, RSED.p117, sa, BDBH.2547, sa, AG08.p649, ja, HLKS.V1,
ɟan, PJDW.p211, , , ja, BSDV3.p216, ja:, BMED.p77, ja:, DHED.p155, ja'',
DSKW.@09330, ja, DSKO.12141, *ɟa(ˀt), additive.particle, \#0079, V1, ,

\begin{itemize}
\tightlist
\item
  Pinnow 1959: V1 / MKCD: ---
\end{itemize}

\subparagraph{\texorpdfstring{\emph{*gam} `say (v)'
(\#0080-2)}{*gam say (v) (\#0080-2)}}\label{gam-say-v-0080-2}

---, ---, gam, RSED.p96, ---, ---, gam, Z1965.121, gam, PKED.p57, gam,
PJDW.p194, ---, ---, gam, CSED.p176, gamu, HLKS.V12, gamu, HLKS.V12,
---, ---, ---, ---, *gam, say (v), \#0080, V12, ,

\begin{itemize}
\tightlist
\item
  Pinnow 1959: V12 / MKCD: ---
\end{itemize}

\subparagraph{\texorpdfstring{\emph{*maraˀk} `peacock' V₁
(\#0081-2)}{*maraˀk peacock V₁ (\#0081-2)}}\label{maraux2c0k-peacock-v-0081-2}

(marraʔ), FR, ma:ra:, RSED.p173, ---, ---, ---, ---, maraʔ, PKED.p131,
marag, PJDW.p242, ---, ---, marak', CSED.p407, ma:ra:, BMED.p114, mara:,
DHED.p225, mara:q, BAHL.p117, mara, NKEV.p324, *maraˀk, peacock, \#0081,
V27, 416,

\begin{itemize}
\tightlist
\item
  Pinnow 1959: V27 / MKCD: 416 \emph{*mraik{[} {]}}
\end{itemize}

Gorum \emph{marraʔ} `husband' probably belongs to another etymon.

\subparagraph{\texorpdfstring{\emph{*ɲam} `get (v)'
(\#0088-2)}{*ɲam get (v) (\#0088-2)}}\label{ux272am-get-v-0088-2}

---, ---, ɲam, RSED.p186, ---, ---, ---, ---, ɲam, PKED.p140, ---, ---,
---, ---, ɲam, CSED.p434, na:m, BMED.p126, nam, DHED.p241, ɲa:m,
BAHL.p66, na, NKEV.p327, *ɲam, get (v), \#0088, 5(?), 1243(?),

\begin{itemize}
\tightlist
\item
  Pinnow 1959: 5(?) / MKCD: ---
\end{itemize}

possibly \emph{*a₄}

\begin{longtable}[]{@{}lllllllllllll@{}}
\toprule
Gorum & Sora & Remo & Gutob & Kharia & Juang & Gtaʔ & Santali & Mundari
& Ho & Korwa & Korku & Set\tabularnewline
\midrule
\endhead
-- & a & -- & -- & a & -- & -- & a & a: & a & a: & a &
0088-2\tabularnewline
-- & -- & o & o & -- & ɔ & a & a & a: & a & a: & a: &
0048-2\tabularnewline
-- & -- & o & o & a & -- & a & a & a: & a & a: & a &
0056-2\tabularnewline
\bottomrule
\end{longtable}

\paragraph{\texorpdfstring{\emph{*a₂}}{*a₂}}\label{a-1}

\begin{longtable}[]{@{}llllllllllll@{}}
\toprule
Gorum & Sora & Remo & Gutob & Kharia & Juang & Gtaʔ & Santali & Mundari
& Ho & Korwa & Korku\tabularnewline
\midrule
\endhead
a & a & a & a & a & a & ia & a & a(:) & a(:) & a(:) & a\tabularnewline
\bottomrule
\end{longtable}

The diphthongization in Gtaʔ occured before a velar coda (i.e. \emph{*ŋ}
and \emph{*ˀk}), so that in Gtaʔ the proto-Munda coda \emph{*aˀk} became
Gtaʔ \emph{iaʔ} while pM \emph{*aŋ} became Gtaʔ \emph{ia}.

Defining context: velar coda -- \emph{*aˀk and }*aŋ\_

\subparagraph{\texorpdfstring{\emph{*daˀk} `water'
(\#0001-2)}{*daˀk water (\#0001-2)}}\label{daux2c0k-water-0001-2}

ɖaʔ, FR, daʔ, RSED.p70, dak', BDBH.1179, ɖaʔ, ZG63.85, ɖaʔ, PKED.p41,
ɖag, PJDW.p185, ndiaʔ, PGEG.p36, dak', CDES.p217, da:, BMED.p31, daʔ,
DHED.p73, da:ʔ, BAHL.p87, ɖa, NKEV.p300, da(a)ˀk, water, \#0001, V2,
274, 75

\begin{itemize}
\tightlist
\item
  Pinnow 1959: V2 / MKCD: 274 \emph{*diʔaak} \textgreater{} \emph{*ɗaak}
\end{itemize}

\subparagraph{\texorpdfstring{\emph{*ɟaŋ} `bone'
(\#0002-2)}{*ɟaŋ bone (\#0002-2)}}\label{ux25faux14b-bone-0002-2}

za̰ŋ, FR, əɟaŋ, RSED.p6, siʔsaŋ, BDBH.2614, sisaŋ, AG08.p651, jaŋ,
PKED.p83, ɟaŋ, PJDW.p210, ncia, PGEG.p36, jaŋ, CDES.p19, ja:ŋ, BMED.p80,
jaŋ, HOGV.p150, ja:ŋ, BAHL.p60, ---, , ɟa(a)ŋ, bone, \#0002, V7, 488, 31

\begin{itemize}
\tightlist
\item
  Pinnow 1959: V7 / MKCD: 488 \emph{*cʔaaŋ} ; \emph{*cʔaiŋ};
  \emph{*cʔi{[} {]}ŋ}
\end{itemize}

\subparagraph{\texorpdfstring{\emph{*laŋ} `tongue'
(\#0003-2)}{*laŋ tongue (\#0003-2)}}\label{laux14b-tongue-0003-2}

laŋ, FR, əlaŋ, RSED.p158, leaŋ, BDBH.2423, laʔŋ, AG08.p638, laŋ,
PKED.p173, elaŋ, PJDW.p191, nlia, PGEG.p36, alaŋ, CDES.p203, a:la:ŋ,
BMED.p5, (leʔ), DHED.p208, a:la:ŋ, BAHL.p11, laŋ, NKEV.p322, la(a)ŋ,
tongue, \#0003, V14, , 44

\begin{itemize}
\tightlist
\item
  Pinnow 1959: V14 / MKCD: ---
\end{itemize}

\subparagraph{\texorpdfstring{\emph{*Olaaˀ}/\emph{*Olaˀk} `leaf' V₂
(\#0035-3)}{*Olaaˀ/*Olaˀk leaf V₂ (\#0035-3)}}\label{olaaux2c0olaux2c0k-leaf-v-0035-3}

olaʔ, FR, o:la:, RSED.p192, ulak', BDBH.169, olag, AG08.p633, ulaʔ,
PKED.p298, olag, PJDW.p254, uliaʔ, PGEG.p124, palha, CDES.p111,
pa:lha:o, BMED.p142, pala, DHED.p259, (sakam), BAHL.pdfp129, pa:la,
NKEV.p331, , leaf, \#0035, V50, 230,

\begin{itemize}
\tightlist
\item
  Pinnow 1959: V50 / MKCD: 230 \emph{*slaʔ}
\end{itemize}

\subparagraph{\texorpdfstring{\emph{*(saŋ)saŋ} `tumeric'
(\#0072-2)}{*(saŋ)saŋ tumeric (\#0072-2)}}\label{saux14bsaux14b-tumeric-0072-2}

saŋsaŋ, FR, sansaŋ, RSED.249, saŋsaŋ, BDBH.400, saŋsaŋ, GZ63.226,
saŋsaŋ, PKED.p176, saŋsaŋ, PJDW.p268, ssia, PGEG.p42, sasaŋ, CDES.p230,
sasaŋ, BMED.p157, sasaŋ, DHED.p307, ---, ---, sasan, Korku.txt.24491,
*saŋsaŋ, turmeric/yellow, \#0072, V271, ,

\begin{itemize}
\tightlist
\item
  Pinnow 1959: V271 / MKCD: ---
\end{itemize}

\subparagraph{\texorpdfstring{\emph{*maraˀk} `peacock' V₂
(\#0081-4)}{*maraˀk peacock V₂ (\#0081-4)}}\label{maraux2c0k-peacock-v-0081-4}

(marraʔ), FR, ma:ra:, RSED.p173, ---, ---, ---, ---, maraʔ, PKED.p131,
marag, PJDW.p242, ---, ---, marak', CSED.p407, ma:ra:, BMED.p114, mara:,
DHED.p225, mara:q, BAHL.p117, mara, NKEV.p324, *maraˀk, peacock, \#0081,
V27, 416,

\begin{itemize}
\tightlist
\item
  Pinnow 1959: V27 / MKCD: 416 \emph{*mraik{[} {]}}
\end{itemize}

Gorum \emph{marraʔ} `husband' probably belongs to another etymon
connected with MKCD 183 \emph{*mraʔ} `person'.

\subparagraph{\texorpdfstring{\emph{*laˀk} `to scrape'
(\#0093-2)}{*laˀk to scrape (\#0093-2)}}\label{laux2c0k-to-scrape-0093-2}

laʔ, FR, ---, ---, ---, ---, lag, Z1965.205, laʔ, PKED.p118, lag,
PJDW.p235, liaʔ, PGEG.p31, lak', CSED.p359, ---, ---, laʔ, DHED.p203,
---, ---, laʔ, DSKO.17551, *laˀk, scrape (v), \#0093, ---, 418,

\begin{itemize}
\tightlist
\item
  Pinnow 1959: --- / MKCD: 418 \emph{*l{[}a{]}k}
\end{itemize}

\paragraph{\texorpdfstring{\emph{*a₃}}{*a₃}}\label{a-2}

\begin{longtable}[]{@{}llllllllllll@{}}
\toprule
Gorum & Sora & Remo & Gutob & Kharia & Juang & Gtaʔ & Santali & Mundari
& Ho & Korwa & Korku\tabularnewline
\midrule
\endhead
a & a & a & a & a & a & æ & e/e̠ & e & e & e & ?\tabularnewline
\bottomrule
\end{longtable}

Correnspondence set for pM *a in a rhyme with a palatal coda (i.e.~ɲ or
ˀc). South Munda has consistently \emph{a} except Gtaʔ, which has
\emph{æ}. North Munda has consistently \emph{e} (although there is so
far no data from Korku for these etyma).

Defining context: palatal coda -- *aɲ\_ and \emph{*aˀc}

\subparagraph{\texorpdfstring{\emph{*taɲ} `to weave'
(\#0005-2)}{*taɲ to weave (\#0005-2)}}\label{taux272-to-weave-0005-2}

taɲ, FR, taɲ, RSED.p281, taNy, BDBH.1358, taɲ, GZ65.369, taɲ, PKED.p196,
---, ---, tæ, PGEG.p45, teɲ, CDES.p219, teŋ, BMED.p183, teɲ, HOGV.p187,
---, ---, ---, ---, taɲ, weave (v), \#0005, V301, 898,

\begin{itemize}
\tightlist
\item
  Pinnow 1959: V301 / MKCD: 898 \emph{*t₁aaɲ}
\end{itemize}

\subparagraph{\texorpdfstring{\emph{*daˀc} `to climb'
(\#0006-2)}{*daˀc to climb (\#0006-2)}}\label{daux2c0c-to-climb-0006-2}

ɖaˀɟ, FR, daɟ, RSED.p72, ɖaĭ, BDBH.1168, ɖaj, GZ65.79, ---, ---, ɖaɲ,
PJDW.p186, ɖæʔ, PGEG.p16, de̠c', CDES.p32, dej', BMED.p40, deʔ, DHED.p81,
deʔ, BAHL.p89, (cuɖe), NKEV.p298, daˀɟ, climb (v), \#0006, V333, ,

\begin{itemize}
\tightlist
\item
  Pinnow 1959: V333 / MKCD: ---
\end{itemize}

\subparagraph{\texorpdfstring{\emph{*gaˀc} `to fry'
(\#0096-2)}{*gaˀc to fry (\#0096-2)}}\label{gaux2c0c-to-fry-0096-2}

gaˀɟ ,FR ,gaɟ ,RSED.p95 ,gaĭ ,BDBH.766 ,gaj ,Z1965.120 ,gaˀj ,PKED.p165
,gaj ,DSJU\#10461 ,gæʔ ,PGEG.p19 ,ge̠c' ,CSED.p184 ,geʔ ,EMV5.p1411 ,---
,--- ,--- ,--- ,--- ,--- ,*gaˀc ,fry/scrape (v) ,\#0096 ,V15 ,(338a) ,

\begin{itemize}
\tightlist
\item
  Pinnow 1959: V15 / MKCD: (338a)
\end{itemize}

\paragraph{\texorpdfstring{\emph{*a₄}}{*a₄}}\label{a-3}

\begin{longtable}[]{@{}lllllllllllll@{}}
\toprule
Gorum & Sora & Remo & Gutob & Kharia & Juang & Gtaʔ & Santali & Mundari
& Ho & Korwa & Korku & Set\tabularnewline
\midrule
\endhead
-- & -- & o & o & -- & ɔ & a & a & a: & a & a: & a: &
0048-2\tabularnewline
-- & -- & o & o & a & -- & a & a & a: & a & a: & a &
0056-2\tabularnewline
\bottomrule
\end{longtable}

Defining context: bilabial stop coda --\emph{*Vˀp}

Probable reflexes of \emph{*am} are inconsistent. \emph{*gam} `say (v)'
(\#0080-2) is clearly part of \emph{*a₁} while \emph{*ɲam} `get (v)'
(\#0088-2) is too poorly atttested to decide whether it belongs to
\emph{*a₁} or \emph{*a₄}. (In my notes I have \emph{*tam} `to wash,
rinse, hit' that would fit best into \_*a₄, but Kharia /o/ would be
inconsitent with \#0056-2.)

\subparagraph{\texorpdfstring{\emph{*saˀp} `grab (v)'
(\#0048-2)}{*saˀp grab (v) (\#0048-2)}}\label{saux2c0p-grab-v-0048-2}

---, ---, (sakab), RSED.p246, sop', BDBH.2748, sob, GGEG.p113, (suˀb),
PKED.p188, sɔb, PJDW.p277, saʔ, PGEG.p42, sap', CDES.p28, sa:b,
BMED.p163, sab, DHED.p296, sa:b, BAHL.pdfp131, sa:p, NKEV.p337, *sxˀp,
grab (v), \#0048, , ,

\begin{itemize}
\tightlist
\item
  MKCD 1236 \emph{*{[}c{]}kiip}; \emph{*{[}c{]}kiəp};
  \emph{*t{[}₁{]}kiəp}; \emph{*ckap}; \emph{*t₁kap}; \emph{ckuəp}
\item
  MKCD 1243 *cap; *caap; *ciəp; *cip; *cup
\end{itemize}

The connection to MKCD 1236 is not strong. Reflexes of \emph{*t₁} should
remain a stop, while the reflexes of the cluster \emph{*{[}c{]}k} are
not well understood. It could be a case of type 2a cluster splitting by
second consonant loss (CᵢCᵢᵢ → Cᵢ). Thus \emph{*ckap} → \emph{*cap} →
\emph{*saˀp} or \emph{*ckap} → \emph{*skap} → \emph{*saˀp}.

\subparagraph{\texorpdfstring{\emph{*Kʰaˀp} `bite (v)'
(\#0056-2)}{*Kʰaˀp bite (v) (\#0056-2)}}\label{kux2b0aux2c0p-bite-v-0056-2}

(kuˀb), FR, (küb/kib/kaib), RSED.p144, op, BDBH.337, op, ZG63.7, hapkay,
PKED.p73, ---, ---, haʔ, PGEG.p24, hap', CDES.p17, ha:b, BMED.p64, hab,
DHED.p124, ha:p, BAHL.p146, khap, NKEV.p320, *Kʰaˀp, bite (v), \#0056,
V294, 1231,

\begin{itemize}
\tightlist
\item
  Pinnow 1959: V294 / MKCD: 1231 \emph{*kap}/\emph{*kaap}
\end{itemize}

\paragraph{\texorpdfstring{\emph{*a₅}}{*a₅}}\label{a-4}

\begin{longtable}[]{@{}lllllllllllll@{}}
\toprule
Gorum & Sora & Remo & Gutob & Kharia & Juang & Gtaʔ & Santali & Mundari
& Ho & Korwa & Korku & Set\tabularnewline
\midrule
\endhead
a & a & ∅ & ∅ & o & o & ∅ & ∅ & a: & a & ∅ & a: & 0004-4\tabularnewline
a & a & ɔ & o & o & o & o & ∅ & ∅ & ∅ & ∅ & ∅ & 0014-3\tabularnewline
a & a & o & o/u & o/a & ɔ & ∅ & a & a: & a & a & a &
0038-4\tabularnewline
\bottomrule
\end{longtable}

Delineation of \emph{*a₅} and \emph{*a₄} is difficult. In both sets Sora
and North Munda has consistently \emph{a}, while Remo, Gutob, Kharia,
and Juang have o/ɔ. The only difference is that in \emph{*a₄} Gorum has
\emph{o} and in \emph{*a₅} Gorum has \emph{a} and Gtaʔ has \emph{a}
\emph{*a₄} and \emph{o} in \emph{*a₅}.

The set \emph{*a₅} cannot be regarded as definite, because set \#0014-3,
which establishes the reflex in Gtaʔ and contains a full set of reflexes
for all but North Munda, lacks reflexes in North Munda completely. So
far (0100), no other set, besides \#0014-3, has an /a/ reflex in
Sora-Gorum and /o/ in Gtaʔ.

\subparagraph{\texorpdfstring{\emph{*kᵊla} `tiger' V₂
(\#0004-4)}{*kᵊla tiger V₂ (\#0004-4)}}\label{kux1d4ala-tiger-v-0004-4}

kulaʔ, FR, kina:, RSED.p140, ŋku, MVol.p733, gikil, AG08.p651, kiɽoʔ,
PKED.p102, kiɭog, PJDW.p224, nku, PGEG.p36, kul, CDES.p201, kula:,
BMED.p98, kula, HOGV.p183, ku:l, BAHL.p33, kula, NKEV.p319, *kᵊla,
tiger, \#0004, V281, 197,

\begin{itemize}
\tightlist
\item
  Pinnow 1959: V281 / MKCD: 197 \emph{*klaʔ}
\end{itemize}

\subparagraph{\texorpdfstring{\emph{*v₍₉₎laŋ} `thatch' V₂
(\#0014-3)}{*v₍₉₎laŋ thatch V₂ (\#0014-3)}}\label{vlaux14b-thatch-v-0014-3}

alaŋ, FR, əlaŋ, RSED.p158, lɔŋ, BDBH.2437, uloŋ, AG08.p644, oloŋ,
PKED.p214, oloŋ, PJDW.p254, nlo, PGEG.p36, ---, ---, ---, ---, ---, ---,
---, ---, ---, ---, *v₍₉₎laŋ, thatch, \#0014, V270, 749,

\begin{itemize}
\tightlist
\item
  Pinnow 1959: V270 / MKCD: 749 \emph{*{[}p{]}laŋ}; \emph{*{[}p{]}laiŋ}
\end{itemize}

\subparagraph{\texorpdfstring{\emph{*sVmaŋ} `forehead/front' V₂
(\#0038-4)}{*sVmaŋ forehead/front V₂ (\#0038-4)}}\label{svmaux14b-foreheadfront-v-0038-4}

amaŋ, FR, ammaŋ, RSED.p31, gutumoŋ, BDBH.885, sumoŋ/amuŋ, GZ65.21,
somoŋ/somo/sumaŋ, PKED.p185, ɛmɔŋ, PJDW.p191, ssæ, PGEG.p44, samaŋ,
CDES.p79, sa:ma:ŋ, BMED.p167, sanamaŋ, HOGV.p159, samaŋ, BAHL.pdfp130,
samma, NKEV.p336, , forehead/front, \#0038, V269, ,

\begin{itemize}
\tightlist
\item
  Pinnow 1959: V269 / MKCD: ---
\end{itemize}

Gtaʔ /æ/ is treated as a reflex of V₁ here.

\subsubsection{\texorpdfstring{Proto-Munda
\emph{*i}}{Proto-Munda *i}}\label{proto-munda-i}

\begin{itemize}
\tightlist
\item
  \emph{*i₁}
\item
  \emph{*i₂} (palatal onset or coda?)
\item
  \emph{*i₃} unclear condition for differences to \emph{*i₁}
\end{itemize}

Main motivation to keep \emph{*i₃} as a reflex of proto-Munda \emph{*i}
that most branches clear point to \emph{*i}.

\begin{longtable}[]{@{}lllllllllllll@{}}
\toprule
Gorum & Sora & Remo & Gutob & Kharia & Juang & Gtaʔ & Santali & Mundari
& Ho & Korwa & Korku &\tabularnewline
\midrule
\endhead
i & i(:) & i & i & i & i & i & i & i & i(:) & i(:) & i &
\emph{*i₁}\tabularnewline
i & i & i & i & i & i/ɛ & æ & i & i & i & i: & i &
\emph{*i₂}\tabularnewline
--- & i & i & i & e & ɛ & i & e & i & i & i: & i &
\emph{*i₃}\tabularnewline
\bottomrule
\end{longtable}

\paragraph{\texorpdfstring{\emph{*i₁}}{*i₁}}\label{i}

\begin{longtable}[]{@{}llllllllllll@{}}
\toprule
Gorum & Sora & Remo & Gutob & Kharia & Juang & Gtaʔ & Santali & Mundari
& Ho & Korwa & Korku\tabularnewline
\midrule
\endhead
i & i(:) & i & i & i & i & i & i & i & i(:) & i(:) & i\tabularnewline
\bottomrule
\end{longtable}

\subparagraph{\texorpdfstring{\emph{*tiiˀ} `hand'
(\#0008-2)}{*tiiˀ hand (\#0008-2)}}\label{tiiux2c0-hand-0008-2}

siʔ, FR, si:ʔ, RSED.p254, titi, BDBH.1370, titi, GZ65.p29, tiʔ,
PKED.p199, iti, PJDW.p208, nti, PGEG.p37, ti, CDES.p89, ti, BMED.p186,
ti:, DHED.p350, tiʔi:, BAHL.p63, ʈi, NKEV.p343, tiiˀ, hand, \#0008, V75,
66, 48

\begin{itemize}
\tightlist
\item
  Pinnow 1959: V75 / MKCD: 66 \emph{*t₁iiʔ}
\end{itemize}

\subparagraph{\texorpdfstring{\emph{*riˀt} `to grind'
(\#0025-2)}{*riˀt to grind (\#0025-2)}}\label{riux2c0t-to-grind-0025-2}

riˀd, FR, rid, RSED.p233, riʔ, BDBH.2276, riɽ, GZ63.15, riɖ, PKED.p169,
riɖ, PJDW.p266, rig, PGEG.p4, rit', CDES.p86, ri'd, BMED.p159, riɖ,
DHED.p288, ri:ɖ, BAHL.p124, --, ---, *riˀt, grind (v), \#0025, V76,
1056,

\begin{itemize}
\tightlist
\item
  Pinnow 1959: V76 / MKCD: 1056 \emph{*riit}, \emph{*riət}
\end{itemize}

\subparagraph{\texorpdfstring{\emph{*xsin} `to boil' V₂
(\#0046-3)}{*xsin to boil V₂ (\#0046-3)}}\label{xsin-to-boil-v-0046-3}

asin, FR, əsin, RSED.p16, nsiŋ, BDBH.1641, isin, GZ65.173, isin,
PKED.p81, isinɔ, JLIC.v65, nsiŋ, PGEG.p37, isin, CDES.p39, isin,
BMED.p77, isin, DHED.p153, isiŋ, BAHL.p12, isin, Korku.txt.12071, *xsin,
boil (v), \#0046, V86, ,

\begin{itemize}
\tightlist
\item
  Pinnow 1959: V86 / MKCD: 1137 \emph{*ciinʔ} (\textgreater{}
  Pre-Bahnaric \emph{*cin}); \emph{*ciən{[} {]}}; \emph{*cain{[} {]}};
  `cooked'
\end{itemize}

\subparagraph{\texorpdfstring{\emph{*xli} `liquor' V₂
(\#0067-3)}{*xli liquor V₂ (\#0067-3)}}\label{xli-liquor-v-0067-3}

ali, FR, əli/ali, RSED.p8, ili, BDBH.120, ili, AG08.p672, ---, ---, ---,
---, ---, ---, ---, ---, ili, BMED.p75, ili, DHED.p151, ---, ---, ---,
---, *xlx, liquor, \#0067, V85, ,

\begin{itemize}
\tightlist
\item
  Pinnow 1959: V85 / MKCD: ---
\end{itemize}

\subparagraph{\texorpdfstring{\emph{*siŋi} `sun' V₁ and V₂ (\#0075-2 and
\#0075-4)}{*siŋi sun V₁ and V₂ (\#0075-2 and \#0075-4)}}\label{siux14bi-sun-v-and-v-0075-2-and-0075-4}

---, ---, ---, ---, siŋi, BDBH.2543, siN, AG08660, siŋ, PKED.p183, siŋ,
PJDW.p244, sni, PGEG.p34, siɲ, CDES.p193, siŋi, BMED.p174, siŋi,
DHED.p319, si:ŋ, BAHL.p136, ---, ---, *siŋi, sun, \#0075, V286, 31,

\begin{itemize}
\tightlist
\item
  Pinnow 1959: V286 / MKCD: 31 \emph{*t₂ŋiiʔ}
\end{itemize}

The reflexes of V₂ (\#0075-4) are consistent with \emph{*i₁}, but not
definite, while the reflexes of V₁ (\#0075-2) are definite.

\paragraph{\texorpdfstring{\emph{*i₂}}{*i₂}}\label{i-1}

\begin{longtable}[]{@{}lllllllllllll@{}}
\toprule
Gorum & Sora & Remo & Gutob & Kharia & Juang & Gtaʔ & Santali & Mundari
& Ho & Korwa & Korku & Set\tabularnewline
\midrule
\endhead
--- & i: & i & i & i & ɛ & æ & i & i & i & i: & i &
0028-2\tabularnewline
i & i & i & i & i & (i) & æ & i & i & i & i: & i & 0091-1\tabularnewline
--- & ɪ: & i & --- & i & i & æi & i & i & i & i: & i &
0095-2\tabularnewline
\bottomrule
\end{longtable}

The correspondence set \emph{*i₂} is treated as a continuation of
proto-Munda \emph{*i}. The reflexes are dominantly close front unrounded
vowels. However, Juang /ɛ/ and Gtaʔ /æ/ constitute a considerable
deviation from the Juang /i/ and Gtaʔ /i/ in \emph{*i₁}. What makes the
difference to \emph{*i₁} problematic for a continuation of proto-Munda
\emph{*i} is the fact that the lowering to an open-mid vowel can not be
motivated. The reconstructed forms from Shorto open up the possibility
that \emph{*i₂} is a reflex of proto-Austroasiatic \emph{*iə}.

More etymons belonging to \emph{*i₂} are needed to decide whether there
are factors allowing to motivate the different reflexes of \emph{*i} in
\emph{*i₂} as opposed to \emph{*i₁}.

\subparagraph{\texorpdfstring{\emph{*si₂m} `chicken'
(\#0028-2)}{*si₂m chicken (\#0028-2)}}\label{sim-chicken-0028-2}

---, ---, kənsi:m, RSED.p131, gisiŋ, BDBH.856, gisiŋ, AG08.p651, siŋkoy,
PKED.p183, sɛŋkɔe, PJDW.p275, gsæŋ, PGEG.p23, sim, CDES.p30, sim,
BMED.p173, sim, HOGV.p151, si:m, BAHL.p135, ---, ---, *sim, chicken,
\#0028, V315, 1324,

\begin{itemize}
\tightlist
\item
  Pinnow 1959: V315 / MKCD: 1324 \emph{*cim}; \emph{*ciim};
  \emph{*ciəm}; \emph{*caim}; \emph{*cum}
\end{itemize}

\subparagraph{\texorpdfstring{\emph{*i₂ˀc} `defecate (v)'
(\#0091-1)}{*i₂ˀc defecate (v) (\#0091-1)}}\label{iux2c0c-defecate-v-0091-1}

ḭj / iˀɟ, FR, gad-iɟ, RSED.p94, ik', BDBH.88, ig, AG08.p652, iˀj,
DSKH\#12711, (ica), DSJU\#13321, æg, PGEG.p4, ic', CSED.p244, ij',
BMED.p75, iiʔ, DHED.p150, i:q, BAHL.p16, ich, NKEV.p310, *iˀc, defecate
(v), \#0091, V81, 794,

\begin{itemize}
\tightlist
\item
  Pinnow 1959: V281 / MKCD: 794 \emph{*ʔic}; \emph{*ʔiə{[}c{]}};
  \emph{*ʔ{[}ə{]}c}
\end{itemize}

\subparagraph{\texorpdfstring{\emph{*ɟi₂ŋ(k)} `porcupine'
(\#0095-2)}{*ɟi₂ŋ(k) porcupine (\#0095-2)}}\label{ux25fiux14bk-porcupine-0095-2}

---, ---, kənɟɪ:ŋ, RSED.p131, gisiŋreʔe, BDBH.858, ---, ---, jiŋray,
PKED.p86, ɟiŋɛ, PJDW.p212, gcæiŋ, PGEG.p22, jhĩk, CSED.p268, jiki,
BMED.p82, jiki, DHED.p165, ji:k, DSKW@09500, jikɽa, NKEV.p313, *ɟiŋ(k),
porcupine, \#0095, V318, 528/1883,

\begin{itemize}
\tightlist
\item
  Pinnow 1959: V318 / MKCD: 528 \emph{*cu{[}ə{]}ŋ}; \emph{*cəŋ};
  \emph{*ciəŋ}
\end{itemize}

Looks like a combination of MKCD 528 and 1883 \emph{*{[}r{]}kus};
\emph{*{[}r{]}kuus}; \emph{*{[}r{]}kuəs}; \emph{*{[}r{]}k{[}iə{]}s},
e.g. \emph{*ciəŋ+{[}r{]}kuəs}.

\paragraph{\texorpdfstring{\emph{*i₃}}{*i₃}}\label{i-2}

\begin{longtable}[]{@{}lllllllllllll@{}}
\toprule
Gorum & Sora & Remo & Gutob & Kharia & Juang & Gtaʔ & Santali & Mundari
& Ho & Korwa & Korku & Set\tabularnewline
\midrule
\endhead
--- & i & i & i & e & ɛ & i & e & i & i & i: & i & 0009-2\tabularnewline
--- & ∅ & i & i & --- & (ɛ) & i & ∅ & i & i & ∅ & --- &
0070-1\tabularnewline
\bottomrule
\end{longtable}

\subparagraph{\texorpdfstring{\emph{*sii₃ˀ} `louse'
(\#0009-2)}{*sii₃ˀ louse (\#0009-2)}}\label{siiux2c0-louse-0009-2}

(aŋiˀd), FR, iʔi, RSED.p109, gisi, BDBH.855, gisi, AG08.p651, seʔ,
PKED.p258, ɛsɛ, PJDW.p192, gsi, PGEG.p23, se, CDES.p116, siku,
BMED.p173, siku, HOGV.p165, guhi:, BAHL.p45, siku, NKEV.p338, sii₂ˀ,
louse, \#0009, V341, 39, 22

\begin{itemize}
\tightlist
\item
  Pinnow 1959: V341 UM: e,ɛ / MKCD: 39 \emph{*ciiʔ} (\& \emph{*ciʔ}?)
\end{itemize}

Unexplained variation of \emph{*i₁}, especially the constrast to *tiiˀ\_
`hand' (\#0008-2) is striking. Kharia /e/, Juang /ɛ/, and Santali /e/
cannot be explained. Pinnow (1959, p.~164 and p.~195) reconstructs
proto-Munda \emph{*e}/\emph{*ɛ}. However, positing \#0009-2 as a
continuation of proto-Munda \emph{*e} (\emph{*ɛ}) is also not
consistent. MKCD 39 \emph{*ciiʔ} also suggests proto-Munda \emph{*siiˀ}.

\subparagraph{\texorpdfstring{\emph{*uli} `mango (ripe)' V₂
(\#0070-3)}{*uli mango (ripe) V₂ (\#0070-3)}}\label{uli-mango-ripe-v-0070-3}

---, ---, u:l, RSED.p304, uli, BDBH.171, ili, DSGU\#4032, ---, ---,
holɛ, PJDW.p205, uli, PGEG.p7, ul, CDES.p118, uli, BMED.p192, uli,
DHED.p370, u:l, BAHL.p19, ---, ---, *uli, mango (ripe), \#0070,
V144/V400e/K496, ,

\begin{itemize}
\tightlist
\item
  Pinnow 1959: V144;V400e;K496 / MKCD: ---
\end{itemize}

If Juang /ɛ/ is genuine, \#0070-3 belongs to \emph{*i₃}. Otherwise,
\emph{*i₁} is also possible.

\subsubsection{\texorpdfstring{Proto-Munda
\emph{*u}}{Proto-Munda *u}}\label{proto-munda-u}

Proto-Munda \emph{*u} is consistently reflected as high back rounded
vowels, except for Gutob where it is consistently reflected as /i/. The
sets \emph{*u₁ₐ} and \emph{*u₂} are very close to \emph{*u₁}. So far
they are based on one etymon, \emph{*buluuˀ} `thigh' (\#0017), where the
fronting of /u/ in Gutob-Remo, spreads into Remo. The V₁ of
\emph{*buluuˀ} `thigh' (\#0017-2) is optionally fronted, while V₂
\emph{*buluuˀ} `thigh' (\#0017-4) is mandatorily fronted.

\begin{itemize}
\tightlist
\item
  \emph{*u₁} general reflex of \emph{*u}
\item
  \emph{*u₁ₐ} difference between \emph{*u₁ₐ} and \emph{*u₁} derives from
  variance due to vowel harmony in Remo (dependent on \emph{*u₂})
\item
  \emph{*u₂} Remo /i/, difference to etyma in \emph{*u₁} unclear
\item
  \emph{*u₂ₐ} /o/ in Santali unique and unexplained
\item
  \emph{*u₃} \emph{*ɲ} coda
\item
  \emph{*u₄} \emph{*m} coda maybe \emph{*u₂} if Gtaʔ /o/ secondary after
  shift from /m/ to /ŋ/
\item
  \emph{*u₅} actually \emph{*ɨ}? (/i/ in Gutob-Remo, Kharia and Gtaʔ)
\item
  \emph{*u₆} probably \emph{*u₂}
\item
  \emph{*u₇} Gutob /u/ unexplained, else compatible with \emph{*u₁} and
  \emph{*u₄}
\end{itemize}

\begin{longtable}[]{@{}lllllllllllll@{}}
\toprule
Gorum & Sora & Remo & Gutob & Kharia & Juang & Gtaʔ & Santali & Mundari
& Ho & Korwa & Korku &\tabularnewline
\midrule
\endhead
u & u & u & i & u & u & u & u & u & u & u & u &
\emph{*u₁}\tabularnewline
u & u & u/i & i & u & u & u & u & u & u & u & u &
\emph{*u₁ₐ}\tabularnewline
u & u & i & i & u & u & u & u & u & u & u & u &
\emph{*u₂}\tabularnewline
u & u & i & i & u & u & u & o & u & u & u & u &
\emph{*u₂ₐ}\tabularnewline
i & u & i & i & u & u & wi & u & ui & u & -- & u &
\emph{*u₃}\tabularnewline
u & u & -- & i & u & u & o & u & u & u & u & u &
\emph{*u₄}\tabularnewline
u & u & i & i & i & -- & i & u & u & u & u & u &
\emph{*u₅}\tabularnewline
u & u & i & i & (o) & (o) & u & u & u & u & u & u &
\emph{*u₆}\tabularnewline
u & ʊ & u & u & u & u & ∅ & --- & (u) & --- & --- & --- &
\emph{*u₇}\tabularnewline
\bottomrule
\end{longtable}

\paragraph{\texorpdfstring{\emph{*u₁}}{*u₁}}\label{u}

\begin{longtable}[]{@{}llllllllllll@{}}
\toprule
Gorum & Sora & Remo & Gutob & Kharia & Juang & Gtaʔ & Santali & Mundari
& Ho & Korwa & Korku\tabularnewline
\midrule
\endhead
u & u & u & i & u & u & u & u & u & u & u & u\tabularnewline
\bottomrule
\end{longtable}

\subparagraph{\texorpdfstring{\emph{*bul} `drunk (v)'
(\#0016-2)}{*bul drunk (v) (\#0016-2)}}\label{bul-drunk-v-0016-2}

bṵl, FR, buʔul, Sora.txt.18922, bu, BDBH.1900, bil, AG08.p672, bul,
PKED.p39, buli, PJDW.p174, busaʔ, PGEG.p13, bul, CDES.p58, bul,
BMED.p25, bul, HOGV.p155, bubul, BAHL.p108, bubul, NKEV.p70, , drunk,
\#0016, V105, 1765,

\begin{itemize}
\tightlist
\item
  Pinnow 1959: V105 / MKCD: 1765 \emph{*ɓul}; \emph{*ɓuul}
\end{itemize}

\subparagraph{\texorpdfstring{\emph{*usal} `skin' V₁
(\#0036-1)}{*usal skin V₁ (\#0036-1)}}\label{usal-skin-v-0036-1}

usal, FR, usal, RSED.p308, usa, BDBH.173, isa:l, HLKS.V149, usal,
PKED.p300, chalo, PJDW.p180, ugsa, PGEG.p6, chal, CDES.p176, , , , , , ,
sa:li, NKEV.p337, , skin, \#0036, V149, ,

\begin{itemize}
\tightlist
\item
  Pinnow 1959: V149 / MKCD: ---
\end{itemize}

\begin{longtable}[]{@{}llllllllllll@{}}
\toprule
Gorum & Sora & Remo & Gutob & Kharia & Juang & Gtaʔ & Santali & Mundari
& Ho & Korwa & Korku\tabularnewline
\midrule
\endhead
u & u & u & i & u & ∅ & u & ∅ & --- & --- & --- & ∅\tabularnewline
\bottomrule
\end{longtable}

The irregular ∅ reflexes are probably due to a second etymon in Juang,
Santali and Korku. This would also account for the irregular palatals in
Juang and Santali.

\subparagraph{\texorpdfstring{\emph{*uli} `mango (ripe)' V₁
(\#0070-1)}{*uli mango (ripe) V₁ (\#0070-1)}}\label{uli-mango-ripe-v-0070-1}

---, ---, u:l, RSED.p304, uli, BDBH.171, ili, DSGU\#4032, ---, ---,
holɛ, PJDW.p205, uli, PGEG.p7, ul, CDES.p118, uli, BMED.p192, uli,
DHED.p370, u:l, BAHL.p19, ---, ---, *xlx, mango (ripe), \#0070,
V144/V400e/K496, ,

\begin{itemize}
\tightlist
\item
  Pinnow 1959: V144;V400e;K496 / MKCD: ---
\end{itemize}

Juang /o/ is unexpected, however the whole form /holɛ/ is as a whole
problematic. The initial /h/ is unexpected.

\begin{longtable}[]{@{}lllllllllllll@{}}
\toprule
Gorum & Sora & Remo & Gutob & Kharia & Juang & Gtaʔ & Santali & Mundari
& Ho & Korwa & Korku & Set\tabularnewline
\midrule
\endhead
--- & u: & u & i & --- & (o) & u & u & u & u & u: & --- &
0070-1\tabularnewline
\bottomrule
\end{longtable}

\subparagraph{\texorpdfstring{\emph{*lutu(uˀ)r} `ear' V₁ and V₂
(\#0073-2 and
\#0073-4)}{*lutu(uˀ)r ear V₁ and V₂ (\#0073-2 and \#0073-4)}}\label{lutuuux2c0r-ear-v-and-v-0073-2-and-0073-4}

luˀd, FR, luˀd, RSED.p165, luntur, BDBH.2386, litir, AG08.p652, lutur,
PKED.p127, lutur/lutuʔ, PJDW.p239, nlug, PGEG.p36, lutur, CDES.p60,
lutur, BMED.p110, lutur, DHED.p216, lutur, BAHL.p128, lutur, NKEV.p324,
*lutu(uˀ)r, ear, \#0073, V147, 1621,

\begin{itemize}
\tightlist
\item
  Pinnow 1959: V147 / MKCD: 1621 \emph{*kt₂uur}; \emph{*kt₂uər}
\end{itemize}

\subparagraph{\texorpdfstring{\emph{*gur} `fall/rain (v)'
(\#0089-2)}{*gur fall/rain (v) (\#0089-2)}}\label{gur-fallrain-v-0089-2}

gur, FR, gur, RSED.p92, gur, BDBH.914, gir, Z1965.132, gur, PKED.p68,
gur, PJDW.p200, gur, PGEG.p21, gur, CSED.p207, gur, EMV5.p1535, gur,
DHED.p122, ---, ---, guru, DSKO\#10541, *gur, fall/rain (v), \#0089,
V106, 1579,

\begin{itemize}
\tightlist
\item
  Pinnow 1959: V106 / MKCD: 1579 \emph{*guur}
\end{itemize}

\subparagraph{\texorpdfstring{\emph{*uˀt} `drink/swallow (v)'
(\#0090-1)}{*uˀt drink/swallow (v) (\#0090-1)}}\label{uux2c0t-drinkswallow-v-0090-1}

---, ---, ---, ---, uʔ, BDBH.181, iɖ, AG08.p664, uˀd, PKED.p205, ur/uɖ,
PJDW.p292, ug, PGEG.p6, ut', CSED.p674, ud', BMED.p191, uɖ, DHED.p369,
u:ɖ, BAHL.p18, u:ʈ, NKEV.346, *uˀt, drink/swallow (v), \#0090, V142,
1106,

\begin{itemize}
\tightlist
\item
  Pinnow 1959: V142 / MKCD: 1106 \emph{*hut}; \emph{*huut};
  \emph{*huət}; \emph{*huc}; \emph{*huuc}; \emph{*huəc}
\end{itemize}

\subparagraph{\texorpdfstring{\emph{*uˀp} (\emph{*uˀk/*uuˀ}) `hair'
(\#0099-1)}{*uˀp (*uˀk/*uuˀ) hair (\#0099-1)}}\label{uux2c0p-uux2c0kuuux2c0-hair-0099-1}

---, FR, uʔ/(uppur), RSED.p308(307), ugbok', BDBH.135, iʔboʔ,
DSGU\#9411, (ului), DSKH\#32441, ---, ---, ugboʔ/ogboʔ, PGEG.p6, up',
CSED.p670, ub', BMED.p191, ub, DHED.p369, u:b, BAHL.p18, hu:p,
NKEV.p310, *uˀp, hair, \#0099, V143, ,

\begin{itemize}
\tightlist
\item
  Pinnow 1959: V143 / MKCD: ---
\end{itemize}

The vowel set is consistent with \emph{*u₁} if Gtaʔ /ugboʔ/ is taken as
the main reflex and compatible with \emph{*u₄} if based on Gtaʔ /ogboʔ/.

\begin{longtable}[]{@{}llllllllllll@{}}
\toprule
Gorum & Sora & Remo & Gutob & Kharia & Juang & Gtaʔ & Santali & Mundari
& Ho & Korwa & Korku\tabularnewline
\midrule
\endhead
--- & u & u & i & (u) & u & u/o & u & u & u & u: & u:\tabularnewline
\bottomrule
\end{longtable}

\paragraph{\texorpdfstring{\emph{*u₁ₐ}}{*u₁ₐ}}\label{uux2090}

\begin{longtable}[]{@{}llllllllllll@{}}
\toprule
Gorum & Sora & Remo & Gutob & Kharia & Juang & Gtaʔ & Santali & Mundari
& Ho & Korwa & Korku\tabularnewline
\midrule
\endhead
u & u & u/i & i & u & u & u & u & u & u & u & u\tabularnewline
\bottomrule
\end{longtable}

\subparagraph{\texorpdfstring{\emph{*buluuˀ} `thigh' V₁
(\#0017-2)}{*buluuˀ thigh V₁ (\#0017-2)}}\label{buluuux2c0-thigh-v-0017-2}

bulu, FR, bulu:, RSED.p64, buli/bili, BDBH.1949/1890, bili, DSGU.2681,
bhulu, PKED.p32, bulu, PJDW.p174, bulu, PGEG.p13, bulu, CDES.p199, bulu,
BMED.p25, bulu, HOGV.p183, bu:l, BAHL.p109, bulu, NKEV.p295, , thigh,
\#0017, V145, 223,

\begin{itemize}
\tightlist
\item
  Pinnow 1959: V145 / MKCD: 223 \emph{*bluuʔ}
\end{itemize}

\paragraph{\texorpdfstring{\emph{*u₂}}{*u₂}}\label{u-1}

\begin{longtable}[]{@{}llllllllllll@{}}
\toprule
Gorum & Sora & Remo & Gutob & Kharia & Juang & Gtaʔ & Santali & Mundari
& Ho & Korwa & Korku\tabularnewline
\midrule
\endhead
u & u & i & i & u & u & u & u & u & u & u & u\tabularnewline
\bottomrule
\end{longtable}

\subparagraph{\texorpdfstring{\emph{*buluuˀ} `thigh' V₂
(\#0017-4)}{*buluuˀ thigh V₂ (\#0017-4)}}\label{buluuux2c0-thigh-v-0017-4}

bulu, FR, bulu:, RSED.p64, buli/bili, BDBH.1949/1890, bili, DSGU.2681,
bhulu, PKED.p32, bulu, PJDW.p174, bulu, PGEG.p13, bulu, CDES.p199, bulu,
BMED.p25, bulu, HOGV.p183, bu:l, BAHL.p109, bulu, NKEV.p295, , thigh,
\#0017, V145, 223,

\begin{itemize}
\tightlist
\item
  Pinnow 1959: V145 / MKCD: 223 \emph{*bluuʔ}
\end{itemize}

\paragraph{\texorpdfstring{\emph{*u₂ₐ}}{*u₂ₐ}}\label{uux2090-1}

\begin{longtable}[]{@{}llllllllllll@{}}
\toprule
Gorum & Sora & Remo & Gutob & Kharia & Juang & Gtaʔ & Santali & Mundari
& Ho & Korwa & Korku\tabularnewline
\midrule
\endhead
u & u & i & i & u & u & u & o & u & u & u & u\tabularnewline
\bottomrule
\end{longtable}

Santali /o/ is unexpected.

\subparagraph{\texorpdfstring{\emph{*KʰVsu} `fever/pain' V₁
(\#0026-4)}{*KʰVsu fever/pain V₁ (\#0026-4)}}\label{kux2b0vsu-feverpain-v-0026-4}

asu, FR, asu:/əsu:, RSED.p42, siʔ, BDBH.2610, isi, GGEG.p93, kosu/kusu,
PKED.p107, kasu, PJDW.p220, aʔsu, PGEG.p4, haso, CDES.p135, ha:su,
BMED.p67, hasu, HOGV.p147, hasu:, BAHL.p145, kaSu, NKEV.p315, *Kʰxsu,
fever/pain, \#0026, V247, 44,

\begin{itemize}
\tightlist
\item
  Pinnow 1959: V247 / MKCD: 44 \emph{*{[}c{]}uuʔ}
\end{itemize}

\paragraph{\texorpdfstring{\emph{*u₃}}{*u₃}}\label{u-2}

\begin{longtable}[]{@{}llllllllllll@{}}
\toprule
Gorum & Sora & Remo & Gutob & Kharia & Juang & Gtaʔ & Santali & Mundari
& Ho & Korwa & Korku\tabularnewline
\midrule
\endhead
i & u & i & i & u & u & wi & u & ui & u & ? & u\tabularnewline
\bottomrule
\end{longtable}

Defining context: palatal nasal coda \emph{*uɲ}

\subparagraph{\texorpdfstring{\emph{*tuɲ} `shoot (v)'
(\#0027-2)}{*tuɲ shoot (v) (\#0027-2)}}\label{tuux272-shoot-v-0027-2}

tiŋ, FR, tuɲ, RSED.p299, tiŋ, BDBH.1368, tiŋ, GZ63.190, tuɲ, PKED.p196,
tuɲ, PJDW.p288, ʈwiŋ, PGEG.p46, tuɲ, CDES.p173, tuiŋ, BMED.p180, tuŋ,
HOGV.p177, , , ʈuɲj, NKEV.p343, , shoot (v), \#0027, V107, 896a?,

\begin{itemize}
\tightlist
\item
  Pinnow 1959: V107 / MKCD: 896a?
\end{itemize}

MKCD 896a \emph{*t₁iɲ}; \emph{*t₁iiɲ}; \emph{*t₁iəɲ}; \emph{*t₁əɲ} `to
pluck, twang' could be related. Its meaning is generally `to pluck a
(stringed) instrument' which is rather close to `to shoot with bow and
arrow.'

\paragraph{\texorpdfstring{\emph{*u₄}}{*u₄}}\label{u-3}

\begin{longtable}[]{@{}llllllllllll@{}}
\toprule
Gorum & Sora & Remo & Gutob & Kharia & Juang & Gtaʔ & Santali & Mundari
& Ho & Korwa & Korku\tabularnewline
\midrule
\endhead
u & u & -- & i & u & u & o & u & u & u & u & u\tabularnewline
\bottomrule
\end{longtable}

\subparagraph{\texorpdfstring{\emph{*gum} `winnow (v)'
(\#0044-2)}{*gum winnow (v) (\#0044-2)}}\label{gum-winnow-v-0044-2}

gumar, FR, gum, RSED.p105, (giteʔ), BDBH.864, gim, GZ63.134, gum,
PKED.p67, guŋ/guɲ, PJDW.p199, goŋ, PGEG.p20, gum, BSDV2.p490, gum,
BMED.p214, gum, DHED.p120, gum, BAHL.p45, gum, NKEV.p307, *gum, winnow
(v), \#0044, K159, 1317,

\begin{itemize}
\tightlist
\item
  Pinnow 1959: K159 / MKCD: 1317 \emph{*gum}; \emph{*guum};
  \emph{*g{[}əə{]}m}
\end{itemize}

If Remo \emph{gumi} `heap of unclean paddy before winnowing' (BDBH.908)
can be connected, it would connected these reflexes to \emph{*u₁}, with
Gtaʔ \emph{goŋ} displaying exceptional reflexes of proto-Munda \emph{*u}
and proto-Munda \emph{*ŋ}.

\paragraph{\texorpdfstring{\emph{*u₅}
(\emph{*ɨ}?)}{*u₅ (*ɨ?)}}\label{u-ux268}

\begin{longtable}[]{@{}lllllllllllll@{}}
\toprule
Gorum & Sora & Remo & Gutob & Kharia & Juang & Gtaʔ & Santali & Mundari
& Ho & Korwa & Korku & Set\tabularnewline
\midrule
\endhead
u & u & i & i & i & -- & i & u & u & u & u & u & 0033-3\tabularnewline
u & u & i & i & i & -- & i & -- & -- & -- & -- & -- &
0034-4\tabularnewline
-- & u: & i & i & -- & -- & i & ∅ & u & u & -- & u &
0069-4\tabularnewline
\bottomrule
\end{longtable}

The set \emph{*u₅} features /u/ in North Munda and Sora-Gorum and /i/ in
the other branches, while \emph{*u₅ₐ} features /u/ in Sora-Gorum and /i/
in all other branches.

\subparagraph{\texorpdfstring{\emph{*vdʲu₅ˀp} `night' V₂
(\#0033-3)}{*vdʲu₅ˀp night V₂ (\#0033-3)}}\label{vdux2b2uux2c0p-night-v-0033-3}

uɖuˀb, FR, orub, RSED.p195, minɖip', BDBH.2087, noNdib, GZ65.260, iɖiˀb,
PKED.p79, ---, ---, minɖig, PGEG.p33, ayup', CDES.p128, a:yub, BMED.p14,
ayub, HOGV.p157, ayub, BAHL.p3, ayup, NKEV.p290, , night, \#0033, V280,
1268,

\begin{itemize}
\tightlist
\item
  Pinnow 1959: V280 / MKCD: 1268 \emph{*yup}; \emph{*y{[}uu{]}p};
  \emph{*yəp}
\end{itemize}

\subparagraph{\texorpdfstring{\emph{*tVru₅ˀp} `cloud' V₂
(\#0034-4)}{*tVru₅ˀp cloud V₂ (\#0034-4)}}\label{tvruux2c0p-cloud-v-0034-4}

taruˀb, FR, tarub, RSED.p283, tirib, BDBH.1387, tirib, GZ65.416, tiriˀb,
PKED.p287, ---, ---, trig, PGEG.p46, rimil, CDES.p33, rimil, BMED.p160,
rimil, HOGV.p152, liNbir, BAHL.p127, ---, ---, , cloud, \#0034, V285a, ,

\begin{itemize}
\tightlist
\item
  Pinnow 1959: V285a / MKCD: ---
\end{itemize}

If /rim/ in Santali, Munda, and Ho is parallel to /ri(ˀ)b/ in Remo,
Gutob, and Kharia, \emph{*tVru₅ˀp} `cloud' belongs quite clearly to
\emph{*bVˀt} `sow (v)', else it could belong to \emph{*vdʲu₅ˀp(???)}
`night' or \emph{*bVˀt} `sow (v)'.

\subparagraph{\texorpdfstring{\emph{*ɟəlu₅} `meat' V₂
(\#0069-4)}{*ɟəlu₅ meat V₂ (\#0069-4)}}\label{ux25fux259lu-meat-v-0069-4}

---, ---, ɟelu:, RSED.p123, sili/seli, BDBH.2599/2731, seli, AG08.p674,
―, ---, ---, ---, cili, PGEG.p15, jel, CDES.p120, jilu, BMED.p83, jilu,
DHED.p165, ---, ---, jilu, NKEV.p311, *ɟəlu₅, meat, \#0069, V228, ,

\begin{itemize}
\tightlist
\item
  Pinnow 1959: V228 / MKCD: ---
\end{itemize}

A possibly connected MKCD etymon is MKCD 204 \emph{*{[}c{]}nlu{[}u{]}ʔ}
`edible grub' only atttested in Bahnaric.

\paragraph{\texorpdfstring{\emph{*u₆}}{*u₆}}\label{u-4}

\begin{longtable}[]{@{}lllllllllllll@{}}
\toprule
Gorum & Sora & Remo & Gutob & Kharia & Juang & Gtaʔ & Santali & Mundari
& Ho & Korwa & Korku & Set\tabularnewline
\midrule
\endhead
u & u & i & i & (o) & (o) & u & u & u & u & u & u &
0074-2\tabularnewline
\bottomrule
\end{longtable}

The correspondence set is based on the etymon \emph{*muuˀ} `nose'.
Kharia \emph{romoŋ}/\emph{romoˀɖ} and Juang \emph{motɛɟ} feature a /o/
instead of the /u/ attested in set \emph{u₂}. The dental coda in both
Kharia and Juang is unexplained and might indicate that these words are
not straightforward reflexes of proto-Munda \emph{*muuˀ}.

\begin{longtable}[]{@{}lllllllllllll@{}}
\toprule
Gorum & Sora & Remo & Gutob & Kharia & Juang & Gtaʔ & Santali & Mundari
& Ho & Korwa & Korku & Set\tabularnewline
\midrule
\endhead
u & u & i & i & (o) & (o) & u & u & u & u & u & u &
\emph{u₆}\tabularnewline
u & u & i & i & u & u & u & u & u & u & u & u & \emph{u₂}\tabularnewline
\bottomrule
\end{longtable}

\subparagraph{\texorpdfstring{\emph{*muuˀ} `nose'
(\#0074-2)}{*muuˀ nose (\#0074-2)}}\label{muuux2c0-nose-0074-2}

muʔ, FR, mu:ʔ, RSED.p179, nseʔmiʔ, BDBH.1653, miʔ, GZ63.262,
romoŋ/romoˀɖ, PKED.p170, motɛɟ, PJDW.p245, mmu, PGEG.p34, muN,
CDES.p129, mu/muhu, BMED.p121, muwa/muʈa, DHED.p238, hu:mu:, DSKW@23180,
mu:, NKEV.p327, *mxxˀ, nose, \#0074, , ,

\begin{itemize}
\tightlist
\item
  Pinnow 1959: V387 / MKCD: 2045 \emph{*muh}; \emph{*muuh}; \emph{*muus}
\end{itemize}

\paragraph{\texorpdfstring{\emph{*u₇}}{*u₇}}\label{u-5}

\begin{longtable}[]{@{}lllllllllllll@{}}
\toprule
Gorum & Sora & Remo & Gutob & Kharia & Juang & Gtaʔ & Santali & Mundari
& Ho & Korwa & Korku & Set\tabularnewline
\midrule
\endhead
u & ʊ & u & u & u & u & ∅ & --- & (u) & --- & --- & --- &
0068-2\tabularnewline
\bottomrule
\end{longtable}

The one attested set where /u/ in Gorum, Sora, Remo, Kharia and Juang
correlates with Gutob /u/ and not /i/. Otherwise, the set is compatible
with \emph{*u₁} and \emph{*u₄}. MCKD suggests an origin in \emph{*ᵊ}
(probably \emph{*rᵊ(N)kvˀk})

\subparagraph{\texorpdfstring{\emph{*ruNkOˀk} `husked rice'
(\#0068-2)}{*ruNkOˀk husked rice (\#0068-2)}}\label{runkoux2c0k-husked-rice-0068-2}

ruŋk, FR, rʊŋkʊ, RSED.p239, ruŋku, BDBH.2291, rukug, AG08.p672,
ruŋkuˀb/rumkuˀb, PKED.p171, ruŋkub, PJDW.p269, rkoʔ, PGEG.p41, ---, ---,
(rukhaɽ), BMED.p163, ---, ---, ---, ---, ---, ---, *ruNkOˀk, husked
rice, \#0068, V139, 1820,

\begin{itemize}
\tightlist
\item
  Pinnow 1959: V139 / MKCD: 1820 \emph{*rk{[}aw{]}ʔ}
\end{itemize}

\subsubsection{\texorpdfstring{Proto-Munda
\emph{*e}}{Proto-Munda *e}}\label{proto-munda-e}

Context for variation is very unclear. Maybe separate into \emph{*ɛ} and
\emph{*e}? Probably best along the lines \emph{*e₁} as a continuation of
\emph{*ɛ}; \emph{*e₂} and \emph{*e₃} as a continuation of \emph{*e}.

\begin{itemize}
\tightlist
\item
  \emph{*e₁}
\item
  \emph{*e₂} \emph{*l} coda?, but \#0065-3, \#0054-5 are \emph{*e₁}
\item
  \emph{*e₃}
\end{itemize}

\begin{longtable}[]{@{}lllllllllllll@{}}
\toprule
Gorum & Sora & Remo & Gutob & Kharia & Juang & Gtaʔ & Santali & Mundari
& Ho & Korwa & Korku &\tabularnewline
\midrule
\endhead
e & e & e & e & e & -- & i & e̠ & e & e & e & e &
\emph{*e₁}\tabularnewline
i & e/ɪ: & e & e & e & ɛ & e & e & i & i & e/i & i &
\emph{*e₂}\tabularnewline
i & e & e & e & ɛ & -- & i & e & e & e & e & -- &
\emph{*e₃}\tabularnewline
\bottomrule
\end{longtable}

\paragraph{\texorpdfstring{\emph{*e₁}}{*e₁}}\label{e}

\begin{longtable}[]{@{}lllllllllllll@{}}
\toprule
Gorum & Sora & Remo & Gutob & Kharia & Juang & Gtaʔ & Santali & Mundari
& Ho & Korwa & Korku & Set\tabularnewline
\midrule
\endhead
e & e & -- & -- & e & -- & -- & -- & e & e & -- & -- &
0020-4\tabularnewline
e & e & e & e & ɛ & -- & i & e̠ & -- & -- & -- & -- &
0057-2\tabularnewline
e & -- & -- & -- & e & -- & -- & e̠ & e & e & e & e &
0065-3\tabularnewline
\bottomrule
\end{longtable}

\subparagraph{\texorpdfstring{\emph{*səreŋ} `stone' V₂
(\#0020-4)}{*səreŋ stone V₂ (\#0020-4)}}\label{sux259reux14b-stone-v-0020-4}

areŋ, FR, areŋ, RSED.p39, ---, ---, ---, ---, soreŋ, PKED.p187, ---,
---, ---, ---, ---, ---, sereŋ, BMED.p172, sereɲ, HOGV.p175, ---, ---,
---, ---, *səreŋ, stone, \#0020, V183, ,

\begin{itemize}
\tightlist
\item
  Pinnow 1959: V183 / MKCD: ---
\end{itemize}

MKCD B42 (Palaungic) \emph{*{[} {]}rʔa{[}a{]}ŋ} might compare to
\emph{*səreŋ}.

\subparagraph{\texorpdfstring{\emph{*peˀt} `blow (v)'
(\#0057-2)}{*peˀt blow (v) (\#0057-2)}}\label{peux2c0t-blow-v-0057-2}

peˀd, ped, RSED.p212, peʔ, BDBH.1759, ped, ZG65.293, pɛˀɖ, PJED.p156,
---, , piʔ, PGEG.p38, phe̠t', CDES.p142, ---, , ---, , ---, , ---, ,
*pxˀt, blow (v), \#0057, V157, 1028,

\begin{itemize}
\tightlist
\item
  Pinnow 1959: V157 / MKCD: 1028 \emph{*puut}; \emph{*p{[}əə{]}t}
\end{itemize}

\subparagraph{\texorpdfstring{\emph{*əsel} `white' V₂
(\#0065-3)}{*əsel white V₂ (\#0065-3)}}\label{ux259sel-white-v-0065-3}

asel, FR, ---, ---, ---, ---, ---, ---, osel, PKED.p216, ---, ---, ---,
---, e̠se̠l, BSDV2.p343, esel, BMED.p56, esel, DHED.p102, hesel,
BAHL.p149, esel, HLKS.V255, *əsel, white, \#0065, V255, ,

\begin{itemize}
\tightlist
\item
  Pinnow 1959: V255 / MKCD: ---
\end{itemize}

\subparagraph{\texorpdfstring{\emph{*boŋtel}/\emph{*bitkil} `buffalo' V₂
(\#0054-5)}{*boŋtel/*bitkil buffalo V₂ (\#0054-5)}}\label{boux14btelbitkil-buffalo-v-0054-5}

boŋtel, FR, boŋtel, RSED.p62, buŋte, BDBH.1917, boŋtel, AG08.p647,
boŋtel, PKED.p36, ---, ---, buNʈi, PGEG.p13, bitkil, CDES.p23, ---, ---,
---, ---, ---, ---, biʈkhil, NKEV.p294, *bxŋtxl, buffalo, \#0054, , ,

North Munda /i/ has to be taken out. There is no regular sound change
that will result in the southern forms and North Munda \emph{*bitkil}.
When the North Munda forms are excluded, the remaining set is consistent
with \emph{*e₁} and \emph{*ə₂}.

\begin{longtable}[]{@{}lllllllllllll@{}}
\toprule
Gorum & Sora & Remo & Gutob & Kharia & Juang & Gtaʔ & Santali & Mundari
& Ho & Korwa & Korku & Set\tabularnewline
\midrule
\endhead
e & e & e & e & e & -- & i & (i) & -- & -- & -- & (i) &
0054-5\tabularnewline
e & e & e & e & e & -- & i & e̠ & e & e & e & e &
\emph{*e₁}\tabularnewline
e & e & e & e & e & -- & i & e̠ & i & i & e & i &
\emph{*ə₂}\tabularnewline
\bottomrule
\end{longtable}

\subparagraph{\texorpdfstring{\emph{*geˀp} `to burn (vi)'
(\#0058-2)}{*geˀp to burn (vi) (\#0058-2)}}\label{geux2c0p-to-burn-vi-0058-2}

geˀb ,FR ,tuŋge:b ,RSED.p298 ,gep' ,BDBH.967 ,geb ,GZ65.123 ,geb
,PKED.p61 ,--- ,--- ,giʔ ,PGEG.p19 ,~--- ,--- ,--- ,--- ,--- ,--- ,---
,--- ,--- ,--- ,*geˀp ,burn (vi) ,\#0058 , 156, ,

\begin{itemize}
\tightlist
\item
  Pinnow 1959: 156 / MKCD: ---
\end{itemize}

\begin{longtable}[]{@{}lllllllllllll@{}}
\toprule
Gorum & Sora & Remo & Gutob & Kharia & Juang & Gtaʔ & Santali & Mundari
& Ho & Korwa & Korku & Set\tabularnewline
\midrule
\endhead
e & e: & e & e & e & -- & i & -- & -- & -- & -- & -- &
0058-2\tabularnewline
e & e & e & e & e & -- & i & e̠ & e & e & e & e &
\emph{*e₁}\tabularnewline
e & e & e & e & e & -- & i & e̠ & i & i & e & i &
\emph{*ə₂}\tabularnewline
\bottomrule
\end{longtable}

\subparagraph{\texorpdfstring{\emph{*per} `to burn (of chilies) (vi)'
(\#0097-2)}{*per to burn (of chilies) (vi) (\#0097-2)}}\label{per-to-burn-of-chilies-vi-0097-2}

per ,FR ,--- ,--- ,per ,BDBH.1756 ,per ,Z1975.294 ,--- ,--- ,--- ,---
,pir ,PGEG.p38 ,pe̠ṛen ,CSED.p500 ,--- ,--- ,(pertol) ,DHED.p266 ,---
,--- ,--- ,--- ,*per ,burn(chilies) (v) ,\#0097 ,--- , ,

\begin{itemize}
\tightlist
\item
  Pinnow 1959: --- / MKCD: ---
\end{itemize}

\begin{longtable}[]{@{}lllllllllllll@{}}
\toprule
Gorum & Sora & Remo & Gutob & Kharia & Juang & Gtaʔ & Santali & Mundari
& Ho & Korwa & Korku & Set\tabularnewline
\midrule
\endhead
e & -- & e & e & -- & -- & i & e̠ & -- & (e) & -- & -- &
0097-2\tabularnewline
e & e & e & e & e & -- & i & e̠ & e & e & e & e &
\emph{*e₁}\tabularnewline
e & e & e & e & e & -- & i & e̠ & i & i & e & i &
\emph{*ə₂}\tabularnewline
\bottomrule
\end{longtable}

\paragraph{\texorpdfstring{\emph{*e₂}}{*e₂}}\label{e-1}

\begin{longtable}[]{@{}llllllllllll@{}}
\toprule
Gorum & Sora & Remo & Gutob & Kharia & Juang & Gtaʔ & Santali & Mundari
& Ho & Korwa & Korku\tabularnewline
\midrule
\endhead
i & e/ɪ: & e & e & e & ɛ & e & e & i & i & e & i\tabularnewline
i & e: & e & e & e & ɛ & e & e & i & i & i & --\tabularnewline
-- & e & e & -- & e/i & ɛ & e & e & i & i & -- & --\tabularnewline
\bottomrule
\end{longtable}

Defining context: \emph{*el} (not sufficient, cf. *əsel\_ `white' in
\emph{*e₁})

some variation: Sora e/ɪ:\textasciitilde{}e(:), Kharia
e\textasciitilde{}e/i, Korwa e\textasciitilde{}i

\subparagraph{\texorpdfstring{\emph{*bel} `spread (v)'
(\#0022-2)}{*bel spread (v) (\#0022-2)}}\label{bel-spread-v-0022-2}

bil, FR, bel/bɪ:l, RSED.p56/58, be-sak', BDBH.1982, be(d), GZ65.50, bel,
PKED.p18, bɛn, PJDW.p166, beʔ, PGEG.p11, bel, CDES.p184, bil, BMED.p24,
bil, HOGV.p179, bel, BAHL.p111, (bi)bil, NKEV.p293, *bel, spread (vt),
\#0022, V221, 1761,

\begin{itemize}
\tightlist
\item
  Pinnow 1959: V221 / MKCD: 1761 \emph{*b{[}e{]}l} (\emph{*beel}?)
\end{itemize}

\subparagraph{\texorpdfstring{\emph{*xrel} `hail/pebble' V₂
(\#0032-3)}{*xrel hail/pebble V₂ (\#0032-3)}}\label{xrel-hailpebble-v-0032-3}

aril, FR, are:l, RSED.p39, are, BDBH.43, arel, HLKS.V225, arel, PKED.p7,
aɭɛn, PJDW.p158, hare, PGEG.p24, arel, CDES.p88, a:ɽil, BMED.p10, aril,
HOGV.p161, a:ril, BAHL.p10, ---, ---, *xrel, hail/pebble, \#0032, V225,
1791,

\begin{itemize}
\tightlist
\item
  Pinnow 1959: V225 / MKCD: 1791 \emph{*pril}; \emph{*priəl}
\end{itemize}

\subparagraph{\texorpdfstring{\emph{*tVrel} `ebony' V₂
(\#0083-4)}{*tVrel ebony V₂ (\#0083-4)}}\label{tvrel-ebony-v-0083-4}

---, ---, tarel, RSED.p138, tire, BDBH.1390, ---, ---, ti(ɽr)(ei)l,
PKED.p200, tɛrɛn, PJDW.p285, tre, PGEG.p46, terel, CSED.p626, tiril,
BMED.p188, tiril, DHED.p355, ---, ---, ---, ---, *txrel, ebony, \#0083,
V227, ,

\begin{itemize}
\tightlist
\item
  Pinnow 1959: V227 / MKCD: ---
\end{itemize}

\paragraph{\texorpdfstring{\emph{*e₃}}{*e₃}}\label{e-2}

\begin{longtable}[]{@{}lllllllllllll@{}}
\toprule
Gorum & Sora & Remo & Gutob & Kharia & Juang & Gtaʔ & Santali & Mundari
& Ho & Korwa & Korku &\tabularnewline
\midrule
\endhead
i & e & e & e & ɛ & -- & -- & e & e & e & e & -- & 0077-4\tabularnewline
i & -- & -- & e & e & -- & (e) & e & e & e & e & e &
0086-2\tabularnewline
i & e & e & e & e & -- & i & -- & -- & -- & -- & -- &
0053-4\tabularnewline
\bottomrule
\end{longtable}

Some variation: Kharia ɛ\textasciitilde{}e probably artefact of
different descriptions.

\subparagraph{\texorpdfstring{\emph{*gəle} `ear of corn' V₂
(\#0077-4)}{*gəle ear of corn V₂ (\#0077-4)}}\label{gux259le-ear-of-corn-v-0077-4}

gali, FR, gale, RSED.p96, gileker, DSBO.11781, gile, GTXT.7791, gɔlɛ,
HLKS.V182, (ɔnɔ), PJDW.p255, (konto-ja), PGEG.p28, gele, CDES.p185,
gele, EM.p1418, gele, DHED.p111, geleʔ, BAHL.p45, (kelʈa), NKEV.p317,
*gxle, ear of corn, \#0077, V182, 1577,

\begin{itemize}
\tightlist
\item
  Pinnow 1959: V182 / MKCD: 1577 \emph{*gur}; \emph{*guər}
\end{itemize}

\subparagraph{\texorpdfstring{\emph{*ten} `trample (v)'
(\#0086-2)}{*ten trample (v) (\#0086-2)}}\label{ten-trample-v-0086-2}

tin, FR, ---, ---, ---, ---, ten, Z1965.402, ten, PKED.p199, ---, ---,
(ʈe), PGEG.p46, ten, CSED.p624, ʈhen, BMED.p185, ten, DHED.p347, ten,
DSKW@1275, ten, DSKO\#26831, *txn, trample (v), \#0086, K306, 1153a,

\begin{itemize}
\tightlist
\item
  Pinnow 1959: K306 / MKCD: 1153a \emph{*t₁een}
\end{itemize}

Gtaʔ \emph{ʈe} is problematic. The retroflex /ʈ/ is unexplained and the
vowel /e/ is not expected in Gtaʔ in \emph{*e₃} reflex sets.

\paragraph{\texorpdfstring{\emph{*tVme} `new'
(\#0053-4)}{*tVme new (\#0053-4)}}\label{tvme-new-0053-4}

tḛmi, FR, tamme, RSED.p277, time, BDBH.1383, time, ZG65.410, tonme,
PKED.p289, ---, ---, tmi, PGEG.p125, (nãwã), CDES.p128, (nawã),
BMED.p127, (nama), HOGV.p168, ---, ---, (uni), NKEV.345, *txmx, new,
\#0053, V184, 144,

\begin{itemize}
\tightlist
\item
  Pinnow 1959: V182 / MKCD: 144 \emph{*t₁miʔ}
\end{itemize}

\subsubsection{\texorpdfstring{Proto-Munda
\emph{*o}}{Proto-Munda *o}}\label{proto-munda-o}

In all languages /o/ except Remo consistently /u/, Gtaʔ /u/ in
\emph{*o₁} and /o/ in \emph{*o₂}.

\begin{itemize}
\tightlist
\item
  \emph{*o₁} general
\item
  \emph{*o₂} so far: velar coda \emph{*ˀk} and \emph{*ŋ} (but
  \emph{*boŋtel})
\end{itemize}

\begin{longtable}[]{@{}lllllllllllll@{}}
\toprule
Gorum & Sora & Remo & Gutob & Kharia & Juang & Gtaʔ & Santali & Mundari
& Ho & Korwa & Korku &\tabularnewline
\midrule
\endhead
o & o & u & o & o & o & u & o̠ & o & o & o & o &
\emph{*o₁}\tabularnewline
o & o & u & o & o & o & o & o̠ & o & o & o & u &
\emph{*o₂}\tabularnewline
\bottomrule
\end{longtable}

\paragraph{\texorpdfstring{\emph{*o₁}}{*o₁}}\label{o}

\begin{longtable}[]{@{}llllllllllll@{}}
\toprule
Gorum & Sora & Remo & Gutob & Kharia & Juang & Gtaʔ & Santali & Mundari
& Ho & Korwa & Korku\tabularnewline
\midrule
\endhead
o & o & u & o & o & o & u & o̠ & o & o & o & o\tabularnewline
\bottomrule
\end{longtable}

\subparagraph{\texorpdfstring{\emph{*tol} `tie (v)'
(\#0024-2)}{*tol tie (v) (\#0024-2)}}\label{tol-tie-v-0024-2}

tol, FR, tol, RSED.p292, tu, BDBH.1398, tol, AG08.647, tol, PKED.p288,
tor, PJDW.p287, tu, PGEG.p46, to̠l, CDES.p201, tol, BMED.p186, tol,
HOGV.p183, tol, BAHL.p84, ʈol, NKEV.p343, *tol, tie (v), \#0024, V191, ,

\begin{itemize}
\tightlist
\item
  Pinnow 1959: V191 / MKCD: ---
\end{itemize}

\subparagraph{\texorpdfstring{\emph{*ɟoˀt} `wipe (v)'
(\#0029-2)}{*ɟoˀt wipe (v) (\#0029-2)}}\label{ux25foux2c0t-wipe-v-0029-2}

zoˀd, FR, ɟoˀd, RSED.p126, susuʔ, BDBH.2698, sosod, GZ65.374, joˀɖ,
PKED.p87, ---, ---, cuʔ, PGEG.p15, jo̠t', CDES.p221, jod', BMED.p84, joɖ,
HOGV.p88, joɖ, BAHL.p62, o:jo, NKEV.p329, *ɟoˀt, wipe (v), \#0029, V190,
994,

\begin{itemize}
\tightlist
\item
  Pinnow 1959: V190 / MKCD: 994 \emph{*{[} {]}jut}; \emph{*{[} {]}juut}
\end{itemize}

\subparagraph{\texorpdfstring{\emph{*ɟo(o)ˀ} `fruit; bear fruit (v)'
(\#0030-2)}{*ɟo(o)ˀ fruit; bear fruit (v) (\#0030-2)}}\label{ux25fooux2c0-fruit-bear-fruit-v-0030-2}

zoʔ, FR, ɟo:ʔ, RSED.p125, suʔ, BDBH.2701, ---, ---, ---, ---, ---, ---,
cu, PGEG.p15, jo̠, CDES.p80, jo, BMED.p83, jo:, DHED.p83, joʔ, BAHL.p63,
jo:, NKEV.p313, *ɟo(o)ˀ, fruit / to bear fruit (v), \#0030, V188, ,

\begin{itemize}
\tightlist
\item
  Pinnow 1959: V188 / MKCD: ---
\end{itemize}

\subparagraph{\texorpdfstring{\emph{*boŋtel} `buffalo' V₁
(\#0054-2)}{*boŋtel buffalo V₁ (\#0054-2)}}\label{boux14btel-buffalo-v-0054-2}

boŋtel, FR, boŋtel, RSED.p62, buŋte, BDBH.1917, boŋtel, AG08.p647,
boŋtel, PKED.p36, ---, ---, buNʈi, PGEG.p13, bitkil, CDES.p23, ---, ---,
---, ---, ---, ---, biʈkhil, NKEV.p294, *bxŋtxl, buffalo, \#0054, , ,

North Munda /i/ has to be taken out. There is no regular sound change
that will result in the southern forms and North Munda \emph{*bitkil}.

\paragraph{\texorpdfstring{\emph{*o₂}}{*o₂}}\label{o-1}

\begin{longtable}[]{@{}llllllllllll@{}}
\toprule
Gorum & Sora & Remo & Gutob & Kharia & Juang & Gtaʔ & Santali & Mundari
& Ho & Korwa & Korku\tabularnewline
\midrule
\endhead
o & o & u & o & o & o & o & o̠ & o & o & o & u\tabularnewline
\bottomrule
\end{longtable}

Defining context: velar coda -- \emph{*oˀk}/\emph{*oŋ}

\subparagraph{\texorpdfstring{\emph{*ɟoˀk} `sweep (v)'
(\#0031-2)}{*ɟoˀk sweep (v) (\#0031-2)}}\label{ux25foux2c0k-sweep-v-0031-2}

zoʔ, FR, ɟo:, RSED.p126, suk', BDBH.2624, sog, AG08.p650, joʔ, PKED.p87,
ɟɛnɔg, PJDW.p211, coʔ, PGEG.p15, jo̠k', CDES.p194, joʔ, BMED.p85, joʔ,
DHED.p167, ---, ---, ju-khɽi, NKEV.p313, *ɟoˀk, sweep (v), \#0031, 190,
,

\begin{itemize}
\tightlist
\item
  Pinnow 1959: V190 / MKCD: ---
\end{itemize}

\subparagraph{\texorpdfstring{\emph{*bVtoŋ} `fear' V₂
(\#0039-4)}{*bVtoŋ fear V₂ (\#0039-4)}}\label{bvtoux14b-fear-v-0039-4}

butoŋ, FR, bato:ŋ, RSED.p55, butuŋ, BDBH.1922, butoŋ, GZ65.76, bɔtɔŋ
(P), HLKS.V261, betɔŋan, JLIC.v239, bʈoʔ, PGEG.p14, ---, ---, botoŋ,
BMED.p25, ---, ---, (bor), BAHL.p112, ---, ---, *bxtoŋ, fear, \#0039,
V261, 552,

\begin{itemize}
\tightlist
\item
  Pinnow 1959: V261 / MKCD: 552 \emph{*ʔt₁uuŋ}
\end{itemize}

The connection to MKCD 552 \emph{*ʔt₁uuŋ} (already made by Shorto) is
questionable.

\subparagraph{\texorpdfstring{\emph{*soŋ} `buy/sell (v)'
(\#0060-2)}{*soŋ buy/sell (v) (\#0060-2)}}\label{soux14b-buysell-v-0060-2}

oŋ, FR, ---, ---, suŋ, BDBH.2635, soŋ, GZ65.370, soŋ, PKED.p185, soŋ,
PJDW.p278, so, PGEG.p42, ---, ---, ---, ---, ---, ---, ―, ---, ---, ---,
*soŋ, buy/sell (vt), \#0060, K209, ,

\begin{itemize}
\tightlist
\item
  Pinnow 1959: K209 / MKCD: ―
\end{itemize}

Pinnow (1959, p.~224) connects Pal(aung) \emph{jʌːŋ}, \emph{jɔːŋ} `to
sell' and Mon \emph{swʌ̃} `to sell'.

\paragraph{\texorpdfstring{\emph{*o₃}}{*o₃}}\label{o-2}

\begin{longtable}[]{@{}lllllllllllll@{}}
\toprule
Gorum & Sora & Remo & Gutob & Kharia & Juang & Gtaʔ & Santali & Mundari
& Ho & Korwa & Korku & Set\tabularnewline
\midrule
\endhead
--- & --- & o & o & o & ɔ & we & o̠ & o & o & oe & o &
0051-2\tabularnewline
o & o: & (o) & o & (ɔ) & --- & oe & o̠ & o & o & o & u &
0071-2\tabularnewline
(a) & (a) & i & o & o & o & we & o & oe & o:e & oe & o: &
0094-2\tabularnewline
\bottomrule
\end{longtable}

Proto-Munda \emph{*o} in context of a palatal coda
(\emph{*oˀc}/\emph{*oj}; not attested so far \emph{*oɲ}). Some
interesting -- if not crucial -- unexplained variation are the occurence
of Remo /i/ in \#0094-2 as oppoed to the expected Remo and Gutob /o/ as
well as the different distribution of the coarticulatory
diphthongization /oe/ in Kherwarian.

\subparagraph{\texorpdfstring{\emph{*goˀc/goj} `die (v)'
(\#0051-2)}{*goˀc/goj die (v) (\#0051-2)}}\label{goux2c0cgoj-die-v-0051-2}

(kiˀd), FR, (kajed), RSED.p133, goĭ, BDBH.975, goj, ZG65.139, goˀj,
PKED.p63, gɔɟ, PJDW.p197, gweʔ, PGEG.p23, go̠c', CDES.p51, goj, BMED.p61,
goj, DHED.p115, goej, BAHL.p46, go, NKEV.p306, *goj, die (v), \#0051,
K67 , ,

\begin{itemize}
\tightlist
\item
  Pinnow 1959: K67 / MKCD: ---
\end{itemize}

Connect MKCD 1543 \emph{*ghuuy}; \emph{*ghuəy} `spirit, soul' or less
likely MKCD 805 \emph{*guc}; \emph{*guc} `to burn'?

\subparagraph{\texorpdfstring{\emph{*roj}/\emph{*roˀk} `fly'
(\#0071-2)}{*roj/*roˀk fly (\#0071-2)}}\label{rojroux2c0k-fly-0071-2}

aroj, FR, əro:j, RSED.p14, (ayoŋ/ayuŋ), BDBH.39, uroj, GGEG.p93,
(kɔnɖɔi), HLKS.K356, ---, ---, nɖroe, PGEG.p36, ro̠, CDES.p76, roko,
BMED.p161, roko, DHED.p291, roʔo, DSKW.19600, ruku, NKEV.p335, *roj,
fly, \#0071, K356, 1534,

\begin{itemize}
\tightlist
\item
  Pinnow 1959: K356 / MKCD: 1534 Pre-Proto-Mon-Khmer \emph{*ru{[}wa{]}y}
  \textgreater{} \emph{*ruy}; \emph{*ruuy}; \emph{*ruəy};
  Pre-Proto-Mon-Khmer \emph{*ruhay}
\end{itemize}

Gtaʔ /nɖroe/ derives from pre-Gtaʔ \emph{*n(ɖ)roj}

\subparagraph{\texorpdfstring{\emph{*roˀc} `squeeze/milk (v)'
(\#0094-2)}{*roˀc squeeze/milk (v) (\#0094-2)}}\label{roux2c0c-squeezemilk-v-0094-2}

(ra'd), FR, (rad), RSED.p226, riʔ, BDBH.2276, roj, DSGU\#2071, roˀj,
PKED.p170, roɟ, PJDW.p268, rweʔ, PGEG.p41, roco, BSDV5.p98, roeʔ,
EMV12.p3628, ro:eʔ, DHED.p290, roej, DSKW@19520, ro(:)c, NKEV.p335,
*roˀc, squeeze/milk (v), \#0094, V381, 1061,

\begin{itemize}
\tightlist
\item
  Pinnow 1959: V381 / MKCD: 1061 \emph{*ruut}; \emph{*ruət};
  \emph{*rət}; \emph{*rat}; \emph{*rit}; \emph{*riit}; \emph{*riət}
\end{itemize}

\paragraph{\texorpdfstring{incomplete
\emph{*o}}{incomplete *o}}\label{incomplete-o}

\begin{longtable}[]{@{}lllllllllllll@{}}
\toprule
Gorum & Sora & Remo & Gutob & Kharia & Juang & Gtaʔ & Santali & Mundari
& Ho & Korwa & Korku & Set\tabularnewline
\midrule
\endhead
∅ & ∅ & ∅ & ∅ & o & o & ∅ & o̠ & ∅ & ∅ & ∅ & --- & 0011-2\tabularnewline
\bottomrule
\end{longtable}

\subparagraph{\texorpdfstring{\emph{*b(oKʰ)Vˀp} `head'
(\#0011-2)}{*b(oKʰ)Vˀp head (\#0011-2)}}\label{bokux2b0vux2c0p-head-0011-2}

baˀb, FR, bo:ˀb, RSED.p60, bob, BDBH.2007, bob, GZ63.50, bokoˀb,
PKED.p24, bokob, PJDW.p169, bhaʔ, PGEG.p13, bo̠ho̠k', CDES.p90, bo,
BMED.p24, bo:ʔ, DHED.p40, boʔ, BAHL.p113, ---, ---, , head, \#0011,
V206, 361, 38

\begin{itemize}
\tightlist
\item
  Pinnow 1959: V206 / MKCD: 361 \emph{*{[}b{]}uuk}
\end{itemize}

The reflexes of V₁ in \emph{*b(oKʰ)xˀp} `head' (\#0011) -- Kharia /o/,
Juang /o/,Santali /o̠/ -- are too incomplete to assign the set to any
correspondence set, unequivocally. \#0011-2 is consistent with
\emph{*o₁} and \emph{*o₂}.

\subsubsection{\texorpdfstring{Proto-Munda
\emph{*ɨ}}{Proto-Munda *ɨ}}\label{proto-munda-ux268}

\paragraph{\texorpdfstring{\emph{*ɨ₁}}{*ɨ₁}}\label{ux268}

\begin{longtable}[]{@{}lllllllllllll@{}}
\toprule
Gorum & Sora & Remo & Gutob & Kharia & Juang & Gtaʔ & Santali & Mundari
& Ho & Korwa & Korku & Set\tabularnewline
\midrule
\endhead
u & ü/i & i & i & i & i & i & i & i & i & -- & i & 0045-2\tabularnewline
\bottomrule
\end{longtable}

\subparagraph{\texorpdfstring{\emph{*bɨˀt} `sow (v)'
(\#0045-2)}{*bɨˀt sow (v) (\#0045-2)}}\label{bux268ux2c0t-sow-v-0045-2}

buˀd, FR, büd/bid, RSED.p63, biʔ, BDBH.1898, biɽ, GZ63.67, biˀɖ,
PKED.p20, bir, PJDW.p167, big, PGEG.p11, bit', CDES.p142, bid',
BMED.p22, biɖ, DHED.p35, , , biʈ, NKEV.p294, *bɨˀt, sow (v), \#0045,
V285, ,

\begin{itemize}
\tightlist
\item
  Pinnow 1959: V285 / MKCD: ---
\end{itemize}

\subsubsection{\texorpdfstring{Proto-Munda
\emph{*ə}}{Proto-Munda *ə}}\label{proto-munda-ux259}

\begin{itemize}
\tightlist
\item
  \emph{*ə₁} some unexplained variation in \emph{*ə₁} (Sora, Kharia)
\end{itemize}

\emph{*ə₂} has been moved to \emph{*ᵊ}

\begin{longtable}[]{@{}lllllllllllll@{}}
\toprule
Gorum & Sora & Remo & Gutob & Kharia & Juang & Gtaʔ & Santali & Mundari
& Ho & Korwa & Korku &\tabularnewline
\midrule
\endhead
a & a(ə/o:0 & o & o & o(a) & o & wa & e̠ & e & e & e & -- &
\emph{*ə₁}\tabularnewline
e & e & i/e & e & e & -- & i & e & i & i & e & i &
\emph{*ə₂}\tabularnewline
\bottomrule
\end{longtable}

\paragraph{\texorpdfstring{\emph{*ə₁}}{*ə₁}}\label{ux259}

\begin{longtable}[]{@{}lllllllllllll@{}}
\toprule
Gorum & Sora & Remo & Gutob & Kharia & Juang & Gtaʔ & Santali & Mundari
& Ho & Korwa & Korku & Set\tabularnewline
\midrule
\endhead
a & \textbf{o:/a} & o & o & o & ɔ & ua & e̠ & e & e & e & e &
0012-2\tabularnewline
a & a & o & o & \textbf{a} & --- & wa & e̠ & e & e & e & e &
0013-2\tabularnewline
--- & --- & o & o & o & o & wa & e̠ & e & e & e & e &
0019-4\tabularnewline
a & a & --- & --- & o & --- & --- & -- & e & e & -- & -- &
0020-2\tabularnewline
a & a & o & o & o & o & wa & -- & -- & -- & -- & -- &
0055-3\tabularnewline
a & -- & -- & -- & o & --- & -- & e̠ & e & e & e & e &
0065-1\tabularnewline
a & \textbf{ə} & o & o & o & o & (u)a & e̠ & e & e & e & -- &
0021-4\tabularnewline
a & a & o & o & --- & --- & wa & -- & -- & -- & -- & -- &
0085-2\tabularnewline
a & a & o & o & --- & --- & oa & e̠ & e & e & -- & -- &
0092-2\tabularnewline
\bottomrule
\end{longtable}

\emph{*ə} is a grouping of four closely related correpondece sets, all
defined by Gtaʔ diphthong /ua/ (/wa/) and North Munda /e/ (Santali /e̠/).
The variation is restricted to Sora /o:/ (v1: \#0012-2)
\textasciitilde{} /a/ (v2 and v3) \textasciitilde{} /ə/ (v4) and Kharia
/o/ (v1, v3, and v4) and /a/ (v2). The consistency of Gtaʔ and North
munda, /a/ ing Gorum and the consistency of /o/ in Remo, Gutob, Juang
and with one exception Kharia, suggest that these sets are indeed
reflexes of a single proto-phoneme. The nature of this proto-phoneme
remains unclear.

MKCD 1045\_ *mat\_ for `eye' (*məˀt\_ `eye' \emph{*ə\_v1) suggest
proto-Munda }*a\_, but defining the contexts that allow us to link this
group of sets continuing proto-Munda \emph{*a} are unclear.

MKCD 972 \emph{*sguut}; \emph{*{[}s{]}gət}; \emph{*sgat} `cut (v)'
(\emph{*gVˀt} `cut (v)' v2) also allows for proto-Munda \emph{*a}, but
also add \emph{*u} and \emph{*ə} inti the list of likely candidates.
MKCD 1723 \emph{*j{[}n{]}ŋəl} would further strengthen the case for
proto-Munda \emph{*ə}.

The reflexes especially with no apparent motivation for the front vowel
in North Munda makes a proto-phoneme, other than \emph{*a} more likely
and

Some variation: Sora o:\textasciitilde{}a seems real, Juang
ɔ\textasciitilde{}o artefact of my inconsistent treatment of PJDW, Gtaʔ
ua\textasciitilde{}wa seems to be the artefact of inconsistent
represeantation of the diphthong in PGEG.

\subparagraph{\texorpdfstring{\emph{*məˀt} `eye' (\#0012-2)
\_*ə₁\_v1}{*məˀt eye (\#0012-2) \_*ə₁\_v1}}\label{mux259ux2c0t-eye-0012-2-_ux259_v1}

maˀd, FR, mo:ˀd/mad, RSED.p168, moʔ, BDBH.220, moʔ, AG08.p642, moˀɖ,
PKED.p195, ɛmɔɖ, PJDW.p191, muaʔ, PGEG.p34, mẽ̠t', CDES.p67, med',
BMED.p117, meɖ, DHED.p228, meɖ, BAHL.p120, med, ZKPM.p48, məˀt, eye,
\#0012, V250, 1045, 40

\begin{itemize}
\tightlist
\item
  Pinnow 1959: V250 / MKCD: 1045 *mat
\end{itemize}

\subparagraph{\texorpdfstring{\emph{*gəˀt} `cut (v)' (\#0013-2)
\_*ə₁\_v2}{*gəˀt cut (v) (\#0013-2) \_*ə₁\_v2}}\label{gux259ux2c0t-cut-v-0013-2-_ux259_v2}

gaˀd, FR, gad, RSED.p93, goʔ, BDBH.1018, goʔ, AG08.p669, gaˀɖ, PKED.p60,
, , gwaʔ, PGEG.p21, ge̠t', CDES.p44, ged', EMV5.1411, geɖ, DHED.p111,
geɖ, BAHL.p46, geʈ, NKEV.p306, gəˀt, cut (v), \#0013, V334, 972,

\begin{itemize}
\tightlist
\item
  Pinnow 1959: V334 / MKCD: MKCD 972 \emph{*sguut}; \emph{*{[}s{]}gət};
  \emph{*sgat}
\end{itemize}

\subparagraph{\texorpdfstring{\emph{*bVrəl} `raw' V₂ (\#0019-4)
\_*ə₁\_v1/3/4}{*bVrəl raw V₂ (\#0019-4) \_*ə₁\_v1/3/4}}\label{bvrux259l-raw-v-0019-4-_ux259_v134}

---, ---, ---, ---, buro, BDBH.1937, burol, GZ65.74, borol, PKED.p25,
boron, PJDW.p171, brwa, PGEG.p14, be̠re̠l, CDES.p211, berel, BMED.p21,
berel, HOGV.p185, berel, BAHL.p111, boboɽ, NKEV.p294, *bxrəl, raw,
\#0019, V253, ,

\begin{itemize}
\tightlist
\item
  Pinnow 1959: V253 / MKCD: ---
\end{itemize}

\subparagraph{\texorpdfstring{\emph{*səreŋ} `stone' V₁ (\#0020-2)
\_*ə₁\_v3}{*səreŋ stone V₁ (\#0020-2) \_*ə₁\_v3}}\label{sux259reux14b-stone-v-0020-2-_ux259_v3}

areŋ, FR, areŋ, RSED.p39, ---, ---, ---, ---, soreŋ, PKED.p187, ---,
---, ---, ---, ---, ---, sereŋ, BMED.p172, sereɲ, HOGV.p175, ---, ---,
---, ---, *səreŋ, stone, \#0020, V183, ,

\begin{itemize}
\tightlist
\item
  Pinnow 1959: V183 / MKCD: ---
\end{itemize}

\subparagraph{\texorpdfstring{\emph{*xsər} `dry' V₂ (\#0055-3)
\_*ə₁\_v3}{*xsər dry V₂ (\#0055-3) \_*ə₁\_v3}}\label{xsux259r-dry-v-0055-3-_ux259_v3}

asar, FR, asar, RSED.p42, nsor, BDBH.1657, usor, AG08.p650, kosor,
PKED.p155, kosor, PJDW.p229, nswar, PGEG.p37, ---, ---, ---, ---, ---,
---, ---, ---, ---, ---, *xsxr, dry, \#0055, V260, 160,

\begin{itemize}
\tightlist
\item
  Pinnow 1959: V183 / MKCD: 160 \emph{*rɔʔ}; \emph{*rɔs}, ( \emph{*rɔs
  rɔs} \textgreater{}?) \emph{*srɔs}
\end{itemize}

\subparagraph{\texorpdfstring{\emph{*əsel} `white' V₁ (\#0065-1)
\_*ə₁\_v1/3/4}{*əsel white V₁ (\#0065-1) \_*ə₁\_v1/3/4}}\label{ux259sel-white-v-0065-1-_ux259_v134}

asel, FR, ---, ---, ---, ---, ---, ---, osel, PKED.p216, ---, ---, ---,
---, e̠se̠l, BSDV2.p343, esel, BMED.p56, esel, DHED.p102, hesel,
BAHL.p149, esel, HLKS.V255, *əsel, white, \#0065, V255, ,

\begin{itemize}
\tightlist
\item
  Pinnow 1959: V255 / MKCD: ---
\end{itemize}

\subparagraph{\texorpdfstring{\emph{*sVŋəl} `fuel' V₂ (\#0021-4)
\_*ə₁\_v4}{*sVŋəl fuel V₂ (\#0021-4) \_*ə₁\_v4}}\label{svux14bux259l-fuel-v-0021-4-_ux259_v4}

aŋal, FR, aŋəl, RSED.p37, suŋo, BDBH.2638, suõl, GZ63.216, soŋgol,
PKED.p186, sɛŋon, PJDW.p276, sua, PGEG.p43, se̠ŋge̠l, CDES.p73, seŋgel,
BMED.p172, seŋgel, HOGV.p158, seNgel, BAHL.p137, ---, ---, *sxŋxl, fuel,
\#0021, V252, 1723,

\begin{itemize}
\tightlist
\item
  Pinnow 1959: V252 / MKCD 1723 \emph{*j{[}n{]}ŋəl}
\end{itemize}

If Gtaʔ /sua/ is to be interpreted as a /ua/ parallel to
/ua/\textasciitilde{}/wa? in the set above, V₂ of \emph{*sVŋVl} `fuel'
belongs to the set \emph{*ə₁} above only deviating reflex is Sora /ə/.

\emph{sua}/\emph{su.a} \textless{} \emph{*suŋal} \textless{}
\emph{*sVŋVl} with l-loss and intervocalic ŋ-loss or \emph{sua}
\textless{} \emph{*sŋual} \textless{} \emph{*sVŋVl} with intervocalic
ŋ-loss, V₁loss, and l-loss.

\subparagraph{\texorpdfstring{\emph{*pəˀt}/\emph{*par(om)} `cross (v)'
V₁
(\#0085-2)}{*pəˀt/*par(om) cross (v) V₁ (\#0085-2)}}\label{pux259ux2c0tparom-cross-v-v-0085-2}

pa'd, FR, pad, RSED.p200, poʔ, BDBH.1793, poɖ, DSGU\#18931, paro(m),
PKED.p155, (pakea), DSJU\#25131, pwaʔ, PGEG.p39, par, CSED.p474, pa:rom,
BMED.p144, parom, DHED.p262, parom, BAHL.p97, pa:r, NKEV.p331, *pəˀt,
cross (v), \#0085, , ,

While the reflexes in Gorum, Sora, Remo, Gutob, and Gtaʔ are consistent
with a reconstruction of proto-Munda \emph{*pəˀt}. The attested vowels
are consistent with the correspondence sets \emph{*ə\_v2 and }*ə\_v3.

\begin{longtable}[]{@{}llllllllllll@{}}
\toprule
Gorum & Sora & Remo & Gutob & Kharia & Juang & Gtaʔ & Santali & Mundari
& Ho & Korwa & Korku\tabularnewline
\midrule
\endhead
a & a & o & o & --- & --- & wa & -- & -- & -- & -- & --\tabularnewline
\bottomrule
\end{longtable}

The onset and final consonants of these forms are also consistent. So
that for these forms proto-Munda \emph{*pəˀt} can be posited,
confidently. If the North Munda and Kharia forms \emph{pa(:)r(om)} are
added, the reflexes do not fit any known vowel set and the final rhotics
are also inconsistent with any regular reflexes. This suggests that we
have to assume a second etymon \emph{*par(om)} `cross (v)', probably of
Indo-Aryan origin.

\subparagraph{\texorpdfstring{\emph{*bəˀt} `contain/block (v)'
(\#0092-2)}{*bəˀt contain/block (v) (\#0092-2)}}\label{bux259ux2c0t-containblock-v-0092-2}

baˀd ,FR ,bad ,RSED.p47 ,boʔ ,BDBH.2027 ,bod ,Z1965.59 ,--- ,--- ,---
,--- ,boaʔ ,PGEG.p11 ,be̠t' ,BSDV1.p275 ,bed' ,BMED.p21 ,beɖ ,DHED.p33
,--- ,--- ,--- ,--- ,--- ,contain/block (v) ,\#0092 ,--- , 1032,

\begin{itemize}
\tightlist
\item
  Pinnow 1959: --- / MKCD 1032 \emph{*bat}; \emph{*buət}
\end{itemize}

\subsubsection{\texorpdfstring{Superset: \emph{*ᵊ} epenthetic
schwa}{Superset: *ᵊ epenthetic schwa}}\label{superset-ux1d4a-epenthetic-schwa}

\paragraph{\texorpdfstring{\emph{ə₂} Superset: \emph{*ᵊ} epenthetic
schwa}{ə₂ Superset: *ᵊ epenthetic schwa}}\label{ux259-superset-ux1d4a-epenthetic-schwa}

\begin{longtable}[]{@{}lllllllllllll@{}}
\toprule
Gorum & Sora & Remo & Gutob & Kharia & Juang & Gtaʔ & Santali & Mundari
& Ho & Korwa & Korku & Set\tabularnewline
\midrule
\endhead
e & e & e & -- & e & -- & i & e & i & i & e & -- & 0007-2\tabularnewline
-- & e & i/e & e & -- & -- & i & e & i & i & -- & i &
0069-2\tabularnewline
\bottomrule
\end{longtable}

close to reflexes of \emph{*e}, but rather different to reflexes of ə₁

\subparagraph{\texorpdfstring{\emph{*dəraŋ} `horn' V₁
(\#0007-2)}{*dəraŋ horn V₁ (\#0007-2)}}\label{dux259raux14b-horn-v-0007-2}

ɖeraŋ, FR, deraŋ, RSED.p78, deruŋ, BDBH.1266, ---, ---, ɖereŋ, PKED.p44,
---, ---, ɖiraŋ, PGEG.p17, dereɲ, CDSE.p171, diriŋ, BMED.p49, diriɲ,
HOGV.p162, dereŋ, BAHL.p89, ---, ---, dəraŋ, horn, \#0007, V347, 699, 34

\begin{itemize}
\tightlist
\item
  Pinnow 1959: V347 / MKCD 699 \emph{*d₂raŋ}
\end{itemize}

\subparagraph{\texorpdfstring{\emph{*ɟəlu₅} `meat' V₂
(\#0069-2)}{*ɟəlu₅ meat V₂ (\#0069-2)}}\label{ux25fux259lu-meat-v-0069-2}

---, ---, ɟelu:, RSED.p123, sili/seli, BDBH.2599/2731, seli, AG08.p674,
―, ---, ---, ---, cili, PGEG.p15, jel, CDES.p120, jilu, BMED.p83, jilu,
DHED.p165, ---, ---, jilu, NKEV.p311, *ɟəlu₅, meat, \#0069, V228, ,

\begin{itemize}
\tightlist
\item
  Pinnow 1959: V228 / MKCD: ---
\end{itemize}

A possibly connected MKCD etymon is MKCD 204 \emph{*{[}c{]}nlu{[}u{]}ʔ}
`edible grub' only attested in Bahnaric.

\paragraph{\texorpdfstring{VS-005 \emph{*u:i} Superset: \emph{*ᵊ}
epenthetic
schwa}{VS-005 *u:i Superset: *ᵊ epenthetic schwa}}\label{vs-005-ui-superset-ux1d4a-epenthetic-schwa}

\begin{longtable}[]{@{}lllllllllllll@{}}
\toprule
Gorum & Sora & Remo & Gutob & Kharia & Juang & Gtaʔ & Santali & Mundari
& Ho & Korwa & Korku & Set\tabularnewline
\midrule
\endhead
u & i & u & i & i & i & u & u & u & u & u: & u & 0004-2\tabularnewline
\bottomrule
\end{longtable}

\subparagraph{\texorpdfstring{\emph{*kᵊla} `tiger' V₂
(\#0004-2)}{*kᵊla tiger V₂ (\#0004-2)}}\label{kux1d4ala-tiger-v-0004-2}

kulaʔ, FR, kina:, RSED.p140, ŋku, MVol.p733, gikil, AG08.p651, kiɽoʔ,
PKED.p102, kiɭog, PJDW.p224, nku, PGEG.p36, kul, CDES.p201, kula:,
BMED.p98, kula, HOGV.p183, ku:l, BAHL.p33, kula, NKEV.p319,
\emph{*kᵊla}, tiger, \#0004, V281, 197,

\begin{itemize}
\tightlist
\item
  Pinnow 1959: V281 / MKCD: 197 \emph{*klaʔ}
\end{itemize}

Reflexes and in particular the corresponding etymon in MKCD suggest that
\#0004-2 are reflexes of a cluster-splitting epenthetic schwa.

\paragraph{\texorpdfstring{VS-006 \emph{*a:u:e:i} Superset: \emph{*ᵊ}
epenthetic
schwa}{VS-006 *a:u:e:i Superset: *ᵊ epenthetic schwa}}\label{vs-006-auei-superset-ux1d4a-epenthetic-schwa}

\begin{longtable}[]{@{}lllllllllllll@{}}
\toprule
Gorum & Sora & Remo & Gutob & Kharia & Juang & Gtaʔ & Santali & Mundari
& Ho & Korwa & Korku & Set\tabularnewline
\midrule
\endhead
a & a & u & --- & e & --- & a & e & i & i & e & --- &
0007-4\tabularnewline
--- & --- & u & u & e & --- & --- & e & i & i & e & i &
0061-2\tabularnewline
a & ə & u & u & e & i & ∅ & e & i & i & e & --- & 0084-2\tabularnewline
\bottomrule
\end{longtable}

This set is characterised by Pinnow as ``Lautgesetzlich Kh ɛ (e), Sa e,
Mu i, nur So abweichend ɑ statt e, daneben teilweise auch e.'' Pinnow
(1959, p.~163)

Pinnow (1959, p.~164) says ``Der alte Vokal \emph{*e} oder *ɛ\_ im UM,
aus \emph{*ɛ},\emph{*i} im UA. Die z.T. lautlic sehr stark abweichenden
Formen sind dennoh nicht zu trennen.''

\subparagraph{\texorpdfstring{\emph{*dərv₍₆₎ŋ} `horn' V₂
(\#0007-4)}{*dərv₍₆₎ŋ horn V₂ (\#0007-4)}}\label{dux259rvux14b-horn-v-0007-4}

ɖeraŋ, FR, deraŋ, RSED.p78, deruŋ, BDBH.1266, ---, ---, ɖereŋ, PKED.p44,
---, ---, ɖiraŋ, PGEG.p17, dereɲ, CDSE.p171, diriŋ, BMED.p49, diriɲ,
HOGV.p162, dereŋ, BAHL.p89, ---, ---, *dərv₍₆₎ŋ, horn, \#0007, V347,
699, 34

\begin{itemize}
\tightlist
\item
  Pinnow 1959: V347 UM: \emph{*e},\emph{*ɛ}/ MKCD 699 \emph{*d₂raŋ}
\end{itemize}

The correspondence set \#0007-4 for V₂ in \emph{*dərv₍₆₎ŋ} `horn' is
unique and not well understood. If Shorto's reconstruction \emph{*d₂raŋ}
is correct and were continued in proto-Munda as \emph{*dᵊraŋ} (or
\emph{*dəraŋ}), regular correspondences consistent with \emph{*a₂} (pM
\emph{*a} with velar coda) would be expected. However, the
correspondences are inconsistent with \emph{*a₂} and do not connect well
to any established proto-phoneme.

\subparagraph{\texorpdfstring{\emph{*sv₍₆₎bVl} `sweet' V₁
(\#0061-2)}{*sv₍₆₎bVl sweet V₁ (\#0061-2)}}\label{svbvl-sweet-v-0061-2}

---, ---, ---, ---, subu, BDBH.2665, subul, AG08.p651, sebol, PKED.p180,
---, ---, ―, ---, sebel, CDES.p1194, sibil, EMV13.p3943, sibil,
DHED.p316, sebel, DSKW.@21820, simil, NKEV.p338, *sv₍₆₎bxl, sweet,
\#0061, V257, ,

\begin{itemize}
\tightlist
\item
  Pinnow 1959: V257 / MKCD ---
\end{itemize}

\subparagraph{\texorpdfstring{\emph{*sv₍₆₎lVˀp} `gazelle' V₁
(\#0084-2)}{*sv₍₆₎lVˀp gazelle V₁ (\#0084-2)}}\label{svlvux2c0p-gazelle-v-0084-2}

alu'b, FR, əle:b, RSED.p7, sulup, BDBH.2688, sulub, GGEG.p116, selhob,
PKED.p180, silib, PJDW.p278, sloʔ, PGEG.p43, selep', CSED.p571, silib,
BMED.p173, silib, DHED.p317, seleb, DSKW@21960, ---, ---, *sxlxˀp,
gazelle, \#0084, V233, ,

\begin{itemize}
\tightlist
\item
  Pinnow 1959: V233 / MKCD ---
\end{itemize}

Gtaʔ ∅ is unproblematic. Sora /ə/ is a variation contrastin with /a/ in
\#0007-4.

\subsubsection{Unassigned correspondent
sets}\label{unassigned-correspondent-sets}

\paragraph{\texorpdfstring{VS-001 \emph{*o:u₁} Superset:
\emph{*O}}{VS-001 *o:u₁ Superset: *O}}\label{vs-001-ou-superset-o}

\begin{longtable}[]{@{}llllllllllll@{}}
\toprule
Gorum & Sora & Remo & Gutob & Kharia & Juang & Gtaʔ & Santali & Mundari
& Ho & Korwa & Korku\tabularnewline
\midrule
\endhead
o & o: & u & o & u & o & u & -- & -- & -- & -- & --\tabularnewline
\bottomrule
\end{longtable}

\subparagraph{\texorpdfstring{\emph{*Olaaˀ}/\emph{*Ola(ˀk)} `leaf' V₁
(\#0035-1)}{*Olaaˀ/*Ola(ˀk) leaf V₁ (\#0035-1)}}\label{olaaux2c0olaux2c0k-leaf-v-0035-1}

olaʔ, FR, o:la:, RSED.p192, ulak', BDBH.169, olag, AG08.p633, ulaʔ,
PKED.p298, olag, PJDW.p254, uliaʔ, PGEG.p124, palha, CDES.p111,
pa:lha:o, BMED.p142, pala, DHED.p259, (sakam), BAHL.pdfp129, pa:la,
NKEV.p331, , leaf, \#0035, V50, 230,

\begin{itemize}
\tightlist
\item
  Pinnow 1959: V50 / MKCD: 230 \emph{*slaʔ}
\end{itemize}

If, as assumed here, the Southern forms with initial vowel and the forms
of North Munda with an inital \emph{p} belong to distinct etyma, the set
is very close to \emph{*o₁}.

Hypothesis: V₁\_*u\_ in the context of V₂\_*a\_? Alternatively, V₁\_*o\_
in the context of V₂\_*a\_?

\paragraph{\texorpdfstring{VS-002
\emph{*a:u:æ}}{VS-002 *a:u:æ}}\label{vs-002-auuxe6}

\begin{longtable}[]{@{}llllllllllll@{}}
\toprule
Gorum & Sora & Remo & Gutob & Kharia & Juang & Gtaʔ & Santali & Mundari
& Ho & Korwa & Korku\tabularnewline
\midrule
\endhead
a & a & (u) & u/a & o/u & ɛ & æ & a & a: & a & a & a\tabularnewline
\bottomrule
\end{longtable}

\subparagraph{\texorpdfstring{\emph{*sVmaŋ} `forehead/front' V₁
(\#0038-2)}{*sVmaŋ forehead/front V₁ (\#0038-2)}}\label{svmaux14b-foreheadfront-v-0038-2}

amaŋ, FR, ammaŋ, RSED.p31, gutumoŋ, BDBH.885, sumoŋ/amuŋ, GZ65.21,
somoŋ/somo/sumaŋ, PKED.p185, ɛmɔŋ, PJDW.p191, ssæ, PGEG.p44, samaŋ,
CDES.p79, sa:ma:ŋ, BMED.p167, sanamaŋ, HOGV.p159, samaŋ, BAHL.pdfp130,
samma, NKEV.p336, , forehead/front, \#0038, V269, ,

\begin{itemize}
\tightlist
\item
  Pinnow 1959: V269 / MKCD: ---
\end{itemize}

The reflexes in V₁ position are irregular and the material preceding
\emph{*maŋ} is structurally divers. Especially, Remo \emph{gutumoŋ}
(\emph{gutu-}), and the lack of initial /s/ in Gutob \emph{amuŋ} and
Juang \emph{ɛmɔŋ} cannot be explained in the ssame way as the s-loss in
Sora and Gorum can be explained.

Correspondence between Juang /ɛ/ and Gtaʔ /æ/ is also attested in the
\emph{*i₂} sets of reflexes, but the reflexes in other languages are
decidedly distinct.

\paragraph{\texorpdfstring{VS-003
\emph{*i:u:o}}{VS-003 *i:u:o}}\label{vs-003-iuo}

\begin{longtable}[]{@{}lllllllllllll@{}}
\toprule
Gorum & Sora & Remo & Gutob & Kharia & Juang & Gtaʔ & Santali & Mundari
& Ho & Korwa & Korku & Set\tabularnewline
\midrule
\endhead
(u) & i & u & ui & u & u & o & i & i & i & i: & i &
0041-2\tabularnewline
u & u & -- & -- & u & -- & o & i & i & i & -- & i &
0062-2\tabularnewline
\bottomrule
\end{longtable}

Comparison with \emph{*tuɲ} `shoot (v)' (\#0027-2):

\begin{longtable}[]{@{}lllllllllllll@{}}
\toprule
Gorum & Sora & Remo & Gutob & Kharia & Juang & Gtaʔ & Santali & Mundari
& Ho & Korwa & Korku & etymon\tabularnewline
\midrule
\endhead
u & u & -- & -- & u & -- & o & i & i & i & -- & i &
0062-2\tabularnewline
(u) & i & u & ui & u & u & o & i & i & i & i: & i &
0041-2\tabularnewline
i & u & i & i & u & u & wi & u & ui & u & ? & u & 0027-2\tabularnewline
\bottomrule
\end{longtable}

\subparagraph{\texorpdfstring{\emph{*bVɲˀ(*bVVˀɲ?)} `snake'
(\#0041-2)}{*bVɲˀ(*bVVˀɲ?) snake (\#0041-2)}}\label{bvux272ux2c0bvvux2c0ux272-snake-0041-2}

bubuˀd, FR, biɲ/biŋ, RSED.p59, bubuʔ, BDBH.1931, buɽbui, GGEG.p108,
buɲam, PKED.p4, bubuŋ, PJDW.p172, boʔ, PGEG.p12, biɲ, CDES.p179, biŋ,
BMED.p23, biɲ, HOGV.p178, bi:ŋ, BAHL.p108, biɲj, NKEV.p294, , snake,
\#0041, V353, 937,

\begin{itemize}
\tightlist
\item
  Pinnow 1959: V353; ; VW u/i; UM:i / MKCD: 937 \emph{*{[}b{]}saɲʔ}
\end{itemize}

MKCD 937 \emph{*{[}b{]}saɲʔ} less likely MKCD 1921a \emph{*ɓəs}

close back rounded vowels and the close-mid back rounded vowel in Gtaʔ
suggest a rounded vowel followed by a palatal, nasal, and glottalized
coda. The best candidate for this vowel phoneme is /o/, thus
\emph{*boɲˀ} or maybe \emph{*booˀɲ}. Other possibilities are
\emph{*buɲˀ}/\emph{*buuˀɲ} or more problematic \emph{*ɨ} or \emph{*ʉ}.

\subparagraph{\texorpdfstring{\emph{*tVŋ} `kindle (v)'
(\#0062-2)}{*tVŋ kindle (v) (\#0062-2)}}\label{tvux14b-kindle-v-0062-2}

tuŋ, FR, tuŋa:l, RSED.p297, ---, ---, ---, ---, tuŋgal, HLKS.V324, ---,
---, toŋ, PGEG.p42, tiŋgi, BSDV5.p461, tiŋ, BMED.p187, tiɲ, DHED.p353,
---, ---, ʈingi, NKEV.p343, *tVŋ, kindle (v), \#0062, V324, 549,

\begin{itemize}
\tightlist
\item
  Pinnow 1959: V324; VW i/u; UM:i/ MKCD: 549 \emph{*t₁uuŋ}
\end{itemize}

\paragraph{VS-004 a/ə:(i):(o/u):a}\label{vs-004-aux259ioua}

\begin{longtable}[]{@{}lllllllllllll@{}}
\toprule
Gorum & Sora & Remo & Gutob & Kharia & Juang & Gtaʔ & Santali & Mundari
& Ho & Korwa & Korku & Set\tabularnewline
\midrule
\endhead
a & a/ə & ∅ & i & o/u & a & a & a & a: & a & a & a &
0026-2\tabularnewline
a & a/ə & (u) & --- & --- & --- & ∅ & a & a & a & -- & -- &
0064-2\tabularnewline
\bottomrule
\end{longtable}

not clear that this constitutes a single correpondence set.

\subparagraph{\texorpdfstring{\emph{*Kʰv₍₄₎su} `fever/pain' V₁
(\#0026-2)}{*Kʰv₍₄₎su fever/pain V₁ (\#0026-2)}}\label{kux2b0vsu-feverpain-v-0026-2}

asu, FR, asu:/əsu:, RSED.p42, siʔ, BDBH.2610, isi, GGEG.p93, kosu/kusu,
PKED.p107, kasu, PJDW.p220, aʔsu, PGEG.p4, haso, CDES.p135, ha:su,
BMED.p67, hasu, HOGV.p147, hasu:, BAHL.p145, kaSu, NKEV.p315, *Kʰxsu,
fever/pain, \#0026, V247, 44,

\begin{itemize}
\tightlist
\item
  Pinnow 1959: V247 / MKCD: 44 \emph{*{[}c{]}uuʔ}
\end{itemize}

\subparagraph{\texorpdfstring{\emph{*mv₍₄₎raŋ} `big' V₁
(\#0064-2)}{*mv₍₄₎raŋ big V₁ (\#0064-2)}}\label{mvraux14b-big-v-0064-2}

---, ---, maraŋ/məraŋ, RSED.p173/167, munaʔ, BDBH.2121, (moɖo),
AG08.p663, ---, ---, ---, ---, mnaʔ, PGEG.35, maraŋ, CDES.p17, maraŋ,
BMED.p220, maraŋ, DHED.p225, ---, ---, ---, ---, *mxrxŋ, big, \#0064,
K107, ,

\begin{itemize}
\tightlist
\item
  Pinnow 1959: K107 / MKCD: ---
\end{itemize}

Gtaʔ \emph{mnaʔ} and Remo \emph{munaʔ} are irregular reflexes of
\emph{*mv₍₄₎raŋ}, especially the Gtaʔ form \emph{mnaʔ} should be
different, given our current understanding of the phonological
developments, since a velar coda \emph{*aŋ} results in Gtaʔ /ia/. Remo
and Gtaʔ /n/ are also inconsistent as reflexes or either \emph{*r} or
\emph{*ŋ}. Gtaʔ \emph{mnaʔ} and Remo \emph{munaʔ} are consistently
parallel to one another.

\paragraph{\texorpdfstring{VS-007
\emph{*a:e:i:o:u}}{VS-007 *a:e:i:o:u}}\label{vs-007-aeiou}

\begin{longtable}[]{@{}lllllllllllll@{}}
\toprule
Gorum & Sora & Remo & Gutob & Kharia & Juang & Gtaʔ & Santali & Mundari
& Ho & Korwa & Korku & Set\tabularnewline
\midrule
\endhead
i & e: & u & u & u & i & o & a & a & --- & --- & (a) &
0010-2\tabularnewline
\bottomrule
\end{longtable}

\subparagraph{\texorpdfstring{\emph{*ɟv₍₇₎ŋ} `foot'
(\#0010-2)}{*ɟv₍₇₎ŋ foot (\#0010-2)}}\label{ux25fvux14b-foot-0010-2}

zḭŋ, FR, ɟe:ˀŋ, RSED.p123, suŋ, BDBH.1363, suŋ, GZ63.205, juŋ, PKED.p66,
iɟiŋ, PJDW.p208, nco, PGEG.p114, jaŋga, CDES.p76, jaŋga, HLKS.182, ---,
---, ---, ---, (naŋga), NKEV.p327, , foot, \#0010, V365, 538,

\begin{itemize}
\tightlist
\item
  Pinnow 1959: V365 / MKCD 538 \emph{*juŋ}; \emph{*juəŋ}; \emph{*jəŋ};
  \emph{*jəəŋ}
\end{itemize}

unique set with unclear

Pinnow (1959, p.~169) says ``\ldots{}so bleibt der Vokalwechsel des
Wortes für Fuß, Bein ein gänzlich ungelöstes Rätsel der
austroasiatischen Sprachwissenschaft,\ldots{}''

\paragraph{\texorpdfstring{VS-008 \emph{*a:o} Superset:
\emph{*O}}{VS-008 *a:o Superset: *O}}\label{vs-008-ao-superset-o}

\begin{longtable}[]{@{}lllllllllllll@{}}
\toprule
Gorum & Sora & Remo & Gutob & Kharia & Juang & Gtaʔ & Santali & Mundari
& Ho & Korwa & Korku & Set\tabularnewline
\midrule
\endhead
a & o: & o & o & o & o & a & o̠ & o & o: & o & --- &
0011-4\tabularnewline
\bottomrule
\end{longtable}

\subparagraph{\texorpdfstring{\emph{*b(oKʰ)Oˀp} `head'
(\#0011-4)}{*b(oKʰ)Oˀp head (\#0011-4)}}\label{bokux2b0oux2c0p-head-0011-4}

baˀb, FR, bo:ˀb, RSED.p60, bob, BDBH.2007, bob, GZ63.50, bokoˀb,
PKED.p24, bokob, PJDW.p169, bhaʔ, PGEG.p13, bo̠ho̠k', CDES.p90, bo,
BMED.p24, bo:ʔ, DHED.p40, boʔ, BAHL.p113, ---, ---, *b(oKʰ)Oˀp, head,
\#0011, V206, 361, 38

\begin{itemize}
\tightlist
\item
  Pinnow 1959: V206 UM *ɔ/ MKCD: 361 \emph{*{[}b{]}uuk}
\end{itemize}

The reflexes suggest a mid to open central to back vowel for \#0011-4.
The reflexes are unique and show no clear affinity to any particular
other set. The bilabial coda \emph{*ˀp} would suggest \emph{*a₄}, if
VS-008 were a continuation of proto-Munda \emph{*a}. Pinnow (1959,
p.~112) posits \emph{*ɔ} and for the complete word \emph{*bɔkɔp},
\emph{*bɔkɔk}, \emph{*bɔp}, \emph{*bɔk}. Analysing VS-008 as reflexes of
either \emph{*ɔ} or \emph{*o}, maybe with a bilabial coda -- so
\emph{*ɔˀp} or \emph{*oˀp} -- is possible. However, an attestation in
more than one etymon would be desirable.

\paragraph{VS-009}\label{vs-009}

\begin{longtable}[]{@{}lllllllllllll@{}}
\toprule
Gorum & Sora & Remo & Gutob & Kharia & Juang & Gtaʔ & Santali & Mundari
& Ho & Korwa & Korku & Set\tabularnewline
\midrule
\endhead
a & ə & ∅ & u & o & o & ∅ & --- & --- & --- & --- & --- &
0014-1\tabularnewline
\bottomrule
\end{longtable}

\subparagraph{\texorpdfstring{\emph{*v₍₉₎laŋ} `thatch' V₁
(\#0014-1)}{*v₍₉₎laŋ thatch V₁ (\#0014-1)}}\label{vlaux14b-thatch-v-0014-1}

alaŋ, FR, əlaŋ, RSED.p158, lɔŋ, BDBH.2437, uloŋ, AG08.p644, oloŋ,
PKED.p214, oloŋ, PJDW.p254, nlo, PGEG.p36, ---, ---, ---, ---, ---, ---,
---, ---, ---, ---, *v₍₉₎laŋ, thatch, \#0014, V270, 749,

\begin{itemize}
\tightlist
\item
  Pinnow 1959: V270 / MKCD: 749 \emph{*{[}p{]}laŋ}; \emph{*{[}p{]}laiŋ}
\end{itemize}

Incomplete set, due to absence of V₁ in Remo and Gtaʔ and the absence of
this etymon in North Munda. The reflexes suggest a central or bac vowel.
MKCD: 749 \emph{*{[}p{]}laŋ}/\emph{*{[}p{]}laiŋ} would favour epenthetic
\emph{*ə}. (The loss of initial \emph{*p} seems from the current
understanding irregular.)

\paragraph{VS-010}\label{vs-010}

\begin{longtable}[]{@{}lllllllllllll@{}}
\toprule
Gorum & Sora & Remo & Gutob & Kharia & Juang & Gtaʔ & Santali & Mundari
& Ho & Korwa & Korku & Set\tabularnewline
\midrule
\endhead
u & o & u & o & o & i & o & o̠ & o & o & o & o & 0015-2\tabularnewline
\bottomrule
\end{longtable}

\paragraph{\texorpdfstring{\emph{*ɟv₍₁₀₎m} `eat (v)'
(\#0015-2)}{*ɟv₍₁₀₎m eat (v) (\#0015-2)}}\label{ux25fvm-eat-v-0015-2}

zum, FR, ɟom, RSED.p128, sum, BDBH.2667, som, GZ63.212, jom, HLKS.K274,
ɟim, PJDW.p212, coŋ, PGEG.p15, jo̠m, CDES.p60, jom, BMED.p84, jom,
HOGV.p156, jom, BAHL.p63, jom, NKEV.p313, , eat (v), \#0015, V385, 1327,
55

\begin{itemize}
\tightlist
\item
  Pinnow 1959: V385 / MKCD: 1327 \emph{*cuum}; \emph{*cuəm};
  \emph{*cəm}; (\emph{*cim cim} \textgreater{}) \emph{*ncim};
  \emph{*ciəm} (\& \emph{*nciəm}?); \emph{*caim}
\end{itemize}

Comparison with:

\begin{itemize}
\tightlist
\item
  \emph{*o₂}: \emph{*ɟoˀk} `sweep (v)' (\#0031-2);
\item
  \emph{*o₁}: \emph{*ɟoˀt} `wipe (v)' (\#0029-2); MKCD: 994 \emph{*{[}
  {]}jut}; \emph{*{[} {]}juut}
\item
  \emph{*o₁}: \emph{*ɟo(o)ˀ} `fruit; bear fruit (v)' (\#0030-2); MCKD
  ---
\end{itemize}

\begin{longtable}[]{@{}lllllllllllll@{}}
\toprule
Gorum & Sora & Remo & Gutob & Kharia & Juang & Gtaʔ & Santali & Mundari
& Ho & Korwa & Korku & Set\tabularnewline
\midrule
\endhead
\textbf{u} & o & u & o & o & \textbf{i} & o & o̠ & o & o & o & o &
0015-2\tabularnewline
o & o: & u & o & o & \textbf{ɔ} & o & o̠ & o & o & o & \textbf{u} &
0031-2\tabularnewline
o & o & u & o & o & --- & u & o̠ & o & o & o & o & 0029-2\tabularnewline
o & o & u & --- & --- & --- & u & o̠ & o & o: & o & o: &
0030-2\tabularnewline
\bottomrule
\end{longtable}

Maybe proto-Munda \emph{*o} under certain conditions? (\emph{*m} coda,
cannot be \emph{*ɟ} onset)

\paragraph{VS-011}\label{vs-011}

\begin{longtable}[]{@{}lllllllllllll@{}}
\toprule
Gorum & Sora & Remo & Gutob & Kharia & Juang & Gtaʔ & Santali & Mundari
& Ho & Korwa & Korku & Set\tabularnewline
\midrule
\endhead
--- & --- & u & u & o & i & e & e & i & i & i: & i &
0018-4\tabularnewline
\bottomrule
\end{longtable}

The vowels sets V-011 and VS-012 are closely related. The sets
constitution VS-012 -- \#0061-4 and \#0084-4 -- are identical. \#0018-4
(VS-011) differs by Gtaʔ /e/ -- as opposed to VS-012 /o/ -- and Korwa
/i:/ -- as opposed to VS-012 /e/.

\begin{longtable}[]{@{}lllllllllllll@{}}
\toprule
Gorum & Sora & Remo & Gutob & Kharia & Juang & Gtaʔ & Santali & Mundari
& Ho & Korwa & Korku & Set\tabularnewline
\midrule
\endhead
--- & --- & u & u & o & i & \textbf{e} & e & i & i & \textbf{i:} & i &
0018-4\tabularnewline
--- & --- & u & u & o & --- & --- & e & i & i & \textbf{e} & i &
0061-4\tabularnewline
u & e: & u & u & o & i & \textbf{o} & e̠ & i & i & \textbf{e} & --- &
0084-4\tabularnewline
\bottomrule
\end{longtable}

All three sets occur in bisyllabic words and the vowel patterns attested
in these words are informative.

\begin{longtable}[]{@{}lllllllllllll@{}}
\toprule
Gorum & Sora & Remo & Gutob & Kharia & Juang & Gtaʔ & Santali & Mundari
& Ho & Korwa & Korku & Set\tabularnewline
\midrule
\endhead
--- & --- & u-u & u-u & e-o & i-i & ∅-e & e-e & i-i & i-i & i:-i: & i-i
& 0018\tabularnewline
--- & --- & u-u & u-u & e-o & --- & --- & e-e & i-i & i-i & e-e & i-i &
0061\tabularnewline
a-u & ə-e: & u-u & u-u & e-o & i-i & ∅-o & e-e & i-i & i-i & e-e & --- &
0084\tabularnewline
\bottomrule
\end{longtable}

\subparagraph{\texorpdfstring{\emph{*bVlv₍₁₁₎} `ripe' V₂
(\#0018-4)}{*bVlv₍₁₁₎ ripe V₂ (\#0018-4)}}\label{bvlv-ripe-v-0018-4}

---, ---, ---, ---, bulu, BDBH.1591, bulu, AG08.p644, belom, PKED.p19,
bilim, PJDW.p167, ble, PGEG.p13, bele, CDES.p161, bili, BMED.p23, bili,
HOGV.p156, bhi:li:, BAHL.p115, bili, NKEV.p293, *bxlx, ripe, \#0018,
V232, ,

\begin{itemize}
\item
  Pinnow 1959: V232 / MKCD: ---
\item
  MKCD 2080 \emph{*bl{[}ɔ{]}h} `finished'
\item
  MKCD 1878 \emph{*lʔas} `ripe'
\end{itemize}

\paragraph{VS-012}\label{vs-012}

\begin{longtable}[]{@{}lllllllllllll@{}}
\toprule
Gorum & Sora & Remo & Gutob & Kharia & Juang & Gtaʔ & Santali & Mundari
& Ho & Korwa & Korku & Set\tabularnewline
\midrule
\endhead
--- & --- & u & u & o & --- & --- & e & i & i & e & i &
0061-4\tabularnewline
u & e: & u & u & o & i & o & e̠ & i & i & e & --- & 0084-4\tabularnewline
\bottomrule
\end{longtable}

\subparagraph{\texorpdfstring{\emph{*sv₍₆₎bv₍₁₂₎l} `sweet' V₂
(\#0061-4)}{*sv₍₆₎bv₍₁₂₎l sweet V₂ (\#0061-4)}}\label{svbvl-sweet-v-0061-4}

---, ---, ---, ---, subu, BDBH.2665, subul, AG08.p651, sebol, PKED.p180,
---, ---, ―, ---, sebel, CDES.p1194, sibil, EMV13.p3943, sibil,
DHED.p316, sebel, DSKW.@21820, simil, NKEV.p338, *sv₍₆₎bxl, sweet,
\#0061, V257, ,

\begin{itemize}
\tightlist
\item
  Pinnow 1959: V257 / MKCD: ---
\end{itemize}

\subparagraph{\texorpdfstring{\emph{*sv₍₆₎lv₍₁₂₎ˀp} `gazelle' V₂
(\#0084-4)}{*sv₍₆₎lv₍₁₂₎ˀp gazelle V₂ (\#0084-4)}}\label{svlvux2c0p-gazelle-v-0084-4}

alu'b, FR, əle:b, RSED.p7, sulup, BDBH.2688, sulub, GGEG.p116, selhob,
PKED.p180, silib, PJDW.p278, sloʔ, PGEG.p43, selep', CSED.p571, silib,
BMED.p173, silib, DHED.p317, seleb, DSKW@21960, ---, ---, *sxlxˀp,
gazelle, \#0084, V233, ,

\begin{itemize}
\tightlist
\item
  Pinnow 1959: V233 / MKCD ---
\end{itemize}

\subparagraph{VS-013 (=VS-006?)}\label{vs-013-vs-006}

\begin{longtable}[]{@{}lllllllllllll@{}}
\toprule
Gorum & Sora & Remo & Gutob & Kharia & Juang & Gtaʔ & Santali & Mundari
& Ho & Korwa & Korku & Set\tabularnewline
\midrule
\endhead
--- & --- & u & u & e & i & ∅ & e & i & i & i: & i &
0018-2\tabularnewline
\bottomrule
\end{longtable}

\subparagraph{\texorpdfstring{\emph{*bv₍₁₃₎lv₍₁₁₎} `ripe' V₂
(\#0018-4)}{*bv₍₁₃₎lv₍₁₁₎ ripe V₂ (\#0018-4)}}\label{bvlv-ripe-v-0018-4-1}

---, ---, ---, ---, bulu, BDBH.1591, bulu, AG08.p644, belom, PKED.p19,
bilim, PJDW.p167, ble, PGEG.p13, bele, CDES.p161, bili, BMED.p23, bili,
HOGV.p156, bhi:li:, BAHL.p115, bili, NKEV.p293, *bxlx, ripe, \#0018,
V232, ,

\begin{itemize}
\item
  Pinnow 1959: V232 / MKCD: ---
\item
  MKCD 2080 \emph{*bl{[}ɔ{]}h} `finished'
\item
  MKCD 1878 \emph{*lʔas} `ripe'
\end{itemize}

In all likelihood ultimately the reflects of an epenthetic schwa,
\emph{*bᵊlv₍₁₁₎}.

\paragraph{\texorpdfstring{VS-014
(\emph{*ə}?)}{VS-014 (*ə?)}}\label{vs-014-ux259}

\begin{longtable}[]{@{}lllllllllllll@{}}
\toprule
Gorum & Sora & Remo & Gutob & Kharia & Juang & Gtaʔ & Santali & Mundari
& Ho & Korwa & Korku & Set\tabularnewline
\midrule
\endhead
--- & --- & u & u & o & o & ∅ & e̠ & e & e & e & (o) &
0019-2\tabularnewline
a & a & u & u & o & ɛ & ∅ & e̠ & e & e & e & --- & 0021-2\tabularnewline
\bottomrule
\end{longtable}

V₁ of \emph{*bVrəl} `raw' (\#0019-2) and \emph{*sVŋəl} `fuel' (\#0021-2)
are consistent with \emph{*ə₁\_v3, in particular \#0020-2 V₂ of
}*səreŋ\_ `stone' and \#0065-1 V₂ of \emph{*əsel} `white':

\begin{longtable}[]{@{}lllllllllllll@{}}
\toprule
Gorum & Sora & Remo & Gutob & Kharia & Juang & Gtaʔ & Santali & Mundari
& Ho & Korwa & Korku & Set\tabularnewline
\midrule
\endhead
--- & --- & u & u & o & o & ∅ & e̠ & e & e & e & (o) &
0019-2\tabularnewline
a & a & u & u & o & ɛ & ∅ & e̠ & e & e & e & --- & 0021-2\tabularnewline
a & a & --- & --- & o & --- & --- & -- & e & e & -- & -- &
0020-2\tabularnewline
a & -- & -- & -- & o & --- & -- & e̠ & e & e & e & e &
0065-1\tabularnewline
\bottomrule
\end{longtable}

However, all other reflexes of the set \emph{*ə₁} feature Remo and Gutob
/o/, when one of these two languages provides a reflex. Furthermore, V₂
of \emph{*bVrəl} `raw' (\#0019-4) shows distinct reflexes from V₁
(\#0019-1) and is itself consitent with \_*ə₁\_v1/3/4.

\begin{longtable}[]{@{}lllllllllllll@{}}
\toprule
Gorum & Sora & Remo & Gutob & Kharia & Juang & Gtaʔ & Santali & Mundari
& Ho & Korwa & Korku & Set\tabularnewline
\midrule
\endhead
--- & --- & u & u & o & o & ∅ & e̠ & e & e & e & (o) &
0019-2\tabularnewline
--- & --- & o & o & o & o & wa & e̠ & e & e & e & e &
0019-4\tabularnewline
\bottomrule
\end{longtable}

This raises the possibility, that the reflexes of V₁ and V₂ of
\emph{*bVrəl} `raw' are both reflexes of proto-Munda \emph{*ə}, only in
different positions (and with different histories of stress and stress
shift). We could thus posit \emph{*bərəl} or \emph{*bᵊrəl}.

\begin{longtable}[]{@{}lllllllllllll@{}}
\toprule
Gorum & Sora & Remo & Gutob & Kharia & Juang & Gtaʔ & Santali & Mundari
& Ho & Korwa & Korku & Set\tabularnewline
\midrule
\endhead
--- & --- & u-o & u-o & o-o & o-o & ∅-wa & e̠-e̠ & e-e & e-e & e-e & (o-o)
& 0019\tabularnewline
a-a & a-a & u-o & u-o & o-o & ɛ-o & ∅-ua & e̠-e̠ & e-e & e-e & e-e & --- &
0021\tabularnewline
a-a & a-a & ∅-o & u-o & o-o & o-o & ∅-ua & --- & --- & --- & --- & --- &
0055\tabularnewline
a-a & ə-a & ∅-ɔ & u-o & o-o & o-o & ∅-o & --- & --- & --- & --- & --- &
0014\tabularnewline
\bottomrule
\end{longtable}

\emph{*v₍₉₎laŋ} `thatch' V₁ (\#0014-1) VS-009 is very close.

\subparagraph{\texorpdfstring{\emph{*bVrəl} `raw' V₁
(\#0019-2)}{*bVrəl raw V₁ (\#0019-2)}}\label{bvrux259l-raw-v-0019-2}

\begin{longtable}[]{@{}lllllllllllll@{}}
\toprule
Gorum & Sora & Remo & Gutob & Kharia & Juang & Gtaʔ & Santali & Mundari
& Ho & Korwa & Korku & Set\tabularnewline
\midrule
\endhead
--- & --- & u & u & o & o & ∅ & e̠ & e & e & e & (o) &
0019-2\tabularnewline
\bottomrule
\end{longtable}

---, ---, ---, ---, buro, BDBH.1937, burol, GZ65.74, borol, PKED.p25,
boron, PJDW.p171, brwa, PGEG.p14, be̠re̠l, CDES.p211, berel, BMED.p21,
berel, HOGV.p185, berel, BAHL.p111, boboɽ, NKEV.p294, *bxrəl, raw,
\#0019, V253, ,

\begin{itemize}
\tightlist
\item
  Pinnow 1959: V253 / MKCD: ---
\end{itemize}

\subparagraph{\texorpdfstring{\emph{*sVŋəl} `fuel' V₁
(\#0021-2)}{*sVŋəl fuel V₁ (\#0021-2)}}\label{svux14bux259l-fuel-v-0021-2}

\begin{longtable}[]{@{}lllllllllllll@{}}
\toprule
Gorum & Sora & Remo & Gutob & Kharia & Juang & Gtaʔ & Santali & Mundari
& Ho & Korwa & Korku & Set\tabularnewline
\midrule
\endhead
a & a & u & u & o & ɛ & ∅ & e̠ & e & e & e & --- & 0021-2\tabularnewline
\bottomrule
\end{longtable}

aŋal, FR, aŋəl, RSED.p37, suŋo, BDBH.2638, suõl, GZ63.216, soŋgol,
PKED.p186, sɛŋon, PJDW.p276, sua, PGEG.p43, se̠ŋge̠l, CDES.p73, seŋgel,
BMED.p172, seŋgel, HOGV.p158, seNgel, BAHL.p137, ---, ---, *sxŋəl, fuel,
\#0021, V252, 1723,

\begin{itemize}
\tightlist
\item
  Pinnow 1959: V252 / MKCD 1723 \emph{*j{[}n{]}ŋəl}
\end{itemize}

If Gtaʔ /sua/ is to be interpreted as a /ua/ parallel to
/ua/\textasciitilde{}/wa? in the set above, V₂ of \emph{*sVŋVl} `fuel'
belongs to the set \emph{*ə₁} above only deviating reflex is Sora /ə/.

\begin{longtable}[]{@{}lllllllllllll@{}}
\toprule
Gorum & Sora & Remo & Gutob & Kharia & Juang & Gtaʔ & Santali & Mundari
& Ho & Korwa & Korku & Set\tabularnewline
\midrule
\endhead
--- & --- & u-o & u-o & o-o & o-o & ∅-wa & e̠-e̠ & e-e & e-e & e-e & (o-o)
& 0019\tabularnewline
a-a & a-a & u-o & u-o & o-o & ɛ-o & ∅-ua & e̠-e̠ & e-e & e-e & e-e & --- &
0021\tabularnewline
\bottomrule
\end{longtable}

\subparagraph{\texorpdfstring{\emph{*xsər} `dry' V₂
(\#0055-1)}{*xsər dry V₂ (\#0055-1)}}\label{xsux259r-dry-v-0055-1}

asar, FR, asar, RSED.p42, nsor, BDBH.1657, usor, AG08.p650, kosor,
PKED.p155, kosor, PJDW.p229, nswar, PGEG.p37, ---, ---, ---, ---, ---,
---, ---, ---, ---, ---, *xsxr, dry, \#0055, V260, 160,

\begin{itemize}
\tightlist
\item
  Pinnow 1959: V183 / MKCD: 160 \emph{*rɔʔ}; \emph{*rɔs}, ( \emph{*rɔs
  rɔs} \textgreater{}?) \emph{*srɔs}
\end{itemize}

\begin{longtable}[]{@{}lllllllllllll@{}}
\toprule
Gorum & Sora & Remo & Gutob & Kharia & Juang & Gtaʔ & Santali & Mundari
& Ho & Korwa & Korku & Set\tabularnewline
\midrule
\endhead
a & a & ∅ & u & o & o & ∅ & --- & --- & --- & --- & --- &
0055-1\tabularnewline
\bottomrule
\end{longtable}

\paragraph{VS-015}\label{vs-015}

\begin{longtable}[]{@{}lllllllllllll@{}}
\toprule
Gorum & Sora & Remo & Gutob & Kharia & Juang & Gtaʔ & Santali & Mundari
& Ho & Korwa & Korku & Set\tabularnewline
\midrule
\endhead
a-u & a-u & i-i & i-i & i-i & --- & ∅-i & --- & --- & --- & --- & --- &
0034\tabularnewline
a-i & a-i & i-i & i-i & --- & --- & --- & --- & i-i & i-i & --- & --- &
0067\tabularnewline
--- & a-e & i-e & --- & i-i/e & ɛ-ɛ & ∅-e & e-e & i-i & i-i & --- & ---
& 0083\tabularnewline
\bottomrule
\end{longtable}

\paragraph{\texorpdfstring{\emph{*tVru₅ˀp} `cloud' V₁
(\#0034-2)}{*tVru₅ˀp cloud V₁ (\#0034-2)}}\label{tvruux2c0p-cloud-v-0034-2}

taruˀb, FR, tarub, RSED.p283, tirib, BDBH.1387, tirib, GZ65.416, tiriˀb,
PKED.p287, ---, ---, trig, PGEG.p46, rimil, CDES.p33, rimil, BMED.p160,
rimil, HOGV.p152, liNbir, BAHL.p127, ---, ---, , cloud, \#0034, V285a, ,

\begin{itemize}
\tightlist
\item
  Pinnow 1959: V285a / MKCD: ---
\end{itemize}

If /rim/ in Santali, Munda, and Ho is parallel to /ri(ˀ)b/ in Remo,
Gutob, and Kharia, \emph{*tVru₅ˀp} `cloud' belongs quite clearly to
\emph{*bVˀt} `sow (v)', else it could belong to \emph{*vdʲu₅ˀp} `night'
or \emph{*bVˀt} `sow (v)'.

\subparagraph{\texorpdfstring{\emph{*xli} `liquor' V₁
(\#0067-1)}{*xli liquor V₁ (\#0067-1)}}\label{xli-liquor-v-0067-1}

ali, FR, əli/ali, RSED.p8, ili, BDBH.120, ili, AG08.p672, ---, ---, ---,
---, ---, ---, ---, ---, ili, BMED.p75, ili, DHED.p151, ---, ---, ---,
---, *xlx, liquor, \#0067, V85, ,

\begin{itemize}
\tightlist
\item
  Pinnow 1959: V85 / MKCD: ---
\end{itemize}

\subparagraph{\texorpdfstring{\emph{*tVrel} `ebony' V₁
(\#0083-2)}{*tVrel ebony V₁ (\#0083-2)}}\label{tvrel-ebony-v-0083-2}

---, ---, tarel, RSED.p138, tire, BDBH.1390, ---, ---, ti(ɽr)(ei)l,
PKED.p200, tɛrɛn, PJDW.p285, tre, PGEG.p46, terel, CSED.p626, tiril,
BMED.p188, tiril, DHED.p355, ---, ---, ---, ---, *txrel, ebony, \#0083,
V227, ,

\begin{itemize}
\tightlist
\item
  Pinnow 1959: V227 / MKCD: ---
\end{itemize}

\paragraph{VS-016}\label{vs-016}

\begin{longtable}[]{@{}lllllllllllll@{}}
\toprule
Gorum & Sora & Remo & Gutob & Kharia & Juang & Gtaʔ & Santali & Mundari
& Ho & Korwa & Korku & Set\tabularnewline
\midrule
\endhead
u & a & u & u & ɔ & e & ∅ & --- & o & --- & --- & --- &
0039-2\tabularnewline
u & ə & u & u & --- & (u) & ∅ & o̠ & (o) & (o) & o & --- &
0066-2\tabularnewline
\bottomrule
\end{longtable}

\subparagraph{\texorpdfstring{\emph{*bVtoŋ} `fear' V₁
(\#0039-2)}{*bVtoŋ fear V₁ (\#0039-2)}}\label{bvtoux14b-fear-v-0039-2}

butoŋ, FR, bato:ŋ, RSED.p55, butuŋ, BDBH.1922, butoŋ, GZ65.76, bɔtɔŋ
(P), HLKS.V261, betɔŋan, JLIC.v239, bʈoʔ, PGEG.p14, ---, ---, botoŋ,
BMED.p25, ---, ---, (bor), BAHL.p112, ---, ---, *bxtoŋ, fear, \#0039,
V261, 552,

\begin{itemize}
\tightlist
\item
  Pinnow 1959: V261 / MKCD: 552 \emph{*ʔt₁uuŋ}
\end{itemize}

\subparagraph{\texorpdfstring{\emph{*bVrV(ˀp/ˀk)} `lung' V₁
(\#0066-2)}{*bVrV(ˀp/ˀk) lung V₁ (\#0066-2)}}\label{bvrvux2c0pux2c0k-lung-v-0066-2}

buroˀb, FR, bəro:, RSED.p46, buruk', BDBH.1936, ---, ---, ---, ---,
(buku), JLIC.n49, breʔ, PGEG.p14, bo̠ro̠, CDES.p116, (borkod'), BMED.p25,
(borkoɖ), DHED.p45, boro, BAHL.p112, , , , lungs, \#0066, , ,

\begin{itemize}
\tightlist
\item
  Pinnow 1959: --- / MKCD: ---
\end{itemize}

\paragraph{VS-017}\label{vs-017}

\begin{longtable}[]{@{}lllllllllllll@{}}
\toprule
Gorum & Sora & Remo & Gutob & Kharia & Juang & Gtaʔ & Santali & Mundari
& Ho & Korwa & Korku & Set\tabularnewline
\midrule
\endhead
o & o: & u & --- & --- & (u) & e & o̠ & (o) & (o) & o & --- &
0066-4\tabularnewline
\bottomrule
\end{longtable}

\paragraph{\texorpdfstring{\emph{*bVrV(ˀp/ˀk)} `lungs' V₂
(\#0066-4)}{*bVrV(ˀp/ˀk) lungs V₂ (\#0066-4)}}\label{bvrvux2c0pux2c0k-lungs-v-0066-4}

buroˀb, FR, bəro:, RSED.p46, buruk', BDBH.1936, ---, ---, ---, ---,
(buku), JLIC.n49, breʔ, PGEG.p14, bo̠ro̠, CDES.p116, (borkod'), BMED.p25,
(borkoɖ), DHED.p45, boro, BAHL.p112, , , , lungs, \#0066, , ,

\begin{itemize}
\tightlist
\item
  Pinnow 1959: --- / MKCD: ---
\end{itemize}

\begin{longtable}[]{@{}lllllllllllll@{}}
\toprule
Gorum & Sora & Remo & Gutob & Kharia & Juang & Gtaʔ & Santali & Mundari
& Ho & Korwa & Korku & Set\tabularnewline
\midrule
\endhead
u-o & a-o: & u-u & u-o & ɔ-ɔ & e-ɔ & ∅-o & --- & o-o & --- & --- & --- &
0039\tabularnewline
u-o & ə-o: & u-u & --- & --- & (u-u) & ∅-e & o̠-o̠ & (o-o) & (o-o) & o-o &
--- & 0066\tabularnewline
\bottomrule
\end{longtable}

\paragraph{VS-018}\label{vs-018}

\begin{longtable}[]{@{}lllllllllllll@{}}
\toprule
Gorum & Sora & Remo & Gutob & Kharia & Juang & Gtaʔ & Santali & Mundari
& Ho & Korwa & Korku & Set\tabularnewline
\midrule
\endhead
i & --- & --- & u & a & a & wa & a & a: & a & a: & a &
0037-2\tabularnewline
\bottomrule
\end{longtable}

The sets VS-018 and VS-019 are (so far) the only sets with Gtaʔ /wa/
(notational variation /ua/ and /oa/) that do not seem to constinue
\emph{*ə}. However, the two sets are as different from each other as
they are from the goup of sets \emph{*ə₁} and \emph{*ə₂}.

\subparagraph{\texorpdfstring{\emph{*lv₍₁₈₎(N)dx} `laugh (v)' V₁
(\#0037-2)}{*lv₍₁₈₎(N)dx laugh (v) V₁ (\#0037-2)}}\label{lvndx-laugh-v-v-0037-2}

liɖa, FR, ---, ---, (ɖoɖo), BDBH.1283, luɖo, GZ65.228, laɖa, PKED.p202,
lara, PJDW.p236, lwaʔ, PGEG.p32, lanɖa, CDES.p110, la:nɖa:, BMED.p102,
landa, HOGV.p166, la:Nd, BAHL.p127, lanɖa, NKEV.p322, , laugh (v),
\#0037, V302, ,

\begin{itemize}
\tightlist
\item
  Pinnow 1959: V302 / MKCD: ---
\end{itemize}

\paragraph{VS-019}\label{vs-019}

\begin{longtable}[]{@{}lllllllllllll@{}}
\toprule
Gorum & Sora & Remo & Gutob & Kharia & Juang & Gtaʔ & Santali & Mundari
& Ho & Korwa & Korku & Set\tabularnewline
\midrule
\endhead
i & a & o & u & u & o & wa & e & i & i & --- & e & 0050-2\tabularnewline
\bottomrule
\end{longtable}

\subparagraph{\texorpdfstring{\emph{*tv₍₁₉₎ŋVn/tv₍₁₉₎nVŋ} `stand (v)' V₁
(\#0050-2)}{*tv₍₁₉₎ŋVn/tv₍₁₉₎nVŋ stand (v) V₁ (\#0050-2)}}\label{tvux14bvntvnvux14b-stand-v-v-0050-2}

tinaŋ, FR, tanaŋ, RSED.p, toŋ, BDBH.1490, tunon, AG08.p662, tuŋon,
PKED.p201, toŋon, PJDW.p287, thwaN, PGEG.p46, teŋgon, CDES.p186, tiŋun,
BMED.p187, tiŋgu, HOGV.p180, ---, ---, ʈengene, NKEV.p342, *txŋxn, stand
(v), \#0050, V258, 1824,

\begin{itemize}
\tightlist
\item
  Pinnow 1959: V258 / MKCD: 1824 \emph{*taaw}
\end{itemize}

\paragraph{VS-020}\label{vs-020}

\begin{longtable}[]{@{}lllllllllllll@{}}
\toprule
Gorum & Sora & Remo & Gutob & Kharia & Juang & Gtaʔ & Santali & Mundari
& Ho & Korwa & Korku & Set\tabularnewline
\midrule
\endhead
a & --- & --- & o & a & a & ∅ & a & a: & a & ∅ & a &
0037-5\tabularnewline
\bottomrule
\end{longtable}

The correspondence set VS-020 is

\subparagraph{\texorpdfstring{\emph{*lv₍₁₈₎(N)dv₍₂₀₎} `laugh (v)' V₂
(\#0037-5)}{*lv₍₁₈₎(N)dv₍₂₀₎ laugh (v) V₂ (\#0037-5)}}\label{lvndv-laugh-v-v-0037-5}

liɖa, FR, ---, ---, (ɖoɖo), BDBH.1283, luɖo, GZ65.228, laɖa, PKED.p202,
lara, PJDW.p236, lwaʔ, PGEG.p32, lanɖa, CDES.p110, la:nɖa:, BMED.p102,
landa, HOGV.p166, la:Nd, BAHL.p127, lanɖa, NKEV.p322, , laugh (v),
\#0037, V302, ,

\begin{itemize}
\tightlist
\item
  Pinnow 1959: V302 / MKCD: ---
\end{itemize}

\paragraph{\texorpdfstring{VS-021 Superset:
\emph{*O}}{VS-021 Superset: *O}}\label{vs-021-superset-o}

\begin{longtable}[]{@{}lllllllllllll@{}}
\toprule
Gorum & Sora & Remo & Gutob & Kharia & Juang & Gtaʔ & Santali & Mundari
& Ho & Korwa & Korku & Set\tabularnewline
\midrule
\endhead
e & i & o & o & e & ɛ & ue & --- & --- & --- & --- & --- &
0040-2\tabularnewline
\bottomrule
\end{longtable}

VS-021 is a problematic correspondence set. Gtaʔ /we/ (variations: /oe/
and /ue/) seems to be a reflex of \emph{*oɲ} in pre-Gtaʔ. Gtaʔ /we/ very
conistenly correlates with /o/ in Remo and Gutob. However, no other
correspondence set features Remo and Gutob /e/ as well as Kharia and
Juang /e\textasciitilde{}ɛ/.

Given the presence of back vowels in Remo-Gutob and Gtaʔ and with the
palatal coda a motivation for fronting, exlaining the front vowels in
Sora-Gorum, Kharia, and Juang we posit a back vowel \emph{*O} (more
likely \emph{*o} than \emph{*u}).

\subparagraph{\texorpdfstring{\emph{*dO₍₂₁₎ɲ} `cook (v)'
(\#0040-2)}{*dO₍₂₁₎ɲ cook (v) (\#0040-2)}}\label{doux272-cook-v-0040-2}

ɖeɲ, FR, diɲ, RSED.p80, ɖoNĭ, BDBH.1302, ɖoɲ, AG08.p664, ɖeɲ, PKED.p63,
ɖɛɲ, PJDW.p187, ɖue, PGEG.p17, ---, ---, ---, ---, ---, ---, ---, ---,
---, ---, *dxɲ, cook (v), \#0040, V342, 583,

\begin{itemize}
\tightlist
\item
  Pinnow 1959: V342 / MKCD: 583 \emph{*kɗaŋ}
\end{itemize}

\paragraph{VS-022}\label{vs-022}

\begin{longtable}[]{@{}lllllllllllll@{}}
\toprule
Gorum & Sora & Remo & Gutob & Kharia & Juang & Gtaʔ & Santali & Mundari
& Ho & Korwa & Korku & Set\tabularnewline
\midrule
\endhead
--- & --- & e & ui & i & i & i & i & iu & i: & i: & --- &
0042-2\tabularnewline
\bottomrule
\end{longtable}

VS-022 is a unique set. While Kharia, Juang, Gtaʔ and all Kherwarian
except Mundari feature a close front vowel (/i/), Mundari /iu/ is a
unique exception, unattested in any other set. Gutob /ui/ also occurs in
\#0041-2 (\emph{*bVɲˀ}/\emph{*bVVˀɲ} `snake'), but does not correlate
with Remo /e/ in this case nor with unrounded front vowels in North
Munda.

\paragraph{\texorpdfstring{\emph{*sv₍₂₂₎} `plough (v)'
(\#0042-2)}{*sv₍₂₂₎ plough (v) (\#0042-2)}}\label{sv-plough-v-0042-2}

(or), FR, (or), RSED.p195, se, BDBH.2706, sui, AG08.p650, si, PKED.p143,
si, PJDW.p276, si, PGEG.p42, si, CDES.p143, siu, BMED.p175, si:,
HOGV.p170, si:, BAHL.p135, ---, ---, *sx, plough (v), \#0042, V99, ,

\begin{itemize}
\tightlist
\item
  Pinnow 1959: V99 / MKCD: ---
\end{itemize}

MKCD 1599 \emph{*bcuər} is not a good candidate.

\paragraph{\texorpdfstring{VS-023
\emph{*I₍₂₃₎}}{VS-023 *I₍₂₃₎}}\label{vs-023-i}

\begin{longtable}[]{@{}lllllllllllll@{}}
\toprule
Gorum & Sora & Remo & Gutob & Kharia & Juang & Gtaʔ & Santali & Mundari
& Ho & Korwa & Korku & Set\tabularnewline
\midrule
\endhead
a & ə & ∅ & i & i & i & ∅ & i & i & i & i & i & 0046-1\tabularnewline
\bottomrule
\end{longtable}

VS-023 is unique and the all languages that have a vowel in this postion
have an unrounded closed front vowel, except for Gorum /a/ and Sora /ə/.

\subparagraph{\texorpdfstring{\emph{*I₍₂₃₎sin} `to boil' V₁
(\#0046-1)}{*I₍₂₃₎sin to boil V₁ (\#0046-1)}}\label{isin-to-boil-v-0046-1}

asin, FR, əsin, RSED.p16, nsiŋ, BDBH.1641, isin, GZ65.173, isin,
PKED.p81, isinɔ, JLIC.v65, nsiŋ, PGEG.p37, isin, CDES.p39, isin,
BMED.p77, isin, DHED.p153, isiŋ, BAHL.p12, isin, Korku.txt.12071,
*I₍₂₃₎sin, boil (v), \#0046, V86, ,

\begin{itemize}
\tightlist
\item
  Pinnow 1959: V86 / MKCD: 1137 \emph{*ciinʔ} (\textgreater{}
  Pre-Bahnaric \emph{*cin}); \emph{*ciən{[} {]}}; \emph{*cain{[} {]}};
  `cooked'
\end{itemize}

\paragraph{VS-024 *I₍₂₄₎\_}\label{vs-024-i_}

\begin{longtable}[]{@{}lllllllllllll@{}}
\toprule
Gorum & Sora & Remo & Gutob & Kharia & Juang & Gtaʔ & Santali & Mundari
& Ho & Korwa & Korku & Set\tabularnewline
\midrule
\endhead
u & i & i & i & i & i & --- & (i) & (i) & --- & i: & --- &
0049-1\tabularnewline
\bottomrule
\end{longtable}

VS-024 is very close to the \emph{*i₁} set continuing \emph{*i}.
However, Gorum /u/ is unexplained. The best candidate is currently
\emph{*ɨ}.

\subparagraph{\texorpdfstring{\emph{*tI₍₂₄₎l} `bury
(v)'}{*tI₍₂₄₎l bury (v)}}\label{til-bury-v}

tul, FR, til, RSED.p288, ti, BDBH.1360, til, GZ65.408, til, PKED.p199,
tir, PJDW.p284, ---, ---, (til), RSED.p288, (til), RSED.p288, ---, ---,
ti:l, BAHL.p82, ---, ---, *txl, bury (v), \#0049, ---, ---,

\begin{itemize}
\tightlist
\item
  Pinnow 1959: --- / MKCD: ---
\end{itemize}

\paragraph{VS-025}\label{vs-025}

\begin{longtable}[]{@{}lllllllllllll@{}}
\toprule
Gorum & Sora & Remo & Gutob & Kharia & Juang & Gtaʔ & Santali & Mundari
& Ho & Korwa & Korku & Set\tabularnewline
\midrule
\endhead
a & a & (o) & o & o & o & (wa) & o & u & u & --- & e &
0050-4\tabularnewline
\bottomrule
\end{longtable}

VS-025

Comparison \emph{*a₅} (\#0014-3) \emph{*ə₁} (\#0055-3)

\begin{longtable}[]{@{}lllllllllllll@{}}
\toprule
Gorum & Sora & Remo & Gutob & Kharia & Juang & Gtaʔ & Santali & Mundari
& Ho & Korwa & Korku & Set\tabularnewline
\midrule
\endhead
a & a & (o) & o & o & o & (wa) & o & u & u & --- & e &
0050-4\tabularnewline
a & a & ɔ & o & o & o & o & ∅ & ∅ & ∅ & ∅ & ∅ & 0014-3\tabularnewline
a & a & o & o & o & o & wa & -- & -- & -- & -- & -- &
0055-3\tabularnewline
\bottomrule
\end{longtable}

\subparagraph{\texorpdfstring{\emph{*tv₍₁₉₎ŋv₍₂₅₎n/tv₍₁₉₎nv₍₂₅₎ŋ} `stand
(v)' V₂
(\#0050-4)}{*tv₍₁₉₎ŋv₍₂₅₎n/tv₍₁₉₎nv₍₂₅₎ŋ stand (v) V₂ (\#0050-4)}}\label{tvux14bvntvnvux14b-stand-v-v-0050-4}

tinaŋ, FR, tanaŋ, RSED.p, toŋ, BDBH.1490, tunon, AG08.p662, tuŋon,
PKED.p201, toŋon, PJDW.p287, thwaN, PGEG.p46, teŋgon, CDES.p186, tiŋun,
BMED.p187, tiŋgu, HOGV.p180, ---, ---, ʈengene, NKEV.p342, *txŋxn, stand
(v), \#0050, V258, 1824,

\begin{itemize}
\tightlist
\item
  Pinnow 1959: V258 / MKCD: 1824 \emph{*taaw}
\end{itemize}

\paragraph{VS-026}\label{vs-026}

\begin{longtable}[]{@{}lllllllllllll@{}}
\toprule
Gorum & Sora & Remo & Gutob & Kharia & Juang & Gtaʔ & Santali & Mundari
& Ho & Korwa & Korku & Set\tabularnewline
\midrule
\endhead
e & e & u & --- & --- & --- & i & i & i & i & i & i &
0052-2\tabularnewline
\bottomrule
\end{longtable}

\subparagraph{\texorpdfstring{\emph{*ɲv₍₂₆₎r} `run (v)'
(\#0052-2)}{*ɲv₍₂₆₎r run (v) (\#0052-2)}}\label{ux272vr-run-v-0052-2}

jer, FR, jer, RSED.p88, ur, BDBH.155, ---, ---, yar, DSKH\#12601, ---,
---, wir, PGEG.p9, ɲir, CDES.p164, nir, BMED.p132, nir, DHED.p246, ɲir,
BAHL.p66, niɽe, NKEV.p328, *ɲxr, run (v), \#0052, K294, 1602,

\begin{itemize}
\tightlist
\item
  Pinnow 1959: K294 / MKCD: 1602 \emph{*jarʔ}
\end{itemize}

Maybe two forms North Munda \emph{*ɲvr} and in the southern languages
\emph{*jvr}?

\paragraph{\texorpdfstring{VS-027
\emph{*ᵊ}}{VS-027 *ᵊ}}\label{vs-027-ux1d4a}

\begin{longtable}[]{@{}lllllllllllll@{}}
\toprule
Gorum & Sora & Remo & Gutob & Kharia & Juang & Gtaʔ & Santali & Mundari
& Ho & Korwa & Korku & Set\tabularnewline
\midrule
\endhead
e & a & i & i & o & --- & ∅ & --- & --- & --- & --- & --- &
0053-2\tabularnewline
\bottomrule
\end{longtable}

\begin{longtable}[]{@{}lllllllllllll@{}}
\toprule
Gorum & Sora & Remo & Gutob & Kharia & Juang & Gtaʔ & Santali & Mundari
& Ho & Korwa & Korku & Set\tabularnewline
\midrule
\endhead
e & a & i & i & o & --- & ∅ & --- & --- & --- & --- & --- &
0053-2\tabularnewline
a & a & i & i & ɔ & --- & --- & e & e & e & e & --- &
0077-2\tabularnewline
\bottomrule
\end{longtable}

\begin{longtable}[]{@{}lllllllllllll@{}}
\toprule
Gorum & Sora & Remo & Gutob & Kharia & Juang & Gtaʔ & Santali & Mundari
& Ho & Korwa & Korku & Set\tabularnewline
\midrule
\endhead
e-i & a-e & i-e & i-e & o-e & --- & ∅-i & --- & --- & --- & --- & --- &
0053\tabularnewline
a-i & a-e & i-e & i-e & ɔ-ɛ & --- & --- & e-e & e-e & e-e & e-e & --- &
0077\tabularnewline
\bottomrule
\end{longtable}

\subparagraph{\texorpdfstring{\emph{*tᵊme} `new' V₁
(\#0053-2)}{*tᵊme new V₁ (\#0053-2)}}\label{tux1d4ame-new-v-0053-2}

tḛmi, FR, tamme, RSED.p277, time, BDBH.1383, time, ZG65.410, tonme,
PKED.p289, ---, ---, tmi, PGEG.p125, (nãwã), CDES.p128, (nawã),
BMED.p127, (nama), HOGV.p168, ---, ---, (uni), NKEV.345, *tᵊme, new,
\#0053, V184, 144,

\begin{itemize}
\tightlist
\item
  Pinnow 1959: V182 / MKCD: 144 \emph{*t₁miʔ}
\end{itemize}

\paragraph{VS-028}\label{vs-028}

\subparagraph{\texorpdfstring{\emph{*gəle} `ear of corn' V₁
(\#0077-2)}{*gəle ear of corn V₁ (\#0077-2)}}\label{gux259le-ear-of-corn-v-0077-2}

gali, FR, gale, RSED.p96, gileker, DSBO.11781, gile, GTXT.7791, gɔlɛ,
HLKS.V182, (ɔnɔ), PJDW.p255, (konto-ja), PGEG.p28, gele, CDES.p185,
gele, EM.p1418, gele, DHED.p111, geleʔ, BAHL.p45, (kelʈa), NKEV.p317,
*gxle, ear of corn, \#0077, V182, 1577,

\begin{itemize}
\tightlist
\item
  Pinnow 1959: V182 / MKCD: 1577 \emph{*gur}; \emph{*guər}
\end{itemize}

\paragraph{\texorpdfstring{VS-029
\emph{*a}?}{VS-029 *a?}}\label{vs-029-a}

\begin{longtable}[]{@{}lllllllllllll@{}}
\toprule
Gorum & Sora & Remo & Gutob & Kharia & Juang & Gtaʔ & Santali & Mundari
& Ho & Korwa & Korku & Set\tabularnewline
\midrule
\endhead
--- & (o:) & o & o & a(i) & (a) & we & a & a: & a: & a: & a: &
0063-2\tabularnewline
\bottomrule
\end{longtable}

Gtaʔ /we/ suggests \emph{*oˀc} (or \emph{*oj}) in pre-Gtaʔ. This is
supported by Remo and Gutob /oj/. The reflexes suggest \emph{*a} similar
to \emph{*a₄} or \emph{*a₅}, however Gtaʔ \emph{*weʔ} or the assumed
pre-Gtaʔ \emph{*oˀc} consititutes a deviation from the attested reflexes
of \emph{*a}.

\begin{longtable}[]{@{}lllllllllllll@{}}
\toprule
Gorum & Sora & Remo & Gutob & Kharia & Juang & Gtaʔ & Santali & Mundari
& Ho & Korwa & Korku & Set\tabularnewline
\midrule
\endhead
--- & (o:) & o & o & a(i) & (a) & we & a & a: & a: & a: & a: &
0063-2\tabularnewline
--- & --- & o & o & o & ɔ & we & o̠ & o & o & oe & o &
0051-2\tabularnewline
o & o: & (o) & o & (ɔ) & --- & oe & o̠ & o & o & o & u &
0071-2\tabularnewline
(a) & (a) & i & o & o & o & we & o & oe & o:e & oe & o: &
0094-2\tabularnewline
\bottomrule
\end{longtable}

\subparagraph{\texorpdfstring{\emph{*lv₍₂₉₎ˀc} `stomach' (\#0063-2)
\emph{*a}?}{*lv₍₂₉₎ˀc stomach (\#0063-2) *a?}}\label{lvux2c0c-stomach-0063-2-a}

---, ---, (lo:ˀɟ), RSED.p163, suloĭ, BDBH.2692, suloj, AG08.p651,
la(i)ˀj, PKED.p119, (lai), JLIC.n57, slweʔ, PGEG.p43, lac', CDES.p188,
la:iˀ, BMED.p101, la:iʔ, DHED.p204, la:i:ʔ, BAHL.p123, la:j, NKEV.p323,
*lv₍₂₉₎jˀ, stomach, \#0063, K282, ,

\begin{itemize}
\tightlist
\item
  Pinnow 1959: K282 / MKCD: ---
\end{itemize}

\paragraph{\texorpdfstring{VS-038 \emph{*O}}{VS-038 *O}}\label{vs-038-o}

\begin{longtable}[]{@{}lllllllllllll@{}}
\toprule
Gorum & Sora & Remo & Gutob & Kharia & Juang & Gtaʔ & Santali & Mundari
& Ho & Korwa & Korku & Set\tabularnewline
\midrule
\endhead
∅ & ʊ & u & u & u & u & o & --- & (a) & --- & --- & --- &
0068-5\tabularnewline
\bottomrule
\end{longtable}

VS-029 is the only set that features /u/ in Sora, Remo, Gutob, Kharia,
and Juang and /o/ in Gtaʔ.

The set shows some similarity with \#0062-2 of \emph{*tVŋ} `kindle (v)'.
However, if Mundari /a/ belongs to this set, Kherwarian /a/ would be in
sharp contrast to Khewarian /i/ in \#0062-2.

\begin{longtable}[]{@{}lllllllllllll@{}}
\toprule
Gorum & Sora & Remo & Gutob & Kharia & Juang & Gtaʔ & Santali & Mundari
& Ho & Korwa & Korku & Set\tabularnewline
\midrule
\endhead
∅ & ʊ & u & u & u & u & o & --- & (a) & --- & --- & --- &
0068-5\tabularnewline
u & u & -- & -- & u & -- & o & i & i & i & -- & i &
0062-2\tabularnewline
\bottomrule
\end{longtable}

\subparagraph{\texorpdfstring{\emph{*ruNkOˀk} `husked rice'
(\#0068-5)}{*ruNkOˀk husked rice (\#0068-5)}}\label{runkoux2c0k-husked-rice-0068-5}

ruŋk, FR, rʊŋkʊ, RSED.p239, ruŋku, BDBH.2291, rukug, AG08.p672,
ruŋkuˀb/rumkuˀb, PKED.p171, ruŋkub, PJDW.p269, rkoʔ, PGEG.p41, ---, ---,
(rukhaɽ), BMED.p163, ---, ---, ---, ---, ---, ---, *ruNkOˀk, husked
rice, \#0068, V139, 1820,

\begin{itemize}
\tightlist
\item
  Pinnow 1959: V139 / MKCD: 1820 \emph{*rk{[}aw{]}ʔ}
\end{itemize}

Gtaʔ /rkoʔ/ is surprisingly close th Shorto's \emph{*rk{[}aw{]}ʔ}.

\paragraph{\texorpdfstring{VS-030
\emph{*Ə}}{VS-030 *Ə}}\label{vs-030-ux259}

\begin{longtable}[]{@{}lllllllllllll@{}}
\toprule
Gorum & Sora & Remo & Gutob & Kharia & Juang & Gtaʔ & Santali & Mundari
& Ho & Korwa & Korku & Set\tabularnewline
\midrule
\endhead
a & a & e & --- & a & (a) & --- & e & e & e & e & e &
0087-2\tabularnewline
\bottomrule
\end{longtable}

Gorum, Sora, and Kharia /a/ and North Munda /e/ are consistent with
\emph{*a₃} and \emph{*ə₁}. However, Remo /e/ is not compatible with any
of these. In set \emph{*a₃}, Remo features /a/ and in \emph{*ə₁} /o/.
Furthermore, the palatal context motivating \emph{*a₃} is probably
absent in the case of \#0087-2.

\subparagraph{\texorpdfstring{\emph{*sƏn} `chase (v)'
(\#0087-2)}{*sƏn chase (v) (\#0087-2)}}\label{sux259n-chase-v-0087-2}

san, FR, san, RSED.p248, sensen, BDBH.2714, ---, ---, san, PKED.p176,
(saŋgem), PJDW.p273, ---, ---, sen, CSED.p572, sen, BMED.p172, sen,
DHED.p311, sen, BAHL.p138, sen(e), NKEV.p337, *sƏn, chase (v), \#0087,
V300, 899,

\begin{itemize}
\tightlist
\item
  Pinnow 1959: V300 / MKCD: 899 \emph{*təɲ}
\end{itemize}

\subsubsection{VS-031}\label{vs-031}

\begin{longtable}[]{@{}lllllllllllll@{}}
\toprule
Gorum & Sora & Remo & Gutob & Kharia & Juang & Gtaʔ & Santali & Mundari
& Ho & Korwa & Korku & Set\tabularnewline
\midrule
\endhead
u & e & u & u & e & i & ∅ & i & i: & i: & i: & --- &
0098-2\tabularnewline
\bottomrule
\end{longtable}

\subparagraph{\texorpdfstring{\emph{*bv₍₃₁₎/bv₍₃₁₎sv₍₃₂₎} `sated (v)'
(\#0098-2)}{*bv₍₃₁₎/bv₍₃₁₎sv₍₃₂₎ sated (v) (\#0098-2)}}\label{bvbvsv-sated-v-0098-2}

buʔ, FR, beˀ, RSED.p56, busu, BDBH.1960, busu, Z1965.72, beso/u,
PKED.p20, bisu, PJDW.p14, bse, PGEG.p14, bi(ʔ), CSED.p67, bi:(ʔ)/biu,
BMED.22, bi:, DHED.p35, bi:, BAHL.p106, ---, ---, *bx, be sated (v),
\#0098, V319, 259,

\begin{itemize}
\tightlist
\item
  Pinnow 1959: V319 / MKCD: 259 \emph{*bhiiʔ}
\end{itemize}

It remains unknown why \#0098-2 is not a straightforward reflex. The
correspondence set shows some similarity with VS-003 (\#0041-2 and
\#0062-2), but the attested words in this group have a palatal coda. The
other similar set is VS-026 (\#0052-2). If we take the /u/ of Gorum as
an effect

maybe via \emph{*bhiiʔ} \textgreater{} \emph{*bᵊhiiʔ} \textgreater{}
\emph{*bəhiiʔ} \textgreater{} \emph{*bəiiʔ} \textgreater{}
\emph{*bv₍₃₁₎}

\begin{longtable}[]{@{}lllllllllllll@{}}
\toprule
Gorum & Sora & Remo & Gutob & Kharia & Juang & Gtaʔ & Santali & Mundari
& Ho & Korwa & Korku & Set\tabularnewline
\midrule
\endhead
u & e & u & u & e & i & ∅ & i & i: & i: & i: & --- &
0098-2\tabularnewline
(u) & i & u & ui & u & u & o & i & i & i & i: & i &
0041-2\tabularnewline
u & u & -- & -- & u & -- & o & i & i & i & -- & i &
0062-2\tabularnewline
e & e & u & --- & --- & --- & i & i & i & i & i & i &
0052-2\tabularnewline
\bottomrule
\end{longtable}

If MKCD 259 \emph{*bhiiʔ} is correct the reflex should be consistent
with \emph{*i}, but these vowels are not at all clear reflex of
proto-Munda *i\_.

\subsubsection{VS-032}\label{vs-032}

\begin{longtable}[]{@{}lllllllllllll@{}}
\toprule
Gorum & Sora & Remo & Gutob & Kharia & Juang & Gtaʔ & Santali & Mundari
& Ho & Korwa & Korku & Set\tabularnewline
\midrule
\endhead
--- & --- & u & u & o/u & u & e & --- & --- & --- & --- & --- &
0098-5\tabularnewline
\bottomrule
\end{longtable}

\subparagraph{\texorpdfstring{\emph{*bv₍₃₁₎/bv₍₃₁₎sv₍₃₂₎} `sated (v)'
(\#0098-5)}{*bv₍₃₁₎/bv₍₃₁₎sv₍₃₂₎ sated (v) (\#0098-5)}}\label{bvbvsv-sated-v-0098-5}

buʔ, FR, beˀ, RSED.p56, busu, BDBH.1960, busu, Z1965.72, beso/u,
PKED.p20, bisu, PJDW.p14, bse, PGEG.p14, bi(ʔ), CSED.p67, bi:(ʔ)/biu,
BMED.22, bi:, DHED.p35, bi:, BAHL.p106, ---, ---, *bx, be sated (v),
\#0098, V319, 259,

\begin{itemize}
\tightlist
\item
  Pinnow 1959: V319 / MKCD: 259 \emph{*bhiiʔ}
\end{itemize}

Only in the /sV/ part attested in in Gutob, Remo, Kharia, Juang, and
Gtaʔ, probably not going back to proto-Munda, especially if MKCD 259 is
the same etymon.

\subsubsection{\texorpdfstring{VS-033
\emph{*U}}{VS-033 *U}}\label{vs-033-u}

\begin{longtable}[]{@{}lllllllllllll@{}}
\toprule
Gorum & Sora & Remo & Gutob & Kharia & Juang & Gtaʔ & Santali & Mundari
& Ho & Korwa & Korku & Set\tabularnewline
\midrule
\endhead
u & a & i & i & i & --- & i & u & u & u & --- & i &
0100-2\tabularnewline
\bottomrule
\end{longtable}

Comparison to \emph{*u₅} and in particular *vdʲu₅ˀp\_ `night' V₂
(\#0033-3). The difference between Korku /i/ in \#0100-2, as opposed to
/u/ in \#0033-3, is less problematic as is Sora /a/ in \#0100-2, as
opposed to /u/ in \#0033-3.

\begin{longtable}[]{@{}lllllllllllll@{}}
\toprule
Gorum & Sora & Remo & Gutob & Kharia & Juang & Gtaʔ & Santali & Mundari
& Ho & Korwa & Korku & Set\tabularnewline
\midrule
\endhead
u & \textbf{a} & i & i & i & --- & i & u & u & u & --- & \textbf{i} &
0100-2\tabularnewline
u & \textbf{u} & i & i & i & -- & i & u & u & u & u & \textbf{u} &
0033-3\tabularnewline
\bottomrule
\end{longtable}

\subparagraph{\texorpdfstring{\emph{*ɲUm} `name'
(\#0100-2)}{*ɲUm name (\#0100-2)}}\label{ux272um-name-0100-2}

inum, FR, əɲam, RSED.p12, nimi, BDBH.1588, imi, AG.p645, (i)ɲimi,
PKED.p140, ---, ---, mni, PGEG.p35, ɲum/ɲutum, CSED.p451/452, num/nutum,
BMED.134, numu/nutum, DHED.p249, ---, ---, jimu, , *ɲUm, name, \#0100,
V279, 147,

\begin{itemize}
\tightlist
\item
  Pinnow 1959: V319 / MKCD: 259 \emph{*{[}hy{]}muʔ}
\end{itemize}

\subsubsection{Words with problematic
vocalism}\label{words-with-problematic-vocalism}

\begin{itemize}
\tightlist
\item
  \#0059-2: VS-034
\item
  \#0059-4: VS-035
\item
  \#0082-2: VS-036
\item
  \#0082-4: VS-037
\end{itemize}

\paragraph{\texorpdfstring{\emph{*miɲam}/\emph{*mayOm} `blood'
(\#0059-2/\#0059-4)}{*miɲam/*mayOm blood (\#0059-2/\#0059-4)}}\label{miux272ammayom-blood-0059-20059-4}

miam, FR, miɲam, RSED.p177, ---, ---, iam, Z1963.325, iɲam, PKED.p115,
iɲam/iɲɑm, PJDW.p208, mia, PGEG.p33, maNyaNm, CDES.p18, ma:yom,
BMED.p116, mayom, HOGV.p149, , , mayum, NKEV.p325, , blood, \#0059,
V303, 1430,

\begin{itemize}
\tightlist
\item
  Pinnow 1959: V303 / MKCD: 1430 \emph{*jhaam}; \emph{*jhiim}
\end{itemize}

\begin{longtable}[]{@{}lllllllllllll@{}}
\toprule
Gorum & Sora & Remo & Gutob & Kharia & Juang & Gtaʔ & Santali & Mundari
& Ho & Korwa & Korku &\tabularnewline
\midrule
\endhead
i-a & i-a & --- & i-a & i-a & i-a/ɑ & i-a & a-a & a-o: & a-o & --- & a-u
& 0059\tabularnewline
\bottomrule
\end{longtable}

North Munda a-a/o/u Southern Languages i-a

Go-So, Re-Gu, Kh, Ju, Gt: \emph{*miɲam} NM: \emph{*mayOm}

V₂ in North Munda languages unique so far.

\paragraph{\texorpdfstring{\emph{*ɟVlVN} `long/tall'
(\#0082-2/\#0082-4)}{*ɟVlVN long/tall (\#0082-2/\#0082-4)}}\label{ux25fvlvn-longtall-0082-20082-4}

zuleŋa, FR, ɟele:n, RSED.p123, sileŋ, BDBH.2601, silej, AG08.p651,
jhelo(g, b, m), PKED.p92, ɟaliŋ, PJDW.210, clæ, PGEG.p15, jeleɲ,
CSED.p260, jiliŋ, BMED.p83, jiliɲ, DHED.p165, ---, ---, ---, ---,
*ɟxlxN, long/tall, \#0082, V340, 740,

\begin{itemize}
\tightlist
\item
  Pinnow 1959: V340 / MKCD: 740 \emph{*jiliiŋ} (\& \emph{*jiliŋ}?);
  \emph{*jla{[}i{]}ŋ} `long'
\end{itemize}

The set \emph{*ɟVlVN} `long/tall' displays unclear reflexes of a final
nasal. The problems are aggravated by the fact that this might be a
fused set of two or more etyma meaning \emph{long}, \emph{tall},
\emph{high}, \emph{slim}, and related concepts all based in the
consonantal frame \emph{*ɟVlVN}, but with different vowels. If we assume
that the character of the vowel preceding the nasal may influence the
the form in certain languages, the problem of the different etyma with
different vowels is fundamental for the reconstruction of the final
nasal.

\subsection{Consonants}\label{consonants}

\begin{longtable}[]{@{}llllll@{}}
\toprule
& bilabial & dental/alveolar & palatal & velar & glottal\tabularnewline
\midrule
\endhead
voiceless stop & *p & *t & (*c???) & *k & (*Vˀ?)\tabularnewline
voiced stop & *b & *d & *ɟ & *g &\tabularnewline
glottalized stop & *ˀp & *ˀt & *ˀc & *ˀk &\tabularnewline
nasal & *m & *n & *ɲ & *ŋ &\tabularnewline
sibilant & & *s & & &\tabularnewline
lateral & & *l & & &\tabularnewline
rhotic & & *r & & &\tabularnewline
approximants & & & *j & &\tabularnewline
\bottomrule
\end{longtable}

\subsubsection{Issues}\label{issues-1}

\begin{itemize}
\tightlist
\item
  \emph{*Kʰ} Pinnow (1959 p.232-234) \emph{*q} etc., seems to
  konsistently reflect MKCD \emph{*k} with no apparent reason for the
  variation /k/\textsubscript{/h/}∅.
\item
  \emph{*dʲ}
\end{itemize}

\subsubsection{Bilabials}\label{bilabials}

\begin{longtable}[]{@{}llll@{}}
\toprule
voiceless & voiced & glottalized & nasal\tabularnewline
\midrule
\endhead
\emph{*p} & \emph{*b} & \emph{*ˀp} & \emph{*m}\tabularnewline
\bottomrule
\end{longtable}

\begin{longtable}[]{@{}lllllllllllll@{}}
\toprule
Gorum & Sora & Remo & Gutob & Kharia & Juang & Gtaʔ & Santali & Mundari
& Ho & Korwa & Korku &\tabularnewline
\midrule
\endhead
p & p & p & p & p & --- & p & p(h) & p & p & p & p &
\emph{*p}\tabularnewline
\bottomrule
\end{longtable}

\begin{longtable}[]{@{}lllllllllllll@{}}
\toprule
Gorum & Sora & Remo & Gutob & Kharia & Juang & Gtaʔ & Santali & Mundari
& Ho & Korwa & Korku &\tabularnewline
\midrule
\endhead
b & b & b & b & b & b & b & b & b & b & b(h) & b &
\emph{*b}\tabularnewline
--- & --- & b & b & b & --- & --- & b & b & b & b & m &
\emph{*b₂}\tabularnewline
\bottomrule
\end{longtable}

\begin{longtable}[]{@{}lllllllllllll@{}}
\toprule
Gorum & Sora & Remo & Gutob & Kharia & Juang & Gtaʔ & Santali & Mundari
& Ho & Korwa & Korku &\tabularnewline
\midrule
\endhead
ˀb & b & p(') & b & (ˀ)b & b & ʔ/g & p' & b & b & b/p & p &
\emph{*ˀp}\tabularnewline
\bottomrule
\end{longtable}

\begin{longtable}[]{@{}lllllllllllll@{}}
\toprule
Gorum & Sora & Remo & Gutob & Kharia & Juang & Gtaʔ & Santali & Mundari
& Ho & Korwa & Korku &\tabularnewline
\midrule
\endhead
m & m & m & m & m & m & m & m & m & m & m & m & \emph{*m₁}
onset\tabularnewline
m & m & --- & ∅ & ∅ & ∅ & m & m & m & m & m & m & \emph{*m₂}
onset\tabularnewline
m & m(m) & m & m & m & m & m/∅ & m & m & m & m & m(m) & \emph{*m}
medial\tabularnewline
m & m & m & m & m & m & ŋ & m & m & m & m & m & \emph{*m₁}
coda\tabularnewline
m & m & --- & m & m & m & ∅ & m & m & m & --- & m & \emph{*m₂}
coda\tabularnewline
m & m & m & m & m & ŋ/ɲ & ŋ & m & m & m & m & m & \emph{*m₃}
coda\tabularnewline
--- & m & ŋ & ŋ & ŋ & ŋ & ŋ & m & m & m & m & m & \emph{*m₄}
coda\tabularnewline
\bottomrule
\end{longtable}

\paragraph{\texorpdfstring{\emph{*p}}{*p}}\label{p}

\begin{longtable}[]{@{}llllllllllll@{}}
\toprule
Gorum & Sora & Remo & Gutob & Kharia & Juang & Gtaʔ & Santali & Mundari
& Ho & Korwa & Korku\tabularnewline
\midrule
\endhead
p & p & p & p & p & p & p & p(h) & p & p & p & p\tabularnewline
\bottomrule
\end{longtable}

Proto-Munda \emph{*p} is surprisingly poorly attested in widespread
etyma. The reconstruction is exclusively based on /pVC/ verbs, no well
attested intervocalic instance of \emph{*p} has been identified so far.

\subparagraph{\texorpdfstring{\emph{*peˀt} `blow (v)'
(\#0057-1)}{*peˀt blow (v) (\#0057-1)}}\label{peux2c0t-blow-v-0057-1}

peˀd, FR, ped, RSED.p212, peʔ, BDBH.1759, ped, ZG65.293, pɛˀɖ,
PJED.p156, ---, , piʔ, PGEG.p38, phe̠t', CDES.p142, ---, , ---, , ---, ,
---, , *pxˀt, blow (v), \#0057, V157, 1028,

\begin{itemize}
\tightlist
\item
  Pinnow 1959: V157 / MKCD: 1028 \emph{*puut}; \emph{*p{[}əə{]}t}
\end{itemize}

\subparagraph{\texorpdfstring{\emph{*pəˀt}/\emph{*par(om)} `cross (v)'
(\#0085-1)}{*pəˀt/*par(om) cross (v) (\#0085-1)}}\label{pux259ux2c0tparom-cross-v-0085-1}

pa'd, FR, pad, RSED.p200, poʔ, BDBH.1793, poɖ, DSGU\#18931, paro(m),
PKED.p155, (pakea), DSJU\#25131, pwaʔ, PGEG.p39, par, CSED.p474, pa:rom,
BMED.p144, parom, DHED.p262, parom, BAHL.p97, pa:r, NKEV.p331, *pəˀt,
cross (v), \#0085, , ,

Althought the initial bilabials constitute a consistent correspondence
set, evidence from the vowel and the dental/retroflex second consonant
suggest that we have to assume *pəˀt\_ for Sora, Gorum, Remo, Gutob, and
Gtaʔ and a second etymon \emph{*par(om)} `cross (v)' for Kharia and
North Munda, probably of Indo-Aryan origin.

\subparagraph{\texorpdfstring{\emph{*per} `to burn (of chilies) (vi)'
(\#0097-2)}{*per to burn (of chilies) (vi) (\#0097-2)}}\label{per-to-burn-of-chilies-vi-0097-2-1}

per ,FR ,--- ,--- ,per ,BDBH.1756 ,per ,Z1975.294 ,--- ,--- ,--- ,---
,pir ,PGEG.p38 ,pe̠ṛen ,CSED.p500 ,--- ,--- ,(pertol) ,DHED.p266 ,---
,--- ,--- ,--- ,*per ,burn(chilies) (v) ,\#0097 ,--- , ,

\begin{itemize}
\tightlist
\item
  Pinnow 1959: --- / MKCD: ---
\end{itemize}

\paragraph{\texorpdfstring{\emph{*b}}{*b}}\label{b}

\begin{longtable}[]{@{}llllllllllll@{}}
\toprule
Gorum & Sora & Remo & Gutob & Kharia & Juang & Gtaʔ & Santali & Mundari
& Ho & Korwa & Korku\tabularnewline
\midrule
\endhead
b & b & b & b & b & b & b & b & b & b & b(h) & b\tabularnewline
\bottomrule
\end{longtable}

\subparagraph{\texorpdfstring{\emph{*b(oKʰ)Oˀp} `head'
(\#0011-4)}{*b(oKʰ)Oˀp head (\#0011-4)}}\label{bokux2b0oux2c0p-head-0011-4-1}

baˀb, FR, bo:ˀb, RSED.p60, bob, BDBH.2007, bob, Z1963.50, bokoˀb,
PKED.p24, bokob, PJDW.p169, bhaʔ, PGEG.p13, bo̠ho̠k', CDES.p90, bo,
BMED.p24, bo:ʔ, DHED.p40, boʔ, BAHL.p113, ---, ---, *b(oKʰ)Oˀp, head,
\#0011, V206, 361, 38

\begin{itemize}
\tightlist
\item
  Pinnow 1959: V206 UM *ɔ/ MKCD: 361 \emph{*{[}b{]}uuk}
\end{itemize}

\subparagraph{\texorpdfstring{\emph{*bul} `drunk (v)'
(\#0016-1)}{*bul drunk (v) (\#0016-1)}}\label{bul-drunk-v-0016-1}

bṵl, FR, buʔul, Sora.txt.18922, bu, BDBH.1900, bil, AG08.p672, bul,
PKED.p39, buli, PJDW.p174, busaʔ, PGEG.p13, bul, CDES.p58, bul,
BMED.p25, bul, HOGV.p155, bubul, BAHL.p108, bubul, NKEV.p70, , drunk,
\#0016, V105, 1765,

\begin{itemize}
\tightlist
\item
  Pinnow 1959: V105 / MKCD: 1765 \emph{*ɓul}; \emph{*ɓuul}
\end{itemize}

\subparagraph{\texorpdfstring{\emph{*buluuˀ} `thigh' V₂
(\#0017-1)}{*buluuˀ thigh V₂ (\#0017-1)}}\label{buluuux2c0-thigh-v-0017-1}

bulu, FR, bulu:, RSED.p64, buli/bili, BDBH.1949/1890, bili, DSGU.2681,
bhulu, PKED.p32, bulu, PJDW.p174, bulu, PGEG.p13, bulu, CDES.p199, bulu,
BMED.p25, bulu, HOGV.p183, bu:l, BAHL.p109, bulu, NKEV.p295, *buluuˀ,
thigh, \#0017, V145, 223,

\begin{itemize}
\tightlist
\item
  Pinnow 1959: V145 / MKCD: 223 \emph{*bluuʔ}
\end{itemize}

\subparagraph{\texorpdfstring{\emph{*bv₍₁₃₎lv₍₁₁₎} `ripe'
(\#0018-1)}{*bv₍₁₃₎lv₍₁₁₎ ripe (\#0018-1)}}\label{bvlv-ripe-0018-1}

---, ---, ---, ---, bulu, BDBH.1591, bulu, AG08.p644, belom, PKED.p19,
bilim, PJDW.p167, ble, PGEG.p13, bele, CDES.p161, bili, BMED.p23, bili,
HOGV.p156, bhi:li:, BAHL.p115, bili, NKEV.p293, *bxlx, ripe, \#0018,
V232, ,

\begin{itemize}
\item
  Pinnow 1959: V232 / MKCD: ---
\item
  MKCD 2080 \emph{*bl{[}ɔ{]}h} `finished'
\item
  MKCD 1878 \emph{*lʔas} `ripe'
\end{itemize}

\subparagraph{\texorpdfstring{\emph{*bVrəl} `raw'
(\#0019-1)}{*bVrəl raw (\#0019-1)}}\label{bvrux259l-raw-0019-1}

---, ---, ---, ---, buro, BDBH.1937, burol, Z1965.74, borol, PKED.p25,
boron, PJDW.p171, brwa, PGEG.p14, be̠re̠l, CDES.p211, berel, BMED.p21,
berel, HOGV.p185, berel, BAHL.p111, boboɽ, NKEV.p294, *bxrəl, raw,
\#0019, V253, ,

\begin{itemize}
\tightlist
\item
  Pinnow 1959: V253 / MKCD: ---
\end{itemize}

\subparagraph{\texorpdfstring{\emph{*bel} `spread (v)'
(\#0022-1)}{*bel spread (v) (\#0022-1)}}\label{bel-spread-v-0022-1}

bil, FR, bel/bɪ:l, RSED.p56/58, be-sak', BDBH.1982, be(d), Z1965.50,
bel, PKED.p18, bɛn, PJDW.p166, beʔ, PGEG.p11, bel, CDES.p184, bil,
BMED.p24, bil, HOGV.p179, bel, BAHL.p111, (bi)bil, NKEV.p293, *bel,
spread (vt), \#0022, V221, 1761,

\begin{itemize}
\tightlist
\item
  Pinnow 1959: V221 / MKCD: 1761 \emph{*b{[}e{]}l} (\emph{*beel}?)
\end{itemize}

\subparagraph{\texorpdfstring{\emph{*bal} `to burn'
(\#0023-1)}{*bal to burn (\#0023-1)}}\label{bal-to-burn-0023-1}

Go. \emph{bal}; So. \emph{ba:l} (RSED.p49); Gu. \emph{bal} (Z1965.43);
Gt. \emph{ba} (PGEG.p9); Sa. \emph{bal} (BSDV1.p1840); Mu. \emph{bal}
(BMED.p18); Ho \emph{bal} (HOGV.p151); Kw. \emph{ba:l} (BAHL.p1050; Ko.
\emph{ba:l} (NKEV.p292)

\begin{itemize}
\tightlist
\item
  Pinnow 1959 --- / MKCD ---
\end{itemize}

A connection to any of the etyma attested in MKCD remains unclear. The
best candidate is MKCD 1671 \emph{*waar}, \emph{*wər}. However, neither
pAA *w nor pAA *r match the reconstructed pM *b and *l.

\subparagraph{\texorpdfstring{\emph{*bVtoŋ} `fear'
(\#0039-1)}{*bVtoŋ fear (\#0039-1)}}\label{bvtoux14b-fear-0039-1}

butoŋ, FR, bato:ŋ, RSED.p55, butuŋ, BDBH.1922, butoŋ, GZ65.76, bɔtɔŋ
(P), HLKS.V261, betɔŋan, JLIC.v239, bʈoʔ, PGEG.p14, ---, ---, botoŋ,
BMED.p25, ---, ---, (bor), BAHL.p112, ---, ---, *bxtoŋ, fear, \#0039,
V261, 552,

\begin{itemize}
\tightlist
\item
  Pinnow 1959: V261 / MKCD: 552 \emph{*ʔt₁uuŋ}
\end{itemize}

MKCD 552 \emph{*ʔt₁uuŋ} is a close -- but not perfect -- match. Intial
/b/ is attested in all languages. Shorto's initial cluster \emph{*ʔt₁}
is not easily connected to any proto-Munda \emph{*bVt}.

\subparagraph{\texorpdfstring{\emph{*bVɲˀ(*bVVˀɲ)} `snake'
(\#0041-1)}{*bVɲˀ(*bVVˀɲ) snake (\#0041-1)}}\label{bvux272ux2c0bvvux2c0ux272-snake-0041-1}

bubuˀd, FR, biɲ/biŋ, RSED.p59, bubuʔ, BDBH.1931, buɽbui, GGEG.p108,
buɲam, PKED.p4, bubuŋ, PJDW.p172, boʔ, PGEG.p12, biɲ, CDES.p179, biŋ,
BMED.p23, biɲ, HOGV.p178, bi:ŋ, BAHL.p108, biɲj, NKEV.p294, , snake,
\#0041, V353, 937,

\begin{itemize}
\tightlist
\item
  Pinnow 1959: V353; ; VW u/i; UM:i / MKCD: 937 \emph{*{[}b{]}saɲʔ}
\end{itemize}

MKCD 937 \emph{*{[}b{]}saɲʔ} less likely MKCD 1921a \emph{*ɓəs}

\subparagraph{\texorpdfstring{\emph{*bɨˀt} `sow (v)'
(\#0045-1)}{*bɨˀt sow (v) (\#0045-1)}}\label{bux268ux2c0t-sow-v-0045-1}

buˀd, FR, büd/bid, RSED.p63, biʔ, BDBH.1898, biɽ, GZ63.67, biˀɖ,
PKED.p20, bir, PJDW.p167, big, PGEG.p11, bit', CDES.p142, bid',
BMED.p22, biɖ, DHED.p35, , , biʈ, NKEV.p294, *bɨˀt, sow (v), \#0045,
V285, ,

\begin{itemize}
\tightlist
\item
  Pinnow 1959: V285 / MKCD: ---
\end{itemize}

\subparagraph{\texorpdfstring{\emph{*boŋtel}/\emph{*bitkil} `buffalo'
(\#0054-1)}{*boŋtel/*bitkil buffalo (\#0054-1)}}\label{boux14btelbitkil-buffalo-0054-1}

boŋtel, FR, boŋtel, RSED.p62, buŋte, BDBH.1917, boŋtel, AG08.p647,
boŋtel, PKED.p36, ---, ---, buNʈi, PGEG.p13, bitkil, CDES.p23, ---, ---,
---, ---, ---, ---, biʈkhil, NKEV.p294, *bxŋtxl, buffalo, \#0054, , ,

Probably two separate etyma \emph{*boŋtel} and North Munda *bitkil\_.
The form suggests some relation, but the two forms cannot be derived
from proto-Munda by regular sound change.

\subparagraph{\texorpdfstring{\emph{*bVrV(ˀp/ˀk)} `lung'
(\#0066-1)}{*bVrV(ˀp/ˀk) lung (\#0066-1)}}\label{bvrvux2c0pux2c0k-lung-0066-1}

buroˀb, FR, bəro:, RSED.p46, buruk', BDBH.1936, ---, ---, ---, ---,
(buku), JLIC.n49, breʔ, PGEG.p14, bo̠ro̠, CDES.p116, (borkod'), BMED.p25,
(borkoɖ), DHED.p45, boro, BAHL.p112, , , , lungs, \#0066, , ,

\begin{itemize}
\tightlist
\item
  Pinnow 1959: --- / MKCD: ---
\end{itemize}

\subparagraph{\texorpdfstring{\emph{*bar} `two'
(\#0078-1)}{*bar two (\#0078-1)}}\label{bar-two-0078-1}

bagu, FR, bar, RSED.p48, mbaʔr, BDBH.2214, umbar, AG08.p646, ubar,
PKED.p205, umba, PJDW.p291, mbar, PGEG.p34, bar, CSED.p42, baria,
BMED.p20, bar, DHED.p27, ---, ---, ba:r, NKEV.293, *bar, two, \#0078,
V49, 1562,

\begin{itemize}
\tightlist
\item
  Pinnow 1959: V49 / MKCD: 1562 \emph{*biʔaar} \textgreater{}
  \emph{*ɓaar}, Pre-Khmer \emph{*{[}ɓ{]}ir}, Pre-Palaungic \&c.
  \emph{*ʔaar}
\end{itemize}

\subparagraph{\texorpdfstring{\emph{*bəˀt} `contain/block (v)'
(\#0092-1)}{*bəˀt contain/block (v) (\#0092-1)}}\label{bux259ux2c0t-containblock-v-0092-1}

baˀd ,FR ,bad ,RSED.p47 ,boʔ ,BDBH.2027 ,bod ,Z1965.59 ,--- ,--- ,---
,--- ,boaʔ ,PGEG.p11 ,be̠t' ,BSDV1.p275 ,bed' ,BMED.p21 ,beɖ ,DHED.p33
,--- ,--- ,--- ,--- ,--- ,contain/block (v) ,\#0092 ,--- , 1032,

\begin{itemize}
\tightlist
\item
  Pinnow 1959: --- / MKCD 1032 \emph{*bat}; \emph{*buət}
\end{itemize}

\subparagraph{\texorpdfstring{\emph{*bv₍₃₁₎/bv₍₃₁₎v₍₃₁₎ˀ/bv₍₃₁₎sv₍₃₂₎}
`sated (v)'
(\#0098-1)}{*bv₍₃₁₎/bv₍₃₁₎v₍₃₁₎ˀ/bv₍₃₁₎sv₍₃₂₎ sated (v) (\#0098-1)}}\label{bvbvvux2c0bvsv-sated-v-0098-1}

buʔ, FR, beˀ, RSED.p56, busu, BDBH.1960, busu, Z1965.72, beso/u,
PKED.p20, bisu, PJDW.p14, bse, PGEG.p14, bi(ʔ), CSED.p67, bi:(ʔ)/biu,
BMED.22, bi:, DHED.p35, bi:, BAHL.p106, ---, ---, *bx, be sated (v),
\#0098, V319, 259,

\begin{itemize}
\tightlist
\item
  Pinnow 1959: V319 / MKCD: 259 \emph{*bhiiʔ}
\end{itemize}

\paragraph{\texorpdfstring{\emph{*b₂}}{*b₂}}\label{b-1}

\begin{longtable}[]{@{}llllllllllll@{}}
\toprule
Gorum & Sora & Remo & Gutob & Kharia & Juang & Gtaʔ & Santali & Mundari
& Ho & Korwa & Korku\tabularnewline
\midrule
\endhead
--- & --- & b & b & b & --- & --- & b & b & b & b & m\tabularnewline
\bottomrule
\end{longtable}

\subparagraph{\texorpdfstring{\emph{*sv₍₆₎bv₍₁₂₎l} `sweet' V₂
(\#0061-3)}{*sv₍₆₎bv₍₁₂₎l sweet V₂ (\#0061-3)}}\label{svbvl-sweet-v-0061-3}

---, ---, ---, ---, subu, BDBH.2665, subul, AG08.p651, sebol, PKED.p180,
---, ---, ―, ---, sebel, CDES.p1194, sibil, EMV13.p3943, sibil,
DHED.p316, sebel, DSKW.@21820, simil, NKEV.p338, *sv₍₆₎bxl, sweet,
\#0061, V257, ,

\begin{itemize}
\tightlist
\item
  Pinnow 1959: V257 / MKCD: ---
\end{itemize}

\paragraph{\texorpdfstring{\emph{*ˀp}}{*ˀp}}\label{ux2c0p}

\begin{longtable}[]{@{}llllllllllll@{}}
\toprule
Gorum & Sora & Remo & Gutob & Kharia & Juang & Gtaʔ & Santali & Mundari
& Ho & Korwa & Korku\tabularnewline
\midrule
\endhead
ˀb & b & p(') & b & (ˀ)b & b & ʔ/g & p' & b & b & b/p & p\tabularnewline
\bottomrule
\end{longtable}

\subparagraph{\texorpdfstring{\emph{*vdʲu₅ˀp} `night'
(\#0033-4)}{*vdʲu₅ˀp night (\#0033-4)}}\label{vdux2b2uux2c0p-night-0033-4}

uɖuˀb, FR, orub, RSED.p195, minɖip', BDBH.2087, noNdib, GZ65.260, iɖiˀb,
PKED.p79, ---, ---, minɖig, PGEG.p33, ayup', CDES.p128, a:yub, BMED.p14,
ayub, HOGV.p157, ayub, BAHL.p3, ayup, NKEV.p290, , night, \#0033, V280,
1268,

\begin{itemize}
\tightlist
\item
  Pinnow 1959: V280 / MKCD: 1268 \emph{*yup}; \emph{*y{[}uu{]}p};
  \emph{*yəp}
\end{itemize}

\subparagraph{\texorpdfstring{\emph{*tVru₅ˀp} `cloud'
(\#0034-5)}{*tVru₅ˀp cloud (\#0034-5)}}\label{tvruux2c0p-cloud-0034-5}

taruˀb, FR, tarub, RSED.p283, tirib, BDBH.1387, tirib, GZ65.416, tiriˀb,
PKED.p287, ---, ---, trig, PGEG.p46, rimil, CDES.p33, rimil, BMED.p160,
rimil, HOGV.p152, liNbir, BAHL.p127, ---, ---, , cloud, \#0034, V285a, ,

\begin{itemize}
\tightlist
\item
  Pinnow 1959: V285a / MKCD: ---
\end{itemize}

Kherwarian \emph{rimil} cannot be directly related to \emph{*tVru₅ˀp} .
If /rim/ in Santali, Mundari, and Ho is related to /ɖi(ˀ)b/ in Remo,
Gutob, and Kharia, the two etyma might be ultimately related. However,
they cannot be derived from a common proto-Munda form. Korwa
\emph{liNbir} is probably not connected to either.

\subparagraph{\texorpdfstring{\emph{*saˀp} `grab (v)'
(\#0048-3)}{*saˀp grab (v) (\#0048-3)}}\label{saux2c0p-grab-v-0048-3}

---, ---, (sakab), RSED.p246, sop', BDBH.2748, sob, GGEG.p113, (suˀb),
PKED.p188, sɔb, PJDW.p277, saʔ, PGEG.p42, sap', CDES.p28, sa:b,
BMED.p163, sab, DHED.p296, sa:b, BAHL.pdfp131, sa:p, NKEV.p337, *sxˀp,
grab (v), \#0048, , ,

\begin{itemize}
\tightlist
\item
  MKCD 1236 \emph{*{[}c{]}kiip}; \emph{*{[}c{]}kiəp};
  \emph{*t{[}₁{]}kiəp}; \emph{*ckap}; \emph{*t₁kap}; \emph{ckuəp}
\item
  MKCD 1243 *cap; *caap; *ciəp; *cip; *cup
\end{itemize}

The connection to MKCD 1236 is not strong. Reflexes of \emph{*t₁} should
remain a stop, while the reflexes of the cluster \emph{*{[}c{]}k} are
not well understood. It could be a case of type 2a cluster splitting by
second consonant loss (CᵢCᵢᵢ → Cᵢ). Thus \emph{*ckap} → \emph{*cap} →
\emph{*saˀp} or \emph{*ckap} → \emph{*skap} → \emph{*saˀp}.

\subparagraph{\texorpdfstring{\emph{*Kʰaˀp} `bite (v)'
(\#0056-3)}{*Kʰaˀp bite (v) (\#0056-3)}}\label{kux2b0aux2c0p-bite-v-0056-3}

(kuˀb), FR, (küb/kib/kaib), RSED.p144, op, BDBH.337, op, ZG63.7, hapkay,
PKED.p73, ---, ---, haʔ, PGEG.p24, hap', CDES.p17, ha:b, BMED.p64, hab,
DHED.p124, ha:p, BAHL.p146, khap, NKEV.p320, *Kʰaˀp, bite (v), \#0056,
V294, 1231,

\begin{itemize}
\tightlist
\item
  Pinnow 1959: V294 / MKCD: 1231 \emph{*kap}/\emph{*kaap}
\end{itemize}

\subparagraph{\texorpdfstring{\emph{*geˀp} `to burn (vi)'
(\#0058-3)}{*geˀp to burn (vi) (\#0058-3)}}\label{geux2c0p-to-burn-vi-0058-3}

geˀb ,FR ,tuŋge:b ,RSED.p298 ,gep' ,BDBH.967 ,geb ,GZ65.123 ,geb
,PKED.p61 ,--- ,--- ,giʔ ,PGEG.p19 ,~--- ,--- ,--- ,--- ,--- ,--- ,---
,--- ,--- ,--- ,*geˀp ,burn (vi) ,\#0058 , 156, ,

\begin{itemize}
\tightlist
\item
  Pinnow 1959: 156 / MKCD: ---
\end{itemize}

\subparagraph{\texorpdfstring{\emph{*sVlVˀp} `gazelle'
(\#0084-5)}{*sVlVˀp gazelle (\#0084-5)}}\label{svlvux2c0p-gazelle-0084-5}

alu'b, FR, əle:b, RSED.p7, sulup, BDBH.2688, sulub, GGEG.p116, selhob,
PKED.p180, silib, PJDW.p278, sloʔ, PGEG.p43, selep', CSED.p571, silib,
BMED.p173, silib, DHED.p317, seleb, DSKW@21960, ---, ---, *sxlxˀp,
gazelle, \#0084, V233, ,

\paragraph{\texorpdfstring{\emph{*ˀb} \textasciitilde{} \emph{*ˀk}
(\emph{*ˀb/k})}{*ˀb \textasciitilde{} *ˀk (*ˀb/k)}}\label{ux2c0b-ux2c0k-ux2c0bk}

There is a small but remarkable number of etyma in which some variation
between reflexes of \emph{*ˀb} and \emph{*ˀk}. It is not clear whether
this is a case of coda neutralization or there is an underlying
variation in proto-Munda and the etymo have to be reconstructed as
having a variant with \emph{*ˀb} and one with \emph{*ˀk}.

Gtaʔ shows consistent coda neutralization as \emph{*ˀb} is generally
continued as /ʔ/ or -- probably a notational variation -- as /g/. If
etymoa are to be reconstructed

See also Pinnow (1959, p.~378)

\begin{longtable}[]{@{}lllllllllllll@{}}
\toprule
Gorum & Sora & Remo & Gutob & Kharia & Juang & Gtaʔ & Santali & Mundari
& Ho & Korwa & Korku &\tabularnewline
\midrule
\endhead
ˀb & ˀb & b & b & ˀb & b & ʔ & k' & ∅ & ʔ & ʔ & --- &
0011-5\tabularnewline
ˀb & ∅ & k' & --- & --- & (k) & ʔ & ∅ & (k) & (k) & ∅ & --- &
0068-6\tabularnewline
∅ & ∅ & ∅ & g & ˀb & b & ʔ & --- & --- & --- & --- & --- &
0068-6\tabularnewline
∅ & ʔ & g & ʔ & (∅) & --- & g & p' & b' & b & b & p &
0099-2\tabularnewline
\bottomrule
\end{longtable}

\begin{longtable}[]{@{}lllllllllllll@{}}
\toprule
Gorum & Sora & Remo & Gutob & Kharia & Juang & Gtaʔ & Santali & Mundari
& Ho & Korwa & Korku &\tabularnewline
\midrule
\endhead
B & B & B & B & B & B & B/V & V & ∅ & V & V & --- &
0011-5\tabularnewline
B & ∅ & V & --- & --- & (V) & B/V & ∅ & (V) & (V) & ∅ & --- &
0068-6\tabularnewline
∅ & ∅ & ∅ & V & B & B & B/V & --- & --- & --- & --- & --- &
0068-6\tabularnewline
∅ & G & G & G & (∅) & --- & B/V & B & B & B & B & B &
0099-2\tabularnewline
\bottomrule
\end{longtable}

\subparagraph{\texorpdfstring{\emph{*b(oKʰ)Oˀp} `head'
(\#0011-5)}{*b(oKʰ)Oˀp head (\#0011-5)}}\label{bokux2b0oux2c0p-head-0011-5}

baˀb, FR, bo:ˀb, RSED.p60, bob, BDBH.2007, bob, GZ63.50, bokoˀb,
PKED.p24, bokob, PJDW.p169, bhaʔ, PGEG.p13, bo̠ho̠k', CDES.p90, bo,
BMED.p24, bo:ʔ, DHED.p40, boʔ, BAHL.p113, ---, ---, *b(oKʰ)Oˀp, head,
\#0011, V206, 361, 38

\begin{itemize}
\tightlist
\item
  Pinnow 1959: V206 UM *ɔ/ MKCD: 361 \emph{*{[}b{]}uuk}
\end{itemize}

\begin{longtable}[]{@{}llllllllllll@{}}
\toprule
Gorum & Sora & Remo & Gutob & Kharia & Juang & Gtaʔ & Santali & Mundari
& Ho & Korwa & Korku\tabularnewline
\midrule
\endhead
ˀb & ˀb & b & b & ˀb & b & ʔ & k' & ∅ & ʔ & ʔ & ---\tabularnewline
\bottomrule
\end{longtable}

\subparagraph{\texorpdfstring{\emph{*bVrV(ˀp/ˀk)} `lung'
(\#0066-1)}{*bVrV(ˀp/ˀk) lung (\#0066-1)}}\label{bvrvux2c0pux2c0k-lung-0066-1-1}

buroˀb, FR, bəro:, RSED.p46, buruk', BDBH.1936, ---, ---, ---, ---,
(buku), JLIC.n49, breʔ, PGEG.p14, bo̠ro̠, CDES.p116, (borkod'), BMED.p25,
(borkoɖ), DHED.p45, boro, BAHL.p112, , , , lungs, \#0066, , ,

\begin{itemize}
\tightlist
\item
  Pinnow 1959: --- / MKCD: ---
\end{itemize}

\begin{longtable}[]{@{}llllllllllll@{}}
\toprule
Gorum & Sora & Remo & Gutob & Kharia & Juang & Gtaʔ & Santali & Mundari
& Ho & Korwa & Korku\tabularnewline
\midrule
\endhead
ˀb & ∅ & k' & --- & --- & (k) & ʔ & ∅ & (k) & (k) & ∅ &
---\tabularnewline
\bottomrule
\end{longtable}

\subparagraph{\texorpdfstring{\emph{*ruNkO(ˀp)} `husked rice'
(\#0068-6)}{*ruNkO(ˀp) husked rice (\#0068-6)}}\label{runkoux2c0p-husked-rice-0068-6}

ruŋk, FR, rʊŋkʊ, RSED.p239, ruŋku, BDBH.2291, rukug, AG08.p672,
ruŋkuˀb/rumkuˀb, PKED.p171, ruŋkub, PJDW.p269, rkoʔ, PGEG.p41, ---, ---,
(rukhaɽ), BMED.p163, ---, ---, ---, ---, ---, ---, *ruNkO(ˀp), husked
rice, \#0068, V139, 1820,

\begin{itemize}
\tightlist
\item
  Pinnow 1959: V139 / MKCD: 1820 \emph{*rk{[}aw{]}ʔ}
\end{itemize}

Gtaʔ /rkoʔ/ is surprisingly close th Shorto's \emph{*rk{[}aw{]}ʔ}.

\begin{longtable}[]{@{}llllllllllll@{}}
\toprule
Gorum & Sora & Remo & Gutob & Kharia & Juang & Gtaʔ & Santali & Mundari
& Ho & Korwa & Korku\tabularnewline
\midrule
\endhead
∅ & ∅ & ∅ & g & ˀb & b & ʔ & --- & --- & --- & --- & ---\tabularnewline
\bottomrule
\end{longtable}

\subparagraph{\texorpdfstring{\emph{*uˀp} (\emph{*uˀk/*uuˀ}) `hair'
(\#0099-2)}{*uˀp (*uˀk/*uuˀ) hair (\#0099-2)}}\label{uux2c0p-uux2c0kuuux2c0-hair-0099-2}

---, FR, uʔ/(uppur), RSED.p308(307), ugbok', BDBH.135, iʔboʔ,
DSGU\#9411, (ului), DSKH\#32441, ---, ---, ugboʔ/ogboʔ, PGEG.p6, up',
CSED.p670, ub', BMED.p191, ub, DHED.p369, u:b, BAHL.p18, hu:p,
NKEV.p310, *uˀp, hair, \#0099, V143, ,

\begin{itemize}
\tightlist
\item
  Pinnow 1959: V143 / MKCD: ---
\end{itemize}

\begin{longtable}[]{@{}llllllllllll@{}}
\toprule
Gorum & Sora & Remo & Gutob & Kharia & Juang & Gtaʔ & Santali & Mundari
& Ho & Korwa & Korku\tabularnewline
\midrule
\endhead
∅ & ʔ & g & ʔ & (∅) & --- & g & p' & b' & b & b & p\tabularnewline
\bottomrule
\end{longtable}

\paragraph{\texorpdfstring{\emph{*m₁}
(Onset)}{*m₁ (Onset)}}\label{m-onset}

\begin{longtable}[]{@{}llllllllllll@{}}
\toprule
Gorum & Sora & Remo & Gutob & Kharia & Juang & Gtaʔ & Santali & Mundari
& Ho & Korwa & Korku\tabularnewline
\midrule
\endhead
m & m & m & m & m & m & m & m & m & m & m & m\tabularnewline
\bottomrule
\end{longtable}

\subparagraph{\texorpdfstring{\emph{*məˀt} `eye'
(\#0012-1)}{*məˀt eye (\#0012-1)}}\label{mux259ux2c0t-eye-0012-1}

maˀd, FR, mo:ˀd/mad, RSED.p168, moʔ, BDBH.220, moʔ, AG08.p642, moˀɖ,
PKED.p195, ɛmɔɖ, PJDW.p191, muaʔ, PGEG.p34, mẽ̠t', CDES.p67, med',
BMED.p117, meɖ, DHED.p228, meɖ, BAHL.p120, med, ZKPM.p48, məˀt, eye,
\#0012, V250, 1045, 40

\begin{itemize}
\tightlist
\item
  Pinnow 1959: V250 / MKCD: 1045 *mat
\end{itemize}

\subparagraph{\texorpdfstring{\emph{*mv₍₄₎raŋ} `big'
(\#0064-1)}{*mv₍₄₎raŋ big (\#0064-1)}}\label{mvraux14b-big-0064-1}

---, ---, maraŋ/məraŋ, RSED.p173/167, munaʔ, BDBH.2121, (moɖo),
AG08.p663, ---, ---, ---, ---, mnaʔ, PGEG.35, maraŋ, CDES.p17, maraŋ,
BMED.p220, maraŋ, DHED.p225, ---, ---, ---, ---, *mxrxŋ, big, \#0064,
K107, ,

\begin{itemize}
\tightlist
\item
  Pinnow 1959: K107 / MKCD: ---
\end{itemize}

Gtaʔ \emph{mnaʔ} and Remo \emph{munaʔ} are irregular reflexes of
\emph{*mv₍₄₎raŋ}, especially the Gtaʔ form \emph{mnaʔ} should be
different, given our current understanding of the phonological
developments.

\subparagraph{\texorpdfstring{\emph{*muuˀ} `nose'
(\#0074-1)}{*muuˀ nose (\#0074-1)}}\label{muuux2c0-nose-0074-1}

muʔ, FR, mu:ʔ, RSED.p179, nseʔmiʔ, BDBH.1653, miʔ, GZ63.262,
romoŋ/romoˀɖ, PKED.p170, motɛɟ, PJDW.p245, mmu, PGEG.p34, muN,
CDES.p129, mu/muhu, BMED.p121, muwa/muʈa, DHED.p238, hu:mu:, DSKW@23180,
mu:, NKEV.p327, *mxxˀ, nose, \#0074, , ,

\begin{itemize}
\tightlist
\item
  Pinnow 1959: V387 / MKCD: 2045 \emph{*muh}; \emph{*muuh}; \emph{*muus}
\end{itemize}

\subparagraph{\texorpdfstring{\emph{*maraˀk} `peacock' V₂
(\#0081-1)}{*maraˀk peacock V₂ (\#0081-1)}}\label{maraux2c0k-peacock-v-0081-1}

(marraʔ), FR, ma:ra:, RSED.p173, ---, ---, ---, ---, maraʔ, PKED.p131,
marag, PJDW.p242, ---, ---, marak', CSED.p407, ma:ra:, BMED.p114, mara:,
DHED.p225, mara:q, BAHL.p117, mara, NKEV.p324, *maraˀk, peacock, \#0081,
V27, 416,

\begin{itemize}
\tightlist
\item
  Pinnow 1959: V27 / MKCD: 416 \emph{*mraik{[} {]}}
\end{itemize}

Gorum \emph{marraʔ} `husband' probably belongs to another etymon
connected with MKCD 183 \emph{*mraʔ} `person'.

\paragraph{\texorpdfstring{\emph{*m₂}
(Onset)}{*m₂ (Onset)}}\label{m-onset-1}

\begin{longtable}[]{@{}llllllllllll@{}}
\toprule
Gorum & Sora & Remo & Gutob & Kharia & Juang & Gtaʔ & Santali & Mundari
& Ho & Korwa & Korku\tabularnewline
\midrule
\endhead
m & m & --- & ∅ & ∅ & ∅ & m & m & m & m & m & m\tabularnewline
\bottomrule
\end{longtable}

\subparagraph{\texorpdfstring{\emph{*miɲam}/\emph{*mayOm} `blood'
(\#0059-1)}{*miɲam/*mayOm blood (\#0059-1)}}\label{miux272ammayom-blood-0059-1}

miam, FR, miɲam, RSED.p177, ---, ---, iam, GZ63.325, iɲam, PKED.p115,
iɲam/iɲɑm, PJDW.p208, mia, PGEG.p33, maNyaNm, CDES.p18, ma:yom,
BMED.p116, mayom, HOGV.p149, , , mayum, NKEV.p325, , blood, \#0059,
V303, 1430,

\begin{itemize}
\tightlist
\item
  Pinnow 1959: V303 / MKCD: 1430 \emph{*jhaam}; \emph{*jhiim}
\end{itemize}

\paragraph{\texorpdfstring{\emph{*m}
(Medial)}{*m (Medial)}}\label{m-medial}

\begin{longtable}[]{@{}llllllllllll@{}}
\toprule
Gorum & Sora & Remo & Gutob & Kharia & Juang & Gtaʔ & Santali & Mundari
& Ho & Korwa & Korku\tabularnewline
\midrule
\endhead
m & m(m) & m & m & m & m & m/∅ & m & m & m & m & m(m)\tabularnewline
\bottomrule
\end{longtable}

\subparagraph{\texorpdfstring{\emph{*sVmaŋ} `forehead/front'
(\#0038-3)}{*sVmaŋ forehead/front (\#0038-3)}}\label{svmaux14b-foreheadfront-0038-3}

amaŋ, FR, ammaŋ, RSED.p31, gutumoŋ, BDBH.885, sumoŋ/amuŋ, GZ65.21,
somoŋ/somo/sumaŋ, PKED.p185, ɛmɔŋ, PJDW.p191, ssæ, PGEG.p44, samaŋ,
CDES.p79, sa:ma:ŋ, BMED.p167, sanamaŋ, HOGV.p159, samaŋ, BAHL.pdfp130,
samma, NKEV.p336, , forehead/front, \#0038, V269, ,

\begin{itemize}
\tightlist
\item
  Pinnow 1959: V269 / MKCD: ---
\end{itemize}

\subparagraph{\texorpdfstring{\emph{*tᵊme} `new' V₁
(\#0053-3)}{*tᵊme new V₁ (\#0053-3)}}\label{tux1d4ame-new-v-0053-3}

tḛmi, FR, tamme, RSED.p277, time, BDBH.1383, time, ZG65.410, tonme,
PKED.p289, ---, ---, tmi, PGEG.p125, (nãwã), CDES.p128, (nawã),
BMED.p127, (nama), HOGV.p168, ---, ---, (uni), NKEV.345, *tᵊme, new,
\#0053, V184, 144,

\begin{itemize}
\tightlist
\item
  Pinnow 1959: V182 / MKCD: 144 \emph{*t₁miʔ}
\end{itemize}

\paragraph{\texorpdfstring{\emph{*m₁} (Coda)}{*m₁ (Coda)}}\label{m-coda}

\begin{longtable}[]{@{}llllllllllll@{}}
\toprule
Gorum & Sora & Remo & Gutob & Kharia & Juang & Gtaʔ & Santali & Mundari
& Ho & Korwa & Korku\tabularnewline
\midrule
\endhead
m & m & m & m & m & m & ŋ & m & m & m & m & m\tabularnewline
\bottomrule
\end{longtable}

Gtaʔ /ŋ/ is the coda neutralization also attested in \emph{*ˀp}.

\subparagraph{\texorpdfstring{\emph{*ɟv₍₁₀₎m} `eat (v)'
(\#0015-3)}{*ɟv₍₁₀₎m eat (v) (\#0015-3)}}\label{ux25fvm-eat-v-0015-3}

zum, FR, ɟom, RSED.p128, sum, BDBH.2667, som, GZ63.212, jom, HLKS.K274,
ɟim, PJDW.p212, coŋ, PGEG.p15, jo̠m, CDES.p60, jom, BMED.p84, jom,
HOGV.p156, jom, BAHL.p63, jom, NKEV.p313, , eat (v), \#0015, V385, 1327,
55

\begin{itemize}
\tightlist
\item
  Pinnow 1959: V385 / MKCD: 1327 \emph{*cuum}; \emph{*cuəm};
  \emph{*cəm}; (\emph{*cim cim} \textgreater{}) \emph{*ncim};
  \emph{*ciəm} (\& \emph{*nciəm}?); \emph{*caim}
\end{itemize}

\subparagraph{\texorpdfstring{\emph{*gam} `say (v)'
(\#0080-3)}{*gam say (v) (\#0080-3)}}\label{gam-say-v-0080-3}

---, ---, gam, RSED.p96, ---, ---, gam, Z1965.121, gam, PKED.p57, gam,
PJDW.p194, ---, ---, gam, CSED.p176, gamu, HLKS.V12, gamu, HLKS.V12,
---, ---, ---, ---, *gam, say (v), \#0080, V12, ,

\begin{itemize}
\tightlist
\item
  Pinnow 1959: V12 / MKCD: ---
\end{itemize}

\subparagraph{\texorpdfstring{\emph{*ɲam} `get (v)'
(\#0088-3)}{*ɲam get (v) (\#0088-3)}}\label{ux272am-get-v-0088-3}

---, ---, ɲam, RSED.p186, ---, ---, ---, ---, ɲam, PKED.p140, ---, ---,
---, ---, ɲam, CSED.p434, na:m, BMED.p126, nam, DHED.p241, ɲa:m,
BAHL.p66, na, NKEV.p327, *ɲxm, get (v), \#0088, 5(?), 1243(?),

\begin{itemize}
\tightlist
\item
  Pinnow 1959: 5(?) / MKCD: ---
\end{itemize}

\paragraph{\texorpdfstring{\emph{*m₂}
(Coda)}{*m₂ (Coda)}}\label{m-coda-1}

\begin{longtable}[]{@{}llllllllllll@{}}
\toprule
Gorum & Sora & Remo & Gutob & Kharia & Juang & Gtaʔ & Santali & Mundari
& Ho & Korwa & Korku\tabularnewline
\midrule
\endhead
m & m & --- & m & m & m & ∅ & m & m & m & --- & m\tabularnewline
\bottomrule
\end{longtable}

\subparagraph{\texorpdfstring{\emph{*miɲam}/\emph{*mayOm} `blood'
(\#0059-5)}{*miɲam/*mayOm blood (\#0059-5)}}\label{miux272ammayom-blood-0059-5}

miam, FR, miɲam, RSED.p177, ---, ---, iam, GZ63.325, iɲam, PKED.p115,
iɲam/iɲɑm, PJDW.p208, mia, PGEG.p33, maNyaNm, CDES.p18, ma:yom,
BMED.p116, mayom, HOGV.p149, , , mayum, NKEV.p325, , blood, \#0059,
V303, 1430,

\begin{itemize}
\tightlist
\item
  Pinnow 1959: V303 / MKCD: 1430 \emph{*jhaam}; \emph{*jhiim}
\end{itemize}

\subparagraph{\texorpdfstring{\emph{*ɲUm} `name'
(\#0100-3)}{*ɲUm name (\#0100-3)}}\label{ux272um-name-0100-3}

inum, FR, əɲam, RSED.p12, nimi, BDBH.1588, imi, AG.p645, (i)ɲimi,
PKED.p140, ---, ---, mni, PGEG.p35, ɲum/ɲutum, CSED.p451/452, num/nutum,
BMED.134, numu/nutum, DHED.p249, ---, ---, jimu, , *ɲUm, name, \#0100,
V279, 147,

\begin{itemize}
\tightlist
\item
  Pinnow 1959: V319 / MKCD: 259 \emph{*{[}hy{]}muʔ}
\end{itemize}

\paragraph{\texorpdfstring{\emph{*m₃}
(Coda)}{*m₃ (Coda)}}\label{m-coda-2}

\begin{longtable}[]{@{}llllllllllll@{}}
\toprule
Gorum & Sora & Remo & Gutob & Kharia & Juang & Gtaʔ & Santali & Mundari
& Ho & Korwa & Korku\tabularnewline
\midrule
\endhead
m & m & m & m & m & ŋ/ɲ & ŋ & m & m & m & m & m\tabularnewline
\bottomrule
\end{longtable}

Juang /ŋ/\textasciitilde{}/ɲ/ is unclear. Gtaʔ /ŋ/ is the coda
neutralization also attested in \emph{*m₁} and \emph{*m₂}.

\subparagraph{\texorpdfstring{\emph{*gum} `winnow (v)'
(\#0044-3)}{*gum winnow (v) (\#0044-3)}}\label{gum-winnow-v-0044-3}

gumar, FR, gum, RSED.p105, (giteʔ), BDBH.864, gim, GZ63.134, gum,
PKED.p67, guŋ/guɲ, PJDW.p199, goŋ, PGEG.p20, gum, BSDV2.p490, gum,
BMED.p214, gum, DHED.p120, gum, BAHL.p45, gum, NKEV.p307, *gum, winnow
(v), \#0044, K159, 1317,

\begin{itemize}
\tightlist
\item
  Pinnow 1959: K159 / MKCD: 1317 \emph{*gum}; \emph{*guum};
  \emph{*g{[}əə{]}m}
\end{itemize}

\paragraph{\texorpdfstring{\emph{*m₄}
(Coda)}{*m₄ (Coda)}}\label{m-coda-3}

\begin{longtable}[]{@{}llllllllllll@{}}
\toprule
Gorum & Sora & Remo & Gutob & Kharia & Juang & Gtaʔ & Santali & Mundari
& Ho & Korwa & Korku\tabularnewline
\midrule
\endhead
--- & m & ŋ & ŋ & ŋ & ŋ & ŋ & m & m & m & m & m\tabularnewline
\bottomrule
\end{longtable}

Velar /ŋ/ in Remo, Gutob, Kharia, and probably Juang not a coda
neutralization as it seems to be in Gtaʔ. However, the context affecting
this sound change is unknown so far.

\subparagraph{\texorpdfstring{\emph{*si₂m} `chicken'
(\#0028-3)}{*si₂m chicken (\#0028-3)}}\label{sim-chicken-0028-3}

---, ---, kənsi:m, RSED.p131, gisiŋ, BDBH.856, gisiŋ, AG08.p651, siŋkoy,
PKED.p183, sɛŋkɔe, PJDW.p275, gsæŋ, PGEG.p23, sim, CDES.p30, sim,
BMED.p173, sim, HOGV.p151, si:m, BAHL.p135, ---, ---, *sxm, chicken,
\#0028, V315, 1324,

\begin{itemize}
\tightlist
\item
  Pinnow 1959: V315 / MKCD: 1324 \emph{*cim}; \emph{*ciim};
  \emph{*ciəm}; \emph{*caim}; \emph{*cum}
\end{itemize}

\subsubsection{Alveolar/Dental/Retroflex}\label{alveolardentalretroflex}

\begin{longtable}[]{@{}llllll@{}}
\toprule
voiceless & voiced & glottalized & nasal & sibilant &
unclear\tabularnewline
\midrule
\endhead
\emph{*t₁}/\emph{*t₂} & \emph{*d} & \emph{*ˀt} & \emph{*n} & \emph{*s}
(see below) & (\emph{*dʲ})\tabularnewline
\bottomrule
\end{longtable}

\begin{longtable}[]{@{}lllllllllllll@{}}
\toprule
Gorum & Sora & Remo & Gutob & Kharia & Juang & Gtaʔ & Santali & Mundari
& Ho & Korwa & Korku &\tabularnewline
\midrule
\endhead
t & t & t & t & t & t & t/ʈ & t & t/ʈh & t & t & ʈ & \emph{*t₁}
onset\tabularnewline
s & s & t & t & t & t & t & t & t & t & t & ʈ & \emph{*t₂}
onset\tabularnewline
\bottomrule
\end{longtable}

\begin{longtable}[]{@{}lllllllllllll@{}}
\toprule
Gorum & Sora & Remo & Gutob & Kharia & Juang & Gtaʔ & Santali & Mundari
& Ho & Korwa & Korku &\tabularnewline
\midrule
\endhead
ɖ & d & d/ɖ & ɖ & ɖ & ɖ & ɖ/d & d/ɖ & d/ɖ & d & ɖ & ɖ/d & \emph{*d}
onset\tabularnewline
ɖ & --- & --- & ɖ & ɖ & r & ∅ & ɖ & ɖ & d & d & ɖ & \emph{*d}
medial\tabularnewline
ɖ & r & ɖ & d & ɖ & ɖ & ɖ & y & y & y & y & y & \emph{*dʲ}
medial\tabularnewline
\bottomrule
\end{longtable}

(\emph{*dʲ} is probably spurious and baed n the conflation a etyma with
a /d/ and some with /j/)

\begin{longtable}[]{@{}lllllllllllll@{}}
\toprule
Gorum & Sora & Remo & Gutob & Kharia & Juang & Gtaʔ & Santali & Mundari
& Ho & Korwa & Korku &\tabularnewline
\midrule
\endhead
ˀd & ˀd & --- & ʔ/ɖ/ɽ & ˀɖ & ɖ/r & ʔ/g & t' & d' & ɖ & ɖ & d/ʈ &
\emph{*ˀt}\tabularnewline
\bottomrule
\end{longtable}

\begin{longtable}[]{@{}lllllllllllll@{}}
\toprule
Gorum & Sora & Remo & Gutob & Kharia & Juang & Gtaʔ & Santali & Mundari
& Ho & Korwa & Korku &\tabularnewline
\midrule
\endhead
n & n & n/ŋ & n & n & --- & ŋ? & n & n & n & n/ŋ & n & \emph{*n₁}
coda\tabularnewline
n & n & ŋ & n & n & --- & ŋ & n & n & n & ŋ & n & \emph{*n₂}
coda\tabularnewline
\bottomrule
\end{longtable}

\paragraph{\texorpdfstring{\emph{*t₁}}{*t₁}}\label{t}

\begin{longtable}[]{@{}llllllllllll@{}}
\toprule
Gorum & Sora & Remo & Gutob & Kharia & Juang & Gtaʔ & Santali & Mundari
& Ho & Korwa & Korku\tabularnewline
\midrule
\endhead
t & t & t & t & t & t & t/ʈ & t & t/ʈh & t & t & ʈ\tabularnewline
\bottomrule
\end{longtable}

\subparagraph{\texorpdfstring{\emph{*taɲ} `to weave'
(\#0005-1)}{*taɲ to weave (\#0005-1)}}\label{taux272-to-weave-0005-1}

taɲ, FR, taɲ, RSED.p281, taNy, BDBH.1358, taɲ, GZ65.369, taɲ, PKED.p196,
---, ---, tæ, PGEG.p45, teɲ, CDES.p219, teŋ, BMED.p183, teɲ, HOGV.p187,
---, ---, ---, ---, taɲ, weave (v), \#0005, V301, 898,

\begin{itemize}
\tightlist
\item
  Pinnow 1959: V301 / MKCD: 898 \emph{*t₁aaɲ}
\end{itemize}

\subparagraph{\texorpdfstring{\emph{*tol} `tie (v)'
(\#0024-1)}{*tol tie (v) (\#0024-1)}}\label{tol-tie-v-0024-1}

tol, FR, tol, RSED.p292, tu, BDBH.1398, tol, AG08.647, tol, PKED.p288,
tor, PJDW.p287, tu, PGEG.p46, to̠l, CDES.p201, tol, BMED.p186, tol,
HOGV.p183, tol, BAHL.p84, ʈol, NKEV.p343, *tol, tie (v), \#0024, V191, ,

\begin{itemize}
\tightlist
\item
  Pinnow 1959: V191 / MKCD: ---
\end{itemize}

\subparagraph{\texorpdfstring{\emph{*tuɲ} `shoot (v)'
(\#0027-1)}{*tuɲ shoot (v) (\#0027-1)}}\label{tuux272-shoot-v-0027-1}

tiŋ, FR, tuɲ, RSED.p299, tiŋ, BDBH.1368, tiŋ, GZ63.190, tuɲ, PKED.p196,
tuɲ, PJDW.p288, ʈwiŋ, PGEG.p46, tuɲ, CDES.p173, tuiŋ, BMED.p180, tuŋ,
HOGV.p177, , , ʈuɲj, NKEV.p343, , shoot (v), \#0027, V107, 896a?,

\begin{itemize}
\tightlist
\item
  Pinnow 1959: V107 / MKCD: 896a?
\end{itemize}

MKCD 896a \emph{*t₁iɲ}; \emph{*t₁iiɲ}; \emph{*t₁iəɲ}; \emph{*t₁əɲ} `to
pluck, twang' could be related. Its meaning is generally `to pluck a
(stringed) instrument' which is rather close to `to shoot with bow and
arrow.'

\subparagraph{\texorpdfstring{\emph{*tVru₅ˀp} `cloud'
(\#0034-1)}{*tVru₅ˀp cloud (\#0034-1)}}\label{tvruux2c0p-cloud-0034-1}

taruˀb, FR, tarub, RSED.p283, tirib, BDBH.1387, tirib, GZ65.416, tiriˀb,
PKED.p287, ---, ---, trig, PGEG.p46, rimil, CDES.p33, rimil, BMED.p160,
rimil, HOGV.p152, liNbir, BAHL.p127, ---, ---, , cloud, \#0034, V285a, ,

\begin{itemize}
\tightlist
\item
  Pinnow 1959: V285a / MKCD: ---
\end{itemize}

\subparagraph{\texorpdfstring{\emph{*bVtoŋ} `fear'
(\#0039-3)}{*bVtoŋ fear (\#0039-3)}}\label{bvtoux14b-fear-0039-3}

butoŋ, FR, bato:ŋ, RSED.p55, butuŋ, BDBH.1922, butoŋ, GZ65.76, bɔtɔŋ
(P), HLKS.V261, betɔŋan, JLIC.v239, bʈoʔ, PGEG.p14, ---, ---, botoŋ,
BMED.p25, ---, ---, (bor), BAHL.p112, ---, ---, *bxtoŋ, fear, \#0039,
V261, 552,

\begin{itemize}
\tightlist
\item
  Pinnow 1959: V261 / MKCD: 552 \emph{*ʔt₁uuŋ}
\end{itemize}

MKCD 552 \emph{*ʔt₁uuŋ} is a close -- but not perfect -- match. Intial
/b/ is attested in all languages. Shorto's initial cluster \emph{*ʔt₁}
is not easily connected to any proto-Munda \emph{*bVt}.

\subparagraph{\texorpdfstring{\emph{*tI₍₂₄₎l} `bury (v)'
(\#0049-1)}{*tI₍₂₄₎l bury (v) (\#0049-1)}}\label{til-bury-v-0049-1}

tul, FR, til, RSED.p288, ti, BDBH.1360, til, GZ65.408, til, PKED.p199,
tir, PJDW.p284, ---, ---, (til), RSED.p288, (til), RSED.p288, ---, ---,
ti:l, BAHL.p82, ---, ---, *txl, bury (v), \#0049, ---, ---,

\begin{itemize}
\tightlist
\item
  Pinnow 1959: --- / MKCD: ---
\end{itemize}

\subparagraph{\texorpdfstring{\emph{*tv₍₁₉₎ŋv₍₂₅₎n/tv₍₁₉₎nv₍₂₅₎ŋ} `stand
(v)' V₁
(\#0050-1)}{*tv₍₁₉₎ŋv₍₂₅₎n/tv₍₁₉₎nv₍₂₅₎ŋ stand (v) V₁ (\#0050-1)}}\label{tvux14bvntvnvux14b-stand-v-v-0050-1}

tinaŋ, FR, tanaŋ, RSED.p, toŋ, BDBH.1490, tunon, AG08.p662, tuŋon,
PKED.p201, toŋon, PJDW.p287, thwaN, PGEG.p46, teŋgon, CDES.p186, tiŋun,
BMED.p187, tiŋgu, HOGV.p180, ---, ---, ʈengene, NKEV.p342, *txŋxn, stand
(v), \#0050, V258, 1824,

\begin{itemize}
\tightlist
\item
  Pinnow 1959: V258 / MKCD: 1824 \emph{*taaw}
\end{itemize}

\subparagraph{\texorpdfstring{\emph{*tᵊme} `new' V₁
(\#0053-1)}{*tᵊme new V₁ (\#0053-1)}}\label{tux1d4ame-new-v-0053-1}

tḛmi, FR, tamme, RSED.p277, time, BDBH.1383, time, ZG65.410, tonme,
PKED.p289, ---, ---, tmi, PGEG.p125, (nãwã), CDES.p128, (nawã),
BMED.p127, (nama), HOGV.p168, ---, ---, (uni), NKEV.345, *tᵊme, new,
\#0053, V184, 144,

\begin{itemize}
\tightlist
\item
  Pinnow 1959: V182 / MKCD: 144 \emph{*t₁miʔ}
\end{itemize}

\subparagraph{\texorpdfstring{\emph{*boŋtel}/\emph{*bitkil} `buffalo'
(\#0054-4)}{*boŋtel/*bitkil buffalo (\#0054-4)}}\label{boux14btelbitkil-buffalo-0054-4}

boŋtel, FR, boŋtel, RSED.p62, buŋte, BDBH.1917, boŋtel, AG08.p647,
boŋtel, PKED.p36, ---, ---, buNʈi, PGEG.p13, bitkil, CDES.p23, ---, ---,
---, ---, ---, ---, biʈkhil, NKEV.p294, *bxŋtxl, buffalo, \#0054, , ,

Probably two separate etyma \emph{*boŋtel} and North Munda *bitkil\_.
The form suggests some relation, but the two forms cannot be derived
from proto-Munda by regular sound change.

\subparagraph{\texorpdfstring{\emph{*tVŋ} `kindle (v)'
(\#0062-1)}{*tVŋ kindle (v) (\#0062-1)}}\label{tvux14b-kindle-v-0062-1}

tuŋ, FR, tuŋa:l, RSED.p297, ---, ---, ---, ---, tuŋgal, HLKS.V324, ---,
---, toŋ, PGEG.p42, tiŋgi, BSDV5.p461, tiŋ, BMED.p187, tiɲ, DHED.p353,
---, ---, ʈingi, NKEV.p343, *txŋ, kindle (v), \#0062, V324, 549,

\begin{itemize}
\tightlist
\item
  Pinnow 1959: V324; VW i/u; UM:i/ MKCD: 549 \emph{*t₁uuŋ}
\end{itemize}

\subparagraph{\texorpdfstring{\emph{*tVrel} `ebony' V₁
(\#0083-1)}{*tVrel ebony V₁ (\#0083-1)}}\label{tvrel-ebony-v-0083-1}

---, ---, tarel, RSED.p138, tire, BDBH.1390, ---, ---, ti(ɽr)(ei)l,
PKED.p200, tɛrɛn, PJDW.p285, tre, PGEG.p46, terel, CSED.p626, tiril,
BMED.p188, tiril, DHED.p355, ---, ---, ---, ---, *txrel, ebony, \#0083,
V227, ,

\begin{itemize}
\tightlist
\item
  Pinnow 1959: V227 / MKCD: ---
\end{itemize}

\subparagraph{\texorpdfstring{\emph{*ten} `trample (v)'
(\#0086-1)}{*ten trample (v) (\#0086-1)}}\label{ten-trample-v-0086-1}

tin, FR, ---, ---, ---, ---, ten, Z1965.402, ten, PKED.p199, ---, ---,
(ʈe), PGEG.p46, ten, CSED.p624, ʈhen, BMED.p185, ten, DHED.p347, ten,
DSKW@1275, ten, DSKO\#26831, *txn, trample (v), \#0086, K306, 1153a,

\begin{itemize}
\tightlist
\item
  Pinnow 1959: K306 / MKCD: 1153a \emph{*t₁een}
\end{itemize}

\begin{longtable}[]{@{}llllllllllll@{}}
\toprule
Gorum & Sora & Remo & Gutob & Kharia & Juang & Gtaʔ & Santali & Mundari
& Ho & Korwa & Korku\tabularnewline
\midrule
\endhead
t & --- & --- & t & t & --- & (ʈ) & t & ʈh & t & t & ʈ\tabularnewline
\bottomrule
\end{longtable}

Gtaʔ \emph{ʈe} is problematic. The retroflex /ʈ/ is unexplained, as is
Mundari /ʈh/.

\paragraph{\texorpdfstring{\emph{*t₂}
(Onset)}{*t₂ (Onset)}}\label{t-onset}

\begin{longtable}[]{@{}llllllllllll@{}}
\toprule
Gorum & Sora & Remo & Gutob & Kharia & Juang & Gtaʔ & Santali & Mundari
& Ho & Korwa & Korku\tabularnewline
\midrule
\endhead
s & s & t & t & t & t & t & t & t & t & t & ʈ\tabularnewline
\bottomrule
\end{longtable}

\emph{*t₂} is not well attest enough to be reliably characterized.

\begin{itemize}
\tightlist
\item
  current hypothesis proto-Munda \emph{*ti} \textgreater{}
  proto-Sora-Gorum \emph{*si}
\end{itemize}

\subparagraph{\texorpdfstring{\emph{*tiiˀ} `hand'
(\#0008-1)}{*tiiˀ hand (\#0008-1)}}\label{tiiux2c0-hand-0008-1}

siʔ, FR, si:ʔ, RSED.p254, titi, BDBH.1370, titi, GZ65.p29, tiʔ,
PKED.p199, iti, PJDW.p208, nti, PGEG.p37, ti, CDES.p89, ti, BMED.p186,
ti:, DHED.p350, tiʔi:, BAHL.p63, ʈi, NKEV.p343, tiiˀ, hand, \#0008, V75,
66, 48

\begin{itemize}
\tightlist
\item
  Pinnow 1959: V75 / MKCD: 66 \emph{*t₁iiʔ}
\end{itemize}

\paragraph{\texorpdfstring{\emph{*t₁/*ˀt}}{*t₁/*ˀt}}\label{tux2c0t}

\subparagraph{\texorpdfstring{\emph{*lutu(uˀ)r} `ear' V₁ and V₂
(\#0073-2 and
\#0073-4)}{*lutu(uˀ)r ear V₁ and V₂ (\#0073-2 and \#0073-4)}}\label{lutuuux2c0r-ear-v-and-v-0073-2-and-0073-4-1}

luˀd, FR, luˀd, RSED.p165, luntur, BDBH.2386, litir, AG08.p652, lutur,
PKED.p127, lutur/lutuʔ, PJDW.p239, nlug, PGEG.p36, lutur, CDES.p60,
lutur, BMED.p110, lutur, DHED.p216, lutur, BAHL.p128, lutur, NKEV.p324,
*lutu(uˀ)r, ear, \#0073, V147, 1621,

\begin{itemize}
\tightlist
\item
  Pinnow 1959: V147 / MKCD: 1621 \emph{*kt₂uur}; \emph{*kt₂uər}
\end{itemize}

\paragraph{\texorpdfstring{\emph{*d}
(Onset)}{*d (Onset)}}\label{d-onset}

\begin{longtable}[]{@{}llllllllllll@{}}
\toprule
Gorum & Sora & Remo & Gutob & Kharia & Juang & Gtaʔ & Santali & Mundari
& Ho & Korwa & Korku\tabularnewline
\midrule
\endhead
ɖ & d & d/ɖ & ɖ & ɖ & ɖ & ɖ/d & d/ɖ & d/ɖ & d & ɖ & ɖ/d\tabularnewline
\bottomrule
\end{longtable}

\subparagraph{\texorpdfstring{\emph{*daˀk} `water'
(\#0001-1)}{*daˀk water (\#0001-1)}}\label{daux2c0k-water-0001-1}

ɖaʔ, FR, daʔ, RSED.p70, dak', BDBH.1179, ɖaʔ, ZG63.85, ɖaʔ, PKED.p41,
ɖag, PJDW.p185, ndiaʔ, PGEG.p36, dak', CDES.p217, da:, BMED.p31, daʔ,
DHED.p73, da:ʔ, BAHL.p87, ɖa, NKEV.p300, da(a)ˀk, water, \#0001, V2,
274, 75

\begin{itemize}
\tightlist
\item
  Pinnow 1959: V2 / MKCD: 274 \emph{*diʔaak} \textgreater{} \emph{*ɗaak}
\end{itemize}

\subparagraph{\texorpdfstring{\emph{*daˀc} `to climb'
(\#0006-1)}{*daˀc to climb (\#0006-1)}}\label{daux2c0c-to-climb-0006-1}

ɖaˀɟ, FR, daɟ, RSED.p72, ɖaĭ, BDBH.1168, ɖaj, GZ65.79, ---, ---, ɖaɲ,
PJDW.p186, ɖæʔ, PGEG.p16, de̠c', CDES.p32, dej', BMED.p40, deʔ, DHED.p81,
deʔ, BAHL.p89, (cuɖe), NKEV.p298, daˀɟ, climb (v), \#0006, V333, ,

\begin{itemize}
\tightlist
\item
  Pinnow 1959: V333 / MKCD: ---
\end{itemize}

\subparagraph{\texorpdfstring{\emph{*dərv₍₆₎ŋ} `horn'
(\#0007-1)}{*dərv₍₆₎ŋ horn (\#0007-1)}}\label{dux259rvux14b-horn-0007-1}

ɖeraŋ, FR, deraŋ, RSED.p78, deruŋ, BDBH.1266, ---, ---, ɖereŋ, PKED.p44,
---, ---, ɖiraŋ, PGEG.p17, dereɲ, CDSE.p171, diriŋ, BMED.p49, diriɲ,
HOGV.p162, dereŋ, BAHL.p89, ---, ---, *dərv₍₆₎ŋ, horn, \#0007, V347,
699, 34

\begin{itemize}
\tightlist
\item
  Pinnow 1959: V347 UM: \emph{*e},\emph{*ɛ}/ MKCD 699 \emph{*d₂raŋ}
\end{itemize}

\subparagraph{\texorpdfstring{\emph{*dO₍₂₁₎ɲ} `cook (v)'
(\#0040-1)}{*dO₍₂₁₎ɲ cook (v) (\#0040-1)}}\label{doux272-cook-v-0040-1}

ɖeɲ, FR, diɲ, RSED.p80, ɖoNĭ, BDBH.1302, ɖoɲ, AG08.p664, ɖeɲ, PKED.p63,
ɖɛɲ, PJDW.p187, ɖue, PGEG.p17, ---, ---, ---, ---, ---, ---, ---, ---,
---, ---, *dxɲ, cook (v), \#0040, V342, 583,

\begin{itemize}
\tightlist
\item
  Pinnow 1959: V342 / MKCD: 583 \emph{*kɗaŋ}
\end{itemize}

\subparagraph{\texorpdfstring{\emph{*dal} `to cover'
(\#0047-2)}{*dal to cover (\#0047-2)}}\label{dal-to-cover-0047-2-1}

ɖal, FR, dal, RSED.p73, ɖalu, BDBH.1210, ɖal, GZ65.80, ɖal, PKED.p42,
ɖan, MJTL.p96, ɖa, PGEG.p16, dapal/dalo̠p', CDES.p40, dapal/dālob,
BMED.p35/36, dapal/dalop, HOGV.p153, ---, ---, da:l, NKEV.p299, *dal,
cover (v), \#0047, V3, 1745,

\begin{itemize}
\tightlist
\item
  Pinnow 1959: V3 / MKCD: 1745 \emph{*kdiil}; \emph{*kdiəl};
  \emph{*kdəl}
\end{itemize}

\paragraph{\texorpdfstring{\emph{*d}
(Medial)}{*d (Medial)}}\label{d-medial}

\begin{longtable}[]{@{}llllllllllll@{}}
\toprule
Gorum & Sora & Remo & Gutob & Kharia & Juang & Gtaʔ & Santali & Mundari
& Ho & Korwa & Korku\tabularnewline
\midrule
\endhead
ɖ & --- & --- & ɖ & ɖ & r & ∅ & ɖ & ɖ & d & d & ɖ\tabularnewline
\bottomrule
\end{longtable}

\subparagraph{\texorpdfstring{\emph{*lV(N)dV} `laugh
(v)'}{*lV(N)dV laugh (v)}}\label{lvndv-laugh-v}

liɖa, FR, ---, ---, (ɖoɖo), BDBH.1283, luɖo, GZ65.228, laɖa, PKED.p202,
lara, PJDW.p236, lwaʔ, PGEG.p32, lanɖa, CDES.p110, la:nɖa:, BMED.p102,
landa, HOGV.p166, la:Nd, BAHL.p127, lanɖa, NKEV.p322, , laugh (v),
\#0037, V302, ,

\paragraph{\texorpdfstring{\emph{*dʲ}
(Medial)}{*dʲ (Medial)}}\label{dux2b2-medial}

\begin{longtable}[]{@{}llllllllllll@{}}
\toprule
Gorum & Sora & Remo & Gutob & Kharia & Juang & Gtaʔ & Santali & Mundari
& Ho & Korwa & Korku\tabularnewline
\midrule
\endhead
ɖ & r & ɖ & d & ɖ & ɖ & ɖ & y & y & y & y & y\tabularnewline
\bottomrule
\end{longtable}

Unexplaint correspondence set that is characterized by a voiced alveolar
sound in the southern languages and a palatal glide in North Munda. It
is tempting to separate the two and posit *xdxˀp for the southern
languages and presumably proto-Munda and treat proto-North Munda *ajuˀp
as an innovation. However, even if the North Munda y (/j/) is ignored,
Sora /r/ is not expected in reflexes of a medial *d.

If we take MKCD 1268 \emph{*yup}; \emph{*y{[}uu{]}p}; \emph{*yəp} as the
correct form of the etymon, the reflexes are best explained as
proto-Munda \emph{*ju₅ˀp} with different prefixes:

\begin{itemize}
\tightlist
\item
  Kherwarian: \emph{*a} + \emph{*ju₅ˀp}
\item
  Sora-Gorum and Kharia: \emph{*Vd} + \emph{*ju₅ˀp}
\item
  Gutob-Remo and Gtaʔ: \emph{*NVnd} * \emph{*ju₅ˀp}
\end{itemize}

\subparagraph{\texorpdfstring{\emph{*vdʲu₅ˀp} `night' V₂
(\#0033-3)}{*vdʲu₅ˀp night V₂ (\#0033-3)}}\label{vdux2b2uux2c0p-night-v-0033-3-1}

uɖuˀb, FR, orub, RSED.p195, minɖip', BDBH.2087, noNdib, GZ65.260, iɖiˀb,
PKED.p79, ---, ---, minɖig, PGEG.p33, ayup', CDES.p128, a:yub, BMED.p14,
ayub, HOGV.p157, ayub, BAHL.p3, ayup, NKEV.p290, , night, \#0033, V280,
1268,

\begin{itemize}
\tightlist
\item
  Pinnow 1959: V280 / MKCD: 1268 \emph{*yup}; \emph{*y{[}uu{]}p};
  \emph{*yəp}
\end{itemize}

\paragraph{\texorpdfstring{\emph{*ˀt}}{*ˀt}}\label{ux2c0t}

\begin{longtable}[]{@{}llllllllllll@{}}
\toprule
Gorum & Sora & Remo & Gutob & Kharia & Juang & Gtaʔ & Santali & Mundari
& Ho & Korwa & Korku\tabularnewline
\midrule
\endhead
ˀd & ˀd & --- & ʔ/ɖ/ɽ & ˀɖ & ɖ/r & ʔ/g & t' & d' & ɖ & ɖ &
d/ʈ\tabularnewline
\bottomrule
\end{longtable}

There is some variation among the reflexes. Gutob ʔ/ɖ/ɽ points to a /ɖ/
that is neutralized to /ʔ/ in final position. The three values ʔ/ɖ/ɽ
given in the source data can be analyzed as allophones of /ɖ/ or in the
case of /ʔ/ reflexes of a final /ɖ/ merged with glottal stop allophones
of other phonems into a new phoneme /ʔ/. The pair ʔ/g in Gtaʔ seems to
be the result of inconsistent phonemic represenation among the source
and within single sources. The variation d/ʈ in Korku is so far
unexplained.

The /r/ in Juang in the *bxˀt `sow (v)' set remains unexplained as well.
Given the variantion between /ɖ/ and /r/ attested in \emph{*uˀt}
`drink/swallow (v)', it seems possible, that there is no fundamental
inconsistency here and that the unexplained variation is the result of
our restricted knowledge of Juang in general.

\subparagraph{\texorpdfstring{\emph{*məˀt} `eye'
(\#0012-3)}{*məˀt eye (\#0012-3)}}\label{mux259ux2c0t-eye-0012-3}

maˀd, FR, mo:ˀd/mad, RSED.p168, moʔ, BDBH.220, moʔ, AG08.p642, moˀɖ,
PKED.p195, ɛmɔɖ, PJDW.p191, muaʔ, PGEG.p34, mẽ̠t', CDES.p67, med',
BMED.p117, meɖ, DHED.p228, meɖ, BAHL.p120, med, ZKPM.p48, məˀt, eye,
\#0012, V250, 1045, 40

\begin{itemize}
\tightlist
\item
  Pinnow 1959: V250 / MKCD: 1045 *mat
\end{itemize}

\subparagraph{\texorpdfstring{\emph{*gəˀt} `cut (v)'
(\#0013-3)}{*gəˀt cut (v) (\#0013-3)}}\label{gux259ux2c0t-cut-v-0013-3}

gaˀd, FR, gad, RSED.p93, goʔ, BDBH.1018, goʔ, AG08.p669, gaˀɖ, PKED.p60,
, , gwaʔ, PGEG.p21, ge̠t', CDES.p44, ged', EMV5.1411, geɖ, DHED.p111,
geɖ, BAHL.p46, geʈ, NKEV.p306, gəˀt, cut (v), \#0013, V334, 972,

\begin{itemize}
\tightlist
\item
  Pinnow 1959: V334 / MKCD: MKCD 972 \emph{*sguut}; \emph{*{[}s{]}gət};
  \emph{*sgat}
\end{itemize}

\subparagraph{\texorpdfstring{\emph{*riˀt} `to grind'
(\#0025-3)}{*riˀt to grind (\#0025-3)}}\label{riux2c0t-to-grind-0025-3}

riˀd, FR, rid, RSED.p233, riʔ, BDBH.2276, riɽ, GZ63.15, riɖ, PKED.p169,
riɖ, PJDW.p266, rig, PGEG.p4, rit', CDES.p86, ri'd, BMED.p159, riɖ,
DHED.p288, ri:ɖ, BAHL.p124, --, ---, *riˀt, grind (v), \#0025, V76,
1056,

\begin{itemize}
\tightlist
\item
  Pinnow 1959: V76 / MKCD: 1056 \emph{*riit}, \emph{*riət}
\end{itemize}

\subparagraph{\texorpdfstring{\emph{*ɟoˀt} `wipe (v)'
(\#0029-3)}{*ɟoˀt wipe (v) (\#0029-3)}}\label{ux25foux2c0t-wipe-v-0029-3}

zoˀd, FR, ɟoˀd, RSED.p126, susuʔ, BDBH.2698, sosod, GZ65.374, joˀɖ,
PKED.p87, ---, ---, cuʔ, PGEG.p15, jo̠t', CDES.p221, jod', BMED.p84, joɖ,
HOGV.p88, joɖ, BAHL.p62, o:jo, NKEV.p329, *ɟoˀt, wipe (v), \#0029, V190,
994,

\begin{itemize}
\tightlist
\item
  Pinnow 1959: V190 / MKCD: 994 \emph{*{[} {]}jut}; \emph{*{[} {]}juut}
\end{itemize}

\subparagraph{\texorpdfstring{\emph{*bɨˀt} `sow (v)'
(\#0045-3)}{*bɨˀt sow (v) (\#0045-3)}}\label{bux268ux2c0t-sow-v-0045-3}

buˀd, FR, büd/bid, RSED.p63, biʔ, BDBH.1898, biɽ, GZ63.67, biˀɖ,
PKED.p20, bir, PJDW.p167, big, PGEG.p11, bit', CDES.p142, bid',
BMED.p22, biɖ, DHED.p35, , , biʈ, NKEV.p294, *bɨˀt, sow (v), \#0045,
V285, ,

\begin{itemize}
\tightlist
\item
  Pinnow 1959: V285 / MKCD: ---
\end{itemize}

\subparagraph{\texorpdfstring{\emph{*peˀt} `blow (v)'
(\#0057-3)}{*peˀt blow (v) (\#0057-3)}}\label{peux2c0t-blow-v-0057-3}

peˀd, ped, RSED.p212, peʔ, BDBH.1759, ped, ZG65.293, pɛˀɖ, PJED.p156,
---, , piʔ, PGEG.p38, phe̠t', CDES.p142, ---, , ---, , ---, , ---, ,
*peˀt, blow (v), \#0057, V157, 1028,

\begin{itemize}
\tightlist
\item
  Pinnow 1959: V157 / MKCD: 1028 \emph{*puut}; \emph{*p{[}əə{]}t}
\end{itemize}

\subparagraph{\texorpdfstring{\emph{*uˀt} `drink/swallow (v)'
(\#0090-2)}{*uˀt drink/swallow (v) (\#0090-2)}}\label{uux2c0t-drinkswallow-v-0090-2}

---, ---, ---, ---, uʔ, BDBH.181, iɖ, AG08.p664, uˀd, PKED.p205, ur/uɖ,
PJDW.p292, ug, PGEG.p6, ut', CSED.p674, ud', BMED.p191, uɖ, DHED.p369,
u:ɖ, BAHL.p18, u:ʈ, NKEV.346, *uˀt, drink/swallow (v), \#0090, V142,
1106,

\begin{itemize}
\tightlist
\item
  Pinnow 1959: V142 / MKCD: 1106 \emph{*hut}; \emph{*huut};
  \emph{*huət}; \emph{*huc}; \emph{*huuc}; \emph{*huəc}
\end{itemize}

\subparagraph{\texorpdfstring{\emph{*bəˀt} `contain/block (v)'
(\#0092-3)}{*bəˀt contain/block (v) (\#0092-3)}}\label{bux259ux2c0t-containblock-v-0092-3}

baˀd ,FR ,bad ,RSED.p47 ,boʔ ,BDBH.2027 ,bod ,Z1965.59 ,--- ,--- ,---
,--- ,boaʔ ,PGEG.p11 ,be̠t' ,BSDV1.p275 ,bed' ,BMED.p21 ,beɖ ,DHED.p33
,--- ,--- ,--- ,--- ,--- ,contain/block (v) ,\#0092 ,--- , 1032,

\begin{itemize}
\tightlist
\item
  Pinnow 1959: --- / MKCD 1032 \emph{*bat}; \emph{*buət}
\end{itemize}

\paragraph{\texorpdfstring{\emph{*ˀt}/\emph{*r}
(Coda/Medial)}{*ˀt/*r (Coda/Medial)}}\label{ux2c0tr-codamedial}

\begin{longtable}[]{@{}llllllllllll@{}}
\toprule
Gorum & Sora & Remo & Gutob & Kharia & Juang & Gtaʔ & Santali & Mundari
& Ho & Korwa & Korku\tabularnewline
\midrule
\endhead
ˀd & d & ʔ & ɖ & r & --- & --- & r & r & r & r & r\tabularnewline
\bottomrule
\end{longtable}

\subparagraph{\texorpdfstring{\emph{*pəˀt}/\emph{*par(om)} `cross
(v)'}{*pəˀt/*par(om) cross (v)}}\label{pux259ux2c0tparom-cross-v}

pa'd, FR, pad, RSED.p200, poʔ, BDBH.1793, poɖ, DSGU\#18931, paro(m),
PKED.p155, (pakea), DSJU\#25131, pwaʔ, PGEG.p39, par, CSED.p474, pa:rom,
BMED.p144, parom, DHED.p262, parom, BAHL.p97, pa:r, NKEV.p331, *pəˀt,
cross (v), \#0085, , ,

The Sora and Gorum, Remo and Gutob, as well as Gtaʔ suggest a
\emph{*pəˀt}. However, the \emph{r} in Kharia and North Munda cannot be
explained a a reflex of \emph{*pVˀt} or \emph{*pVd(Vm)}. Since the Sora,
Gorum, Remo, Gutob, and Gtaʔ vowels are also consistent while the Kharia
and North Munda vowels are inconsistent with any correspondece sets.
This suggests that we have to assume a second etymon \emph{*par(om)}
`cross (v)', probably of Indo-Aryan origin.

\paragraph{\texorpdfstring{\emph{*n}
(Onset)}{*n (Onset)}}\label{n-onset}

no set for initial \emph{*n}

synchronic n\textasciitilde{}l variation in Gutob and Ho (check others)

All languages have words with initial /n/ and initial /n/ is
particularly prominent in the deictical and pronominal domain. The
lexemes beginning in /n/ are not well attested across the whole family.

\paragraph{\texorpdfstring{\emph{*n₁} (Coda)}{*n₁ (Coda)}}\label{n-coda}

\begin{longtable}[]{@{}llllllllllll@{}}
\toprule
Gorum & Sora & Remo & Gutob & Kharia & Juang & Gtaʔ & Santali & Mundari
& Ho & Korwa & Korku\tabularnewline
\midrule
\endhead
n & n & n & n & n & --- & --- & n & n & n & n & n\tabularnewline
\bottomrule
\end{longtable}

\subparagraph{\texorpdfstring{\emph{*ten} `trample (v)'
(\#0086-3)}{*ten trample (v) (\#0086-3)}}\label{ten-trample-v-0086-3}

tin, FR, ---, ---, ---, ---, ten, Z1965.402, ten, PKED.p199, ---, ---,
(ʈe), PGEG.p46, ten, CSED.p624, ʈhen, BMED.p185, ten, DHED.p347, ten,
DSKW@1275, ten, DSKO\#26831, *txn, trample (v), \#0086, K306, 1153a,

\begin{itemize}
\tightlist
\item
  Pinnow 1959: K306 / MKCD: 1153a \emph{*t₁een}
\end{itemize}

\subparagraph{\texorpdfstring{\emph{*sƏn} `chase (v)'
(\#0087-3)}{*sƏn chase (v) (\#0087-3)}}\label{sux259n-chase-v-0087-3}

san, FR, san, RSED.p248, sensen, BDBH.2714, ---, ---, san, PKED.p176,
(saŋgem), PJDW.p273, ---, ---, sen, CSED.p572, sen, BMED.p172, sen,
DHED.p311, sen, BAHL.p138, sen(e), NKEV.p337, *sƏn, chase (v), \#0087,
V300, 899,

\begin{itemize}
\tightlist
\item
  Pinnow 1959: V300 / MKCD: 899 \emph{*təɲ}
\end{itemize}

\paragraph{\texorpdfstring{\emph{*n₂}
(Coda)}{*n₂ (Coda)}}\label{n-coda-1}

\begin{longtable}[]{@{}llllllllllll@{}}
\toprule
Gorum & Sora & Remo & Gutob & Kharia & Juang & Gtaʔ & Santali & Mundari
& Ho & Korwa & Korku\tabularnewline
\midrule
\endhead
n & n & ŋ & n & n & --- & ŋ & n & n & n & ŋ & n\tabularnewline
\bottomrule
\end{longtable}

\subparagraph{\texorpdfstring{\emph{*I₍₂₃₎sin} `boil (v)'
(\#0046-4)}{*I₍₂₃₎sin boil (v) (\#0046-4)}}\label{isin-boil-v-0046-4}

asin, FR, əsin, RSED.p16, nsiŋ, BDBH.1641, isin, GZ65.173, isin,
PKED.p81, isinɔ, JLIC.v65, nsiŋ, PGEG.p37, isin, CDES.p39, isin,
BMED.p77, isin, DHED.p153, isiŋ, BAHL.p12, isin, Korku.txt.12071,
\emph{*I₍₂₃₎sin}, boil (v), \#0046, V86, 1137,

\begin{itemize}
\tightlist
\item
  Pinnow 1959: V86 / MKCD: 1137 \emph{*ciinʔ} (\textgreater{}
  Pre-Bahnaric \emph{*cin}); \emph{*ciən{[} {]}}; \emph{*cain{[} {]}};
  `cooked'
\end{itemize}

In contrast to the regular reflexes in \emph{*sƏn} `chase (v)', Remo,
Gtaʔ, and Korwa have \emph{ŋ} instead of the expected \emph{n} in
\emph{*I₍₂₃₎sin} `boil (v)'. The difference between \emph{*sƏn} `chase
(v)' and \emph{*I₍₂₃₎sin} `boil (v)' is unexplained, the crucial
difference is in all likelihood the \emph{*i} preceding the \emph{*n}.

\paragraph{\texorpdfstring{\emph{*N₃}
(Coda)}{*N₃ (Coda)}}\label{n-coda-2}

\begin{longtable}[]{@{}llllllllllll@{}}
\toprule
Gorum & Sora & Remo & Gutob & Kharia & Juang & Gtaʔ & Santali & Mundari
& Ho & Korwa & Korku\tabularnewline
\midrule
\endhead
ŋ & n & ŋ & j/ɲ? & m & ŋ & ∅ & ɲ & ŋ & ɲ & --- & ---\tabularnewline
\bottomrule
\end{longtable}

Problematic set, maybe due to a conflation of distinct etyma with
different etymological pre-nasal vowels.

\subparagraph{\texorpdfstring{\emph{*ɟVlVN} `long/tall'
(\#0082-5)}{*ɟVlVN long/tall (\#0082-5)}}\label{ux25fvlvn-longtall-0082-5}

zuleŋa, FR, ɟele:n, RSED.p123, sileŋ, BDBH.2601, silej, AG08.p651,
jhelo(g, b, m), PKED.p92, ɟaliŋ, PJDW.210, clæ, PGEG.p15, jeleɲ,
CSED.p260, jiliŋ, BMED.p83, jiliɲ, DHED.p165, ---, ---, ---, ---,
*ɟxlxN, long/tall, \#0082, V340, 740,

\begin{itemize}
\tightlist
\item
  Pinnow 1959: V340 / MKCD: 740 \emph{*jiliiŋ} (\& \emph{*jiliŋ}?);
  \emph{*jla{[}i{]}ŋ} `long'
\end{itemize}

The set \emph{*ɟVlVN} `long/tall' displays unclear reflexes of a final
nasal. The problems are aggravated by the fact that this might be a
fused set of two or more etyma meaning long, tall, high, slim, and
realted concepts all besed in the consonantal frame \emph{*ɟVlVN}, but
with different vowels. If we assume that the character of the vowel
preceding the nasal may influence the the form in certain languages, the
problem of the different etyma with different vowels is fundamental for
the reconstruction of the final nasal.

\paragraph{\texorpdfstring{\emph{*n}
(Medial)}{*n (Medial)}}\label{n-medial}

Reflexes of medial \emph{*n} remain unclear. The only well attested
candidates are \#0037 \emph{*lv₍₁₈₎(N)dv₍₂₀₎} and \#0050
\emph{*tv₍₁₉₎ŋv₍₂₅₎n/tv₍₁₉₎nv₍₂₅₎ŋ}. Both have complicating factors. The
nasal in \#0037 \emph{*lv₍₁₈₎(N)dv₍₂₀₎} is only present in North Munda
and part of cluster, while the two nasals in \#0050
\emph{*tv₍₁₉₎ŋv₍₂₅₎n/tv₍₁₉₎nv₍₂₅₎ŋ} seem to under go non-adjacent
metathesis obscuring the reflexes.

\subparagraph{\texorpdfstring{\emph{*lv₍₁₈₎(N)dv₍₂₀₎} `laugh (v)'
(\#0037-3)}{*lv₍₁₈₎(N)dv₍₂₀₎ laugh (v) (\#0037-3)}}\label{lvndv-laugh-v-0037-3}

liɖa, FR, ---, ---, (ɖoɖo), BDBH.1283, luɖo, GZ65.228, laɖa, PKED.p202,
lara, PJDW.p236, lwaʔ, PGEG.p32, lanɖa, CDES.p110, la:nɖa:, BMED.p102,
landa, HOGV.p166, la:Nd, BAHL.p127, lanɖa, NKEV.p322, , laugh (v),
\#0037, V302, ,

\subparagraph{\texorpdfstring{\emph{*tv₍₁₉₎ŋv₍₂₅₎n/tv₍₁₉₎nv₍₂₅₎ŋ} `stand
(v)'
(\#0050-3)}{*tv₍₁₉₎ŋv₍₂₅₎n/tv₍₁₉₎nv₍₂₅₎ŋ stand (v) (\#0050-3)}}\label{tvux14bvntvnvux14b-stand-v-0050-3}

tinaŋ, FR, tanaŋ, RSED.p, toŋ, BDBH.1490, tunon, AG08.p662, tuŋon,
PKED.p201, toŋon, PJDW.p287, thwaN, PGEG.p46, teŋgon, CDES.p186, tiŋun,
BMED.p187, tiŋgu, HOGV.p180, ---, ---, ʈengene, NKEV.p342, *txŋxn, stand
(v), \#0050, , 1824?,

\begin{itemize}
\tightlist
\item
  Pinnow 1959: V258 / MKCD: 1824 \emph{*taaw}
\end{itemize}

\subsubsection{Palatals}\label{palatals}

\begin{longtable}[]{@{}llllll@{}}
\toprule
voiceless & voiced & glottalized & nasal & & glide\tabularnewline
\midrule
\endhead
(*c?) & \emph{*ɟ} & \emph{*ˀc} & \emph{*ɲ} & \emph{*ɲˀ} &
\emph{*j}\tabularnewline
\bottomrule
\end{longtable}

Hypothesis: \emph{*c} merged with \emph{*s} (and \emph{*ɟ}) and reflexes
are in \emph{*s₁} and \emph{*s₂} (as well as \emph{*ɟ}), see the
correlation with data from MKCD below.

\begin{longtable}[]{@{}lllllllllllll@{}}
\toprule
Gorum & Sora & Remo & Gutob & Kharia & Juang & Gtaʔ & Santali & Mundari
& Ho & Korwa & Korku &\tabularnewline
\midrule
\endhead
z & ɟ & s & s & j & ɟ & c & j & j & j & j & j & \emph{*ɟ}\tabularnewline
\bottomrule
\end{longtable}

\emph{*ˀc} (maybe mix of \emph{*ˀc}, \emph{*jˀ}, and \emph{*j}?):

\begin{longtable}[]{@{}lllllllllllll@{}}
\toprule
Gorum & Sora & Remo & Gutob & Kharia & Juang & Gtaʔ & Santali & Mundari
& Ho & Korwa & Korku &\tabularnewline
\midrule
\endhead
ˀɟ & ɟ & ĭ & j & --- & ɲ & ʔ & c' & j & ʔ & ʔ & -- &
\emph{*ˀc}\tabularnewline
-- & -- & ĭ & j & ˀj & ɟ & ʔ & c' & j & j & j & ∅ &
\emph{*ˀc}\tabularnewline
-- & ˀɟ & ĭ & j & ˀj & (i) & ʔ & c' & iˀ & iʔ & i:ʔ & j &
\emph{*ˀc}\tabularnewline
ˀɟ & ɟ & k' & q & ˀj & j & g & c' & j & iʔ & q & ch &
\emph{*ˀc}\tabularnewline
\bottomrule
\end{longtable}

\begin{longtable}[]{@{}lllllllllllll@{}}
\toprule
Gorum & Sora & Remo & Gutob & Kharia & Juang & Gtaʔ & Santali & Mundari
& Ho & Korwa & Korku &\tabularnewline
\midrule
\endhead
j & j & ∅ & ? & ? & ? & ∅/w & ɲ & n & n & ɲ & n & \emph{*ɲ}
onset\tabularnewline
ɲ & ɲ & ɲ (Ny) & ɲ & ɲ & ɲ? & ∅ & ɲ & ŋ & ɲ & ? & ? &
\emph{*ɲ₁}\tabularnewline
ŋ & ɲ & ŋ & ŋ & ɲ & ɲ & ŋ & ɲ & ŋ & ŋ & ? & ɲj &
\emph{*ɲ₂}\tabularnewline
ˀd & ɲ/ŋ & ʔ & 0 & ɲ & ŋ & ʔ & ɲ & ŋ & ɲ & ŋ & ɲj &
\emph{*ɲˀ}\tabularnewline
\bottomrule
\end{longtable}

\begin{longtable}[]{@{}lllllllllllll@{}}
\toprule
Gorum & Sora & Remo & Gutob & Kharia & Juang & Gtaʔ & Santali & Mundari
& Ho & Korwa & Korku &\tabularnewline
\midrule
\endhead
∅ & ɲ & --- & ∅ & ɲ & ɲ & ∅ & y & y & y & --- & y &
\emph{*ɲ₃}\tabularnewline
\bottomrule
\end{longtable}

\begin{longtable}[]{@{}lllllllllllll@{}}
\toprule
Gorum & Sora & Remo & Gutob & Kharia & Juang & Gtaʔ & Santali & Mundari
& Ho & Korwa & Korku &\tabularnewline
\midrule
\endhead
j & j & --- & j & --- & --- & ∅? & --- & --- & --- & --- & --- &
\emph{*j}\tabularnewline
\bottomrule
\end{longtable}

\paragraph{\texorpdfstring{\emph{*c}}{*c}}\label{c}

The symmetry of the consonant system suggests that there was a
proto-Munda \emph{*c} and comparison with Shorto's reconstructions in
the MKCD suggests that it was distinct from \emph{*s}. However, the data
is not clear on this at all. This is rather surprising, given that the
reflexes of proto-Munda \emph{*ɟ} -- discussed in the section below ---
are very clear and regular.

\begin{itemize}
\tightlist
\item
  s-loss in Sora-Gorum unexplained which \emph{*s} became ∅ (but see
  Zide 1987)
\item
  merger of \emph{*s} and \emph{*ɟ} (and \emph{*c}?) in Remo-Gutob
\end{itemize}

Unlike any other consonant phoneme, current Munda languages do not seem
to have preserved proto-Munda \emph{*c} as faithfully as other consonant
phonemes. (holds for both voiceless stops and the other palatal stop)

see \emph{*s} for sets and a more detailed discussion of proto-Munda
\emph{*c}.

\begin{longtable}[]{@{}lllll@{}}
\toprule
& & MKCD & proto-MK & proto-Munda\tabularnewline
\midrule
\endhead
\emph{*c}\textgreater{}\emph{*ɟ} & \#0002 & 488 & \emph{*cʔaaŋ} &
\emph{*ɟaŋ}\tabularnewline
& \#0015 & 1327 & \emph{*cuum} & \emph{*ɟv₍₁₀₎m}\tabularnewline
& \#0043 & 1409 & \emph{*{[}c{]}liəmʔ} & \emph{*ɟal}\tabularnewline
& \#0069 & 204 & \emph{*{[}c{]}nlu{[}u{]}ʔ} &
\emph{*ɟəlu₅}\tabularnewline
& \#0095 & 528 & \emph{*ciəŋ} & \emph{*ɟi₂ŋ(k)}\tabularnewline
\emph{*j}\textgreater{}\emph{*ɟ} & \#0010 & 538 & \emph{*juəŋ} &
\emph{*ɟv₍₇₎ŋ}\tabularnewline
& \#0029 & 994 & \emph{*{[} {]}jut} & \emph{*ɟoˀt}\tabularnewline
& \#0082 & 740 & \emph{*jla{[}i{]}ŋ} & \emph{*ɟVlVN}\tabularnewline
\emph{*c}\textgreater{}\emph{*s} & \#0002 & 488 & \emph{*ciiʔ} &
\emph{*sii₃ˀ}\tabularnewline
& \#0048 & 1243 & \emph{*cap} & \emph{*saˀp}\tabularnewline
& \#0026 & 44 & \emph{*{[}c{]}uuʔ} & \emph{*KʰVsu}\tabularnewline
& \#0028 & 1324 & \emph{*cim} & \emph{*si₂m}\tabularnewline
& \#0046 & 1137 & \emph{*ciinʔ} & \emph{*I₍₂₃₎sin}\tabularnewline
\emph{*j}\textgreater{}\emph{*s} & \#0021 & 1723 & \emph{*j{[}n{]}ŋəl} &
\emph{*sVŋəl}\tabularnewline
\emph{*t₂}\textgreater{}\emph{*s} & \#0075 & 31 & \emph{*t₂ŋiiʔ} &
\emph{*siŋi}\tabularnewline
\emph{*t}\textgreater{}\emph{*s} & \#0087 & 899 & \emph{*təɲ} &
\emph{*sƏn}\tabularnewline
\emph{*s}\textgreater{}\emph{*s} & \#0055 & 160 & \emph{*srɔs} &
\emph{*xsər}\tabularnewline
\bottomrule
\end{longtable}

\paragraph{\texorpdfstring{\emph{*ɟ}}{*ɟ}}\label{ux25f}

\begin{longtable}[]{@{}llllllllllll@{}}
\toprule
Gorum & Sora & Remo & Gutob & Kharia & Juang & Gtaʔ & Santali & Mundari
& Ho & Korwa & Korku\tabularnewline
\midrule
\endhead
z & ɟ & s & s & j & ɟ & c & j & j & j & j & j\tabularnewline
\bottomrule
\end{longtable}

\begin{itemize}
\tightlist
\item
  devoicing and fronting (depalatalization) in Gutob-Remo; pM \emph{*ɟ}
  \textgreater{} proto-Gutob-Remo \emph{*s} (merger of \emph{*ɟ},
  \emph{*s}, and probably \emph{*c})
\item
  fronting in Gorum pM \emph{*ɟ} \textgreater{} pSG \textgreater{}
  \emph{*ɟ} \textgreater{} Gorum /z/
\item
  devoicing in Gtaʔ pM \emph{*ɟ} \textgreater{} /c/
\end{itemize}

\subparagraph{\texorpdfstring{\emph{*ɟaŋ} `bone'
(\#0002-1)}{*ɟaŋ bone (\#0002-1)}}\label{ux25faux14b-bone-0002-1}

za̰ŋ, FR, əɟaŋ, RSED.p6, siʔsaŋ, BDBH.2614, sisaŋ, AG08.p651, jaŋ,
PKED.p83, ɟaŋ, PJDW.p210, ncia, PGEG.p36, jaŋ, CDES.p19, ja:ŋ, BMED.p80,
jaŋ, HOGV.p150, ja:ŋ, BAHL.p60, ---, , ɟa(a)ŋ, bone, \#0002, V7, 488, 31

\begin{itemize}
\tightlist
\item
  Pinnow 1959: V7 / MKCD: 488 \emph{*cʔaaŋ} ; \emph{*cʔaiŋ};
  \emph{*cʔi{[} {]}ŋ}
\end{itemize}

\subparagraph{\texorpdfstring{\emph{*ɟv₍₇₎ŋ} `foot'
(\#0010-2)}{*ɟv₍₇₎ŋ foot (\#0010-2)}}\label{ux25fvux14b-foot-0010-2-1}

zḭŋ, FR, ɟe:ˀŋ, RSED.p123, suŋ, BDBH.1363, suŋ, GZ63.205, juŋ, PKED.p66,
iɟiŋ, PJDW.p208, nco, PGEG.p114, jaŋga, CDES.p76, jaŋga, HLKS.182, ---,
---, ---, ---, (naŋga), NKEV.p327, , foot, \#0010, V365, 538,

\begin{itemize}
\tightlist
\item
  Pinnow 1959: V365 / MKCD 538 \emph{*juŋ}; \emph{*juəŋ}; \emph{*jəŋ};
  \emph{*jəəŋ}
\end{itemize}

\subparagraph{\texorpdfstring{\emph{*ɟv₍₁₀₎m} `eat (v)'
(\#0015-1)}{*ɟv₍₁₀₎m eat (v) (\#0015-1)}}\label{ux25fvm-eat-v-0015-1}

zum, FR, ɟom, RSED.p128, sum, BDBH.2667, som, GZ63.212, jom, HLKS.K274,
ɟim, PJDW.p212, coŋ, PGEG.p15, jo̠m, CDES.p60, jom, BMED.p84, jom,
HOGV.p156, jom, BAHL.p63, jom, NKEV.p313, , eat (v), \#0015, V385, 1327,
55

\begin{itemize}
\tightlist
\item
  Pinnow 1959: V385 / MKCD: 1327 \emph{*cuum}; \emph{*cuəm};
  \emph{*cəm}; (\emph{*cim cim} \textgreater{}) \emph{*ncim};
  \emph{*ciəm} (\& \emph{*nciəm}?); \emph{*caim}
\end{itemize}

\subparagraph{\texorpdfstring{\emph{*ɟoˀt} `wipe (v)'
(\#0029-1)}{*ɟoˀt wipe (v) (\#0029-1)}}\label{ux25foux2c0t-wipe-v-0029-1}

zoˀd, FR, ɟoˀd, RSED.p126, susuʔ, BDBH.2698, sosod, GZ65.374, joˀɖ,
PKED.p87, ---, ---, cuʔ, PGEG.p15, jo̠t', CDES.p221, jod', BMED.p84, joɖ,
HOGV.p88, joɖ, BAHL.p62, o:jo, NKEV.p329, *ɟoˀt, wipe (v), \#0029, V190,
994,

\begin{itemize}
\tightlist
\item
  Pinnow 1959: V190 / MKCD: 994 \emph{*{[} {]}jut}; \emph{*{[} {]}juut}
\end{itemize}

\subparagraph{\texorpdfstring{\emph{*ɟo(o)ˀ} `fruit; bear fruit (v)'
(\#0030-1)}{*ɟo(o)ˀ fruit; bear fruit (v) (\#0030-1)}}\label{ux25fooux2c0-fruit-bear-fruit-v-0030-1}

zoʔ, FR, ɟo:ʔ, RSED.p125, suʔ, BDBH.2701, ---, ---, ---, ---, ---, ---,
cu, PGEG.p15, jo̠, CDES.p80, jo, BMED.p83, jo:, DHED.p83, joʔ, BAHL.p63,
jo:, NKEV.p313, *ɟo(o)ˀ, fruit / to bear fruit (v), \#0030, V188, ,

\begin{itemize}
\tightlist
\item
  Pinnow 1959: V188 / MKCD: ---
\end{itemize}

\subparagraph{\texorpdfstring{\emph{*ɟoˀk} `sweep (v)'
(\#0031-1)}{*ɟoˀk sweep (v) (\#0031-1)}}\label{ux25foux2c0k-sweep-v-0031-1}

zoʔ, FR, ɟo:, RSED.p126, suk', BDBH.2624, sog, AG08.p650, joʔ, PKED.p87,
ɟɛnɔg, PJDW.p211, coʔ, PGEG.p15, jo̠k', CDES.p194, joʔ, BMED.p85, joʔ,
DHED.p167, ---, ---, ju-khɽi, NKEV.p313, *ɟoˀk, sweep (v), \#0031, 190,
,

\begin{itemize}
\tightlist
\item
  Pinnow 1959: V190 / MKCD: ---
\end{itemize}

\subparagraph{\texorpdfstring{\emph{*ɟal} `to lick'
(\#0043-2)}{*ɟal to lick (\#0043-2)}}\label{ux25fal-to-lick-0043-2-1}

zaleˀb, FR, ɟa:l, RSED.p119, salep', BDBH.2523, sal, GZ63.228, jal,
PKED.p82, janɔ, JLIC.v372, cca, PGEG.p14, jal, CDES.p112, jal, EM.p1965,
jal, HOGV.p164, (jaɽa:ʔ), BAHL.p60, jal, NKEV.p312, *ɟal, lick (v),
\#0043, V13, 1409,

\begin{itemize}
\tightlist
\item
  Pinnow 1959: V13 / MKCD: 1409 \emph{*{[}c{]}limʔ};
  \emph{*{[}c{]}liəmʔ}; \emph{*{[}c{]}laim{[} {]}}
\end{itemize}

\subparagraph{\texorpdfstring{\emph{*ɟəlu₅} `meat' V₂
(\#0069-1)}{*ɟəlu₅ meat V₂ (\#0069-1)}}\label{ux25fux259lu-meat-v-0069-1}

---, ---, ɟelu:, RSED.p123, sili/seli, BDBH.2599/2731, seli, AG08.p674,
―, ---, ---, ---, cili, PGEG.p15, jel, CDES.p120, jilu, BMED.p83, jilu,
DHED.p165, ---, ---, jilu, NKEV.p311, *ɟəlu₅, meat, \#0069, V228, 204?,

\begin{itemize}
\tightlist
\item
  Pinnow 1959: V228 / MKCD: ---
\end{itemize}

A possibly connected MKCD etymon is MKCD 204 \emph{*{[}c{]}nlu{[}u{]}ʔ}
`edible grub' only attested in Bahnaric.

\subparagraph{\texorpdfstring{\emph{*ɟa(ˀt)} `additive.particle'
(\#0079-1)}{*ɟa(ˀt) additive.particle (\#0079-1)}}\label{ux25faux2c0t-additive.particle-0079-1}

zaˀd, FR, ɟa:, RSED.p117, sa, BDBH.2547, sa, AG08.p649, ja, HLKS.V1,
ɟan, PJDW.p211, , , ja, BSDV3.p216, ja:, BMED.p77, ja:, DHED.p155, ja'',
DSKW.@09330, ja, DSKO.12141, *ɟa(ˀt), additive.particle, \#0079, V1, ,

\begin{itemize}
\tightlist
\item
  Pinnow 1959: V1 / MKCD: ---
\end{itemize}

\subparagraph{\texorpdfstring{\emph{*ɟVlVN} `long/tall'
(\#0082-1)}{*ɟVlVN long/tall (\#0082-1)}}\label{ux25fvlvn-longtall-0082-1}

zuleŋa, FR, ɟele:n, RSED.p123, sileŋ, BDBH.2601, silej, AG08.p651,
jhelo(g, b, m), PKED.p92, ɟaliŋ, PJDW.210, clæ, PGEG.p15, jeleɲ,
CSED.p260, jiliŋ, BMED.p83, jiliɲ, DHED.p165, ---, ---, ---, ---,
*ɟxlxN, long/tall, \#0082, V340, 740,

\begin{itemize}
\tightlist
\item
  Pinnow 1959: V340 / MKCD: 740 \emph{*jiliiŋ} (\& \emph{*jiliŋ}?);
  \emph{*jla{[}i{]}ŋ} `long'
\end{itemize}

As discussed above, this might be a fused set of two or more etyma
meaning long, tall, high, slim, and related concepts all based in the
consonantal frame \emph{*ɟVlVN}. The vowel alternations do not seem to
affect the reflexes of the consonants \emph{*ɟ} and \emph{*l}.

\subparagraph{\texorpdfstring{\emph{*ɟi₂ŋ(k)} `porcupine'
(\#0095-2)}{*ɟi₂ŋ(k) porcupine (\#0095-2)}}\label{ux25fiux14bk-porcupine-0095-2-1}

---, ---, kənɟɪ:ŋ, RSED.p131, gisiŋreʔe, BDBH.858, ---, ---, jiŋray,
PKED.p86, ɟiŋɛ, PJDW.p212, gcæiŋ, PGEG.p22, jhĩk, CSED.p268, jiki,
BMED.p82, jiki, DHED.p165, ji:k, DSKW@09500, jikɽa, NKEV.p313, *ɟiŋ(k),
porcupine, \#0095, V318, 528/1883,

\begin{itemize}
\tightlist
\item
  Pinnow 1959: V318 / MKCD: 528 \emph{*cu{[}ə{]}ŋ}; \emph{*cəŋ};
  \emph{*ciəŋ}
\end{itemize}

Looks like a combination of MKCD 528 and 1883 \emph{*{[}r{]}kus};
\emph{*{[}r{]}kuus}; \emph{*{[}r{]}kuəs}; \emph{*{[}r{]}k{[}iə{]}s},
e.g. \emph{*ciəŋ+{[}r{]}kuəs}.

\paragraph{\texorpdfstring{\emph{*ˀc}}{*ˀc}}\label{ux2c0c}

The reflexes of the posited proto-Munda \emph{*ˀc} have more irregular
correspondences than other coda obstruents. The palatal obstruent in the
coda also produces more coarticulatory effects in the preceding vowel
than other obstruents.

\begin{longtable}[]{@{}lllllllllllll@{}}
\toprule
Gorum & Sora & Remo & Gutob & Kharia & Juang & Gtaʔ & Santali & Mundari
& Ho & Korwa & Korku &\tabularnewline
\midrule
\endhead
ˀɟ & ɟ & ĭ & j & --- & ɲ & ʔ & c' & j & ʔ & ʔ & -- &
0006-3\tabularnewline
-- & -- & ĭ & j & ˀj & ɟ & ʔ & c' & j & j & j & ∅ &
0051-3\tabularnewline
-- & ˀɟ & ĭ & j & ˀj & (i) & ʔ & c' & iˀ & iʔ & i:ʔ & j &
0063-3\tabularnewline
ˀɟ & ɟ & k' & q & ˀj & j & g & c' & j & iʔ & q & ch &
0091-2\tabularnewline
(ˀd) & (d) & ʔ & j & ˀj & ɟ & ʔ & c & (e)ʔ & eʔ & ej & c &
0094-3\tabularnewline
ˀɟ & ɟ & ĭ & j & ˀj & j & ʔ & c' & -- & -- & -- & -- &
0096-3\tabularnewline
\bottomrule
\end{longtable}

Juang and Mundari branch unclear, not enough data from Korku.

\subparagraph{\texorpdfstring{\emph{*daˀc} `to climb'
(\#0006-3)}{*daˀc to climb (\#0006-3)}}\label{daux2c0c-to-climb-0006-3}

ɖaˀɟ, FR, daɟ, RSED.p72, ɖaĭ, BDBH.1168, ɖaj, GZ65.79, ---, ---, ɖaɲ,
PJDW.p186, ɖæʔ, PGEG.p16, de̠c', CDES.p32, dej', BMED.p40, deʔ, DHED.p81,
deʔ, BAHL.p89, (cuɖe), NKEV.p298, daˀɟ, climb (v), \#0006, V333, ,

\begin{itemize}
\tightlist
\item
  Pinnow 1959: V333 / MKCD: ---
\end{itemize}

\subparagraph{\texorpdfstring{\emph{*goˀc/goj} `die (v)'
(\#0051-3)}{*goˀc/goj die (v) (\#0051-3)}}\label{goux2c0cgoj-die-v-0051-3}

(kiˀd), FR, (kajed), RSED.p133, goĭ, BDBH.975, goj, ZG65.139, goˀj,
PKED.p63, gɔɟ, PJDW.p197, gweʔ, PGEG.p23, go̠c', CDES.p51, goj, BMED.p61,
goj, DHED.p115, goej, BAHL.p46, go, NKEV.p306, *goj, die (v), \#0051,
K67 , ,

\begin{itemize}
\tightlist
\item
  Pinnow 1959: K67 / MKCD: ---
\end{itemize}

Connect MKCD 1543 \emph{*ghuuy}; \emph{*ghuəy} `spirit, soul' or less
likely MKCD 805 \emph{*guc}; \emph{*guc} `to burn'?

\subparagraph{\texorpdfstring{\emph{*i₂ˀc} `defecate (v)'
(\#0091-2)}{*i₂ˀc defecate (v) (\#0091-2)}}\label{iux2c0c-defecate-v-0091-2}

ḭj / iˀɟ, FR, gad-iɟ, RSED.p94, ik', BDBH.88, ig, AG08.p652, iˀj,
DSKH\#12711, ij, DSJU\#13371, æg, PGEG.p4, ic', CSED.p244, ij',
BMED.p75, iiʔ, DHED.p150, i:q, BAHL.p16, ich, NKEV.p310, *ix, defecate
(v), \#0091, V81, 794,

\begin{itemize}
\tightlist
\item
  Pinnow 1959: V281 / MKCD: 794 \emph{*ʔic}; \emph{*ʔiə{[}c{]}};
  \emph{*ʔ{[}ə{]}c}
\end{itemize}

\subparagraph{\texorpdfstring{\emph{*lv₍₂₉₎ˀc} `stomach'
(\#0063-3)}{*lv₍₂₉₎ˀc stomach (\#0063-3)}}\label{lvux2c0c-stomach-0063-3}

---, ---, (lo:ˀɟ), RSED.p163, suloĭ, BDBH.2692, suloj, AG08.p651,
la(i)ˀj, PKED.p119, (lai), JLIC.n57, slweʔ, PGEG.p43, lac', CDES.p188,
la:iˀ, BMED.p101, la:iʔ, DHED.p204, la:i:ʔ, BAHL.p123, la:j, NKEV.p323,
*lv₍₂₉₎ˀc, stomach, \#0063, K282, ,

\begin{itemize}
\tightlist
\item
  Pinnow 1959: K282 / MKCD: ---
\end{itemize}

\subparagraph{\texorpdfstring{\emph{*roˀc} `squeeze/milk (v)'
(\#0094-3)}{*roˀc squeeze/milk (v) (\#0094-3)}}\label{roux2c0c-squeezemilk-v-0094-3}

(ra'd), FR, (rad), RSED.p226, riʔ, BDBH.2276, roj, DSGU\#2071, roˀj,
PKED.p170, roɟ, PJDW.p268, rweʔ, PGEG.p41, roco, BSDV5.p98, roeʔ,
EMV12.p3628, ro:eʔ, DHED.p290, roej, DSKW@19520, ro(:)c, NKEV.p335,
*roˀc, squeeze/milk (v), \#0094, V381, 1061,

\begin{itemize}
\tightlist
\item
  Pinnow 1959: V381 / MKCD: 1061 \emph{*ruut}; \emph{*ruət};
  \emph{*rət}; \emph{*rat}; \emph{*rit}; \emph{*riit}; \emph{*riət}
\end{itemize}

Gorum /raˀd/ and Sora /rad/ fit better to the forms given in MKCD (in
particular \emph{*ruət}, \emph{*rət}, and \emph{*rat}), but these two
reflexes are clearly distinct form the remaining words clearly
reflecting proto-Munda \emph{*roˀc}.

\subparagraph{\texorpdfstring{\emph{*gaˀc} `to fry'
(\#0096-3)}{*gaˀc to fry (\#0096-3)}}\label{gaux2c0c-to-fry-0096-3}

gaˀɟ ,FR ,gaɟ ,RSED.p95 ,gaĭ ,BDBH.766 ,gaj ,Z1965.120 ,gaˀj ,PKED.p165
,gaj ,DSJU\#10461 ,gæʔ ,PGEG.p19 ,ge̠c' ,CSED.p184 ,geʔ ,EMV5.p1411 ,---
,--- ,--- ,--- ,--- ,--- ,*gaˀc ,fry/scrape (v) ,\#0096 ,V15 ,(338a) ,

\begin{itemize}
\tightlist
\item
  Pinnow 1959: V15 / MKCD: (338a)
\end{itemize}

\subsubsection{\texorpdfstring{\emph{*ɲ₁}
(Final)}{*ɲ₁ (Final)}}\label{ux272-final}

\begin{longtable}[]{@{}llllllllllll@{}}
\toprule
Gorum & Sora & Remo & Gutob & Kharia & Juang & Gtaʔ & Santali & Mundari
& Ho & Korwa & Korku\tabularnewline
\midrule
\endhead
ɲ & ɲ & ɲ (Ny) & ɲ & ɲ & ɲ? & ∅ & ɲ & ŋ & ɲ & ? & ?\tabularnewline
\bottomrule
\end{longtable}

\subparagraph{\texorpdfstring{\emph{*taɲ} `to weave'
(\#0005-3)}{*taɲ to weave (\#0005-3)}}\label{taux272-to-weave-0005-3}

taɲ, FR, taɲ, RSED.p281, taNy, BDBH.1358, taɲ, GZ65.369, taɲ, PKED.p196,
---, ---, tæ, PGEG.p45, teɲ, CDES.p219, teŋ, BMED.p183, teɲ, HOGV.p187,
---, ---, ---, ---, taɲ, weave (v), \#0005, V301, 898,

\begin{itemize}
\tightlist
\item
  Pinnow 1959: V301 / MKCD: 898 \emph{*t₁aaɲ}
\end{itemize}

\subparagraph{\texorpdfstring{\emph{*dO₍₂₁₎ɲ} `cook (v)'
(\#0040-3)}{*dO₍₂₁₎ɲ cook (v) (\#0040-3)}}\label{doux272-cook-v-0040-3}

ɖeɲ, FR, diɲ, RSED.p80, ɖoNĭ, BDBH.1302, ɖoɲ, AG08.p664, ɖeɲ, PKED.p63,
ɖɛɲ, PJDW.p187, ɖue, PGEG.p17, ---, ---, ---, ---, ---, ---, ---, ---,
---, ---, *dxɲ, cook (v), \#0040, V342, 583,

\begin{itemize}
\tightlist
\item
  Pinnow 1959: V342 / MKCD: 583 \emph{*kɗaŋ}
\end{itemize}

\paragraph{\texorpdfstring{\emph{*ɲ₂}
(Final)}{*ɲ₂ (Final)}}\label{ux272-final-1}

\begin{longtable}[]{@{}llllllllllll@{}}
\toprule
Gorum & Sora & Remo & Gutob & Kharia & Juang & Gtaʔ & Santali & Mundari
& Ho & Korwa & Korku\tabularnewline
\midrule
\endhead
ŋ & ɲ & ŋ & ŋ & ɲ & ɲ & ŋ & ɲ & ŋ & ŋ & ? & ɲj\tabularnewline
\bottomrule
\end{longtable}

\subparagraph{\texorpdfstring{\emph{*tuɲ} `shoot (v)'
(\#0027-3)}{*tuɲ shoot (v) (\#0027-3)}}\label{tuux272-shoot-v-0027-3}

tiŋ, FR, tuɲ, RSED.p299, tiŋ, BDBH.1368, tiŋ, GZ63.190, tuɲ, PKED.p196,
tuɲ, PJDW.p288, ʈwiŋ, PGEG.p46, tuɲ, CDES.p173, tuiŋ, BMED.p180, tuŋ,
HOGV.p177, , , ʈuɲj, NKEV.p343, , shoot (v), \#0027, V107, 896a?,

\begin{itemize}
\tightlist
\item
  Pinnow 1959: V107 / MKCD: 896a?
\end{itemize}

MKCD 896a \emph{*t₁iɲ}; \emph{*t₁iiɲ}; \emph{*t₁iəɲ}; \emph{*t₁əɲ} `to
pluck, twang' could be related. Its meaning is generally `to pluck a
(stringed) instrument' which is rather close to `to shoot with bow and
arrow.'

\paragraph{\texorpdfstring{\emph{*ɲˀ}
(Final)}{*ɲˀ (Final)}}\label{ux272ux2c0-final}

\begin{longtable}[]{@{}llllllllllll@{}}
\toprule
Gorum & Sora & Remo & Gutob & Kharia & Juang & Gtaʔ & Santali & Mundari
& Ho & Korwa & Korku\tabularnewline
\midrule
\endhead
ˀd & ɲ/ŋ & ʔ & ∅ & ɲ & ŋ & ʔ & ɲ & ŋ & ɲ & ŋ & ɲj\tabularnewline
\bottomrule
\end{longtable}

The sequence /ɲˀ/ would be exceptional.

\subparagraph{\texorpdfstring{\emph{*bVɲˀ(*bVVˀɲ?)} `snake'
(\#0041-3)}{*bVɲˀ(*bVVˀɲ?) snake (\#0041-3)}}\label{bvux272ux2c0bvvux2c0ux272-snake-0041-3}

bubuˀd, FR, biɲ/biŋ, RSED.p59, bubuʔ, BDBH.1931, buɽbui, GGEG.p108,
buɲam, PKED.p4, bubuŋ, PJDW.p172, boʔ, PGEG.p12, biɲ, CDES.p179, biŋ,
BMED.p23, biɲ, HOGV.p178, bi:ŋ, BAHL.p108, biɲj, NKEV.p294, , snake,
\#0041, V353, 937,

\begin{itemize}
\tightlist
\item
  Pinnow 1959: V353; ; VW u/i; UM:i / MKCD: 937 \emph{*{[}b{]}saɲʔ}
\end{itemize}

MKCD 937 \emph{*{[}b{]}saɲʔ} is the prime candidate in MKCD. MKCD 1921a
\emph{*ɓəs} is a less likely option.

This correspondence set differs significantly form the sets \emph{*ɲ₁}
and \emph{*ɲ₂}. However, it remains unclear whether this set represents
a separat proto-Munda phoneme \emph{*ɲˀ} or whether the diverging
correspondences reflect the presence of a glottal element in the coda
and thus ultimately reflect \emph{*bVVˀɲ} or something similar. Either
way, Gorum /ˀd/, Gutob /∅/ as well as Remo and Gtaʔ /ʔ/ need an
explanation.

The sequence \emph{*ɲˀ} would be alluringly close to \emph{*ɲʔ} in MKCD
937. However, all evidence suggests that \emph{*ɲʔ} should simply become
\emph{*ɲ} in proto-Munda.

Alternatively, two etyma -- \emph{*bVɲ} and \emph{*bVˀt} -- could be
posited.

\paragraph{\texorpdfstring{\emph{*ɲ}
(Onset)}{*ɲ (Onset)}}\label{ux272-onset}

\begin{longtable}[]{@{}lllllllllllll@{}}
\toprule
Gorum & Sora & Remo & Gutob & Kharia & Juang & Gtaʔ & Santali & Mundari
& Ho & Korwa & Korku &\tabularnewline
\midrule
\endhead
j & j & ∅ & --- & y & --- & ∅/w & ɲ & n & n & ɲ & n &
0052-1\tabularnewline
--- & ɲ & --- & --- & ɲ & --- & --- & ɲ & n & n & ɲ & n &
0088-1\tabularnewline
n & ɲ & n & ∅ & ɲ & --- & n? & ɲ & n & n & & j & 0100-1\tabularnewline
\bottomrule
\end{longtable}

The reflexes of \emph{*ɲ} in word initial position are decidedly
different from the coda (or word final) reflexes.

\subparagraph{\texorpdfstring{\emph{*ɲv₍₂₆₎r} `run (v)'
(\#0052-1)}{*ɲv₍₂₆₎r run (v) (\#0052-1)}}\label{ux272vr-run-v-0052-1}

jer, FR, jer, RSED.p88, ur, BDBH.155, ---, ---, yar, DSKH\#12601, ---,
---, wir, PGEG.p9, ɲir, CDES.p164, nir, BMED.p132, nir, DHED.p246, ɲir,
BAHL.p66, niɽe, NKEV.p328, *ɲxr, run (v), \#0052, K294, 1602,

\begin{itemize}
\tightlist
\item
  Pinnow 1959: K294 / MKCD: 1602 \emph{*jarʔ}
\end{itemize}

Maybe two forms North Munda \emph{*ɲvr} and in the southern languages
\emph{*jvr}?

\subparagraph{\texorpdfstring{\emph{*ɲam} `get (v)'
(\#0088-1)}{*ɲam get (v) (\#0088-1)}}\label{ux272am-get-v-0088-1}

---, ---, ɲam, RSED.p186, ---, ---, ---, ---, ɲam, PKED.p140, ---, ---,
---, ---, ɲam, CSED.p434, na:m, BMED.p126, nam, DHED.p241, ɲa:m,
BAHL.p66, na, NKEV.p327, *ɲam, get (v), \#0088, 5(?), ,

\begin{itemize}
\tightlist
\item
  Pinnow 1959: 5(?) / MKCD: ---
\end{itemize}

\subparagraph{\texorpdfstring{\emph{*ɲUm} `name'
(\#0100-1)}{*ɲUm name (\#0100-1)}}\label{ux272um-name-0100-1}

inum, FR, əɲam, RSED.p12, nimi, BDBH.1588, imi, AG.p645, (i)ɲimi,
PKED.p140, ---, ---, mni, PGEG.p35, ɲum/ɲutum, CSED.p451/452, num/nutum,
BMED.134, numu/nutum, DHED.p249, ---, ---, jimu, , *ɲUm, name, \#0100,
V279, 147,

\begin{itemize}
\tightlist
\item
  Pinnow 1959: V319 / MKCD: 259 \emph{*{[}hy{]}muʔ}
\end{itemize}

\paragraph{\texorpdfstring{\emph{*ɲ₃} (medial) variation
ɲ\textsubscript{j}∅}{*ɲ₃ (medial) variation ɲj∅}}\label{ux272-medial-variation-ux272j}

\begin{longtable}[]{@{}llllllllllll@{}}
\toprule
Gorum & Sora & Remo & Gutob & Kharia & Juang & Gtaʔ & Santali & Mundari
& Ho & Korwa & Korku\tabularnewline
\midrule
\endhead
∅ & ɲ & --- & ∅ & ɲ & ɲ & ∅ & y & y & y & --- & y\tabularnewline
\bottomrule
\end{longtable}

\paragraph{\texorpdfstring{\emph{*miɲam}/\emph{*mayOm} `blood'
(\#0059-3)}{*miɲam/*mayOm blood (\#0059-3)}}\label{miux272ammayom-blood-0059-3}

miam, FR, miɲam, RSED.p177, ---, ---, iam, GZ63.325, iɲam, PKED.p115,
iɲam/iɲɑm, PJDW.p208, mia, PGEG.p33, maNyaNm, CDES.p18, ma:yom,
BMED.p116, mayom, HOGV.p149, , , mayum, NKEV.p325, , blood, \#0059,
V303, 1430,

\begin{itemize}
\tightlist
\item
  Pinnow 1959: V303 / MKCD: 1430 \emph{*jhaam}; \emph{*jhiim}
\end{itemize}

\paragraph{\texorpdfstring{\emph{*j}}{*j}}\label{j}

\begin{longtable}[]{@{}llllllllllll@{}}
\toprule
Gorum & Sora & Remo & Gutob & Kharia & Juang & Gtaʔ & Santali & Mundari
& Ho & Korwa & Korku\tabularnewline
\midrule
\endhead
j & j & --- & j & --- & --- & ∅ & --- & --- & --- & --- &
---\tabularnewline
\bottomrule
\end{longtable}

\subparagraph{\texorpdfstring{\emph{*roj}/\emph{*roˀk} `fly'
(\#0071-3)}{*roj/*roˀk fly (\#0071-3)}}\label{rojroux2c0k-fly-0071-3}

aroj, FR, əro:j, RSED.p14, (ayoŋ/ayuŋ), BDBH.39, uroj, GGEG.p93,
(kɔnɖɔi), HLKS.K356, ---, ---, nɖroe, PGEG.p36, ro̠, CDES.p76, roko,
BMED.p161, roko, DHED.p291, roʔo, DSKW.19600, ruku, NKEV.p335, *roj,
fly, \#0071, K356, 1534,

\begin{itemize}
\tightlist
\item
  Pinnow 1959: K356 / MKCD: 1534 Pre-Proto-Mon-Khmer \emph{*ru{[}wa{]}y}
  \textgreater{} \emph{*ruy}; \emph{*ruuy}; \emph{*ruəy};
  Pre-Proto-Mon-Khmer \emph{*ruhay}
\end{itemize}

Gtaʔ /nɖroe/ derives from pre-Gtaʔ \emph{*n(ɖ)roj}. Kharia kɔnɖɔi could
derive from /kɔnrɔi/.

\subsubsection{Velars}\label{velars}

\begin{longtable}[]{@{}lllllllllllll@{}}
\toprule
Gorum & Sora & Remo & Gutob & Kharia & Juang & Gtaʔ & Santali & Mundari
& Ho & Korwa & Korku &\tabularnewline
\midrule
\endhead
k & k & k & k & k & k & k & k & k & k & k & k & \emph{*k}\tabularnewline
\bottomrule
\end{longtable}

\emph{*Kʰ} (/k/\textsubscript{/h/}∅): Pinnow (1959 p.232-234) \emph{*q}
etc., seems to konsistently reflect MKCD \emph{*k} with no apparent
reason for the variation /k/\textsubscript{/h/}∅.

\begin{longtable}[]{@{}lllllllllllll@{}}
\toprule
Gorum & Sora & Remo & Gutob & Kharia & Juang & Gtaʔ & Santali & Mundari
& Ho & Korwa & Korku & etymon\tabularnewline
\midrule
\endhead
? & ? & ∅ & ∅ & h & ? & h & h & h & h & h & kh &
\emph{*KʰVˀp}\tabularnewline
∅ & ∅ & ∅ & ∅ & k & k & h & h & ∅ & ∅ & ∅ & ? &
\emph{*b(VKʰ)xˀp}\tabularnewline
∅ & ∅ & ∅ & ∅ & k & k & ∅ & h & h & h & h & k & *Kʰxsu\_\tabularnewline
\bottomrule
\end{longtable}

\begin{longtable}[]{@{}lllllllllllll@{}}
\toprule
Gorum & Sora & Remo & Gutob & Kharia & Juang & Gtaʔ & Santali & Mundari
& Ho & Korwa & Korku &\tabularnewline
\midrule
\endhead
g & g & g & g & g & g & g & g & g & g & g & g & \emph{*g}\tabularnewline
\bottomrule
\end{longtable}

\begin{longtable}[]{@{}lllllllllllll@{}}
\toprule
Gorum & Sora & Remo & Gutob & Kharia & Juang & Gtaʔ & Santali & Mundari
& Ho & Korwa & Korku &\tabularnewline
\midrule
\endhead
ʔ & ʔ/∅ & k' & ʔ/g & ʔ & g & ʔ & k' & ʔ/∅ & ʔ & ʔ & ∅ &
\emph{*ˀk}\tabularnewline
\bottomrule
\end{longtable}

\begin{longtable}[]{@{}lllllllllllll@{}}
\toprule
Gorum & Sora & Remo & Gutob & Kharia & Juang & Gtaʔ & Santali & Mundari
& Ho & Korwa & Korku &\tabularnewline
\midrule
\endhead
ŋ & ŋ & ŋ & ŋ & ŋ & ŋ & ∅ & ŋ & ŋ & ŋ & ŋ & ŋ &
\emph{*ŋ₁}\tabularnewline
ŋ & ŋ & ŋ & ? & ŋ & ? & ŋ & ɲ & ŋ & ɲ & ŋ & ŋ &
\emph{*ŋ₂}\tabularnewline
\bottomrule
\end{longtable}

\paragraph{\texorpdfstring{\emph{*k}}{*k}}\label{k}

\begin{longtable}[]{@{}llllllllllll@{}}
\toprule
Gorum & Sora & Remo & Gutob & Kharia & Juang & Gtaʔ & Santali & Mundari
& Ho & Korwa & Korku\tabularnewline
\midrule
\endhead
k & k & k & k & k & k & k & k & k & k & k & k\tabularnewline
\bottomrule
\end{longtable}

\subparagraph{\texorpdfstring{\emph{*kᵊla} `tiger'
(\#0004-1)}{*kᵊla tiger (\#0004-1)}}\label{kux1d4ala-tiger-0004-1}

kulaʔ, FR, kina:, RSED.p140, ŋku, MVol.p733, gikil, AG08.p651, kiɽoʔ,
PKED.p102, kiɭog, PJDW.p224, nku, PGEG.p36, kul, CDES.p201, kula:,
BMED.p98, kula, HOGV.p183, ku:l, BAHL.p33, kula, NKEV.p319, *kᵊla,
tiger, \#0004, V281, 197,

\begin{itemize}
\tightlist
\item
  Pinnow 1959: V281 / MKCD: 197 \emph{*klaʔ}
\end{itemize}

\subparagraph{\texorpdfstring{\emph{*ruNkO(ˀp)} `husked rice'
(\#0068-4)}{*ruNkO(ˀp) husked rice (\#0068-4)}}\label{runkoux2c0p-husked-rice-0068-4}

ruŋk, FR, rʊŋkʊ, RSED.p239, ruŋku, BDBH.2291, rukug, AG08.p672,
ruŋkuˀb/rumkuˀb, PKED.p171, ruŋkub, PJDW.p269, rkoʔ, PGEG.p41, ---, ---,
(rukhaɽ), BMED.p163, ---, ---, ---, ---, ---, ---, *ruNkO(ˀp), husked
rice, \#0068, V139, 1820,

\begin{itemize}
\tightlist
\item
  Pinnow 1959: V139 / MKCD: 1820 \emph{*rk{[}aw{]}ʔ}
\end{itemize}

Gtaʔ /rkoʔ/ is surprisingly close th Shorto's \emph{*rk{[}aw{]}ʔ}.

\paragraph{\texorpdfstring{\emph{*Kʰ}}{*Kʰ}}\label{kux2b0}

\begin{longtable}[]{@{}lllllllllllll@{}}
\toprule
Gorum & Sora & Remo & Gutob & Kharia & Juang & Gtaʔ & Santali & Mundari
& Ho & Korwa & Korku & etymon\tabularnewline
\midrule
\endhead
? & ? & ∅ & ∅ & h & ? & h & h & h & h & h & kh &
\emph{*KʰVˀp}\tabularnewline
∅ & ∅ & ∅ & ∅ & k & k & h & h & ∅ & ∅ & ∅ & ? &
\emph{*b(xKʰ)Vˀp}\tabularnewline
∅ & ∅ & ∅ & ∅ & k & k & ∅ & h & h & h & h & k &
\emph{*KʰVsu}\tabularnewline
∅ & ∅ & ∅ & ∅ & k & k & h & h & h & h & h & k &
\emph{*Kʰa(aˀ)}\tabularnewline
\bottomrule
\end{longtable}

There are a few instances of k\textsubscript{h}∅ correspondences; see
also Pinnow (1959) p.198-201. Pinnow (1959 p.232-234) \emph{*q} etc.
\textasciitilde{}

ADD: FISH 2WIND

\subparagraph{\texorpdfstring{\emph{*Kʰaˀp} `bite (v)'
(\#0056-1)}{*Kʰaˀp bite (v) (\#0056-1)}}\label{kux2b0aux2c0p-bite-v-0056-1}

(kuˀb), FR, (küb/kib/kaib), RSED.p144, op, BDBH.337, op, ZG63.7, hapkay,
PKED.p73, ---, ---, haʔ, PGEG.p24, hap', CDES.p17, ha:b, BMED.p64, hab,
DHED.p124, ha:p, BAHL.p146, khap, NKEV.p320, *Kʰaˀp, bite (v), \#0056,
V294, 1231,

\begin{itemize}
\tightlist
\item
  Pinnow 1959: V294 / MKCD: 1231 \emph{*kap}/\emph{*kaap}
\end{itemize}

\subparagraph{\texorpdfstring{\emph{*b(oKʰ)Vˀp} `head'
(\#0011-3)}{*b(oKʰ)Vˀp head (\#0011-3)}}\label{bokux2b0vux2c0p-head-0011-3}

baˀb, FR, bo:ˀb, RSED.p60, bob, BDBH.2007, bob, GZ63.50, bokoˀb,
PKED.p24, bokob, PJDW.p169, bhaʔ, PGEG.p13, bo̠ho̠k', CDES.p90, bo,
BMED.p24, bo:ʔ, DHED.p40, boʔ, BAHL.p113, ---, ---, , head, \#0011,
V206, 361, 38

\begin{itemize}
\tightlist
\item
  Pinnow 1959: V206 / MKCD: 361 \emph{*{[}b{]}uuk}
\end{itemize}

\subparagraph{\texorpdfstring{\emph{*Kʰv₍₄₎su} `fever/pain' (\#0026-1)
MISSING in CSV
file}{*Kʰv₍₄₎su fever/pain (\#0026-1) MISSING in CSV file}}\label{kux2b0vsu-feverpain-0026-1-missing-in-csv-file}

asu, FR, asu:/əsu:, RSED.p42, siʔ, BDBH.2610, isi, GGEG.p93, kosu/kusu,
PKED.p107, kasu, PJDW.p220, aʔsu, PGEG.p4, haso, CDES.p135, ha:su,
BMED.p67, hasu, HOGV.p147, hasu:, BAHL.p145, kaSu, NKEV.p315, *Kʰxsu,
fever/pain, \#0026, V247, 44,

\begin{itemize}
\tightlist
\item
  Pinnow 1959: V247 / MKCD: 44 \emph{*{[}c{]}uuʔ}
\end{itemize}

Gtaʔ \emph{aʔsu} should have initial /h/ (so maybe \emph{hasu}), the
presence of /ʔ/ is unexplained and as is the lack of initial /h/. The
two may be connected.

\paragraph{\texorpdfstring{\emph{*g}}{*g}}\label{g}

\begin{longtable}[]{@{}llllllllllll@{}}
\toprule
Gorum & Sora & Remo & Gutob & Kharia & Juang & Gtaʔ & Santali & Mundari
& Ho & Korwa & Korku\tabularnewline
\midrule
\endhead
g & g & g & g & g & g & g & g & g & g & g & g\tabularnewline
\bottomrule
\end{longtable}

\subparagraph{\texorpdfstring{\emph{*gəˀt} `cut (v)'
(\#0013-1)}{*gəˀt cut (v) (\#0013-1)}}\label{gux259ux2c0t-cut-v-0013-1}

gaˀd, FR, gad, RSED.p93, goʔ, BDBH.1018, goʔ, AG08.p669, gaˀɖ, PKED.p60,
, , gwaʔ, PGEG.p21, ge̠t', CDES.p44, ged', EMV5.1411, geɖ, DHED.p111,
geɖ, BAHL.p46, geʈ, NKEV.p306, gəˀt, cut (v), \#0013, V334, 972,

\begin{itemize}
\tightlist
\item
  Pinnow 1959: V334 / MKCD: MKCD 972 \emph{*sguut}; \emph{*{[}s{]}gət};
  \emph{*sgat}
\end{itemize}

\subparagraph{\texorpdfstring{\emph{*gum} `winnow (v)'
(\#0044-1)}{*gum winnow (v) (\#0044-1)}}\label{gum-winnow-v-0044-1}

gumar, FR, gum, RSED.p105, (giteʔ), BDBH.864, gim, GZ63.134, gum,
PKED.p67, guŋ/guɲ, PJDW.p199, goŋ, PGEG.p20, gum, BSDV2.p490, gum,
BMED.p214, gum, DHED.p120, gum, BAHL.p45, gum, NKEV.p307, *gum, winnow
(v), \#0044, K159, 1317,

\begin{itemize}
\tightlist
\item
  Pinnow 1959: K159 / MKCD: 1317 \emph{*gum}; \emph{*guum};
  \emph{*g{[}əə{]}m}
\end{itemize}

\subparagraph{\texorpdfstring{\emph{*goˀc/goj} `die (v)'
(\#0051-1)}{*goˀc/goj die (v) (\#0051-1)}}\label{goux2c0cgoj-die-v-0051-1}

(kiˀd), FR, (kajed), RSED.p133, goĭ, BDBH.975, goj, ZG65.139, goˀj,
PKED.p63, gɔɟ, PJDW.p197, gweʔ, PGEG.p23, go̠c', CDES.p51, goj, BMED.p61,
goj, DHED.p115, goej, BAHL.p46, go, NKEV.p306, *goj, die (v), \#0051,
K67 , ,

\begin{itemize}
\tightlist
\item
  Pinnow 1959: K67 / MKCD: ---
\end{itemize}

Connect MKCD 1543 \emph{*ghuuy}; \emph{*ghuəy} `spirit, soul' or less
likely MKCD 805 \emph{*guc}; \emph{*guc} `to burn'?

\subparagraph{\texorpdfstring{\emph{*geˀp} `to burn (vi)'
(\#0058-3)}{*geˀp to burn (vi) (\#0058-3)}}\label{geux2c0p-to-burn-vi-0058-3-1}

geˀb ,FR ,tuŋge:b ,RSED.p298 ,gep' ,BDBH.967 ,geb ,GZ65.123 ,geb
,PKED.p61 ,--- ,--- ,giʔ ,PGEG.p19 ,~--- ,--- ,--- ,--- ,--- ,--- ,---
,--- ,--- ,--- ,*geˀp ,burn (vi) ,\#0058 , 156, ,

\begin{itemize}
\tightlist
\item
  Pinnow 1959: 156 / MKCD: ---
\end{itemize}

\subparagraph{\texorpdfstring{\emph{*gəle} `ear of corn' V₁
(\#0077-1)}{*gəle ear of corn V₁ (\#0077-1)}}\label{gux259le-ear-of-corn-v-0077-1}

gali, FR, gale, RSED.p96, gileker, DSBO.11781, gile, GTXT.7791, gɔlɛ,
HLKS.V182, (ɔnɔ), PJDW.p255, (konto-ja), PGEG.p28, gele, CDES.p185,
gele, EM.p1418, gele, DHED.p111, geleʔ, BAHL.p45, (kelʈa), NKEV.p317,
*gxle, ear of corn, \#0077, V182, 1577,

\begin{itemize}
\tightlist
\item
  Pinnow 1959: V182 / MKCD: 1577 \emph{*gur}; \emph{*guər}
\end{itemize}

\subparagraph{\texorpdfstring{\emph{*gam} `say (v)'
(\#0080-1)}{*gam say (v) (\#0080-1)}}\label{gam-say-v-0080-1}

---, ---, gam, RSED.p96, ---, ---, gam, Z1965.121, gam, PKED.p57, gam,
PJDW.p194, ---, ---, gam, CSED.p176, gamu, HLKS.V12, gamu, HLKS.V12,
---, ---, ---, ---, *gam, say (v), \#0080, V12, ,

\begin{itemize}
\tightlist
\item
  Pinnow 1959: V12 / MKCD: ---
\end{itemize}

\subparagraph{\texorpdfstring{\emph{*gur} `fall/rain (v)'
(\#0089-1)}{*gur fall/rain (v) (\#0089-1)}}\label{gur-fallrain-v-0089-1}

gur, FR, gur, RSED.p92, gur, BDBH.914, gir, Z1965.132, gur, PKED.p68,
gur, PJDW.p200, gur, PGEG.p21, gur, CSED.p207, gur, EMV5.p1535, gur,
DHED.p122, ---, ---, guru, DSKO\#10541, *gur, fall/rain (v), \#0089,
V106, 1579,

\begin{itemize}
\tightlist
\item
  Pinnow 1959: V106 / MKCD: 1579 \emph{*guur}
\end{itemize}

\subparagraph{\texorpdfstring{\emph{*gaˀc} `to fry'
(\#0096-1)}{*gaˀc to fry (\#0096-1)}}\label{gaux2c0c-to-fry-0096-1}

gaˀɟ ,FR ,gaɟ ,RSED.p95 ,gaĭ ,BDBH.766 ,gaj ,Z1965.120 ,gaˀj ,PKED.p165
,gaj ,DSJU\#10461 ,gæʔ ,PGEG.p19 ,ge̠c' ,CSED.p184 ,geʔ ,EMV5.p1411 ,---
,--- ,--- ,--- ,--- ,--- ,*gaˀc ,fry/scrape (v) ,\#0096 ,V15 ,(338a) ,

\begin{itemize}
\tightlist
\item
  Pinnow 1959: V15 / MKCD: (338a)
\end{itemize}

\paragraph{\texorpdfstring{\emph{*ˀk}}{*ˀk}}\label{ux2c0k}

\begin{longtable}[]{@{}llllllllllll@{}}
\toprule
Gorum & Sora & Remo & Gutob & Kharia & Juang & Gtaʔ & Santali & Mundari
& Ho & Korwa & Korku\tabularnewline
\midrule
\endhead
ʔ & ʔ/∅ & k' & ʔ/g & ʔ & g & ʔ & k' & ʔ/∅ & ʔ & ʔ & ∅\tabularnewline
\bottomrule
\end{longtable}

Hypothesis Remo /k'/ as a reflex of \emph{*ˀk} as opposed to
nono-etymological /ʔ/.

\subparagraph{\texorpdfstring{\emph{*daˀk} `water'
(\#0001-3)}{*daˀk water (\#0001-3)}}\label{daux2c0k-water-0001-3}

ɖaʔ, FR, daʔ, RSED.p70, dak', BDBH.1179, ɖaʔ, ZG63.85, ɖaʔ, PKED.p41,
ɖag, PJDW.p185, ndiaʔ, PGEG.p36, dak', CDES.p217, da:, BMED.p31, daʔ,
DHED.p73, da:ʔ, BAHL.p87, ɖa, NKEV.p300, da(a)ˀk, water, \#0001, V2,
274, 75

\begin{itemize}
\tightlist
\item
  Pinnow 1959: V2 / MKCD: 274 \emph{*diʔaak} \textgreater{} \emph{*ɗaak}
\end{itemize}

\subparagraph{\texorpdfstring{\emph{*ɟoˀk} `sweep (v)'
(\#0031-2)}{*ɟoˀk sweep (v) (\#0031-2)}}\label{ux25foux2c0k-sweep-v-0031-2-1}

zoʔ, FR, ɟo:, RSED.p126, suk', BDBH.2624, sog, AG08.p650, joʔ, PKED.p87,
ɟɛnɔg, PJDW.p211, coʔ, PGEG.p15, jo̠k', CDES.p194, joʔ, BMED.p85, joʔ,
DHED.p167, ---, ---, ju-khɽi, NKEV.p313, *ɟoˀk, sweep (v), \#0031, 190,
,

\begin{itemize}
\tightlist
\item
  Pinnow 1959: V190 / MKCD: ---
\end{itemize}

\subparagraph{\texorpdfstring{\emph{*Olaaˀ}/\emph{*Ola(ˀk)} `leaf' V₁
(\#0035-4)}{*Olaaˀ/*Ola(ˀk) leaf V₁ (\#0035-4)}}\label{olaaux2c0olaux2c0k-leaf-v-0035-4}

olaʔ, FR, o:la:, RSED.p192, ulak', BDBH.169, olag, AG08.p633, ulaʔ,
PKED.p298, olag, PJDW.p254, uliaʔ, PGEG.p124, palha, CDES.p111,
pa:lha:o, BMED.p142, pala, DHED.p259, (sakam), BAHL.pdfp129, pa:la,
NKEV.p331, , leaf, \#0035, V50, 230,

\begin{itemize}
\tightlist
\item
  Pinnow 1959: V50 / MKCD: 230 \emph{*slaʔ}
\end{itemize}

The reflexes of \emph{*ˀk} in this set are irregular, if we assume that
the North Munda reflexes and the reflexes of the other languages
represent the same etymon. However, the loss of initial \emph{*p}
outside of North Munda is irregular as is the correspondence set for V₁,
suggesting two separate etymons --- \emph{*xlaˀk} and North Munda *pxlx
--- both with (more) regular correspondences.

The \emph{pala} forms are probably Indo-Aryan (Turner 7969)

\subparagraph{\texorpdfstring{\emph{*maraˀk} `peacock'
(\#0081-5)}{*maraˀk peacock (\#0081-5)}}\label{maraux2c0k-peacock-0081-5}

(marraʔ), FR, ma:ra:, RSED.p173, ---, ---, ---, ---, maraʔ, PKED.p131,
marag, PJDW.p242, ---, ---, marak', CSED.p407, ma:ra:, BMED.p114, mara:,
DHED.p225, mara:q, BAHL.p117, mara, NKEV.p324, *maraˀk, peacock, \#0081,
V27, 416,

\begin{itemize}
\tightlist
\item
  Pinnow 1959: V27 / MKCD: 416 \emph{*mraik{[} {]}}
\end{itemize}

Gorum \emph{marraʔ} `husband' probably belongs to another etymon
connected with MKCD 183 \emph{*mraʔ} `person'.

\subparagraph{\texorpdfstring{\emph{*laˀk} `to scrape'
(\#0093-3)}{*laˀk to scrape (\#0093-3)}}\label{laux2c0k-to-scrape-0093-3}

laʔ, FR, ---, ---, ---, ---, lag, Z1965.205, laʔ, PKED.p118, lag,
PJDW.p235, liaʔ, PGEG.p31, lak', CSED.p359, ---, ---, laʔ, DHED.p203,
---, ---, laʔ, DSKO.17551, *laˀk, scrape (v), \#0093, ---, 418,

\begin{itemize}
\tightlist
\item
  Pinnow 1959: --- / MKCD: 418 \emph{*l{[}a{]}k}
\end{itemize}

\paragraph{\texorpdfstring{\emph{*ŋ₁}}{*ŋ₁}}\label{ux14b}

\begin{longtable}[]{@{}llllllllllll@{}}
\toprule
Gorum & Sora & Remo & Gutob & Kharia & Juang & Gtaʔ & Santali & Mundari
& Ho & Korwa & Korku\tabularnewline
\midrule
\endhead
ŋ & ŋ & ŋ & ŋ & ŋ & ŋ & ∅ & ŋ & ŋ & ŋ & ŋ & ŋ\tabularnewline
\bottomrule
\end{longtable}

\subparagraph{\texorpdfstring{\emph{*ɟaŋ} `bone'
(\#0002-3)}{*ɟaŋ bone (\#0002-3)}}\label{ux25faux14b-bone-0002-3}

za̰ŋ, FR, əɟaŋ, RSED.p6, siʔsaŋ, BDBH.2614, sisaŋ, AG08.p651, jaŋ,
PKED.p83, ɟaŋ, PJDW.p210, ncia, PGEG.p36, jaŋ, CDES.p19, ja:ŋ, BMED.p80,
jaŋ, HOGV.p150, ja:ŋ, BAHL.p60, ---, , ɟa(a)ŋ, bone, \#0002, V7, 488, 31

\begin{itemize}
\tightlist
\item
  Pinnow 1959: V7 / MKCD: 488 \emph{*cʔaaŋ} ; \emph{*cʔaiŋ};
  \emph{*cʔi{[} {]}ŋ}
\end{itemize}

\subparagraph{\texorpdfstring{\emph{*laŋ} `tongue'
(\#0003-3)}{*laŋ tongue (\#0003-3)}}\label{laux14b-tongue-0003-3}

laŋ, FR, əlaŋ, RSED.p158, leaŋ, BDBH.2423, laʔŋ, AG08.p638, laŋ,
PKED.p173, elaŋ, PJDW.p191, nlia, PGEG.p36, alaŋ, CDES.p203, a:la:ŋ,
BMED.p5, (leʔ), DHED.p208, a:la:ŋ, BAHL.p11, laŋ, NKEV.p322, la(a)ŋ,
tongue, \#0003, V14, , 44

\begin{itemize}
\tightlist
\item
  Pinnow 1959: V14 / MKCD: ---
\end{itemize}

\subparagraph{\texorpdfstring{\emph{*ɟv₍₇₎ŋ} `foot'
(\#0010-3)}{*ɟv₍₇₎ŋ foot (\#0010-3)}}\label{ux25fvux14b-foot-0010-3}

zḭŋ, FR, ɟe:ˀŋ, RSED.p123, suŋ, BDBH.1363, suŋ, GZ63.205, juŋ, PKED.p66,
iɟiŋ, PJDW.p208, nco, PGEG.p114, jaŋga, CDES.p76, jaŋga, HLKS.182, ---,
---, ---, ---, (naŋga), NKEV.p327, , foot, \#0010, V365, 538,

\begin{itemize}
\tightlist
\item
  Pinnow 1959: V365 / MKCD 538 \emph{*juŋ}; \emph{*juəŋ}; \emph{*jəŋ};
  \emph{*jəəŋ}
\end{itemize}

\subparagraph{\texorpdfstring{\emph{*v₍₉₎laŋ} `thatch'
(\#0014-4)}{*v₍₉₎laŋ thatch (\#0014-4)}}\label{vlaux14b-thatch-0014-4}

alaŋ, FR, əlaŋ, RSED.p158, lɔŋ, BDBH.2437, uloŋ, AG08.p644, oloŋ,
PKED.p214, oloŋ, PJDW.p254, nlo, PGEG.p36, ---, ---, ---, ---, ---, ---,
---, ---, ---, ---, *v₍₉₎laŋ, thatch, \#0014, V270, 749,

\begin{itemize}
\tightlist
\item
  Pinnow 1959: V270 / MKCD: 749 \emph{*{[}p{]}laŋ}; \emph{*{[}p{]}laiŋ}
\end{itemize}

\subparagraph{\texorpdfstring{\emph{*sVŋəl} `fuel'
(\#0021-3)}{*sVŋəl fuel (\#0021-3)}}\label{svux14bux259l-fuel-0021-3}

aŋal, FR, aŋəl, RSED.p37, suŋo, BDBH.2638, suõl, GZ63.216, soŋgol,
PKED.p186, sɛŋon, PJDW.p276, sua, PGEG.p43, se̠ŋge̠l, CDES.p73, seŋgel,
BMED.p172, seŋgel, HOGV.p158, seNgel, BAHL.p137, ---, ---, *sxŋəl, fuel,
\#0021, V252, 1723,

\begin{itemize}
\tightlist
\item
  Pinnow 1959: V252 / MKCD 1723 \emph{*j{[}n{]}ŋəl}
\end{itemize}

/ŋg/ in Kharia, Santali, Mundari, Ho, and Korwas is probably secondary.
Espeacially Gtaʔ /sua/ and Gutob /suõl/ suggest that /g/ is not
original.

\subparagraph{\texorpdfstring{\emph{*sVmaŋ} `forehead/front'
(\#0038-5)}{*sVmaŋ forehead/front (\#0038-5)}}\label{svmaux14b-foreheadfront-0038-5}

amaŋ, FR, ammaŋ, RSED.p31, gutumoŋ, BDBH.885, sumoŋ/amuŋ, GZ65.21,
somoŋ/somo/sumaŋ, PKED.p185, ɛmɔŋ, PJDW.p191, ssæ, PGEG.p44, samaŋ,
CDES.p79, sa:ma:ŋ, BMED.p167, sanamaŋ, HOGV.p159, samaŋ, BAHL.pdfp130,
samma, NKEV.p336, , forehead/front, \#0038, V269, ,

\begin{itemize}
\tightlist
\item
  Pinnow 1959: V269 / MKCD: ---
\end{itemize}

Korku ∅ is irregular in this set and unexplained.

\subparagraph{\texorpdfstring{\emph{*bVtoŋ} `fear'
(\#0039-5)}{*bVtoŋ fear (\#0039-5)}}\label{bvtoux14b-fear-0039-5}

butoŋ, FR, bato:ŋ, RSED.p55, butuŋ, BDBH.1922, butoŋ, GZ65.76, bɔtɔŋ
(P), HLKS.V261, betɔŋan, JLIC.v239, bʈoʔ, PGEG.p14, ---, ---, botoŋ,
BMED.p25, ---, ---, (bor), BAHL.p112, ---, ---, *bxtoŋ, fear, \#0039,
V261, 552,

\begin{itemize}
\tightlist
\item
  Pinnow 1959: V261 / MKCD: 552 \emph{*ʔt₁uuŋ}
\end{itemize}

Gtaʔ /ʔ/ is irregular, but probably a secondary addition to the CVCV
word after the loss of proto-Munda \emph{*ŋ}.

\subparagraph{\texorpdfstring{\emph{*soŋ} `buy/sell (v)'
(\#0060-3)}{*soŋ buy/sell (v) (\#0060-3)}}\label{soux14b-buysell-v-0060-3}

oŋ, FR, ---, ---, suŋ, BDBH.2635, soŋ, GZ65.370, soŋ, PKED.p185, soŋ,
PJDW.p278, so, PGEG.p42, ---, ---, ---, ---, ---, ---, ―, ---, ---, ---,
*soŋ, buy/sell (vt), \#0060, K209, ,

\begin{itemize}
\tightlist
\item
  Pinnow 1959: K209 / MKCD: ―
\end{itemize}

Pinnow (1959, p.~224) connects Pal(aung) \emph{jʌːŋ}, \emph{jɔːŋ} `to
sell' and Mon \emph{swʌ̃} `to sell'.

\subparagraph{\texorpdfstring{\emph{*mv₍₄₎raŋ} `big'
(\#0064-5)}{*mv₍₄₎raŋ big (\#0064-5)}}\label{mvraux14b-big-0064-5}

---, ---, maraŋ/məraŋ, RSED.p173/167, munaʔ, BDBH.2121, (moɖo),
AG08.p663, ---, ---, ---, ---, mnaʔ, PGEG.35, maraŋ, CDES.p17, maraŋ,
BMED.p220, maraŋ, DHED.p225, ---, ---, ---, ---, *mxrxŋ, big, \#0064,
K107, ,

\begin{itemize}
\tightlist
\item
  Pinnow 1959: K107 / MKCD: ---
\end{itemize}

Gtaʔ \emph{mnaʔ} and Remo \emph{munaʔ} are irregular reflexes of ,
especially the Gtaʔ form \emph{mnaʔ} should be different, given our
current understanding of the phonological developments, since a velar
coda \emph{*aŋ} results in Gtaʔ /ia/ (as should /aʔ/). Remo and Gtaʔ /n/
are also inconsistent as reflexes or either \emph{*r} or \emph{*ŋ}. Gtaʔ
\emph{mnaʔ} and Remo \emph{munaʔ} are consistently parallel to one
another.

Gtaʔ and Remo /ʔ/ is irregular. Gtaʔ /ʔ/ might be explained as secondary
adition to the CVCV word after the loss of proto-Munda \emph{*ŋ}, see
also \emph{*bVtVŋ} `fear'. Remo /ʔ/ cannot be explained by the same
mechanism and remains unexplained.

\subparagraph{\texorpdfstring{\emph{*(saŋ)saŋ} `tumeric'
(\#0072-3)}{*(saŋ)saŋ tumeric (\#0072-3)}}\label{saux14bsaux14b-tumeric-0072-3}

saŋsaŋ, FR, sansaŋ, RSED.249, saŋsaŋ, BDBH.400, saŋsaŋ, GZ63.226,
saŋsaŋ, PKED.p176, saŋsaŋ, PJDW.p268, ssia, PGEG.p42, sasaŋ, CDES.p230,
sasaŋ, BMED.p157, sasaŋ, DHED.p307, ---, ---, sasan, Korku.txt.24491,
*saŋsaŋ, turmeric/yellow, \#0072, V271, ,

\begin{itemize}
\tightlist
\item
  Pinnow 1959: V271 / MKCD: ---
\end{itemize}

While Sora /n/ in the first coda can be explained as a assimilation to
the following /s/, Korku /n/ is unexpected and remains uneplained.

\paragraph{\texorpdfstring{\emph{*ŋ₂}}{*ŋ₂}}\label{ux14b-1}

\begin{longtable}[]{@{}llllllllllll@{}}
\toprule
Gorum & Sora & Remo & Gutob & Kharia & Juang & Gtaʔ & Santali & Mundari
& Ho & Korwa & Korku\tabularnewline
\midrule
\endhead
ŋ & ŋ & ŋ & ? & ŋ & ? & ŋ & ɲ/ŋ & ŋ & ɲ & ŋ & ŋ\tabularnewline
\bottomrule
\end{longtable}

This correspondence set is defined by a ŋ\textasciitilde{}ɲ variation,
not found in \emph{*ŋ₁}. Gtaʔ reflects \emph{*ŋ₂} as /ŋ/ instead of
\emph{*ŋ₁} ∅.

MKCD 699 \emph{*d₂raŋ} and MKCD 549 \emph{*t₁uuŋ} suggest that these
sets do not reflect a palatal \emph{*ɲ}, but a velar \emph{*ŋ}. However
the set is substantially different from the sets grouped under
\emph{*ŋ₁}. The difference cannot be explained. The palatalization in
Santali and Ho may be explained by the front adjacent front vowels,
although *sxŋəl\_ `fuel' (\#0021-3) seems to be a parallel case without
palatalization. Why \emph{*ŋ₂} does not become ∅ in Gtaʔ is unclear.

This set shows a similar variation of \emph{ŋ} and \emph{ɲ} as
\emph{*ɲ₂}. However, which languages features a velar or a palatal sound
differs considerably.

\begin{longtable}[]{@{}lllllllllllll@{}}
\toprule
Gorum & Sora & Remo & Gutob & Kharia & Juang & Gtaʔ & Santali & Mundari
& Ho & Korwa & Korku & set\tabularnewline
\midrule
\endhead
ŋ & ŋ & ŋ & ? & ŋ & ? & ŋ & ɲ/ŋ & ŋ & ɲ & ŋ & ŋ &
\emph{*ŋ₂}\tabularnewline
ŋ & ɲ & ŋ & ŋ & ɲ & ɲ & ŋ & ɲ & ŋ & ŋ & ? & ɲj &
\emph{*ɲ₂}\tabularnewline
\bottomrule
\end{longtable}

\subparagraph{\texorpdfstring{\emph{*dərv₍₆₎ŋ} `horn'
(\#0007-5)}{*dərv₍₆₎ŋ horn (\#0007-5)}}\label{dux259rvux14b-horn-0007-5}

ɖeraŋ, FR, deraŋ, RSED.p78, deruŋ, BDBH.1266, ---, ---, ɖereŋ, PKED.p44,
---, ---, ɖiraŋ, PGEG.p17, dereɲ, CDSE.p171, diriŋ, BMED.p49, diriɲ,
HOGV.p162, dereŋ, BAHL.p89, ---, ---, *dərv₍₆₎ŋ, horn, \#0007, V347,
699, 34

\begin{itemize}
\tightlist
\item
  Pinnow 1959: V347 UM: \emph{*e},\emph{*ɛ}/ MKCD 699 \emph{*d₂raŋ}
\end{itemize}

\subparagraph{\texorpdfstring{\emph{*səreŋ} `stone'
(\#0020-5)}{*səreŋ stone (\#0020-5)}}\label{sux259reux14b-stone-0020-5}

areŋ, FR, areŋ, RSED.p39, ---, ---, ---, ---, soreŋ, PKED.p187, ---,
---, ---, ---, ---, ---, sereŋ, BMED.p172, sereɲ, HOGV.p175, ---, ---,
---, ---, *səreŋ, stone, \#0020, V183, ,

\begin{itemize}
\tightlist
\item
  Pinnow 1959: V183 / MKCD: ---
\end{itemize}

\subparagraph{\texorpdfstring{\emph{*siŋi} `sun'
(\#0075-3)}{*siŋi sun (\#0075-3)}}\label{siux14bi-sun-0075-3}

---, ---, ---, ---, siŋi, BDBH.2543, siN, AG08660, siŋ, PKED.p183, siŋ,
PJDW.p244, sni, PGEG.p34, siɲ, CDES.p193, siŋi, BMED.p174, siŋi,
DHED.p319, si:ŋ, BAHL.p136, ---, ---, *siŋi, sun, \#0075, V286, 31,

\begin{itemize}
\tightlist
\item
  Pinnow 1959: V286 / MKCD: 31 \emph{*t₂ŋiiʔ}
\end{itemize}

The nasalization in Gutob is unexpected, but consistent with the other
sets in \emph{*ŋ₂} so far. Gtaʔ /n/ is unexpected, but may be explained
by assimilation to the preceding /s/. Ho /ŋ/ is unexpected and
unexplained.

\subparagraph{\texorpdfstring{\emph{*tVŋ} `kindle (v)'
(\#0062-3)}{*tVŋ kindle (v) (\#0062-3)}}\label{tvux14b-kindle-v-0062-3}

tuŋ, FR, tuŋa:l, RSED.p297, ---, ---, ---, ---, tuŋgal, HLKS.V324, ---,
---, toŋ, PGEG.p42, tiŋgi, BSDV5.p461, tiŋ, BMED.p187, tiɲ, DHED.p353,
---, ---, ʈingi, NKEV.p343, *txŋ, kindle (v), \#0062, V324, 549,

\begin{itemize}
\tightlist
\item
  Pinnow 1959: V324; VW i/u; UM:i/ MKCD: 549 \emph{*t₁uuŋ}
\end{itemize}

/ŋg/ in Kharia, Santali, and Korku is probably secondary, cf.~also
\emph{*sVŋVl} `fuel' above.

\paragraph{\texorpdfstring{\emph{*ŋ₃}: irregular
\emph{*ŋ}/\emph{*k}}{*ŋ₃: irregular *ŋ/*k}}\label{ux14b-irregular-ux14bk}

\subparagraph{\texorpdfstring{\emph{*boŋtel}/\emph{*bitkil} `buffalo'
(\#0054-3)}{*boŋtel/*bitkil buffalo (\#0054-3)}}\label{boux14btelbitkil-buffalo-0054-3}

boŋtel, FR, boŋtel, RSED.p62, buŋte, BDBH.1917, boŋtel, AG08.p647,
boŋtel, PKED.p36, ---, ---, buNʈi, PGEG.p13, bitkil, CDES.p23, ---, ---,
---, ---, ---, ---, biʈkhil, NKEV.p294, *boŋtel, buffalo, \#0054, , ,

Probably two separate etyma \emph{*boŋtel} and North Munda *bitkil\_.
The form suggests some relation, but the two forms cannot be derived
from proto-Munda by regular sound change.

/ŋt/ \textasciitilde{} /tk/

\paragraph{\texorpdfstring{\emph{*ŋ₄}: irregular
\emph{*ŋ}/\emph{*ŋk}/\emph{*k}}{*ŋ₄: irregular *ŋ/*ŋk/*k}}\label{ux14b-irregular-ux14bux14bkk}

\subparagraph{\texorpdfstring{\emph{*ɟi₂ŋ(k)} `porcupine'
(\#0095-3)}{*ɟi₂ŋ(k) porcupine (\#0095-3)}}\label{ux25fiux14bk-porcupine-0095-3}

---, ---, kənɟɪ:ŋ, RSED.p131, gisiŋreʔe, BDBH.858, ---, ---, jiŋray,
PKED.p86, ɟiŋɛ, PJDW.p212, gcæiŋ, PGEG.p22, jhĩk, CSED.p268, jiki,
BMED.p82, jiki, DHED.p165, ji:k, DSKW@09500, jikɽa, NKEV.p313, *ɟiŋ(k),
porcupine, \#0095, V318, 528/1883,

\begin{itemize}
\tightlist
\item
  Pinnow 1959: V318 / MKCD: 528 \emph{*cu{[}ə{]}ŋ}; \emph{*cəŋ};
  \emph{*ciəŋ}
\end{itemize}

Looks like a combination of MKCD 528 and 1883 \emph{*{[}r{]}kus};
\emph{*{[}r{]}kuus}; \emph{*{[}r{]}kuəs}; \emph{*{[}r{]}k{[}iə{]}s},
e.g. \emph{*ciəŋ+{[}r{]}kuəs}.

\paragraph{\texorpdfstring{irregular
\emph{*ŋ}/\emph{*n}}{irregular *ŋ/*n}}\label{irregular-ux14bn}

\subparagraph{\texorpdfstring{\emph{*tv₍₁₉₎ŋv₍₂₅₎n/tv₍₁₉₎nv₍₂₅₎ŋ} `stand
(v)'
(\#0050-3)}{*tv₍₁₉₎ŋv₍₂₅₎n/tv₍₁₉₎nv₍₂₅₎ŋ stand (v) (\#0050-3)}}\label{tvux14bvntvnvux14b-stand-v-0050-3-1}

tinaŋ, FR, tanaŋ, RSED.p, toŋ, BDBH.1490, tunon, AG08.p662, tuŋon,
PKED.p201, toŋon, PJDW.p287, thwaN, PGEG.p46, teŋgon, CDES.p186, tiŋun,
BMED.p187, tiŋgu, HOGV.p180, ---, ---, ʈengene, NKEV.p342, *txŋxn, stand
(v), \#0050, , 1824?,

\begin{itemize}
\tightlist
\item
  Pinnow 1959: V258 / MKCD: 1824 \emph{*taaw}
\end{itemize}

\begin{longtable}[]{@{}lllllllllllll@{}}
\toprule
Gorum & Sora & Remo & Gutob & Kharia & Juang & Gtaʔ & Santali & Mundari
& Ho & Korwa & Korku & nasal\tabularnewline
\midrule
\endhead
n & n & ŋ? & n & ŋ & ŋ & N? & ŋ & ŋ & ŋ & ? & ŋ & N₁\tabularnewline
ŋ & ŋ & ŋ? & n & n & n & N? & n & n & ∅ & ? & n & N₂\tabularnewline
\bottomrule
\end{longtable}

(Already discussed under \emph{*n}.)

\paragraph{\texorpdfstring{\emph{*N₃}: irregular
\emph{*ŋ}/\emph{*n}/\emph{*∅}}{*N₃: irregular *ŋ/*n/*∅}}\label{n-irregular-ux14bn}

\subparagraph{\texorpdfstring{\emph{*ruNkO(ˀp)} `husked rice'
(\#0068-4)}{*ruNkO(ˀp) husked rice (\#0068-4)}}\label{runkoux2c0p-husked-rice-0068-4-1}

ruŋk, FR, rʊŋkʊ, RSED.p239, ruŋku, BDBH.2291, rukug, AG08.p672,
ruŋkuˀb/rumkuˀb, PKED.p171, ruŋkub, PJDW.p269, rkoʔ, PGEG.p41, ---, ---,
(rukhaɽ), BMED.p163, ---, ---, ---, ---, ---, ---, *ruNkO(ˀp), husked
rice, \#0068, V139, 1820,

\begin{itemize}
\tightlist
\item
  Pinnow 1959: V139 / MKCD: 1820 \_*rk{[}aw{]}ʔ
\end{itemize}

The nasal cannot be reconstructed as \emph{*ŋ}, securely. The absense in
some of the languages and the /ŋ/\textasciitilde{}/m/ variation in a
velar context in Kharia make a reconstruction as \emph{*ŋ} problematic.

\subsubsection{Sibilants}\label{sibilants}

\begin{longtable}[]{@{}lllllllllllll@{}}
\toprule
Gorum & Sora & Remo & Gutob & Kharia & Juang & Gtaʔ & Santali & Mundari
& Ho & Korwa & Korku &\tabularnewline
\midrule
\endhead
∅ & ∅ & s & s & s & s & s & s & s & s & s & s &
\emph{*s₁}\tabularnewline
s & s & s & s & s & s & s & s & s & s & s & s &
\emph{*s₂}\tabularnewline
\bottomrule
\end{longtable}

The difference between \emph{*s₁} and \emph{*s₂} is unexplained (see
Zide 1987 for an proposal). The symmetry of the consonant system
suggests that there was a proto-Munda \emph{*c} and comparison with
Shorto's reconstructions in the MKCD suggests that it was distinct from
\emph{*s}.

The working hypothesis would be that \emph{*s₁} and \emph{*s₂} are
reflexes of \emph{*c} and \emph{*s}. However, which etymon can be
reconstructed as \emph{*c} and which as \emph{*s} or whether the two had
already merged at the proto-Munda stage is currently unknown.

\paragraph{\texorpdfstring{\emph{*s₁}
(Onset)}{*s₁ (Onset)}}\label{s-onset}

\begin{longtable}[]{@{}llllllllllll@{}}
\toprule
Gorum & Sora & Remo & Gutob & Kharia & Juang & Gtaʔ & Santali & Mundari
& Ho & Korwa & Korku\tabularnewline
\midrule
\endhead
∅ & ∅ & s & s & s & s & s & s & s & s & s & s\tabularnewline
\bottomrule
\end{longtable}

\emph{*s₁} is defined by the s-loss in Sora-Gorum.

\subparagraph{\texorpdfstring{\emph{*sii₃ˀ} `louse'
(\#0009-1)}{*sii₃ˀ louse (\#0009-1)}}\label{siiux2c0-louse-0009-1}

(aŋiˀd), FR, iʔi, RSED.p109, gisi, BDBH.855, gisi, AG08.p651, seʔ,
PKED.p258, ɛsɛ, PJDW.p192, gsi, PGEG.p23, se, CDES.p116, siku,
BMED.p173, siku, HOGV.p165, guhi:, BAHL.p45, siku, NKEV.p338, sii₂ˀ,
louse, \#0009, V341, 39, 22

\begin{itemize}
\tightlist
\item
  Pinnow 1959: V341 UM: e,ɛ / MKCD: 39 \emph{*ciiʔ} (\& \emph{*ciʔ}?)
\end{itemize}

\subparagraph{\texorpdfstring{\emph{*səreŋ} `stone'
(\#0020-1)}{*səreŋ stone (\#0020-1)}}\label{sux259reux14b-stone-0020-1}

areŋ, FR, areŋ, RSED.p39, ---, ---, ---, ---, soreŋ, PKED.p187, ---,
---, ---, ---, ---, ---, sereŋ, BMED.p172, sereɲ, HOGV.p175, ---, ---,
---, ---, *səreŋ, stone, \#0020, V183, ,

\begin{itemize}
\tightlist
\item
  Pinnow 1959: V183 / MKCD: ---
\end{itemize}

\subparagraph{\texorpdfstring{\emph{*sVŋəl} `fuel'
(\#0021-1)}{*sVŋəl fuel (\#0021-1)}}\label{svux14bux259l-fuel-0021-1}

aŋal, FR, aŋəl, RSED.p37, suŋo, BDBH.2638, suõl, GZ63.216, soŋgol,
PKED.p186, sɛŋon, PJDW.p276, sua, PGEG.p43, se̠ŋge̠l, CDES.p73, seŋgel,
BMED.p172, seŋgel, HOGV.p158, seNgel, BAHL.p137, ---, ---, *sxŋəl, fuel,
\#0021, V252, 1723,

\begin{itemize}
\tightlist
\item
  Pinnow 1959: V252 / MKCD 1723 \emph{*j{[}n{]}ŋəl}
\end{itemize}

\subparagraph{\texorpdfstring{\emph{*sVmVŋ} `forehead/front' (\#0038-1)
\emph{*s₂}?}{*sVmVŋ forehead/front (\#0038-1) *s₂?}}\label{svmvux14b-foreheadfront-0038-1-s}

amaŋ, FR, ammaŋ, RSED.p31, gutumoŋ, BDBH.885, sumoŋ/amuŋ, GZ65.21,
somoŋ/somo/sumaŋ, PKED.p185, ɛmɔŋ, PJDW.p191, ssæ, PGEG.p44, samaŋ,
CDES.p79, sa:ma:ŋ, BMED.p167, sanamaŋ, HOGV.p159, samaŋ, BAHL.pdfp130,
samma, NKEV.p336, , forehead/front, \#0038, V269, ,

\begin{itemize}
\tightlist
\item
  Pinnow 1959: V269 / MKCD: ---
\end{itemize}

The Remo form \emph{gutumoŋ} is irregular and seems to be lacking the
initial syllable \emph{*sx}. Gutob \emph{amuŋ} and Juang \emph{ɛmɔŋ} are
also unexpected. Both instances might represent derivational changes and
not a phonological loss of \emph{*s}. This leaves the possibility that
the lack of /s/ in Gorum and Sora is also not the regular s-loss, but
the result of some morphological process.

\paragraph{\texorpdfstring{\emph{*sv₍₂₂₎} `plough (v)' (\#0042-2)
\emph{*s₁} or
\emph{*s₂}}{*sv₍₂₂₎ plough (v) (\#0042-2) *s₁ or *s₂}}\label{sv-plough-v-0042-2-s-or-s}

(or), FR, (or), RSED.p195, se, BDBH.2706, sui, AG08.p650, si, PKED.p143,
si, PJDW.p276, si, PGEG.p42, si, CDES.p143, siu, BMED.p175, si:,
HOGV.p170, si:, BAHL.p135, ---, ---, *sx, plough (v), \#0042, V99, ,

\begin{itemize}
\tightlist
\item
  Pinnow 1959: V99 / MKCD: ---
\end{itemize}

This etymon is unattested in Sora or Gorum. Because of this, it cannot
be decided whether \emph{*sx} belongs to \emph{*s₁} or \emph{*s₂}.
However, s-loss in Sora and Gorum would result in a verb root consisting
of a single vowel. This would motivate the replacement of this lexeme in
the two languages. For this reason, \emph{*sx} `plough (v)' has been
listed under \emph{*s₁}, but the available evidence und the current
unerstanding of \emph{*s₁} or \emph{*s₂} does not allow for a final
decision.

MKCD 1599 \emph{*bcuər} is not a good candidate.

\subparagraph{\texorpdfstring{\emph{*saˀp} `grab (v)' (\#0048-3)
\emph{*s₁} or
\emph{*s₂}}{*saˀp grab (v) (\#0048-3) *s₁ or *s₂}}\label{saux2c0p-grab-v-0048-3-s-or-s}

---, ---, (sakab), RSED.p246, sop', BDBH.2748, sob, GGEG.p113, (suˀb),
PKED.p188, sɔb, PJDW.p277, saʔ, PGEG.p42, sap', CDES.p28, sa:b,
BMED.p163, sab, DHED.p296, sa:b, BAHL.pdfp131, sa:p, NKEV.p337, *sxˀp,
grab (v), \#0048, , ,

\begin{itemize}
\tightlist
\item
  MKCD 1236 \emph{*{[}c{]}kiip}; \emph{*{[}c{]}kiəp};
  \emph{*t{[}₁{]}kiəp}; \emph{*ckap}; \emph{*t₁kap}; \emph{ckuəp}
\item
  MKCD 1243 *cap; *caap; *ciəp; *cip; *cup
\end{itemize}

The connection to MKCD 1236 is not strong. Reflexes of \emph{*t₁} should
remain a stop, while the reflexes of the cluster \emph{*{[}c{]}k} are
not well understood. It could be a case of type 2a cluster splitting by
second consonant loss (CᵢCᵢᵢ → Cᵢ). Thus \emph{*ckap} → \emph{*cap} →
\emph{*saˀp} or \emph{*ckap} → \emph{*skap} → \emph{*saˀp}.

This etymon is unattested in Sora or Gorum. Because of this, it cannot
be decided whether \emph{*saˀp} belongs to \emph{*s₁} or \emph{*s₂}.

\subparagraph{\texorpdfstring{\emph{*soŋ} `buy/sell (v)'
(\#0060-1)}{*soŋ buy/sell (v) (\#0060-1)}}\label{soux14b-buysell-v-0060-1}

oŋ, FR, ---, ---, suŋ, BDBH.2635, soŋ, GZ65.370, soŋ, PKED.p185, soŋ,
PJDW.p278, so, PGEG.p42, ---, ---, ---, ---, ---, ---, ―, ---, ---, ---,
*soŋ, buy/sell (vt), \#0060, K209, ,

\begin{itemize}
\tightlist
\item
  Pinnow 1959: K209 / MKCD: ―
\end{itemize}

Pinnow (1959, p.~224) connects Pal(aung) \emph{jʌːŋ}, \emph{jɔːŋ} `to
sell' and Mon \emph{swʌ̃} `to sell'.

\subparagraph{\texorpdfstring{\emph{*sv₍₆₎bv₍₁₂₎l} `sweet'
(\#0061-1)}{*sv₍₆₎bv₍₁₂₎l sweet (\#0061-1)}}\label{svbvl-sweet-0061-1}

---, ---, ---, ---, subu, BDBH.2665, subul, AG08.p651, sebol, PKED.p180,
---, ---, ―, ---, sebel, CDES.p1194, sibil, EMV13.p3943, sibil,
DHED.p316, sebel, DSKW.@21820, simil, NKEV.p338, *sv₍₆₎bxl, sweet,
\#0061, V257, ,

\begin{itemize}
\tightlist
\item
  Pinnow 1959: V257 / MKCD: ---
\end{itemize}

\subparagraph{\texorpdfstring{\emph{*siŋi} `sun' (\#0075-1) \emph{*s₁}
or
\emph{*s₂}}{*siŋi sun (\#0075-1) *s₁ or *s₂}}\label{siux14bi-sun-0075-1-s-or-s}

---, ---, ---, ---, siŋi, BDBH.2543, siN, AG08660, siŋ, PKED.p183, siŋ,
PJDW.p244, sni, PGEG.p34, siɲ, CDES.p193, siŋi, BMED.p174, siŋi,
DHED.p319, si:ŋ, BAHL.p136, ---, ---, *siŋi, sun, \#0075, V286, 31,

\begin{itemize}
\tightlist
\item
  Pinnow 1959: V286 / MKCD: 31 \emph{*t₂ŋiiʔ}
\end{itemize}

\subparagraph{\texorpdfstring{\emph{*sv₍₆₎lv₍₁₂₎ˀp} `gazelle'
(\#0084-1)}{*sv₍₆₎lv₍₁₂₎ˀp gazelle (\#0084-1)}}\label{svlvux2c0p-gazelle-0084-1}

alu'b, FR, əle:b, RSED.p7, sulup, BDBH.2688, sulub, GGEG.p116, selhob,
PKED.p180, silib, PJDW.p278, sloʔ, PGEG.p43, selep', CSED.p571, silib,
BMED.p173, silib, DHED.p317, seleb, DSKW@21960, ---, ---, *sxlxˀp,
gazelle, \#0084, V233, ,

\begin{itemize}
\tightlist
\item
  Pinnow 1959: V233 / MKCD ---
\end{itemize}

\paragraph{\texorpdfstring{\emph{*s₂}
(Onset)}{*s₂ (Onset)}}\label{s-onset-1}

\begin{longtable}[]{@{}llllllllllll@{}}
\toprule
Gorum & Sora & Remo & Gutob & Kharia & Juang & Gtaʔ & Santali & Mundari
& Ho & Korwa & Korku\tabularnewline
\midrule
\endhead
s & s & s & s & s & s & s & s & s & s & s & s\tabularnewline
\bottomrule
\end{longtable}

\subparagraph{\texorpdfstring{\emph{*Kʰv₍₄₎su} `fever/pain'
(\#0026-3)}{*Kʰv₍₄₎su fever/pain (\#0026-3)}}\label{kux2b0vsu-feverpain-0026-3}

asu, FR, asu:/əsu:, RSED.p42, siʔ, BDBH.2610, isi, GGEG.p93, kosu/kusu,
PKED.p107, kasu, PJDW.p220, aʔsu, PGEG.p4, haso, CDES.p135, ha:su,
BMED.p67, hasu, HOGV.p147, hasu:, BAHL.p145, kaSu, NKEV.p315, *Kʰxsu,
fever/pain, \#0026, V247, 44,

\begin{itemize}
\tightlist
\item
  Pinnow 1959: V247 / MKCD: 44 \emph{*{[}c{]}uuʔ}
\end{itemize}

\subparagraph{\texorpdfstring{\emph{*si₂m} `chicken'
(\#0028-1)}{*si₂m chicken (\#0028-1)}}\label{sim-chicken-0028-1}

---, ---, kənsi:m, RSED.p131, gisiŋ, BDBH.856, gisiŋ, AG08.p651, siŋkoy,
PKED.p183, sɛŋkɔe, PJDW.p275, gsæŋ, PGEG.p23, sim, CDES.p30, sim,
BMED.p173, sim, HOGV.p151, si:m, BAHL.p135, ---, ---, *si₂m, chicken,
\#0028, V315, 1324,

\begin{itemize}
\tightlist
\item
  Pinnow 1959: V315 / MKCD: 1324 \emph{*cim}; \emph{*ciim};
  \emph{*ciəm}; \emph{*caim}; \emph{*cum}
\end{itemize}

\subparagraph{\texorpdfstring{\emph{*usal} `skin' V₁
(\#0036-2)}{*usal skin V₁ (\#0036-2)}}\label{usal-skin-v-0036-2}

usal, FR, usal, RSED.p308, usa, BDBH.173, isa:l, HLKS.V149, usal,
PKED.p300, chalo, PJDW.p180, ugsa, PGEG.p6, chal, CDES.p176, , , , , , ,
sa:li, NKEV.p337, , skin, \#0036, V149, ,

\begin{itemize}
\tightlist
\item
  Pinnow 1959: V149 / MKCD: ---
\end{itemize}

Juang and Santali /ch/ are unexpected and the lack of the initial vowel
in Juang, Santali, and Korku suggest two distinct etyma.

\begin{longtable}[]{@{}llllllllllll@{}}
\toprule
Gorum & Sora & Remo & Gutob & Kharia & Juang & Gtaʔ & Santali & Mundari
& Ho & Korwa & Korku\tabularnewline
\midrule
\endhead
s & s & s & s & s & (ch) & s & (ch) & --- & --- & --- &
(s)\tabularnewline
\bottomrule
\end{longtable}

\subparagraph{\texorpdfstring{\emph{*I₍₂₃₎sin} `boil (v)'
(\#0046-2)}{*I₍₂₃₎sin boil (v) (\#0046-2)}}\label{isin-boil-v-0046-2}

asin, FR, əsin, RSED.p16, nsiŋ, BDBH.1641, isin, GZ65.173, isin,
PKED.p81, isinɔ, JLIC.v65, nsiŋ, PGEG.p37, isin, CDES.p39, isin,
BMED.p77, isin, DHED.p153, isiŋ, BAHL.p12, isin, DSKO.12071, *I₍₂₃₎sin,
boil (v), \#0046, V86, ,

\begin{itemize}
\tightlist
\item
  Pinnow 1959: V86 / MKCD: 1137 \emph{*ciinʔ} (\textgreater{}
  Pre-Bahnaric \emph{*cin}); \emph{*ciən{[} {]}}; \emph{*cain{[} {]}};
  `cooked'
\end{itemize}

\subparagraph{\texorpdfstring{\emph{*xsər} `dry'
(\#0055-2)}{*xsər dry (\#0055-2)}}\label{xsux259r-dry-0055-2}

asar, FR, asar, RSED.p42, nsor, BDBH.1657, usor, AG08.p650, kosor,
PKED.p155, kosor, PJDW.p229, nswar, PGEG.p37, ---, ---, ---, ---, ---,
---, ---, ---, ---, ---, *xsər, dry, \#0055, V260, 160,

\begin{itemize}
\tightlist
\item
  Pinnow 1959: V183 / MKCD: 160 \emph{*rɔʔ}; \emph{*rɔs}, ( \emph{*rɔs
  rɔs} \textgreater{}?) \emph{*srɔs}
\end{itemize}

\subparagraph{\texorpdfstring{\emph{*əsel} `white'
(\#0065-2)}{*əsel white (\#0065-2)}}\label{ux259sel-white-0065-2}

asel, FR, ---, ---, ---, ---, ---, ---, osel, PKED.p216, ---, ---, ---,
---, e̠se̠l, BSDV2.p343, esel, BMED.p56, esel, DHED.p102, hesel,
BAHL.p149, esel, HLKS.V255, *əsel, white, \#0065, V255, ,

\begin{itemize}
\tightlist
\item
  Pinnow 1959: V255 / MKCD: ---
\end{itemize}

\subparagraph{\texorpdfstring{\emph{*(saŋ)saŋ} `tumeric'
(\#0072-1)}{*(saŋ)saŋ tumeric (\#0072-1)}}\label{saux14bsaux14b-tumeric-0072-1}

saŋsaŋ, FR, sansaŋ, RSED.249, saŋsaŋ, BDBH.400, saŋsaŋ, GZ63.226,
saŋsaŋ, PKED.p176, saŋsaŋ, PJDW.p268, ssia, PGEG.p42, sasaŋ, CDES.p230,
sasaŋ, BMED.p157, sasaŋ, DHED.p307, ---, ---, sasan, Korku.txt.24491,
*(saŋ)saŋ, turmeric/yellow, \#0072, V271, ,

\begin{itemize}
\tightlist
\item
  Pinnow 1959: V271 / MKCD: ---
\end{itemize}

\subparagraph{\texorpdfstring{\emph{*sƏn} `chase (v)'
(\#0087-2)}{*sƏn chase (v) (\#0087-2)}}\label{sux259n-chase-v-0087-2-1}

san, FR, san, RSED.p248, sensen, BDBH.2714, ---, ---, san, PKED.p176,
(saŋgem), PJDW.p273, ---, ---, sen, CSED.p572, sen, BMED.p172, sen,
DHED.p311, sen, BAHL.p138, sen(e), NKEV.p337, *sƏn, chase (v), \#0087,
V300, 899,

\begin{itemize}
\tightlist
\item
  Pinnow 1959: V300 / MKCD: 899 \emph{*təɲ}
\end{itemize}

\subparagraph{\texorpdfstring{\emph{*bv₍₃₁₎/bv₍₃₁₎v₍₃₁₎ˀ/bv₍₃₁₎sv₍₃₂₎}
`sated (v)'
(\#0098-3)}{*bv₍₃₁₎/bv₍₃₁₎v₍₃₁₎ˀ/bv₍₃₁₎sv₍₃₂₎ sated (v) (\#0098-3)}}\label{bvbvvux2c0bvsv-sated-v-0098-3}

buʔ, FR, beˀ, RSED.p56, busu, BDBH.1960, busu, Z1965.72, beso/u,
PKED.p20, bisu, PJDW.p14, bse, PGEG.p14, bi(ʔ), CSED.p67, bi:(ʔ)/biu,
BMED.22, bi:, DHED.p35, bi:, BAHL.p106, ---, ---, *bx, be sated (v),
\#0098, V319, 259,

\begin{itemize}
\tightlist
\item
  Pinnow 1959: V319 / MKCD: 259 \emph{*bhiiʔ}
\end{itemize}

\subsubsection{Labials}\label{labials}

\begin{longtable}[]{@{}lllllllllllll@{}}
\toprule
Gorum & Sora & Remo & Gutob & Kharia & Juang & Gtaʔ & Santali & Mundari
& Ho & Korwa & Korku &\tabularnewline
\midrule
\endhead
l & l & l & l & l & l & l & l & l & l & l & l &
\emph{*l₁}\tabularnewline
l & \textbf{n} & ∅ & l & \textbf{ɽ} & \textbf{ɭ} & ∅ & l & l & l & l & l
& \emph{*l₂}\tabularnewline
l & l & ∅ & l & l & \textbf{n} & ∅ & l & l & l & l & l &
\emph{*l₃}\tabularnewline
l & l & ∅ & l & l & \textbf{r} & ∅ & l & l & l & l & l &
\emph{*l₄}\tabularnewline
\bottomrule
\end{longtable}

There is a consistent l-loss in final position in Remo and Gtaʔ. The
reflex set \emph{*l₂} is only attested in one etymon -- \emph{*kᵊla}
`tiger' -- and the conditions for its development are not understood.
The differences between \emph{*l₁}, \emph{*l₃}, and \emph{*l₄} are based
on Juang. In Juang in some circumstances \emph{*l} occurs as /n/
(\emph{*l₃}) and in some as /r/ (\emph{*l₄}). The attested etyma for
\emph{*l₄} are verbs with the form \emph{*tVl}.

\paragraph{\texorpdfstring{\emph{*l₁}}{*l₁}}\label{l}

\begin{longtable}[]{@{}lllllllllllll@{}}
\toprule
Gorum & Sora & Remo & Gutob & Kharia & Juang & Gtaʔ & Santali & Mundari
& Ho & Korwa & Korku &\tabularnewline
\midrule
\endhead
l & l & l & l & l & l & l & l & l & l & l & l &\tabularnewline
l & l & ∅ & l & l & l & ∅ & l & l & l & l & l & final\tabularnewline
\bottomrule
\end{longtable}

The group \emph{*l₁} continues proto-Munda \emph{*l} consistently as
/l/, except for the l-loss in final position in Remo and Gtaʔ.

\subparagraph{\texorpdfstring{\emph{*laŋ} `tongue'
(\#0003-1)}{*laŋ tongue (\#0003-1)}}\label{laux14b-tongue-0003-1}

laŋ, FR, əlaŋ, RSED.p158, leaŋ, BDBH.2423, laʔŋ, AG08.p638, laŋ,
PKED.p173, elaŋ, PJDW.p191, nlia, PGEG.p36, alaŋ, CDES.p203, a:la:ŋ,
BMED.p5, (leʔ), DHED.p208, a:la:ŋ, BAHL.p11, laŋ, NKEV.p322, la(a)ŋ,
tongue, \#0003, V14, , 44

\begin{itemize}
\tightlist
\item
  Pinnow 1959: V14 / MKCD: ---
\end{itemize}

\subparagraph{\texorpdfstring{\emph{*v₍₉₎laŋ} `thatch'
(\#0014-2)}{*v₍₉₎laŋ thatch (\#0014-2)}}\label{vlaux14b-thatch-0014-2}

alaŋ, FR, əlaŋ, RSED.p158, lɔŋ, BDBH.2437, uloŋ, AG08.p644, oloŋ,
PKED.p214, oloŋ, PJDW.p254, nlo, PGEG.p36, ---, ---, ---, ---, ---, ---,
---, ---, ---, ---, *v₍₉₎laŋ, thatch, \#0014, V270, 749,

\begin{itemize}
\tightlist
\item
  Pinnow 1959: V270 / MKCD: 749 \emph{*{[}p{]}laŋ}; \emph{*{[}p{]}laiŋ}
\end{itemize}

\subparagraph{\texorpdfstring{\emph{*bul} `drunk (v)'
(\#0016-3)}{*bul drunk (v) (\#0016-3)}}\label{bul-drunk-v-0016-3}

bṵl, FR, buʔul, Sora.txt.18922, bu, BDBH.1900, bil, AG08.p672, bul,
PKED.p39, buli, PJDW.p174, busaʔ, PGEG.p13, bul, CDES.p58, bul,
BMED.p25, bul, HOGV.p155, bubul, BAHL.p108, bubul, NKEV.p70, , drunk,
\#0016, V105, 1765,

\begin{itemize}
\tightlist
\item
  Pinnow 1959: V105 / MKCD: 1765 \emph{*ɓul}; \emph{*ɓuul}
\end{itemize}

\subparagraph{\texorpdfstring{\emph{*buluuˀ} `thigh'
(\#0017-3)}{*buluuˀ thigh (\#0017-3)}}\label{buluuux2c0-thigh-0017-3}

bulu, FR, bulu:, RSED.p64, buli/bili, BDBH.1949/1890, bili, DSGU.2681,
bhulu, PKED.p32, bulu, PJDW.p174, bulu, PGEG.p13, bulu, CDES.p199, bulu,
BMED.p25, bulu, HOGV.p183, bu:l, BAHL.p109, bulu, NKEV.p295, *buluuˀ,
thigh, \#0017, V145, 223,

\begin{itemize}
\tightlist
\item
  Pinnow 1959: V145 / MKCD: 223 \emph{*bluuʔ}
\end{itemize}

\subparagraph{\texorpdfstring{\emph{*bv₍₁₃₎lv₍₁₁₎} `ripe'
(\#0018-3)}{*bv₍₁₃₎lv₍₁₁₎ ripe (\#0018-3)}}\label{bvlv-ripe-0018-3}

---, ---, ---, ---, bulu, BDBH.1591, bulu, AG08.p644, belom, PKED.p19,
bilim, PJDW.p167, ble, PGEG.p13, bele, CDES.p161, bili, BMED.p23, bili,
HOGV.p156, bhi:li:, BAHL.p115, bili, NKEV.p293, *bv₍₁₃₎lv₍₁₁₎, ripe,
\#0018, V232, ,

\begin{itemize}
\item
  Pinnow 1959: V232 / MKCD: ---
\item
  MKCD 2080 \emph{*bl{[}ɔ{]}h} `finished'
\item
  MKCD 1878 \emph{*lʔas} `ripe'
\end{itemize}

\subparagraph{\texorpdfstring{\emph{*bal} `to burn' (\#0023-3)
\emph{*l₁}, \emph{*l₃}, or
\emph{*l₄}}{*bal to burn (\#0023-3) *l₁, *l₃, or *l₄}}\label{bal-to-burn-0023-3-l-l-or-l}

Go. \emph{bal}; So. \emph{ba:l} (RSED.p49); Gu. \emph{bal} (GZ65.43);
Gt. \emph{ba} (PGEG.p9); Sa. \emph{bal} (BSDV1.p1840); Mu. \emph{bal}
(BMED.p18); Ho \emph{bal} (HOGV.p151); Kw. \emph{ba:l} (BAHL.p1050; Ko.
\emph{ba:l} (NKEV.p292)

\begin{itemize}
\tightlist
\item
  Pinnow 1959 --- / MKCD ---
\end{itemize}

\emph{*bal} is listed here under \emph{*l₁}, but the lack of reflexes in
Remo and Juang makes it possible that it belongs to \emph{*l₃} or
\emph{*l₄}.

\subparagraph{\texorpdfstring{\emph{*Olaaˀ}/\emph{*Ola(ˀk)} `leaf' V₁
(\#0035-2)}{*Olaaˀ/*Ola(ˀk) leaf V₁ (\#0035-2)}}\label{olaaux2c0olaux2c0k-leaf-v-0035-2}

olaʔ, FR, o:la:, RSED.p192, ulak', BDBH.169, olag, AG08.p633, ulaʔ,
PKED.p298, olag, PJDW.p254, uliaʔ, PGEG.p124, palha, CDES.p111,
pa:lha:o, BMED.p142, pala, DHED.p259, (sakam), BAHL.pdfp129, pa:la,
NKEV.p331, , leaf, \#0035, V50, 230,

\begin{itemize}
\tightlist
\item
  Pinnow 1959: V50 / MKCD: 230 \emph{*slaʔ}
\end{itemize}

If the forms with intial \emph{p} in North Munda belong to this set,
Santali and Mundari /lh/ would suggest, that these correspondences
constitute a separate set. However, these form probably belong to a
separate set.

\subparagraph{\texorpdfstring{\emph{*usal} `skin' V₁
(\#0036-4)}{*usal skin V₁ (\#0036-4)}}\label{usal-skin-v-0036-4}

usal, FR, usal, RSED.p308, usa, BDBH.173, isa:l, HLKS.V149, usal,
PKED.p300, chalo, PJDW.p180, ugsa, PGEG.p6, chal, CDES.p176, , , , , , ,
sa:li, NKEV.p337, , skin, \#0036, V149, ,

\begin{itemize}
\tightlist
\item
  Pinnow 1959: V149 / MKCD: ---
\end{itemize}

\subparagraph{\texorpdfstring{\emph{*lv₍₁₈₎(N)dx} `laugh (v)'
(\#0037-1)}{*lv₍₁₈₎(N)dx laugh (v) (\#0037-1)}}\label{lvndx-laugh-v-0037-1}

liɖa, FR, ---, ---, (ɖoɖo), BDBH.1283, luɖo, GZ65.228, laɖa, PKED.p202,
lara, PJDW.p236, lwaʔ, PGEG.p32, lanɖa, CDES.p110, la:nɖa:, BMED.p102,
landa, HOGV.p166, la:Nd, BAHL.p127, lanɖa, NKEV.p322, , laugh (v),
\#0037, V302, ,

\begin{itemize}
\tightlist
\item
  Pinnow 1959: V302 / MKCD: ---
\end{itemize}

\subparagraph{\texorpdfstring{\emph{*boŋtel}/\emph{*bitkil} `buffalo'
(\#0054-5) \emph{*l₁}, \emph{*l₃}, or
\emph{*l₄}}{*boŋtel/*bitkil buffalo (\#0054-5) *l₁, *l₃, or *l₄}}\label{boux14btelbitkil-buffalo-0054-5-l-l-or-l}

boŋtel, FR, boŋtel, RSED.p62, buŋte, BDBH.1917, boŋtel, AG08.p647,
boŋtel, PKED.p36, ---, ---, buNʈi, PGEG.p13, bitkil, CDES.p23, ---, ---,
---, ---, ---, ---, biʈkhil, NKEV.p294, *boŋtel, buffalo, \#0054, , ,

Without an attested reflex of this etymon in Juang, it cannot be
determined whether it belongs to set \emph{*l₁}, \emph{*l₃}, and
\emph{*l₄}.

Probably two separate etyma \emph{*boŋtel} and North Munda *bitkil\_.
The form suggests some relation, but the two forms cannot be derived
from proto-Munda by regular sound change.

\subparagraph{\texorpdfstring{\emph{*sv₍₆₎bv₍₁₂₎l} `sweet'
(\#0061-5)}{*sv₍₆₎bv₍₁₂₎l sweet (\#0061-5)}}\label{svbvl-sweet-0061-5}

---, ---, ---, ---, subu, BDBH.2665, subul, AG08.p651, sebol, PKED.p180,
---, ---, ―, ---, sebel, CDES.p1194, sibil, EMV13.p3943, sibil,
DHED.p316, sebel, DSKW.@21820, simil, NKEV.p338, *sv₍₆₎bxl, sweet,
\#0061, V257, ,

\begin{itemize}
\tightlist
\item
  Pinnow 1959: V257 / MKCD: ---
\end{itemize}

\subparagraph{\texorpdfstring{\emph{*lv₍₂₉₎ˀc} `stomach'
(\#0063-3)}{*lv₍₂₉₎ˀc stomach (\#0063-3)}}\label{lvux2c0c-stomach-0063-3-1}

---, ---, (lo:ˀɟ), RSED.p163, suloĭ, BDBH.2692, suloj, AG08.p651,
la(i)ˀj, PKED.p119, (lai), JLIC.n57, slweʔ, PGEG.p43, lac', CDES.p188,
la:iˀ, BMED.p101, la:iʔ, DHED.p204, la:i:ʔ, BAHL.p123, la:j, NKEV.p323,
*lv₍₂₉₎jˀ, stomach, \#0063, K282, ,

\begin{itemize}
\tightlist
\item
  Pinnow 1959: K282 / MKCD: ---
\end{itemize}

Juang \emph{lai} is uncertain as JLIC gives the meaning navel. However,
*l₃\_, and \emph{*l₄} do not occur in inital position.

\subparagraph{\texorpdfstring{\emph{*əsel} `white' (\#0065-4)
\emph{*l₁}, \emph{*l₃}, or
\emph{*l₄}}{*əsel white (\#0065-4) *l₁, *l₃, or *l₄}}\label{ux259sel-white-0065-4-l-l-or-l}

asel, FR, ---, ---, ---, ---, ---, ---, osel, PKED.p216, ---, ---, ---,
---, e̠se̠l, BSDV2.p343, esel, BMED.p56, esel, DHED.p102, hesel,
BAHL.p149, esel, HLKS.V255, *əsel, white, \#0065, V255, ,

\begin{itemize}
\tightlist
\item
  Pinnow 1959: V255 / MKCD: ---
\end{itemize}

\subparagraph{\texorpdfstring{\emph{*xli} `liquor' (\#0067-2)
\emph{*l₁}, \emph{*l₃}, or
\emph{*l₄}}{*xli liquor (\#0067-2) *l₁, *l₃, or *l₄}}\label{xli-liquor-0067-2-l-l-or-l}

ali, FR, əli/ali, RSED.p8, ili, BDBH.120, ili, AG08.p672, ---, ---, ---,
---, ---, ---, ---, ---, ili, BMED.p75, ili, DHED.p151, ---, ---, ---,
---, *xlx, liquor, \#0067, V85, ,

\begin{itemize}
\tightlist
\item
  Pinnow 1959: V85 / MKCD: ---
\end{itemize}

\subparagraph{\texorpdfstring{\emph{*ɟəlu₅} `meat' (\#0069-2)
\emph{*l₁}, \emph{*l₃}, or
\emph{*l₄}}{*ɟəlu₅ meat (\#0069-2) *l₁, *l₃, or *l₄}}\label{ux25fux259lu-meat-0069-2-l-l-or-l}

---, ---, ɟelu:, RSED.p123, sili/seli, BDBH.2599/2731, seli, AG08.p674,
―, ---, ---, ---, cili, PGEG.p15, jel, CDES.p120, jilu, BMED.p83, jilu,
DHED.p165, ---, ---, jilu, NKEV.p311, *ɟəlu₅, meat, \#0069, V228, 204?,

\begin{itemize}
\tightlist
\item
  Pinnow 1959: V228 / MKCD: ---
\end{itemize}

A possibly connected MKCD etymon is MKCD 204 \emph{*{[}c{]}nlu{[}u{]}ʔ}
`edible grub' only attested in Bahnaric.

\subparagraph{\texorpdfstring{\emph{*uli} `mango (ripe)'
(\#0070-2)}{*uli mango (ripe) (\#0070-2)}}\label{uli-mango-ripe-0070-2}

---, ---, u:l, RSED.p304, uli, BDBH.171, ili, DSGU\#4032, ---, ---,
holɛ, PJDW.p205, uli, PGEG.p7, ul, CDES.p118, uli, BMED.p192, uli,
DHED.p370, u:l, BAHL.p19, ---, ---, *xlx, mango (ripe), \#0070,
V144/V400e/K496, ,

\begin{itemize}
\tightlist
\item
  Pinnow 1959: V144;V400e;K496 / MKCD: ---
\end{itemize}

\subparagraph{\texorpdfstring{\emph{*lutu(uˀ)r} `ear'
(\#0073-1)}{*lutu(uˀ)r ear (\#0073-1)}}\label{lutuuux2c0r-ear-0073-1}

luˀd, FR, luˀd, RSED.p165, luntur, BDBH.2386, litir, AG08.p652, lutur,
PKED.p127, lutur/lutuʔ, PJDW.p239, nlug, PGEG.p36, lutur, CDES.p60,
lutur, BMED.p110, lutur, DHED.p216, lutur, BAHL.p128, lutur, NKEV.p324,
*lutu(uˀ)r, ear, \#0073, V147, 1621,

\begin{itemize}
\tightlist
\item
  Pinnow 1959: V147 / MKCD: 1621 \emph{*kt₂uur}; \emph{*kt₂uər}
\end{itemize}

\subparagraph{\texorpdfstring{\emph{*ɟVlVN} `long/tall'
(\#0082-3)}{*ɟVlVN long/tall (\#0082-3)}}\label{ux25fvlvn-longtall-0082-3}

zuleŋa, FR, ɟele:n, RSED.p123, sileŋ, BDBH.2601, silej, AG08.p651,
jhelo(g, b, m), PKED.p92, ɟaliŋ, PJDW.210, clæ, PGEG.p15, jeleɲ,
CSED.p260, jiliŋ, BMED.p83, jiliɲ, DHED.p165, ---, ---, ---, ---,
*ɟxlxN, long/tall, \#0082, V340, 740,

\begin{itemize}
\tightlist
\item
  Pinnow 1959: V340 / MKCD: 740 \emph{*jiliiŋ} (\& \emph{*jiliŋ}?);
  \emph{*jla{[}i{]}ŋ} `long'
\end{itemize}

As discussed above, this might be a fused set of two or more etyma
meaning long, tall, high, slim, and related concepts all based in the
consonantal frame \emph{*ɟVlVN}. The vowel alternations do not seem to
affect the reflexes of the consonants \emph{*ɟ} and \emph{*l}.

\subparagraph{\texorpdfstring{\emph{*sv₍₆₎lv₍₁₂₎ˀp} `gazelle'
(\#0084-4)}{*sv₍₆₎lv₍₁₂₎ˀp gazelle (\#0084-4)}}\label{svlvux2c0p-gazelle-0084-4}

alu'b, FR, əle:b, RSED.p7, sulup, BDBH.2688, sulub, GGEG.p116, selhob,
PKED.p180, silib, PJDW.p278, sloʔ, PGEG.p43, selep', CSED.p571, silib,
BMED.p173, silib, DHED.p317, seleb, DSKW@21960, ---, ---,
*sv₍₆₎lv₍₁₂₎ˀp, gazelle, \#0084, V233, ,

\begin{itemize}
\tightlist
\item
  Pinnow 1959: V233 / MKCD ---
\end{itemize}

\subparagraph{\texorpdfstring{\emph{*laˀk} `to scrape'
(\#0093-1)}{*laˀk to scrape (\#0093-1)}}\label{laux2c0k-to-scrape-0093-1}

laʔ, FR, ---, ---, ---, ---, lag, Z1965.205, laʔ, PKED.p118, lag,
PJDW.p235, liaʔ, PGEG.p31, lak', CSED.p359, ---, ---, laʔ, DHED.p203,
---, ---, laʔ, DSKO.17551, *laˀk, scrape (v), \#0093, ---, 418,

\begin{itemize}
\tightlist
\item
  Pinnow 1959: --- / MKCD: 418 \emph{*l{[}a{]}k}
\end{itemize}

\paragraph{\texorpdfstring{\emph{*l₂}
(Medial/Final)}{*l₂ (Medial/Final)}}\label{l-medialfinal}

\begin{longtable}[]{@{}llllllllllll@{}}
\toprule
Gorum & Sora & Remo & Gutob & Kharia & Juang & Gtaʔ & Santali & Mundari
& Ho & Korwa & Korku\tabularnewline
\midrule
\endhead
l & n & ∅ & l & ɽ & ɭ & ∅ & l & l & l & l & l\tabularnewline
\bottomrule
\end{longtable}

\subparagraph{\texorpdfstring{\emph{*kᵊla} `tiger'
(\#0004-2)}{*kᵊla tiger (\#0004-2)}}\label{kux1d4ala-tiger-0004-2}

kulaʔ, FR, kina:, RSED.p140, ŋku, MVol.p733, gikil, AG08.p651, kiɽoʔ,
PKED.p102, kiɭog, PJDW.p224, nku, PGEG.p36, kul, CDES.p201, kula:,
BMED.p98, kula, HOGV.p183, ku:l, BAHL.p33, kula, NKEV.p319, *kᵊla,
tiger, \#0004, V281, 197,

\begin{itemize}
\tightlist
\item
  Pinnow 1959: V281 / MKCD: 197 \emph{*klaʔ}
\end{itemize}

\paragraph{\texorpdfstring{\emph{*l₃}
(Medial/Final)}{*l₃ (Medial/Final)}}\label{l-medialfinal-1}

\begin{longtable}[]{@{}llllllllllll@{}}
\toprule
Gorum & Sora & Remo & Gutob & Kharia & Juang & Gtaʔ & Santali & Mundari
& Ho & Korwa & Korku\tabularnewline
\midrule
\endhead
l & l & ∅ & l & l & n & ∅ & l & l & l & l & l\tabularnewline
\bottomrule
\end{longtable}

\subparagraph{\texorpdfstring{\emph{*bVrəl} `raw'
(\#0019-5)}{*bVrəl raw (\#0019-5)}}\label{bvrux259l-raw-0019-5}

---, ---, ---, ---, buro, BDBH.1937, burol, GZ65.74, borol, PKED.p25,
boron, PJDW.p171, brwa, PGEG.p14, be̠re̠l, CDES.p211, berel, BMED.p21,
berel, HOGV.p185, berel, BAHL.p111, boboɽ, NKEV.p294, *bxrəl, raw,
\#0019, V253, ,

\begin{itemize}
\tightlist
\item
  Pinnow 1959: V253 / MKCD: ---
\end{itemize}

\subparagraph{\texorpdfstring{\emph{*sVŋəl} `fuel'
(\#0021-5)}{*sVŋəl fuel (\#0021-5)}}\label{svux14bux259l-fuel-0021-5}

aŋal, FR, aŋəl, RSED.p37, suŋo, BDBH.2638, suõl, GZ63.216, soŋgol,
PKED.p186, sɛŋon, PJDW.p276, sua, PGEG.p43, se̠ŋge̠l, CDES.p73, seŋgel,
BMED.p172, seŋgel, HOGV.p158, seNgel, BAHL.p137, ---, ---, *sxŋəl, fuel,
\#0021, V252, 1723,

\begin{itemize}
\tightlist
\item
  Pinnow 1959: V252 / MKCD 1723 \emph{*j{[}n{]}ŋəl}
\end{itemize}

\subparagraph{\texorpdfstring{\emph{*bel} `spread (v)'
(\#0022-3)}{*bel spread (v) (\#0022-3)}}\label{bel-spread-v-0022-3}

bil, FR, bel/bɪ:l, RSED.p56/58, be-sak', BDBH.1982, be-sag, Z1965.50,
bel, PKED.p18, bɛn, PJDW.p166, beʔ, PGEG.p11, bel, CDES.p184, bil,
BMED.p24, bil, HOGV.p179, bel, BAHL.p111, (bi)bil, NKEV.p293, *bel,
spread (vt), \#0022, V221, 1761,

\begin{itemize}
\tightlist
\item
  Pinnow 1959: V221 / MKCD: 1761 \emph{*b{[}e{]}l} (\emph{*beel}?)
\end{itemize}

Gtaʔ /ʔ/ is unexpected. However, DSGT\#1651 (Chatterji) has \emph{bE}
`to unroll' without a glottal stop. This form is probably best read as
\emph{bɛ} that is PGEG /be/, as opposed to DSGT\#1691 \emph{bEe} `to
send' which corresponds to PGEG /bæ/ `to send' (from pre-Gtaʔ
\emph{*baɲ}).

\subparagraph{\texorpdfstring{\emph{*xrel} `hail/pebble'
(\#0032-4)}{*xrel hail/pebble (\#0032-4)}}\label{xrel-hailpebble-0032-4}

aril, FR, are:l, RSED.p39, are, BDBH.43, arel, HLKS.V225, arel, PKED.p7,
aɭɛn, PJDW.p158, hare, PGEG.p24, arel, CDES.p88, a:ɽil, BMED.p10, aril,
HOGV.p161, a:ril, BAHL.p10, ---, ---, *xrel, hail/pebble, \#0032, V225,
1791,

\begin{itemize}
\tightlist
\item
  Pinnow 1959: V225 / MKCD: 1791 \emph{*pril}; \emph{*priəl}
\end{itemize}

\subparagraph{\texorpdfstring{\emph{*ɟal} `to lick'
(\#0043-3)}{*ɟal to lick (\#0043-3)}}\label{ux25fal-to-lick-0043-3}

zaleˀb, FR, ɟa:l, RSED.p119, salep', BDBH.2523, sal, GZ63.228, jal,
PKED.p82, janɔ, JLIC.v372, cca, PGEG.p14, jal, CDES.p112, jal, EM.p1965,
jal, HOGV.p164, (jaɽa:ʔ), BAHL.p60, jal, NKEV.p312, *ɟal, lick (v),
\#0043, V13, 1409,

\begin{itemize}
\tightlist
\item
  Pinnow 1959: V13 / MKCD: 1409 \emph{*{[}c{]}limʔ};
  \emph{*{[}c{]}liəmʔ}; \emph{*{[}c{]}laim{[} {]}}
\end{itemize}

Korwa \emph{jaɽa:ʔ} with /ɽ/ is problematic. Even if it could be
interpreted as parallel to Gorum and Gutob and thus reflecting
\emph{*ɟalVˀp}, the form remains problematic especially since /ʔ/ is not
an attested reflex of \emph{*ˀp}.

\subparagraph{\texorpdfstring{\emph{*dal} `to cover'
(\#0047-3)}{*dal to cover (\#0047-3)}}\label{dal-to-cover-0047-3}

ɖal, FR, dal, RSED.p73, ɖalu, BDBH.1210, ɖal, GZ65.80, ɖal, PKED.p42,
ɖan, MJTL.p96, ɖa, PGEG.p16, dapal/dalo̠p', CDES.p40, dapal/dālob,
BMED.p35/36, dapal/dalop, HOGV.p153, ---, ---, da:l, NKEV.p299, *dal,
cover (v), \#0047, V3, 1745,

\begin{itemize}
\tightlist
\item
  Pinnow 1959: V3 / MKCD: 1745 \emph{*kdiil}; \emph{*kdiəl};
  \emph{*kdəl}
\end{itemize}

\subparagraph{\texorpdfstring{\emph{*gəle} `ear of corn' V₁ (\#0077-3)
\emph{*l₁}, \emph{*l₃}, or
\emph{*l₄}}{*gəle ear of corn V₁ (\#0077-3) *l₁, *l₃, or *l₄}}\label{gux259le-ear-of-corn-v-0077-3-l-l-or-l}

gali, FR, gale, RSED.p96, gileker, DSBO.11781, gile, GTXT.7791, gɔlɛ,
HLKS.V182, (ɔnɔ), PJDW.p255, (konto-ja), PGEG.p28, gele, CDES.p185,
gele, EM.p1418, gele, DHED.p111, geleʔ, BAHL.p45, (kelʈa), NKEV.p317,
*gxle, ear of corn, \#0077, V182, 1577,

\begin{itemize}
\tightlist
\item
  Pinnow 1959: V182 / MKCD: 1577 \emph{*gur}; \emph{*guər}
\end{itemize}

If Juang \emph{ɔnɔ} is a genuine reflex of \emph{*gVlV}, it belongs to
\emph{*l₃}. However, the lack of an inital /g/ suggests that it is not
connected. Accordingly, *gxlx\_ `ear of corn' can belong to \emph{*l₁},
\emph{*l₃}, or \emph{*l₄}.

\subparagraph{\texorpdfstring{\emph{*tVrel} `ebony'
(\#0083-5)}{*tVrel ebony (\#0083-5)}}\label{tvrel-ebony-0083-5}

---, ---, tarel, RSED.p138, tire, BDBH.1390, ---, ---, ti(ɽr)(ei)l,
PKED.p200, tɛrɛn, PJDW.p285, tre, PGEG.p46, terel, CSED.p626, tiril,
BMED.p188, tiril, DHED.p355, ---, ---, ---, ---, *txrel, ebony, \#0083,
V227, ,

\begin{itemize}
\tightlist
\item
  Pinnow 1959: V227 / MKCD: ---
\end{itemize}

\paragraph{\texorpdfstring{\emph{*l₄}
(Medial/Final)}{*l₄ (Medial/Final)}}\label{l-medialfinal-2}

\begin{longtable}[]{@{}llllllllllll@{}}
\toprule
Gorum & Sora & Remo & Gutob & Kharia & Juang & Gtaʔ & Santali & Mundari
& Ho & Korwa & Korku\tabularnewline
\midrule
\endhead
l & l & ∅ & l & l & r & ∅ & l & l & l & l & l\tabularnewline
\bottomrule
\end{longtable}

\subparagraph{\texorpdfstring{\emph{*tol} `tie (v)'
(\#0024-3)}{*tol tie (v) (\#0024-3)}}\label{tol-tie-v-0024-3}

tol, FR, tol, RSED.p292, tu, BDBH.1398, tol, AG08.647, tol, PKED.p288,
tor, PJDW.p287, tu, PGEG.p46, to̠l, CDES.p201, tol, BMED.p186, tol,
HOGV.p183, tol, BAHL.p84, ʈol, NKEV.p343, *tol, tie (v), \#0024, V191, ,

\begin{itemize}
\tightlist
\item
  Pinnow 1959: V191 / MKCD: ---
\end{itemize}

\subparagraph{\texorpdfstring{\emph{*tI₍₂₄₎l} `bury (v)'
(\#0049-3)}{*tI₍₂₄₎l bury (v) (\#0049-3)}}\label{til-bury-v-0049-3}

tul, FR, til, RSED.p288, ti, BDBH.1360, til, GZ65.408, til, PKED.p199,
tir, PJDW.p284, ---, ---, (til), RSED.p288, (til), RSED.p288, ---, ---,
ti:l, BAHL.p82, ---, ---, *txl, bury (v), \#0049, ---, ---,

\begin{itemize}
\tightlist
\item
  Pinnow 1959: --- / MKCD: ---
\end{itemize}

\subsubsection{Rhotics}\label{rhotics}

\begin{longtable}[]{@{}lllllllllllll@{}}
\toprule
Gorum & Sora & Remo & Gutob & Kharia & Juang & Gtaʔ & Santali & Mundari
& Ho & Korwa & Korku &\tabularnewline
\midrule
\endhead
r & r & r & r & r & r & r & r & r & r & r & r &
\emph{*r₁}\tabularnewline
r & r & r & r & r & r & r & r & r & r & r & \textbf{ɽ} &
\emph{*r₂}\tabularnewline
r & r & r & r & r & \textbf{ɭ} & r & r & \textbf{ɽ} & r & r & --- &
\emph{*r₃}\tabularnewline
--- & r & \textbf{n} & --- & --- & --- & \textbf{n} & r & r & r & --- &
--- & \emph{*r₄}\tabularnewline
\textbf{∅} & r & r & r & r & \textbf{∅} & r & r & r & r & --- & r &
\emph{*r₅}\tabularnewline
\bottomrule
\end{longtable}

\paragraph{\texorpdfstring{\emph{*r₁}}{*r₁}}\label{r}

\begin{longtable}[]{@{}llllllllllll@{}}
\toprule
Gorum & Sora & Remo & Gutob & Kharia & Juang & Gtaʔ & Santali & Mundari
& Ho & Korwa & Korku\tabularnewline
\midrule
\endhead
r & r & r & r & r & r & r & r & r & r & r & r\tabularnewline
\bottomrule
\end{longtable}

\subparagraph{\texorpdfstring{\emph{*dərv₍₆₎ŋ} `horn' (\#0007-5)
\emph{*r₁} or
\emph{*r₂}}{*dərv₍₆₎ŋ horn (\#0007-5) *r₁ or *r₂}}\label{dux259rvux14b-horn-0007-5-r-or-r}

ɖeraŋ, FR, deraŋ, RSED.p78, deruŋ, BDBH.1266, ---, ---, ɖereŋ, PKED.p44,
---, ---, ɖiraŋ, PGEG.p17, dereɲ, CDSE.p171, diriŋ, BMED.p49, diriɲ,
HOGV.p162, dereŋ, BAHL.p89, ---, ---, *dərv₍₆₎ŋ, horn, \#0007, V347,
699, 34

\begin{itemize}
\tightlist
\item
  Pinnow 1959: V347 UM: \emph{*e},\emph{*ɛ}/ MKCD 699 \emph{*d₂raŋ}
\end{itemize}

\subparagraph{\texorpdfstring{\emph{*səreŋ} `stone' (\#0020-3)
\emph{*r₁} or
\emph{*r₂}}{*səreŋ stone (\#0020-3) *r₁ or *r₂}}\label{sux259reux14b-stone-0020-3-r-or-r}

areŋ, FR, areŋ, RSED.p39, ---, ---, ---, ---, soreŋ, PKED.p187, ---,
---, ---, ---, ---, ---, sereŋ, BMED.p172, sereɲ, HOGV.p175, ---, ---,
---, ---, *səreŋ, stone, \#0020, V183, ,

\begin{itemize}
\tightlist
\item
  Pinnow 1959: V183 / MKCD: ---
\end{itemize}

\subparagraph{\texorpdfstring{\emph{*riˀt} `to grind' (\#0025-1)
\emph{*r₁} or
\emph{*r₂}}{*riˀt to grind (\#0025-1) *r₁ or *r₂}}\label{riux2c0t-to-grind-0025-1-r-or-r}

riˀd, FR, rid, RSED.p233, riʔ, BDBH.2276, riɽ, GZ63.15, riɖ, PKED.p169,
riɖ, PJDW.p266, rig, PGEG.p4, rit', CDES.p86, ri'd, BMED.p159, riɖ,
DHED.p288, ri:ɖ, BAHL.p124, --, ---, *riˀt, grind (v), \#0025, V76,
1056,

\subparagraph{\texorpdfstring{\emph{*tVru₅ˀp} `cloud' (\#0034-3)
\emph{*r₁}, \emph{*r₂}, or
\emph{*r₃}}{*tVru₅ˀp cloud (\#0034-3) *r₁, *r₂, or *r₃}}\label{tvruux2c0p-cloud-0034-3-r-r-or-r}

taruˀb, FR, tarub, RSED.p283, tirib, BDBH.1387, tirib, GZ65.416, tiriˀb,
PKED.p287, ---, ---, trig, PGEG.p46, rimil, CDES.p33, rimil, BMED.p160,
rimil, HOGV.p152, liNbir, BAHL.p127, ---, ---, , cloud, \#0034, V285a, ,

\begin{itemize}
\tightlist
\item
  Pinnow 1959: V285a / MKCD: ---
\end{itemize}

\subparagraph{\texorpdfstring{\emph{*xsər} `dry' (\#0055-4) \emph{*r₁}
or
\emph{*r₂}}{*xsər dry (\#0055-4) *r₁ or *r₂}}\label{xsux259r-dry-0055-4-r-or-r}

asar, FR, asar, RSED.p42, nsor, BDBH.1657, usor, AG08.p650, kosor,
PKED.p155, kosor, PJDW.p229, nswar, PGEG.p37, ---, ---, ---, ---, ---,
---, ---, ---, ---, ---, *xsər, dry, \#0055, V260, 160,

\begin{itemize}
\tightlist
\item
  Pinnow 1959: V183 / MKCD: 160 \emph{*rɔʔ}; \emph{*rɔs}, ( \emph{*rɔs
  rɔs} \textgreater{}?) \emph{*srɔs}
\end{itemize}

\subparagraph{\texorpdfstring{\emph{*bVrV(ˀp/ˀk)} `lung' (\#0066-2)
\emph{*r₁}, \emph{*r₂}, or
\emph{*r₃}}{*bVrV(ˀp/ˀk) lung (\#0066-2) *r₁, *r₂, or *r₃}}\label{bvrvux2c0pux2c0k-lung-0066-2-r-r-or-r}

buroˀb, FR, bəro:, RSED.p46, buruk', BDBH.1936, ---, ---, ---, ---,
(buku), JLIC.n49, breʔ, PGEG.p14, bo̠ro̠, CDES.p116, (borkod'), BMED.p25,
(borkoɖ), DHED.p45, boro, BAHL.p112, , , , lungs, \#0066, , ,

\begin{itemize}
\tightlist
\item
  Pinnow 1959: --- / MKCD: ---
\end{itemize}

\subparagraph{\texorpdfstring{\emph{*ruNkO(ˀp)} `husked rice' (\#0068-1)
\emph{*r₁} or
\emph{*r₂}}{*ruNkO(ˀp) husked rice (\#0068-1) *r₁ or *r₂}}\label{runkoux2c0p-husked-rice-0068-1-r-or-r}

ruŋk, FR, rʊŋkʊ, RSED.p239, ruŋku, BDBH.2291, rukug, AG08.p672,
ruŋkuˀb/rumkuˀb, PKED.p171, ruŋkub, PJDW.p269, rkoʔ, PGEG.p41, ---, ---,
(rukhaɽ), BMED.p163, ---, ---, ---, ---, ---, ---, *ruNkO(ˀp), husked
rice, \#0068, V139, 1820,

\begin{itemize}
\tightlist
\item
  Pinnow 1959: V139 / MKCD: 1820 \_*rk{[}aw{]}ʔ
\end{itemize}

\subparagraph{\texorpdfstring{\emph{*roj}/\emph{*roˀk} `fly'
(\#0071-1)}{*roj/*roˀk fly (\#0071-1)}}\label{rojroux2c0k-fly-0071-1}

aroj, FR, əro:j, RSED.p14, (ayoŋ/ayuŋ), BDBH.39, uroj, GGEG.p93,
(kɔnɖɔi), HLKS.K356, ---, ---, nɖroe, PGEG.p36, ro̠, CDES.p76, roko,
BMED.p161, roko, DHED.p291, roʔo, DSKW.19600, ruku, NKEV.p335, *roj,
fly, \#0071, K356, 1534,

\begin{itemize}
\tightlist
\item
  Pinnow 1959: K356 / MKCD: 1534 Pre-Proto-Mon-Khmer \emph{*ru{[}wa{]}y}
  \textgreater{} \emph{*ruy}; \emph{*ruuy}; \emph{*ruəy};
  Pre-Proto-Mon-Khmer \emph{*ruhay}
\end{itemize}

Gtaʔ /nɖroe/ derives from pre-Gtaʔ \emph{*n(ɖ)roj}. Kharia kɔnɖɔi could
derive from /kɔnrɔi/.

\subparagraph{\texorpdfstring{\emph{*lutu(uˀ)r} `ear'
(\#0073-5)}{*lutu(uˀ)r ear (\#0073-5)}}\label{lutuuux2c0r-ear-0073-5}

luˀd, FR, luˀd, RSED.p165, luntur, BDBH.2386, litir, AG08.p652, lutur,
PKED.p127, lutur/lutuʔ, PJDW.p239, nlug, PGEG.p36, lutur, CDES.p60,
lutur, BMED.p110, lutur, DHED.p216, lutur, BAHL.p128, lutur, NKEV.p324,
*lutu(uˀ)r, ear, \#0073, V147, 1621,

\begin{itemize}
\tightlist
\item
  Pinnow 1959: V147 / MKCD: 1621 \emph{*kt₂uur}; \emph{*kt₂uər}
\end{itemize}

Gorum and Sora \emph{luˀd} as well as Gtaʔ \emph{nlug} are not a regular
reflex \emph{*lVtVr}, but reflect \emph{*lxˀt without a final }*r\_.

\subparagraph{\texorpdfstring{\emph{*maraˀk} `peacock'
(\#0081-3)}{*maraˀk peacock (\#0081-3)}}\label{maraux2c0k-peacock-0081-3}

(marraʔ), FR, ma:ra:, RSED.p173, ---, ---, ---, ---, maraʔ, PKED.p131,
marag, PJDW.p242, ---, ---, marak', CSED.p407, ma:ra:, BMED.p114, mara:,
DHED.p225, mara:q, BAHL.p117, mara, NKEV.p324, *maraˀk, peacock, \#0081,
V27, 416,

\begin{itemize}
\tightlist
\item
  Pinnow 1959: V27 / MKCD: 416 \emph{*mraik{[} {]}}
\end{itemize}

Gorum \emph{marraʔ} `husband' probably belongs to another etymon
connected with MKCD 183 \emph{*mraʔ} `person'.

\subparagraph{\texorpdfstring{\emph{*tVrel} `ebony' (\#0083-3)
\emph{*r₁} or
\emph{*r₂}}{*tVrel ebony (\#0083-3) *r₁ or *r₂}}\label{tvrel-ebony-0083-3-r-or-r}

---, ---, tarel, RSED.p138, tire, BDBH.1390, ---, ---, ti(ɽr)(ei)l,
PKED.p200, tɛrɛn, PJDW.p285, tre, PGEG.p46, terel, CSED.p626, tiril,
BMED.p188, tiril, DHED.p355, ---, ---, ---, ---, *txrel, ebony, \#0083,
V227, ,

\begin{itemize}
\tightlist
\item
  Pinnow 1959: V227 / MKCD: ---
\end{itemize}

The variation of /ɽ/ and /r/ in Kharia is not attested elsewhere.

\paragraph{\texorpdfstring{\emph{*gur} `fall/rain (v)'
(\#0089-3)}{*gur fall/rain (v) (\#0089-3)}}\label{gur-fallrain-v-0089-3}

gur, FR, gur, RSED.p92, gur, BDBH.914, gir, Z1965.132, gur, PKED.p68,
gur, PJDW.p200, gur, PGEG.p21, gur, CSED.p207, gur, EMV5.p1535, gur,
DHED.p122, ---, ---, guru, DSKO\#10541, *gur, fall/rain (v), \#0089,
V106, 1579,

\begin{itemize}
\tightlist
\item
  Pinnow 1959: V106 / MKCD: 1579 \emph{*guur}
\end{itemize}

\subparagraph{\texorpdfstring{\emph{*roˀc} `squeeze/milk (v)'
(\#0094-1)}{*roˀc squeeze/milk (v) (\#0094-1)}}\label{roux2c0c-squeezemilk-v-0094-1}

(ra'd), FR, (rad), RSED.p226, riʔ, BDBH.2276, roj, DSGU\#2071, roˀj,
PKED.p170, roɟ, PJDW.p268, rweʔ, PGEG.p41, roco, BSDV5.p98, roeʔ,
EMV12.p3628, ro:eʔ, DHED.p290, roej, DSKW@19520, ro(:)c, NKEV.p335,
*roˀc, squeeze/milk (v), \#0094, V381, 1061,

\begin{itemize}
\tightlist
\item
  Pinnow 1959: V381 / MKCD: 1061 \emph{*ruut}; \emph{*ruət};
  \emph{*rət}; \emph{*rat}; \emph{*rit}; \emph{*riit}; \emph{*riət}
\end{itemize}

\subparagraph{\texorpdfstring{\emph{*per} `to burn (of chilies) (vi)'
(\#0097-3)
\emph{*r₂}?}{*per to burn (of chilies) (vi) (\#0097-3) *r₂?}}\label{per-to-burn-of-chilies-vi-0097-3-r}

per ,FR ,--- ,--- ,per ,BDBH.1756 ,per ,Z1975.294 ,--- ,--- ,--- ,---
,pir ,PGEG.p38 ,pe̠ṛen ,CSED.p500 ,--- ,--- ,(pertol) ,DHED.p266 ,---
,--- ,--- ,--- ,*per ,burn(chilies) (v) ,\#0097 ,--- , ,

\begin{itemize}
\tightlist
\item
  Pinnow 1959: --- / MKCD: ---
\end{itemize}

\paragraph{\texorpdfstring{\emph{*r₂}}{*r₂}}\label{r-1}

\begin{longtable}[]{@{}llllllllllll@{}}
\toprule
Gorum & Sora & Remo & Gutob & Kharia & Juang & Gtaʔ & Santali & Mundari
& Ho & Korwa & Korku\tabularnewline
\midrule
\endhead
r & r & r & r & r & r & r & r & r & r & r & ɽ\tabularnewline
\bottomrule
\end{longtable}

\subparagraph{\texorpdfstring{\emph{*bVrəl} `raw'
(\#0019-3)}{*bVrəl raw (\#0019-3)}}\label{bvrux259l-raw-0019-3}

---, ---, ---, ---, buro, BDBH.1937, burol, GZ65.74, borol, PKED.p25,
boron, PJDW.p171, brwa, PGEG.p14, be̠re̠l, CDES.p211, berel, BMED.p21,
berel, HOGV.p185, berel, BAHL.p111, boboɽ, NKEV.p294, *bxrəl, raw,
\#0019, V253, ,

\begin{itemize}
\tightlist
\item
  Pinnow 1959: V253 / MKCD: ---
\end{itemize}

\subparagraph{\texorpdfstring{\emph{*ɲv₍₂₆₎r} `run (v)'
(\#0052-3)}{*ɲv₍₂₆₎r run (v) (\#0052-3)}}\label{ux272vr-run-v-0052-3}

jer, FR, jer, RSED.p88, ur, BDBH.155, ---, ---, yar, DSKH\#12601, ---,
---, wir, PGEG.p9, ɲir, CDES.p164, nir, BMED.p132, nir, DHED.p246, ɲir,
BAHL.p66, niɽe, NKEV.p328, *ɲxr, run (v), \#0052, K294, 1602,

\begin{itemize}
\tightlist
\item
  Pinnow 1959: K294 / MKCD: 1602 \emph{*jarʔ}
\end{itemize}

\paragraph{\texorpdfstring{\emph{*r₃}}{*r₃}}\label{r-2}

\begin{longtable}[]{@{}llllllllllll@{}}
\toprule
Gorum & Sora & Remo & Gutob & Kharia & Juang & Gtaʔ & Santali & Mundari
& Ho & Korwa & Korku\tabularnewline
\midrule
\endhead
r & r & r & r & r & ɭ & r & r & ɽ & r & r & ---\tabularnewline
\bottomrule
\end{longtable}

\subparagraph{\texorpdfstring{\emph{*xrel} `hail/pebble'
(\#0032-2)}{*xrel hail/pebble (\#0032-2)}}\label{xrel-hailpebble-0032-2}

aril, FR, are:l, RSED.p39, are, BDBH.43, arel, HLKS.V225, arel, PKED.p7,
aɭɛn, PJDW.p158, hare, PGEG.p24, arel, CDES.p88, a:ɽil, BMED.p10, aril,
HOGV.p161, a:ril, BAHL.p10, ---, ---, *xrel, hail/pebble, \#0032, V225,
1791,

\begin{itemize}
\tightlist
\item
  Pinnow 1959: V225 / MKCD: 1791 \emph{*pril}; \emph{*priəl}
\end{itemize}

\paragraph{\texorpdfstring{\emph{*r₄}}{*r₄}}\label{r-3}

\begin{longtable}[]{@{}llllllllllll@{}}
\toprule
Gorum & Sora & Remo & Gutob & Kharia & Juang & Gtaʔ & Santali & Mundari
& Ho & Korwa & Korku\tabularnewline
\midrule
\endhead
--- & r & n & --- & --- & --- & n & r & r & r & --- & ---\tabularnewline
\bottomrule
\end{longtable}

\subparagraph{\texorpdfstring{\emph{*mv₍₄₎raŋ} `big'
(\#0064-3)}{*mv₍₄₎raŋ big (\#0064-3)}}\label{mvraux14b-big-0064-3}

---, ---, maraŋ/məraŋ, RSED.p173/167, munaʔ, BDBH.2121, (moɖo),
AG08.p663, ---, ---, ---, ---, mnaʔ, PGEG.35, maraŋ, CDES.p17, maraŋ,
BMED.p220, maraŋ, DHED.p225, ---, ---, ---, ---, *mxrxŋ, big, \#0064,
K107, ,

\begin{itemize}
\tightlist
\item
  Pinnow 1959: K107 / MKCD: ---
\end{itemize}

Gtaʔ \emph{mnaʔ} and Remo \emph{munaʔ} are irregular reflexes of ,
especially the Gtaʔ form \emph{mnaʔ} should be different, given our
current understanding of the phonological developments, since a velar
coda \emph{*aŋ} results in Gtaʔ /ia/ (as should /aʔ/). Remo and Gtaʔ /n/
are also inconsistent as reflexes or either \emph{*r} or \emph{*ŋ}. Gtaʔ
\emph{mnaʔ} and Remo \emph{munaʔ} are consistently parallel to one
another.

\paragraph{\texorpdfstring{\emph{*r₅}}{*r₅}}\label{r-4}

\begin{longtable}[]{@{}llllllllllll@{}}
\toprule
Gorum & Sora & Remo & Gutob & Kharia & Juang & Gtaʔ & Santali & Mundari
& Ho & Korwa & Korku\tabularnewline
\midrule
\endhead
∅ & r & r & r & r & ∅ & r & r & r & r & --- & r\tabularnewline
\bottomrule
\end{longtable}

\subparagraph{\texorpdfstring{\emph{*bar} `two'
(\#0078-3)}{*bar two (\#0078-3)}}\label{bar-two-0078-3}

bagu, FR, bar, RSED.p48, mbaʔr, BDBH.2214, umbar, AG08.p646, ubar,
PKED.p205, umba, PJDW.p291, mbar, PGEG.p34, bar, CSED.p42, baria,
BMED.p20, bar, DHED.p27, ---, ---, ba:r, NKEV.293, *bar, two, \#0078,
V49, 1562,

\begin{itemize}
\tightlist
\item
  Pinnow 1959: V49 / MKCD: 1562 \emph{*biʔaar} \textgreater{}
  \emph{*ɓaar}, Pre-Khmer \emph{*{[}ɓ{]}ir}, Pre-Palaungic \&c.
  \emph{*ʔaar}
\end{itemize}

\subsubsection{\texorpdfstring{\emph{*ʔ/ˀ}}{*ʔ/ˀ}}\label{ux294ux2c0}

Proto-Austro-Asiatic \emph{*ʔ} was probably lost in Proto-Munda.
However, there are some indications that there was some glottal element
in proto-Munda. Whether this is identical with the assumed \emph{*Vˀ}
discussed in the vocalism section or whether it constitues a separate
phoneme \emph{*ʔ} is unclear.

\subparagraph{\texorpdfstring{\emph{*ɟV(V)ˀ} `fruit; bear fruit
(v)'}{*ɟV(V)ˀ fruit; bear fruit (v)}}\label{ux25fvvux2c0-fruit-bear-fruit-v}

zoʔ, FR, ɟo:ʔ, RSED.p125, suʔ, BDBH.2701, ---, ---, ---, ---, ---, ---,
cu, PGEG.p15, jo̠, CDES.p80, jo, BMED.p83, jo:(ʔ), DHED.p83, joʔ,
BAHL.p63, jo:, NKEV.p313, *ɟx(x)ˀ, fruit / to bear fruit (v), \#0030,
V188, ,

\emph{*ɟV(V)ˀ} `fruit; bear fruit (v)' contrasts with \emph{*ɟVˀk}
`sweep (v)':

\begin{longtable}[]{@{}llllllllllll@{}}
\toprule
Gorum & Sora & Remo & Gutob & Kharia & Juang & Gtaʔ & Santali & Mundari
& Ho & Korwa & Korku\tabularnewline
\midrule
\endhead
zoʔ & ɟo:ʔ & suʔ & & & & cu & jo̠ & jo & jo:(ʔ) & joʔ &
jo:\tabularnewline
zoʔ & ɟo: & suk & sog & joʔ & ɟɛnɔg & coʔ & jo̠k' & joʔ & joʔ & joʔ &
ju\tabularnewline
\bottomrule
\end{longtable}

Both feature a a coda with a very similar vowel and in most languages a
glottal element.

Remo /ʔ/ versus /k/ seems to reflect the distinction. If this represents
a genuine phonological distinction and not inconsistencies in BDBH, this
would be crucial evidence for the reconstruction.

\subparagraph{\texorpdfstring{\emph{*kᵊla} `tiger'
(\#0004-5)}{*kᵊla tiger (\#0004-5)}}\label{kux1d4ala-tiger-0004-5}

kulaʔ, FR, kina:, RSED.p140, ŋku, MVol.p733, gikil, AG08.p651, kiɽoʔ,
PKED.p102, kiɭog, PJDW.p224, nku, PGEG.p36, kul, CDES.p201, kula:,
BMED.p98, kula, HOGV.p183, ku:l, BAHL.p33, kula, NKEV.p319, *kᵊla,
tiger, \#0004, V281, 197,

\begin{itemize}
\tightlist
\item
  Pinnow 1959: V281 / MKCD: 197 \emph{*klaʔ}
\end{itemize}

\subparagraph{\texorpdfstring{\emph{*tiiˀ} `hand'
(\#0008-3)}{*tiiˀ hand (\#0008-3)}}\label{tiiux2c0-hand-0008-3}

siʔ, FR, si:ʔ, RSED.p254, titi, BDBH.1370, titi, GZ65.p29, tiʔ,
PKED.p199, iti, PJDW.p208, nti, PGEG.p37, ti, CDES.p89, ti, BMED.p186,
ti:, DHED.p350, tiʔi:, BAHL.p63, ʈi, NKEV.p343, tiiˀ, hand, \#0008, V75,
66, 48

\begin{itemize}
\tightlist
\item
  Pinnow 1959: V75 / MKCD: 66 \emph{*t₁iiʔ}
\end{itemize}

\subparagraph{\texorpdfstring{\emph{*sii₃ˀ} `louse'
(\#0009-3)}{*sii₃ˀ louse (\#0009-3)}}\label{siiux2c0-louse-0009-3}

(aŋiˀd), FR, iʔi, RSED.p109, gisi, BDBH.855, gisi, AG08.p651, seʔ,
PKED.p258, ɛsɛ, PJDW.p192, gsi, PGEG.p23, se, CDES.p116, siku,
BMED.p173, siku, HOGV.p165, guhi:, BAHL.p45, siku, NKEV.p338, sii₂ˀ,
louse, \#0009, V341, 39, 22

\begin{itemize}
\tightlist
\item
  Pinnow 1959: V341 UM: e,ɛ / MKCD: 39 \emph{*ciiʔ} (\& \emph{*ciʔ}?)
\end{itemize}

\subparagraph{\texorpdfstring{\emph{*muuˀ} `nose'
(\#0074-3)}{*muuˀ nose (\#0074-3)}}\label{muuux2c0-nose-0074-3}

muʔ, FR, mu:ʔ, RSED.p179, nseʔmiʔ, BDBH.1653, miʔ, GZ63.262,
romoŋ/romoˀɖ, PKED.p170, motɛɟ, PJDW.p245, mmu, PGEG.p34, muN,
CDES.p129, mu/muhu, BMED.p121, muwa/muʈa, DHED.p238, hu:mu:, DSKW@23180,
mu:, NKEV.p327, *mxxˀ, nose, \#0074, , ,

\begin{itemize}
\tightlist
\item
  Pinnow 1959: V387 / MKCD: 2045 \emph{*muh}; \emph{*muuh}; \emph{*muus}
\end{itemize}

\subparagraph{\texorpdfstring{\emph{*ɟa}; \emph{*ɟaaˀ}; \emph{*ɟaˀt}
`additive.particle'
(\#0079-3)}{*ɟa; *ɟaaˀ; *ɟaˀt additive.particle (\#0079-3)}}\label{ux25fa-ux25faaux2c0-ux25faux2c0t-additive.particle-0079-3}

zaˀd, FR, ɟa:, RSED.p117, sa, BDBH.2547, sa, AG08.p649, ja, HLKS.V1,
ɟan, PJDW.p211, , , ja, BSDV3.p216, ja:, BMED.p77, ja:, DHED.p155, ja'',
DSKW.@09330, ja, DSKO.12141, *ɟa(ˀt), additive.particle, \#0079, V1, ,

\begin{itemize}
\tightlist
\item
  Pinnow 1959: V1 / MKCD: ---
\end{itemize}

Gorum /ˀd/ and maybe Juang /n/ point to some alveolar element, but the
other forms cosistenly relfect *ɟa\_ or alternatively \emph{*ɟaaˀ}.

\subparagraph{\texorpdfstring{\emph{*bv₍₃₁₎/bv₍₃₁₎v₍₃₁₎ˀ/bv₍₃₁₎sv₍₃₂₎}
`sated (v)'
(\#0098-3)}{*bv₍₃₁₎/bv₍₃₁₎v₍₃₁₎ˀ/bv₍₃₁₎sv₍₃₂₎ sated (v) (\#0098-3)}}\label{bvbvvux2c0bvsv-sated-v-0098-3-1}

buʔ, FR, beˀ, RSED.p56, busu, BDBH.1960, busu, Z1965.72, beso/u,
PKED.p20, bisu, PJDW.p14, bse, PGEG.p14, bi(ʔ), CSED.p67, bi:(ʔ)/biu,
BMED.22, bi:, DHED.p35, bi:, BAHL.p106, ---, ---, *bx, be sated (v),
\#0098, V319, 259,

\begin{itemize}
\tightlist
\item
  Pinnow 1959: V319 / MKCD: 259 \emph{*bhiiʔ}
\end{itemize}

\subsection{Relevant sound changes in branches or individual
languages}\label{relevant-sound-changes-in-branches-or-individual-languages}

to-be-done

s-loss: Sora-Gorum

l-loss: Remo, Gtaʔ

merger of \emph{*s} and \emph{*ɟ} (and \emph{*c}?) Remo-Gutob

palatal nasal and velar: various languages

coda neutralisation: Remo-Gutob, Gtaʔ

nasal loss in coda: Gtaʔ

diphthongisation: Gtaʔ

\subsection{List of Etyma \#0001-\#0100}\label{list-of-etyma-0001-0100}

to-be-done

\subparagraph{\texorpdfstring{\#0001 \emph{*daˀk}
`water'}{\#0001 *daˀk water}}\label{daux2c0k-water}

ɖaʔ, FR, daʔ, RSED.p70, dak', BDBH.1179, ɖaʔ, ZG63.85, ɖaʔ, PKED.p41,
ɖag, PJDW.p185, ndiaʔ, PGEG.p36, dak', CDES.p217, da:, BMED.p31, daʔ,
DHED.p73, da:ʔ, BAHL.p87, ɖa, NKEV.p300, da(a)ˀk, water, \#0001, V2,
274, 75

\begin{itemize}
\item
  Pinnow 1959: V2 UM: \emph{*a} / MKCD: 274 \emph{*diʔaak}
  \textgreater{} \emph{*ɗaak}
\item
  MKCD Pre-Proto-Mon-Khmer \emph{*diʔaak} \textgreater{} \emph{*ɗaak}
  (all branches), \emph{*{[}ɗ{]}ik} (Pre-Khmer)
\item
  Pinnow 1959: V2 UM:\emph{*a}; K89/K179d UM:\emph{*ʔ/k/g}; K398b
  UM:\emph{*d}
\end{itemize}

Proto-Sora-Gorum: \emph{*daʔ}; Proto Remo-Gutob: \emph{*daˀk};
Proto-Khewarian: \_*daˀk

\subparagraph{\texorpdfstring{\#0002 \emph{*ɟaŋ}
`bone'}{\#0002 *ɟaŋ bone}}\label{ux25faux14b-bone}

za̰ŋ, FR, əɟaŋ, RSED.p6, siʔsaŋ, BDBH.2614, sisaŋ, AG08.p651, jaŋ,
PKED.p83, ɟaŋ, PJDW.p210, ncia, PGEG.p36, jaŋ, CDES.p19, ja:ŋ, BMED.p80,
jaŋ, HOGV.p150, ja:ŋ, BAHL.p60, ---, , ɟa(a)ŋ, bone, \#0002, V7, 488, 31

\begin{itemize}
\tightlist
\item
  Pinnow 1959: V7 / MKCD: 488 \emph{*cʔaaŋ} ; \emph{*cʔaiŋ};
  \emph{*cʔi{[} {]}ŋ}
\end{itemize}

\subparagraph{\texorpdfstring{\#0003 \emph{*laŋ}
`tongue'}{\#0003 *laŋ tongue}}\label{laux14b-tongue}

laŋ, FR, əlaŋ, RSED.p158, leaŋ, BDBH.2423, laʔŋ, AG08.p638, laŋ,
PKED.p173, elaŋ, PJDW.p191, nlia, PGEG.p36, alaŋ, CDES.p203, a:la:ŋ,
BMED.p5, (leʔ), DHED.p208, a:la:ŋ, BAHL.p11, laŋ, NKEV.p322, la(a)ŋ,
tongue, \#0003, V14, , 44

\begin{itemize}
\tightlist
\item
  Pinnow 1959: V14 / MKCD: ---
\end{itemize}

\subparagraph{\texorpdfstring{\#0004 \emph{*kᵊla}
`tiger'}{\#0004 *kᵊla tiger}}\label{kux1d4ala-tiger-1}

kulaʔ, FR, kina:, RSED.p140, ŋku, MVol.p733, gikil, AG08.p651, kiɽoʔ,
PKED.p102, kiɭog, PJDW.p224, nku, PGEG.p36, kul, CDES.p201, kula:,
BMED.p98, kula, HOGV.p183, ku:l, BAHL.p33, kula, NKEV.p319, *kᵊla,
tiger, \#0004, V281, 197,

\begin{itemize}
\tightlist
\item
  Pinnow 1959: V281 / MKCD: 197 \emph{*klaʔ}
\end{itemize}

\subparagraph{\texorpdfstring{\#0005 \emph{*taɲ} `to
weave'}{\#0005 *taɲ to weave}}\label{taux272-to-weave}

taɲ, FR, taɲ, RSED.p281, taNy, BDBH.1358, taɲ, GZ65.369, taɲ, PKED.p196,
---, ---, tæ, PGEG.p45, teɲ, CDES.p219, teŋ, BMED.p183, teɲ, HOGV.p187,
---, ---, ---, ---, taɲ, weave (v), \#0005, V301, 898,

\begin{itemize}
\tightlist
\item
  Pinnow 1959: V301 / MKCD: 898 \emph{*t₁aaɲ}
\end{itemize}

\subparagraph{\texorpdfstring{\#0006 \emph{*daˀc} `to
climb'}{\#0006 *daˀc to climb}}\label{daux2c0c-to-climb}

ɖaˀɟ, FR, daɟ, RSED.p72, ɖaĭ, BDBH.1168, ɖaj, GZ65.79, ---, ---, ɖaɲ,
PJDW.p186, ɖæʔ, PGEG.p16, de̠c', CDES.p32, dej', BMED.p40, deʔ, DHED.p81,
deʔ, BAHL.p89, (cuɖe), NKEV.p298, daˀɟ, climb (v), \#0006, V333, ,

\begin{itemize}
\tightlist
\item
  Pinnow 1959: V333 / MKCD: ---
\end{itemize}

\subparagraph{\texorpdfstring{\#0007 \emph{*dəraŋ}
`horn'}{\#0007 *dəraŋ horn}}\label{dux259raux14b-horn}

ɖeraŋ, FR, deraŋ, RSED.p78, deruŋ, BDBH.1266, ---, ---, ɖereŋ, PKED.p44,
---, ---, ɖiraŋ, PGEG.p17, dereɲ, CDSE.p171, diriŋ, BMED.p49, diriɲ,
HOGV.p162, dereŋ, BAHL.p89, ---, ---, dəraŋ, horn, \#0007, V347, 699, 34

\begin{itemize}
\tightlist
\item
  Pinnow 1959: V347 / MKCD 699 \emph{*d₂raŋ}
\end{itemize}

\subparagraph{\texorpdfstring{\#0008 \emph{*tiiˀ}
`hand'}{\#0008 *tiiˀ hand}}\label{tiiux2c0-hand}

siʔ, FR, si:ʔ, RSED.p254, titi, BDBH.1370, titi, GZ65.p29, tiʔ,
PKED.p199, iti, PJDW.p208, nti, PGEG.p37, ti, CDES.p89, ti, BMED.p186,
ti:, DHED.p350, tiʔi:, BAHL.p63, ʈi, NKEV.p343, tiiˀ, hand, \#0008, V75,
66, 48

\begin{itemize}
\tightlist
\item
  Pinnow 1959: V75 / MKCD: 66 \emph{*t₁iiʔ}
\end{itemize}

\subparagraph{\texorpdfstring{\#0009 \emph{*sii₃ˀ}
`louse'}{\#0009 *sii₃ˀ louse}}\label{siiux2c0-louse}

(aŋiˀd), FR, iʔi, RSED.p109, gisi, BDBH.855, gisi, AG08.p651, seʔ,
PKED.p258, ɛsɛ, PJDW.p192, gsi, PGEG.p23, se, CDES.p116, siku,
BMED.p173, siku, HOGV.p165, guhi:, BAHL.p45, siku, NKEV.p338, sii₂ˀ,
louse, \#0009, V341, 39, 22

\begin{itemize}
\tightlist
\item
  Pinnow 1959: V341 UM: e,ɛ / MKCD: 39 \emph{*ciiʔ} (\& \emph{*ciʔ}?)
\end{itemize}

Unexplained variation of \emph{*i₁}, especially the constrast to *tiiˀ\_
`hand' (\#0008-2) is striking. Kharia /e/, Juang /ɛ/, and Santali /e/
cannot be explained. Pinnow (1959, p.~164 and p.~195) reconstructs
proto-Munda \emph{*e}/\emph{*ɛ}. However, positing \#0009-2 as a
continuation of proto-Munda \emph{*e} (\emph{*ɛ}) is also not
consistent. MKCD 39 \emph{*ciiʔ} also suggests proto-Munda \emph{*siiˀ}.

\subparagraph{\texorpdfstring{\#0010 \emph{*ɟv₍₇₎ŋ}
`foot'}{\#0010 *ɟv₍₇₎ŋ foot}}\label{ux25fvux14b-foot}

zḭŋ, FR, ɟe:ˀŋ, RSED.p123, suŋ, BDBH.1363, suŋ, GZ63.205, juŋ, PKED.p66,
iɟiŋ, PJDW.p208, nco, PGEG.p114, jaŋga, CDES.p76, jaŋga, HLKS.182, ---,
---, ---, ---, (naŋga), NKEV.p327, , foot, \#0010, V365, 538,

\begin{itemize}
\tightlist
\item
  Pinnow 1959: V365 / MKCD 538 \emph{*juŋ}; \emph{*juəŋ}; \emph{*jəŋ};
  \emph{*jəəŋ}
\end{itemize}

This is a unique set with unclear reflexes.

Pinnow (1959, p.~169) says ``\ldots{}so bleibt der Vokalwechsel des
Wortes für Fuß, Bein ein gänzlich ungelöstes Rätsel der
austroasiatischen Sprachwissenschaft,\ldots{}''

\subparagraph{\texorpdfstring{\#0011 \emph{*b(oKʰ)Vˀp}
`head'}{\#0011 *b(oKʰ)Vˀp head}}\label{bokux2b0vux2c0p-head}

baˀb, FR, bo:ˀb, RSED.p60, bob, BDBH.2007, bob, GZ63.50, bokoˀb,
PKED.p24, bokob, PJDW.p169, bhaʔ, PGEG.p13, bo̠ho̠k', CDES.p90, bo,
BMED.p24, bo:ʔ, DHED.p40, boʔ, BAHL.p113, ---, ---, , head, \#0011,
V206, 361, 38

\begin{itemize}
\tightlist
\item
  Pinnow 1959: V206 / MKCD: 361 \emph{*{[}b{]}uuk}
\end{itemize}

The reflexes of V₁ in \emph{*b(oKʰ)Vˀp} `head' (\#0011) -- Kharia /o/,
Juang /o/,Santali /o̠/ -- are too incomplete to assign the set to any
correspondence set, unequivocally. \#0011-2 is consistent with
\emph{*o₁} and \emph{*o₂}.

\subparagraph{\texorpdfstring{\#0012 \emph{*məˀt}
`eye'}{\#0012 *məˀt eye}}\label{mux259ux2c0t-eye}

maˀd, FR, mo:ˀd/mad, RSED.p168, moʔ, BDBH.220, moʔ, AG08.p642, moˀɖ,
PKED.p195, ɛmɔɖ, PJDW.p191, muaʔ, PGEG.p34, mẽ̠t', CDES.p67, med',
BMED.p117, meɖ, DHED.p228, meɖ, BAHL.p120, med, ZKPM.p48, məˀt, eye,
\#0012, V250, 1045, 40

\begin{itemize}
\tightlist
\item
  Pinnow 1959: V250 / MKCD: 1045 *mat
\end{itemize}

\subparagraph{\texorpdfstring{\#0013 \emph{*gəˀt} `cut
(v)'}{\#0013 *gəˀt cut (v)}}\label{gux259ux2c0t-cut-v}

gaˀd, FR, gad, RSED.p93, goʔ, BDBH.1018, goʔ, AG08.p669, gaˀɖ, PKED.p60,
, , gwaʔ, PGEG.p21, ge̠t', CDES.p44, ged', EMV5.1411, geɖ, DHED.p111,
geɖ, BAHL.p46, geʈ, NKEV.p306, gəˀt, cut (v), \#0013, V334, 972,

\begin{itemize}
\tightlist
\item
  Pinnow 1959: V334 / MKCD: MKCD 972 \emph{*sguut}; \emph{*{[}s{]}gət};
  \emph{*sgat}
\end{itemize}

Kharia \emph{gaˀɖ} `to reap' seems to be not widely used or even
internally reconstructed from \emph{ganaˀɖ} `sickle'. The vowel /a/
differs from Kharia /o/ in \#0012 \emph{*məˀt} `eye'.

\subparagraph{\texorpdfstring{\#0014 \emph{*v₍₉₎laŋ}
`thatch'}{\#0014 *v₍₉₎laŋ thatch}}\label{vlaux14b-thatch}

alaŋ, FR, əlaŋ, RSED.p158, lɔŋ, BDBH.2437, uloŋ, AG08.p644, oloŋ,
PKED.p214, oloŋ, PJDW.p254, nlo, PGEG.p36, ---, ---, ---, ---, ---, ---,
---, ---, ---, ---, *v₍₉₎laŋ, thatch, \#0014, V270, 749,

\begin{itemize}
\tightlist
\item
  Pinnow 1959: V270 / MKCD: 749 \emph{*{[}p{]}laŋ}; \emph{*{[}p{]}laiŋ}
\end{itemize}

Incomplete set, due to absence of V₁ in Remo and Gtaʔ and the absence of
this etymon in North Munda. The reflexes suggest a central or bac vowel.
MKCD: 749 \emph{*{[}p{]}laŋ}/\emph{*{[}p{]}laiŋ} would favour epenthetic
\emph{*ə}. (The loss of initial \emph{*p} seems from the current
understanding irregular.)

\paragraph{\texorpdfstring{\#0015 \emph{*ɟv₍₁₀₎m} `eat
(v)'}{\#0015 *ɟv₍₁₀₎m eat (v)}}\label{ux25fvm-eat-v}

zum, FR, ɟom, RSED.p128, sum, BDBH.2667, som, GZ63.212, jom, HLKS.K274,
ɟim, PJDW.p212, coŋ, PGEG.p15, jo̠m, CDES.p60, jom, BMED.p84, jom,
HOGV.p156, jom, BAHL.p63, jom, NKEV.p313, , eat (v), \#0015, V385, 1327,
55

\begin{itemize}
\tightlist
\item
  Pinnow 1959: V385 / MKCD: 1327 \emph{*cuum}; \emph{*cuəm};
  \emph{*cəm}; (\emph{*cim cim} \textgreater{}) \emph{*ncim};
  \emph{*ciəm} (\& \emph{*nciəm}?); \emph{*caim}
\end{itemize}

\end{document}
